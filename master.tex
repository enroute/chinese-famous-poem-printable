\documentclass[a4paper,12pt,book]{memoir}

% remove marginnote space for book class in memoir
% \usepackage[nomarginpar]{geometry}
% remove margin note space
% \setlength{\marginparwidth}{0pt}

% NOT WORK!
% \setmarginnotes{1pt}{1pt}{1pt}

% set margin
% \setlrmarginsandblock{left}{right}{ratio}
% \setulmarginsandblock{top}{bottom}{ratio}
\setlrmarginsandblock{3.5cm}{3.5cm}{*}
\setulmarginsandblock{3.5cm}{*}{1}
\checkandfixthelayout 

%% To fix a bug in TeXLive 2018, ref.
%% https://tex.stackexchange.com/questions/427451/dot-between-chapter-number-and-figure-number-disapear-after-update
%% https://tex.stackexchange.com/questions/427995/memoir-on-tex-live-2018-breaks-figure-numbering-after-mainmatter
\makeatletter
\renewcommand{\counterwithin}{\@ifstar{\@csinstar}{\@csin}}
\makeatother

% comment this if don't want to show answer
\def\withanswer{1}

% Make different footmark reference the same footnote
\makeatletter
\newcommand\footnoteref[1]{\protected@xdef\@thefnmark{\ref{#1}}\@footnotemark}
\makeatother

% Turn on subsection numbering in memoir class
\setsecnumdepth{subsection}

\usepackage{pbox}
\usepackage{xpinyin}
\usepackage{metalogo}           %for \XeTeX, \XeLaTeX, \LuaTeX and \LuaLaTeX

%%% \DeclareUnicodeCharacter not compatible with xelatex
% \usepackage[utf8]{inputenc}
% \DeclareUnicodeCharacter{1F019}{\yibing}
% \DeclareUnicodeCharacter{1F01A}{\erbing}
% \DeclareUnicodeCharacter{1F01B}{\sanbing}
% \newcommand{\yibing}{\symbol{"1F019}}
% \newcommand{\yibing}{\fontfamily{Symbola}\selectfont\DeclareUnicodeCharacter{1F019}{kk}}

% Ref. Mahjong tiles in unicode: http://www.unicode.org/charts/PDF/U1F000.pdf
% Symbola font is one of those fonts that implements the mahjong range in unicode
\newcommand{\mahjongsymbol}[1]{\begingroup\Huge\setmainfont{Symbola}%
  \rule[-.3\baselineskip]{0pt}{\baselineskip}%strut to ensure the height
  \symbol{"#1}\endgroup}
\newcommand{\mjbing}[1]{
  \ifthenelse{#1=1}{\mahjongsymbol{1F019}}{% 1 bing
    \ifthenelse{#1=2}{\mahjongsymbol{1F01A}}{% 2 bing
      \ifthenelse{#1=3}{\mahjongsymbol{1F01B}}{% 3 bing
        \ifthenelse{#1=4}{\mahjongsymbol{1F01C}}{% 4 bing
          \ifthenelse{#1=5}{\mahjongsymbol{1F01D}}{% 5 bing
            \ifthenelse{#1=6}{\mahjongsymbol{1F01E}}{% 6 bing
              \ifthenelse{#1=7}{\mahjongsymbol{1F01F}}{% 7 bing
                \ifthenelse{#1=8}{\mahjongsymbol{1F020}}{% 8 bing
                  \ifthenelse{#1=9}{\mahjongsymbol{1F021}}{% 9 bing
                  }}}}}}}}}}
  
\newcommand{\yibing}{\mjbing1}
\newcommand{\erbing}{\mjbing2}
\newcommand{\sanbing}{\mjbing3}
\newcommand{\sibing}{\mjbing4}
\newcommand{\wubing}{\mjbing5}
% \newcommand{\yibing}{\mahjongsymbol{1F019}}
% \newcommand{\erbing}{\mahjongsymbol{1F01A}}
% \newcommand{\sanbing}{\mahjongsymbol{1F01B}}
% \newcommand{\sibing}{\mahjongsymbol{1F01C}}
% \newcommand{\wubing}{\mahjongsymbol{1F01D}}

% \newcommand{\yibing}{\begingroup\Huge\setmainfont{Symbola}\symbol{"1F019}\endgroup}
% \newcommand{\erbing}{\begingroup\Huge\setmainfont{Symbola}\symbol{"1F01A}\endgroup}
% \newcommand{\sanbing}{\begingroup\Huge\setmainfont{Symbola}\symbol{"1F01B}\endgroup}
% \newcommand{\sibing}{\begingroup\Huge\setmainfont{Symbola}\symbol{"1F01C}\endgroup}
% \newcommand{\wubing}{\begingroup\Huge\setmainfont{Symbola}\symbol{"1F01D}\endgroup}

%% subfigure is obsolete and conilict with tocloft package, use subcaption or subfig insted
% \usepackage{subfigure}
% \usepackage[subfigure]{tocloft}
% \PassOptionsToPackage{subfigure}{tocloft}
\usepackage{subcaption}

% toc style
\usepackage{titletoc}% http://ctan.org/pkg/titletoc
\setcounter{tocdepth}{2}% Display up to \subsection in ToC
% \titlecontents*{section}% <section>
\titlecontents*{subsection}% <section>
  [3.8em]% <left>
  {\small}% <above-code>
  {}% <numbered-entry-format>; you could also use {\thecontentslabel. } to show the numbers
  {}% <numberless-entry-format>
  {\ \thecontentspage}% <filler-page-format>
  [,\ \ \ \ ]% <separator>
  []% <end>

% For epigraphs
\usepackage{epigraph}

% Use varwidth to make epigraph length adaptive
\usepackage{varwidth}
\renewcommand{\epigraphsize}{\normalsize}
\setlength{\epigraphwidth}{0.8\textwidth}
\renewcommand{\textflush}{flushright}
\renewcommand{\sourceflush}{flushright}
% A useful addition
\newcommand{\epitextfont}{\kai}
\newcommand{\episourcefont}{\kai}

\makeatletter
\newsavebox{\epi@textbox}
\newsavebox{\epi@sourcebox}
\newlength\epi@finalwidth
\renewcommand{\epigraph}[2]{%
  \vspace{\beforeepigraphskip}
  {\epigraphsize\begin{\epigraphflush}
   \epi@finalwidth=\z@
   \sbox\epi@textbox{%
     \varwidth{\epigraphwidth}
     \begin{\textflush}\epitextfont#1\end{\textflush}
     \endvarwidth
   }%
   \epi@finalwidth=\wd\epi@textbox
   \sbox\epi@sourcebox{%
     \varwidth{\epigraphwidth}
     \begin{\sourceflush}\episourcefont#2\end{\sourceflush}%
     \endvarwidth
   }%
   \ifdim\wd\epi@sourcebox>\epi@finalwidth 
     \epi@finalwidth=\wd\epi@sourcebox
   \fi
   \leavevmode\vbox{
     \hb@xt@\epi@finalwidth{\hfil\box\epi@textbox}
     \vskip1.75ex
     \hrule height \epigraphrule
     \vskip.75ex
     \hb@xt@\epi@finalwidth{\hfil\box\epi@sourcebox}
   }%
   \end{\epigraphflush}
   \vspace{\afterepigraphskip}}}
\makeatother

%%% check mark and x mark
\usepackage{pifont}% http://ctan.org/pkg/pifont
\newcommand{\cmark}{\ding{51}}%
% \newcommand{\xmark}{\ding{55}}%
\newcommand{\xmark}{\times}
\newcommand{\redcross}{{\color{red}\ding{53}}}

\newcommand*\numcircled[1]{\raisebox{.5pt}{\textcircled{\raisebox{-.9pt} {#1}}}}


\usepackage[xetex,
            bookmarksnumbered=true,
            bookmarksopen=true,
            colorlinks=false,
            pdfborder={0 0 1},
            citecolor=blue,
            linkcolor=red,
            anchorcolor=green,
            urlcolor=blue,
            breaklinks=true,
            naturalnames  %与algorithm2e宏包协调
            ]{hyperref}

% For Chinese fonts
% \usepackage{fontspec}
% \setmainfont{SimSun}
% \usepackage[AutoFakeBold,SlantFont]{xeCJK}
% % \usepackage[SlantFont]{xeCJK}   %AutoFakeBold=true makes CJK characters in the generated PDF can't be copied

% \setCJKmainfont{SimSun}

%%% DON'T USE AutoFakeBold, or issues will arise:
%%% https://tex.stackexchange.com/questions/266894/titlesec-autofakebold-making-pdf-figures-bold
%%% https://stackoverflow.com/questions/4422054/why-is-latex-making-my-caption-bold
%%% Use the following font setting instead:
%%% https://www.jianshu.com/p/b1751078e28e
\usepackage{xeCJK}
\setCJKmainfont[ItalicFont={楷体}, BoldFont={黑体}]{宋体}%衬线字体 缺省中文字体为
\setCJKsansfont{黑体}
\setCJKmonofont{仿宋_GB2312}%中文等宽字体


\usepackage{multicol}
\usepackage{multirow}
\usepackage{ulem}               % For \sout, \uwave
\usepackage[usenames,dvipsnames]{pstricks}
\usepackage{pstricks-add}  % For \psrotate, \psbrace
\usepackage{epsfig}
\usepackage{pst-grad} % For gradients
\usepackage{pst-plot} % For axes
\usepackage{pst-node} % For nodes
\usepackage[space]{grffile} % For spaces in paths
\usepackage{etoolbox} % For spaces in paths
\AtBeginEnvironment{quote}{\kai\small} %for quote environment
\AtBeginEnvironment{quotation}{\kai\small} %for quotation environment
\patchcmd{\quotation}{1.5em}{2em}{}{}

\usepackage{pst-eucl}

\usepackage[english]{babel}     % For enquote
\usepackage{csquotes} 

\usepackage{tabularx}
\usepackage{colortbl}
% \usepackage[table]{xcolor}      %this will load colortbl package, for the \cellcolor

\usepackage{hhline}
% from color package?
\definecolor{LightCyan}{rgb}{0.88,1,1}

\usepackage{booktabs}       % 表格,横的粗线;\specialrule{1pt}{0pt}{0pt}, not compactible with vertical line
\usepackage{boldline}       % for \hlineB, bold horizontal lines compactable with vertical line

\usepackage{tikzpeople}
\usepackage{tikz,fp}
% \usetikzlibrary{angles}         % for perpendicular symbols
\usetikzlibrary{calc,intersections,patterns,angles,quotes,shapes}           %
\usepackage{tikz-3dplot}
\usetikzlibrary{positioning}    % right=of (SHAPE)
\usetikzlibrary{3d}
\usetikzlibrary{decorations.markings} %for decorate, arrow in the middle of a line
\usetikzlibrary{decorations.text}     %for text along a path
\usetikzlibrary{through}              %for [circle through] in \node
\usetikzlibrary{arrows.meta}          %for arrow options
\usetikzlibrary{shadings}             %for shadings

%% externalizing tikz graphs
%% https://tex.stackexchange.com/questions/1460/script-to-automate-externalizing-tikz-graphics
% \usetikzlibrary{external}
% \tikzexternalize[prefix=figures/]

\newcommand{\tikzmark}[2]{\tikz[overlay,remember picture,baseline] \node [anchor=base] (#1) {$#2$};}

\newcommand{\DrawVLine}[3][black, thick, opacity=0.5]{%
  \begin{tikzpicture}[overlay,remember picture]
    \draw[shorten <=0.3ex, #1] (#2.north) -- (#3.south);
  \end{tikzpicture}
}

\newcommand{\DrawHLine}[3][black, thick, opacity=0.5]{%
  \begin{tikzpicture}[overlay,remember picture]
    \draw[shorten <=0.2em, #1] (#2.west) -- (#3.east);
  \end{tikzpicture}
}

\usepackage{tkz-euclide} % for perpendicular symbols, \tkzMarkRightAngle, \tkzDefPoint
\usetkzobj{all}

% default style
\tikzset{every picture/.style={line join=round}}

\makeatletter
\def\calcLength(#1,#2)#3{%
  \pgfpointdiff{\pgfpointanchor{#1}{center}}%
               {\pgfpointanchor{#2}{center}}%
  \pgf@xa=\pgf@x%
  \pgf@ya=\pgf@y%
  \FPeval\@temp@a{\pgfmath@tonumber{\pgf@xa}}%
  \FPeval\@temp@b{\pgfmath@tonumber{\pgf@ya}}%
  \FPeval\@temp@sum{(\@temp@a*\@temp@a+\@temp@b*\@temp@b)}%
  \FProot{\FPMathLen}{\@temp@sum}{2}%
  \FPround\FPMathLen\FPMathLen5\relax
  \global\expandafter\edef\csname #3\endcsname{\FPMathLen}
}

\def\drawArc(#1,#2,#3){%
  \pgfmathanglebetweenpoints{\pgfpointanchor{#1}{center}}{\pgfpointanchor{#2}{center}}
  \let\StartAngle\pgfmathresult
  \pgfmathanglebetweenpoints{\pgfpointanchor{#1}{center}}{\pgfpointanchor{#3}{center}}
  \let\EndAngle\pgfmathresult
  \calcLength(#1,#2){myr@dius}
  \draw (#2) arc[start angle=\StartAngle, end angle=\EndAngle, radius=\myr@dius pt];
}

% tikz command to draw arc with center
\def\centerarc[#1](#2)(#3:#4:#5)% Syntax: [draw options] (center) (initial angle:final angle:radius)
{ \draw[#1] ($(#2)+({#5*cos(#3)},{#5*sin(#3)})$) arc (#3:#4:#5); }
\makeatother

\usepackage{pgfplots}           % for polor plot
\usepgfplotslibrary{polar}
% Change to 1.16, since the following output from xelate:
% Package pgfplots notification 'compat/show suggested version=true': you might b
% enefit from \pgfplotsset{compat=1.16} (current compat level: 1.8).
% \pgfplotsset{compat=1.14}


\makeatletter % For spaces in paths
    \patchcmd\Gread@eps{\@inputcheck#1 }{\@inputcheck"#1"\relax}{}{}

    % Font of theorem subtitle
    \def\th@plain{%
      \thm@notefont{}% same as heading font
      \kai % body font
    }
    \def\th@definition{%
      \thm@notefont{}% same as heading font
      % \bfseries \kai % body font
      \bfseries \hei % body font
    }
\makeatother

\def\answer#1{
  \ifx\withanswer\undefined\else 提示:\par\nopagebreak #1\fi
}

\usepackage{mathtools,amsthm,amsfonts,amssymb,bm}
% For \notdivides which has a negating line longer than \nmid in the
% `amssymb' package and shorter than the \centernot in the `centernot'
% package.
\usepackage{mathabx}

\usepackage{extarrows}          % for xlongequal
% use `fc-list|grep -i kai' to get installed font list
\setCJKfamilyfont{kai}{KaiTi}
\setCJKmonofont{NSimSun}
\newcommand{\kai}{\CJKfamily{kai}}
\newcommand{\hei}{\CJKfamily{hei}}

%% chapter style
% \usepackage{titlesec, blindtext, color}
% \definecolor{gray75}{gray}{0.75}
% \newcommand{\hsp}{\hspace{20pt}}
% \titleformat{\chapter}[hang]{\Huge\bfseries}{\thechapter\hsp\textcolor{gray75}{|}\hsp}{0pt}{\Huge\bfseries}
\makechapterstyle{box}{
  \renewcommand*{\printchaptername}{}
  \renewcommand*{\chapnumfont}{\normalfont\sffamily\huge\bfseries}
  \renewcommand*{\printchapternum}{
    \flushright
    \begin{tikzpicture}
      \draw[fill,color=black!30] (0,0) rectangle (2cm,2cm);
      \draw[color=white] (1cm,1cm) node { \chapnumfont\thechapter };
    \end{tikzpicture}
  }
  \renewcommand*{\chaptitlefont}{\normalfont\sffamily\Huge\bfseries}
  \renewcommand*{\printchaptertitle}[1]{\flushright\chaptitlefont##1}
}
\chapterstyle{box}
%% This won't work with the `babel' package
%% \renewcommand\chaptername{第~\thechapter~章}
%% We have to use the following fix
%% w/o using the babel package
%% \renewcommand{\figurename}{\kai 图}
%% w/ babel packge and English as language
\addto\captionsenglish{%
  % \renewcommand\chaptername{第~\thechapter~章}
  \renewcommand\chaptername{}
  \renewcommand\contentsname{目~~~~录}
  \renewcommand\tablename{表}
  \renewcommand\figurename{图}
  \renewcommand\figurename{\kai 图}  
}
% Or use the following
% \AtBeginDocument{%
%   \renewcommand\tablename{表}
% }

% % patch \newtheoremstyle to accept \newline<other tokens>
% \makeatletter
% \patchcmd{\newtheoremstyle}
%  {\def\@tempb{\newline}}
%  {\def\@tempb{\newline}\edef\@tempa{\unexpanded\expandafter{\@car#8\@nil}}}
%  {}{}
% \patchcmd{\newtheoremstyle}
%  {\def\thmheadnl{\newline}}
%  {\def\thmheadnl{#8}}
%  {}{}
% \makeatother

\newtheoremstyle{break}% name
  {15pt}%      Space above, empty = `usual value'
  {}%          Space below
  {\kai}%     Body font
  {}%         Indent amount (empty = no indent, \parindent = para indent)
  {\bfseries\hei}% Thm head font
  {.}%        Punctuation after thm head
  %{\newline\hspace*{\parindent}}% Space after thm head: \newline = linebreak
  {5pt plus 1pt minus 1pt}% Space after thm head: \newline = linebreak
  {}%         Thm head spec
\theoremstyle{break}
\newtheorem{theorem}{\hei 定理}[section]
\newtheorem{example}[theorem]{\hei 例} % Use the same counter as theorem
\newtheorem{property}[theorem]{\hei 性质} % Use the same counter as theorem
\newtheorem{question}[theorem]{\hei 题} % Use the same counter as theorem
\newtheorem{definition}[theorem]{\hei 定义} % Use the same counter as theorem
\newtheorem{corollary}{\hei 推论}[theorem]  % Reset after every theorem
\newtheorem{lemma}[theorem]{\hei 引理}  % Use the same counter as theorem

% \renewcommand*{\proofname}{证明}
\newcommand{\note}{\begingroup\bfseries\kai 注:\endgroup}
\newcommand{\think}{\begingroup\bfseries\kai 思考:\endgroup}
\newcommand{\hints}{\begingroup\bfseries\kai 提示:\endgroup}

\newcommand{\kurschak}{K\"ursch\'ak}
\newcommand{\term}[1]{\begingroup\bfseries\kai{#1}\endgroup}

\DeclareMathOperator{\atan}{atan}

% Change colon(:) in figure caption to period(.)
\usepackage{caption}
\captionsetup{labelsep=period}

\newcommand{\norm}[1]{\left\lVert{#1}\right\rVert}
\renewcommand*{\vec}[1]{\mathbf{#1}}

\newcommand*{\ihat}{\hat{\textbf\i}}
\newcommand*{\jhat}{\hat{\textbf\j}}
\newcommand*{\khat}{\hat{\textbf k}}

\renewcommand*{\le}{\leqslant}
\renewcommand*{\ge}{\geqslant}
\renewcommand*\labelenumi{(\theenumi)}

% Make proof environemtn skip the first line
\makeatletter
\renewenvironment{proof}[1][证明]{\par
  % \vspace{-\topsep}% remove the space after the theorem, HOWEVER A PROOF ENVIRONMENT NOT PRECEDED BY A THEOREM WILL BE TYPESET WRONGLY.
  \pushQED{\qed}%
  \normalfont %\topsep6\p@\@plus6\p@\relax
  \topsep0pt \partopsep0pt % no space before
  \trivlist
  \item[\hskip\labelsep
    \bfseries \kai%\itshape
    #1\@addpunct{.}]%\mbox{}\\*% new line after \proofname
}{%
  \popQED\endtrivlist\@endpefalse
}
\makeatother

% \newcommand*{\max}[1]{\begingroup\mathrm{max}\left(#1\right)\endgroup}
% \newcommand*{\min}[1]{\mathrm{min}\left(#1\right)}

\usepackage{enumitem}
% \setlist{nolistsep}
% \setitemize{labelindent=\parindent,leftmargin=*,align=left,itemindent=2em,noitemsep}%,leftmargin=*,topsep=2pt,partopsep=0pt}
\setitemize{labelindent=\parindent,leftmargin=*,itemsep=0em,partopsep=0em,parsep=0em,topsep=0em,itemindent=3.75em,align=left}
\setenumerate{labelindent=\parindent,leftmargin=*,itemsep=0em,partopsep=0em,parsep=0em,topsep=0em,itemindent=3.75em,align=left}

\setitemize[2]{labelindent=\parindent+2em,leftmargin=*,itemsep=0em,partopsep=0em,parsep=0em,topsep=0em,itemindent=5.75em,align=left}
\setenumerate[2]{labelindent=\parindent+2em,leftmargin=*,itemsep=0em,partopsep=0em,parsep=0em,topsep=0em,itemindent=5.75em,align=left}


\renewcommand{\baselinestretch}{1.1}
\renewcommand*{\arraystretch}{1.5}

% make \arraystrech in pmatrix/vmatrix/bmatrix in amsmath default to .9
\makeatletter
\renewcommand*\env@matrix[1][.9]{%
  \edef\arraystretch{#1}%
  \hskip -\arraycolsep
  \let\@ifnextchar\new@ifnextchar
  \array{*\c@MaxMatrixCols c}}
\makeatother

\usepackage{graphicx}%\resizebox
\graphicspath{{images/}} %Setting the graphicspath

\usepackage[makeroom]{cancel} % for xcancel/cancel/bcancel etc.

\usepackage{fancyvrb}           % for BVerbatim environment

\usepackage{verbatim}
% Center verbatim environment, with the help from verbatim package
\makeatletter
\def\verbatim@font{\normalfont\ttfamily\kai\Large
  \hyphenchar\font\m@ne
  \@noligs}
\makeatother
\newenvironment{pcontent}{%
  \par
  \centering
  \varwidth{\linewidth}%
  \verbatim
}{%
  \endverbatim
  \endvarwidth
  \par\vskip10pt
}

\newcommand{\ptitle}[1]{\par\vskip20pt{\centering\bfseries\kai\Large{#1}\par}\nopagebreak\vskip5pt\nopagebreak}
\newcommand{\pauthor}[1]{\par{\centering\kai {#1}\par}\nopagebreak\vskip1pt}
\newcommand{\premark}[1]{\par\kai\small【注释】{#1}\par\vskip5pt}
\newcommand{\ppreface}[1]{{\setlength{\parindent}{2em}\small\kai{#1}}\par\vskip5pt}

% \par indent
\usepackage{indentfirst}
% parindent to 2 characters
\setlength{\parindent}{2em}
\setlength{\parskip}{3pt}

\usepackage{ifthen}             % for \ifelsethen, \whiledo
% \ifthenelse{CONDITION}{POSITIVE}{NEGATIVE}
%     2 is an \ifthenelse{\isodd{2}}{ODD}{EVEN} number.
% \whiledo{CONDITION}{LOOP CONSTRUCT}
%     \newcounter{X}
%     \whiledo{\value{X}<10}{\value{X},\stepcounter{X}}

\usepackage{units}              % for \nicefrac command, a slashed fraction

\usepackage{harpoon}            % for \overrightharp

% 3 counters for loops
\newcounter{X}
\newcounter{Y}
\newcounter{Z}

% \include{myconfig}

% 选择题
\usepackage{multiplechoice}

\usepackage[chapter]{algorithm}          % the float environment
\usepackage{algorithmic} % algorithmic (NOT algorithmicx) package is suggested for IEEE journals
\floatname{algorithm}{算法}
\renewcommand{\algorithmicrequire}{\textbf{Input:}}
\renewcommand{\algorithmicensure}{\textbf{Output:}}

\newcommand{\dx}{\,\mathrm{d}x}
\newcommand{\dy}{\,\mathrm{d}y}
\newcommand{\dt}{\,\mathrm{d}t}

% \newcommand{\dz}{\mathrm{d}z}
% \newcommand{\gcd}{\mathrm{gcd}}
\newcommand{\lcm}{\mathrm{lcm}}

\newcommand{\trianglewithfournumbers}[4]{
  \coordinate(A) at (0,0);
  \coordinate(C) at (60:2);
  \coordinate(B) at (2,0);
  \coordinate(E) at (60:1);
  \coordinate(F) at (1,0);
  \coordinate(D) at ($(E)+(1,0)$);
  \draw(A)--(B)--(C)--cycle;
  \draw(D)--(E)--(F)--cycle;
  \node at ($1/3*(C)+1/3*(E)+1/3*(D)$) {#1};
  \node at ($1/3*(A)+1/3*(E)+1/3*(F)$) {#2};
  \node at ($1/3*(B)+1/3*(F)+1/3*(D)$) {#3};
  \node at ($1/3*(F)+1/3*(E)+1/3*(D)$) {#4};
}

%%%% 易经符号
% \newCJKfontfamily{\symbolazh}{Symbola} % this font has the required I Ching glyphs
\xeCJKsetcharclass{"4DC0}{"4DFF}{0} % set Yijing Hexagrams Symbols as non-CJK characters, otherwise one has to use the \symbolazh command instead
\newfontface{\symbola}{Symbola}

% 两仪
\newcommand{\yangyao}{{\symbola\char\numexpr"268A}}
\newcommand{\yinyao}{{\symbola\char\numexpr"268B}}
% 四象
\newcommand{\taiyang}{{\symbola\char\numexpr"268C}}
\newcommand{\shaoyin}{{\symbola\char\numexpr"268D}}
\newcommand{\shaoyang}{{\symbola\char\numexpr"268E}}
\newcommand{\taiyin}{{\symbola\char\numexpr"268F}}
% 八卦 0-7
\newcommand{\trigram}[1]{{\symbola\char\numexpr"2630+#1}}
% 六十四卦 0-63
\newcommand{\iching}[1]{{\symbola\char\numexpr"4DC0+#1}}


% comment this when compile the whole document
% \includeonly{basic-operation}
% \includeonly{basic-inequalities}
% \includeonly{proofs-without-words}
% \includeonly{golden-ratio}
% \includeonly{area}
% \includeonly{five-models}
% \includeonly{logic}
% \includeonly{geometry}
% \includeonly{graphs-of-functions}
% \includeonly{proofs-without-words}
% \includeonly{board-coverage}
% \includeonly{graphs}
% \includeonly{traps}
% \includeonly{intuition}
% \includeonly{introduction}
% \includeonly{monotonic-functions}
% \includeonly{algorithm}
% \includeonly{magic-square}
% \includeonly{geometric-inequality}
% \includeonly{number-theory}
% \includeonly{continued-fraction}
% \includeonly{calendar}
% \includeonly{diophantine-equation}
% \includeonly{methods}
% \includeonly{numerical-method}
% \includeonly{identities}
% \includeonly{nim-game}
% \includeonly{cover}
% \includeonly{infinite-series}
% \includeonly{tricks}
% \includeonly{game}
% \includeonly{pattern}
% \includeonly{insight}
% \includeonly{inequality}
% \includeonly{trigonometric-functions}
% \includeonly{set}
% \includeonly{area}
% \includeonly{discrete}
% \includeonly{fun}
% \includeonly{dp}
% \includeonly{matrix}
% \includeonly{color}
% \includeonly{fractal}
% \includeonly{genius-solution-from-the-book}
% \includeonly{genius-solution}
% \includeonly{math-by-physics}
% \includeonly{probability}
% \includeonly{circle}
% \includeonly{number-to-graph}

% allow page break inside align environment
\allowdisplaybreaks
\begin{document}

%\hypersetup{CJKbookmarks=true}  %for \overline in section, NOT WORK!!

\frontmatter
\include{cover}

%% Remvoe the self reference of tableofcontents
\currentpdfbookmark{\contentsname}{tableofcontents}
\tableofcontents*
% \begin{KeepFromToc}
%   \tableofcontents
% \end{KeepFromToc}

\mainmatter

\chapter{脍炙人口的诗词选}
\label{chap:mao}

\clearpage

% enlarge character spacing
% default: {\hskip 0pt plus 0.08\baselineskip}
\renewcommand{\CJKglue}{\hskip 1pt plus 0.08\baselineskip}

%%%%%%%%%%%%%%%%%%%%%%%%%%%%%%%%%%%%%%%
% This file is generated automatically.
% DO NOT MODIFY IT IF YOU DON'T KNOW.
%%%%%%%%%%%%%%%%%%%%%%%%%%%%%%%%%%%%%%%
\chapter{诗经}
\ptitle{关雎}\nopagebreak%
\addcontentsline{toc}{section}{\texorpdfstring{\makebox[10cm]{关雎\dotfill{} 《诗经·周南》}}{关雎\ 《诗经·周南》}}\nopagebreak%
\noindent\begin{minipage}{\linewidth}
  \pauthor{《诗经·周南》}
  \vskip-3pt\begin{table}[H]
    \centering
    \begin{tabular}{@{}l@{}}
关关\xpinyin*{\xpinyin{雎}{jū}}\xpinyin*{\xpinyin{鸠}{jiū}},在河之洲。\xpinyin*{\xpinyin{窈}{yǎo}}\xpinyin*{\xpinyin{窕}{tiǎo}}淑女,君子好\xpinyin*{\xpinyin{逑}{qiú}}。\\
\\
参差\xpinyin*{\xpinyin{荇}{xìng}}菜,左右流之。窈窕淑女,\xpinyin*{\xpinyin{寤}{wù}}\xpinyin*{\xpinyin{寐}{mèi}}求之。\\
\\
求之不得,寤寐思服。悠哉悠哉,\xpinyin*{\xpinyin{辗}{zhǎn}}转反侧。\\
\\
参差荇菜,左右采之。窈窕淑女,琴瑟友之。\\
\\
参差荇菜,左右\xpinyin*{\xpinyin{芼}{mào}}之。窈窕淑女,钟鼓乐之。
    \end{tabular}
  \end{table}
\end{minipage}
\vspace{1cm}


\ptitle{桃夭}\nopagebreak%
\addcontentsline{toc}{section}{\texorpdfstring{\makebox[10cm]{桃夭\dotfill{} 《诗经·周南》}}{桃夭\ 《诗经·周南》}}\nopagebreak%
\noindent\begin{minipage}{\linewidth}
  \pauthor{《诗经·周南》}
  \vskip-3pt\begin{table}[H]
    \centering
    \begin{tabular}{@{}l@{}}
桃之夭夭,灼灼其华。之子于归,宜其室家。\\
\\
桃之夭夭,有\xpinyin*{\xpinyin{蕡}{fén}}其实。之子于归,宜其家室。\\
\\
桃之夭夭,其叶\xpinyin*{\xpinyin{蓁}{zhēn}}蓁。之子于归,宜其家人。
    \end{tabular}
  \end{table}
\end{minipage}
\vspace{1cm}


\ptitle{木瓜}\nopagebreak%
\addcontentsline{toc}{section}{\texorpdfstring{\makebox[10cm]{木瓜\dotfill{} 《诗经·国风·卫风》}}{木瓜\ 《诗经·国风·卫风》}}\nopagebreak%
\noindent\begin{minipage}{\linewidth}
  \pauthor{《诗经·国风·卫风》}
  \vskip-3pt\begin{table}[H]
    \centering
    \begin{tabular}{@{}l@{}}
投我以木瓜,报之以琼\xpinyin*{\xpinyin{琚}{jū}}。匪报也,永以为好也!\\
\\
投我以木桃,报之以琼瑶。匪报也,永以为好也!\\
\\
投我以木李,报之以琼玖。匪报也,永以为好也!
    \end{tabular}
  \end{table}
\end{minipage}
\vspace{1cm}


\ptitle{无衣}\nopagebreak%
\addcontentsline{toc}{section}{\texorpdfstring{\makebox[10cm]{无衣\dotfill{} 《诗经·秦风》}}{无衣\ 《诗经·秦风》}}\nopagebreak%
\noindent\begin{minipage}{\linewidth}
  \pauthor{《诗经·秦风》}
  \vskip-3pt\begin{table}[H]
    \centering
    \begin{tabular}{@{}l@{}}
岂曰无衣?与子同袍。王于兴师,修我戈矛,与子同仇。\\
\\
岂曰无衣?与子同泽。王于兴师,修我矛戟,与子偕作。\\
\\
岂曰无衣?与子同裳。王于兴师,修我甲兵,与子偕行。
    \end{tabular}
  \end{table}
\end{minipage}
\vspace{1cm}


\ptitle{\xpinyin*{\xpinyin{蒹}{jiān}}\xpinyin*{\xpinyin{葭}{jiā}}}\nopagebreak%
\addcontentsline{toc}{section}{\texorpdfstring{\makebox[10cm]{蒹葭\dotfill{} 《诗经·秦风》}}{蒹葭\ 《诗经·秦风》}}\nopagebreak%
\noindent\begin{minipage}{\linewidth}
  \pauthor{《诗经·秦风》}
  \vskip-3pt\begin{table}[H]
    \centering
    \begin{tabular}{@{}l@{}}
蒹葭苍苍,白露为霜。所谓伊人,在水一方。\\
\xpinyin*{\xpinyin{溯}{sù}}\xpinyin*{\xpinyin{洄}{huí}}从之,道阻且长。溯游从之,宛在水中央。\\
\\
蒹葭萋萋,白露未\xpinyin*{\xpinyin{晞}{xī}}。所谓伊人,在水之\xpinyin*{\xpinyin{湄}{méi}}。\\
溯洄从之,道阻且\xpinyin*{\xpinyin{跻}{jī}}。溯游从之,宛在水中\xpinyin*{\xpinyin{坻}{chí}}。\\
\\
蒹葭采采,白露未已。所谓伊人,在水之\xpinyin*{\xpinyin{涘}{sì}}。\\
溯洄从之,道阻且右。溯游从之,宛在水中\xpinyin*{\xpinyin{沚}{zhǐ}}。
    \end{tabular}
  \end{table}
\end{minipage}
\vspace{1cm}


\ptitle{静女}\nopagebreak%
\addcontentsline{toc}{section}{\texorpdfstring{\makebox[10cm]{静女\dotfill{} 《诗经·邶风》}}{静女\ 《诗经·邶风》}}\nopagebreak%
\noindent\begin{minipage}{\linewidth}
  \pauthor{《诗经·邶风》}
  \vskip-3pt\begin{table}[H]
    \centering
    \begin{tabular}{@{}l@{}}
静女其姝,俟我于城隅。爱而不见,搔首踟蹰。\\
\\
静女其娈,贻我彤管。彤管有炜,说怿女美。\\
\\
自牧归荑,洵美且异。匪女之为美,美人之贻。
    \end{tabular}
  \end{table}
\end{minipage}
\vspace{1cm}


\ptitle{采薇}\nopagebreak%
\addcontentsline{toc}{section}{\texorpdfstring{\makebox[10cm]{采薇\dotfill{} 《诗经•小雅》}}{采薇\ 《诗经•小雅》}}\nopagebreak%
\noindent\begin{minipage}{\linewidth}
  \pauthor{《诗经•小雅》}
  \vskip-3pt\begin{table}[H]
    \centering
    \begin{tabular}{@{}l@{}}
采薇采薇,薇亦作止。曰归曰归,岁亦莫止。\\
\xpinyin*{\xpinyin{靡}{mí}}室靡家,\xpinyin*{\xpinyin{猃}{xiǎn}}\xpinyin*{\xpinyin{狁}{yǔn}}之故。不\xpinyin*{\xpinyin{遑}{huáng}}启居,猃狁之故。\\
\\
采薇采薇,薇亦柔止。曰归曰归,心亦忧止。\\
忧心烈烈,载饥载渴。我戍未定,靡使归聘。\\
\\
采薇采薇,薇亦刚止。曰归曰归,岁亦阳止。\\
王事\xpinyin*{\xpinyin{靡}{mí}}\xpinyin*{\xpinyin{盬}{gǔ}},不遑启处。忧心孔疚,我行不来!\\
\\
彼尔维何?维常之华。彼路斯何?君子之车。\\
戎车既驾,四牡业业。岂敢定居?一月三捷。\\
\\
驾彼四牡,四牡\xpinyin*{\xpinyin{骙}{kuí}}\xpinyin*{\xpinyin{骙}{kuí}}。君子所依,小人所\xpinyin*{\xpinyin{腓}{féi}}。\\
四牡翼翼,象弭鱼服。岂不日戒?猃狁孔棘!\\
\\
昔我往矣,杨柳依依。今我来思,雨雪霏霏。\\
行道迟迟,载渴载饥。我心伤悲,莫知我哀!
    \end{tabular}
  \end{table}
\end{minipage}
\vspace{1cm}


\chapter{秦汉}
\ptitle{离骚}\nopagebreak%
\addcontentsline{toc}{section}{\texorpdfstring{\makebox[10cm]{离骚\dotfill{} 【战国】屈原}}{离骚\ 【战国】屈原}}\nopagebreak%
\noindent\begin{minipage}{\linewidth}
  \pauthor{【战国】屈原}
  \vskip-3pt\begin{table}[H]
    \centering
    \begin{tabular}{@{}l@{}}
帝高阳之苗裔兮,朕皇考曰伯庸。\\
摄提贞于孟陬兮,惟庚寅吾以降。\\
皇览揆余初度兮,肇锡余以嘉名。\\
名余曰正则兮,字余曰灵均。\\
纷吾既有此内美兮,又重之以修能。\\
扈江离与辟芷兮,纫秋兰以为佩。\\
汩余若将不及兮,恐年岁之不吾与。\\
朝搴阰之木兰兮,夕揽洲之宿莽。\\
日月忽其不淹兮,春与秋其代序。\\
惟草木之零落兮,恐美人之迟暮。\\
不抚壮而弃秽兮,何不改乎此度?\\
乘骐骥以驰骋兮,来吾道夫先路。
    \end{tabular}
  \end{table}
\end{minipage}
\vspace{1cm}


\ptitle{易水歌}\nopagebreak%
\addcontentsline{toc}{section}{\texorpdfstring{\makebox[10cm]{易水歌\dotfill{} 【先秦】荆轲}}{易水歌\ 【先秦】荆轲}}\nopagebreak%
\noindent\begin{minipage}{\linewidth}
  \pauthor{【先秦】荆轲}
  \vskip-3pt\begin{table}[H]
    \centering
    \begin{tabular}{@{}l@{}}
风萧萧兮易水寒,壮士一去兮不复还。\\
探虎穴兮入蛟宫,仰天呼气兮成白虹。
    \end{tabular}
  \end{table}
\end{minipage}
\vspace{1cm}


\ptitle{\xpinyin*{\xpinyin{垓}{gāi}}下歌}\nopagebreak%
\addcontentsline{toc}{section}{\texorpdfstring{\makebox[10cm]{垓下歌\dotfill{} 【秦】项羽}}{垓下歌\ 【秦】项羽}}\nopagebreak%
\noindent\begin{minipage}{\linewidth}
  \pauthor{【秦】项羽}
  \vskip-3pt\begin{table}[H]
    \centering
    \begin{tabular}{@{}l@{}}
力拔山兮气盖世。时不利兮\xpinyin*{\xpinyin{骓}{zhuī}}不逝。\\
骓不逝兮可奈何!虞兮虞兮奈若何!
    \end{tabular}
  \end{table}
\end{minipage}
\vspace{1cm}


\ptitle{大风歌}\nopagebreak%
\addcontentsline{toc}{section}{\texorpdfstring{\makebox[10cm]{大风歌\dotfill{} 【东汉】刘邦}}{大风歌\ 【东汉】刘邦}}\nopagebreak%
\noindent\begin{minipage}{\linewidth}
  \pauthor{【东汉】刘邦}
  \vskip-3pt\begin{table}[H]
    \centering
    \begin{tabular}{@{}l@{}}
大风起兮云飞扬,\\
威加海内兮归故乡,\\
安得猛士兮守四方!
    \end{tabular}
  \end{table}
\end{minipage}
\vspace{1cm}


\ptitle{十五从军征}\nopagebreak%
\addcontentsline{toc}{section}{\texorpdfstring{\makebox[10cm]{十五从军征\dotfill{} 【汉】乐府}}{十五从军征\ 【汉】乐府}}\nopagebreak%
\noindent\begin{minipage}{\linewidth}
  \pauthor{【汉】乐府}
  \vskip-3pt\begin{table}[H]
    \centering
    \begin{tabular}{@{}l@{}}
十五从军征,八十始得归。道逢乡里人:家中有阿谁?\\
遥看是君家,松柏\xpinyin*{\xpinyin{冢}{zhǒng}}累累。兔从狗\xpinyin*{\xpinyin{窦}{dòu}}入,\xpinyin*{\xpinyin{雉}{zhì}}从梁上飞,\\
中庭生旅谷,井上生旅葵。\xpinyin*{\xpinyin{舂}{chōng}}谷持作饭,采葵持作\xpinyin*{\xpinyin{羹}{gēng}}。\\
羹饭一时熟,不知\xpinyin*{\xpinyin{贻}{yí}}阿谁。出门东向看,泪落沾我衣。
    \end{tabular}
  \end{table}
\end{minipage}
\vspace{1cm}


\ptitle{江南}\nopagebreak%
\addcontentsline{toc}{section}{\texorpdfstring{\makebox[10cm]{江南\dotfill{} 【汉】乐府}}{江南\ 【汉】乐府}}\nopagebreak%
\noindent\begin{minipage}{\linewidth}
  \pauthor{【汉】乐府}
  \vskip-3pt\begin{table}[H]
    \centering
    \begin{tabular}{@{}l@{}}
江南可采莲,莲叶何田田,鱼戏莲叶间。\\
鱼戏莲叶东,鱼戏莲叶西。鱼戏莲叶南,鱼戏莲叶北。
    \end{tabular}
  \end{table}
\end{minipage}
\vspace{1cm}


\ptitle{长歌行}\nopagebreak%
\addcontentsline{toc}{section}{\texorpdfstring{\makebox[10cm]{长歌行\dotfill{} 【汉】乐府}}{长歌行\ 【汉】乐府}}\nopagebreak%
\noindent\begin{minipage}{\linewidth}
  \pauthor{【汉】乐府}
  \vskip-3pt\begin{table}[H]
    \centering
    \begin{tabular}{@{}l@{}}
青青园中葵,朝露待日\xpinyin*{\xpinyin{晞}{xī}}。\\
阳春布德泽,万物生光辉。\\
常恐秋节至,\xpinyin*{\xpinyin{焜}{kūn}}黄华叶衰。\\
百川东到海,何时复西归?\\
少壮不努力,老大徒伤悲!
    \end{tabular}
  \end{table}
\end{minipage}
\vspace{1cm}


\ptitle{上邪}\nopagebreak%
\addcontentsline{toc}{section}{\texorpdfstring{\makebox[10cm]{上邪\dotfill{} 【汉】佚名}}{上邪\ 【汉】佚名}}\nopagebreak%
\noindent\begin{minipage}{\linewidth}
  \pauthor{【汉】佚名}
  \vskip-3pt\begin{table}[H]
    \centering
    \begin{tabular}{@{}l@{}}
上邪,我欲与君相知,长命无绝衰。\\
山无陵,江水为竭。冬雷震震,夏雨雪。\\
天地合,乃敢与君绝。
    \end{tabular}
  \end{table}
\end{minipage}
\vspace{1cm}


\ptitle{古歌}\nopagebreak%
\addcontentsline{toc}{section}{\texorpdfstring{\makebox[10cm]{古歌\dotfill{} 【汉】佚名}}{古歌\ 【汉】佚名}}\nopagebreak%
\noindent\begin{minipage}{\linewidth}
  \pauthor{【汉】佚名}
  \vskip-3pt\begin{table}[H]
    \centering
    \begin{tabular}{@{}l@{}}
秋风萧萧愁杀人,出亦愁,入亦愁。\\
座中何人,谁不怀忧。\\
令我白头。\\
胡地多飚风,树木何修修。\\
离家日趋远,衣带日趋缓。\\
心思不能言,肠中车轮转。
    \end{tabular}
  \end{table}
\end{minipage}
\vspace{1cm}


\ptitle{客从远方来}\nopagebreak%
\addcontentsline{toc}{section}{\texorpdfstring{\makebox[10cm]{客从远方来\dotfill{} 【汉】佚名}}{客从远方来\ 【汉】佚名}}\nopagebreak%
\noindent\begin{minipage}{\linewidth}
  \pauthor{【汉】佚名}
  \vskip-3pt\begin{table}[H]
    \centering
    \begin{tabular}{@{}l@{}}
客从远方来,遗我一端绮。相去万余里,故人心尚尔。\\
文采双鸳鸯,裁为合欢被。著以长相思,缘以结不解。\\
以胶投漆中,谁能别离此?
    \end{tabular}
  \end{table}
\end{minipage}
\vspace{1cm}


\ptitle{明月何皎皎}\nopagebreak%
\addcontentsline{toc}{section}{\texorpdfstring{\makebox[10cm]{明月何皎皎\dotfill{} 【汉】佚名}}{明月何皎皎\ 【汉】佚名}}\nopagebreak%
\noindent\begin{minipage}{\linewidth}
  \pauthor{【汉】佚名}
  \vskip-3pt\begin{table}[H]
    \centering
    \begin{tabular}{@{}l@{}}
明月何皎皎,照我罗床\xpinyin*{\xpinyin{帏}{wéi}}。\\
忧愁不能寐,揽衣起徘徊。\\
客行虽云乐,不如早旋归。\\
出户独彷徨,愁思当告谁!\\
引领还入房,泪下沾裳衣。
    \end{tabular}
  \end{table}
\end{minipage}
\vspace{1cm}


\ptitle{涉江采芙蓉}\nopagebreak%
\addcontentsline{toc}{section}{\texorpdfstring{\makebox[10cm]{涉江采芙蓉\dotfill{} 【汉】佚名}}{涉江采芙蓉\ 【汉】佚名}}\nopagebreak%
\noindent\begin{minipage}{\linewidth}
  \pauthor{【汉】佚名}
  \vskip-3pt\begin{table}[H]
    \centering
    \begin{tabular}{@{}l@{}}
涉江采芙蓉,兰泽多芳草。\\
采之欲遗谁,所思在远道。\\
还顾望旧乡,长路漫浩浩。\\
同心而离居,忧伤以终老。
    \end{tabular}
  \end{table}
\end{minipage}
\vspace{1cm}


\ptitle{生年不满百}\nopagebreak%
\addcontentsline{toc}{section}{\texorpdfstring{\makebox[10cm]{生年不满百\dotfill{} 【汉】佚名}}{生年不满百\ 【汉】佚名}}\nopagebreak%
\noindent\begin{minipage}{\linewidth}
  \pauthor{【汉】佚名}
  \vskip-3pt\begin{table}[H]
    \centering
    \begin{tabular}{@{}l@{}}
生年不满百,常怀千岁忧。昼短苦夜长,何不秉烛游!\\
为乐当及时,何能待来兹?愚者爱惜费,但为後世嗤。\\
仙人王子乔,难可与等期。
    \end{tabular}
  \end{table}
\end{minipage}
\vspace{1cm}


\ptitle{行行重行行}\nopagebreak%
\addcontentsline{toc}{section}{\texorpdfstring{\makebox[10cm]{行行重行行\dotfill{} 【汉】佚名}}{行行重行行\ 【汉】佚名}}\nopagebreak%
\noindent\begin{minipage}{\linewidth}
  \pauthor{【汉】佚名}
  \vskip-3pt\begin{table}[H]
    \centering
    \begin{tabular}{@{}l@{}}
\xpinyin*{\xpinyin{行}{xíng}}行\xpinyin*{\xpinyin{重}{chóng}}行行,与君生别离。\\
相去万余里,各在天一涯。\\
道路阻且长,会面安可知?\\
胡马依北风,越鸟巢南枝。\\
相去日已远,衣带日已缓。\\
浮云蔽白日,游子不顾返。\\
思君令人老,岁月忽已晚。\\
弃捐勿复道,努力加餐饭。
    \end{tabular}
  \end{table}
\end{minipage}
\vspace{1cm}


\ptitle{迢迢牵牛星}\nopagebreak%
\addcontentsline{toc}{section}{\texorpdfstring{\makebox[10cm]{迢迢牵牛星\dotfill{} 【汉】佚名}}{迢迢牵牛星\ 【汉】佚名}}\nopagebreak%
\noindent\begin{minipage}{\linewidth}
  \pauthor{【汉】佚名}
  \vskip-3pt\begin{table}[H]
    \centering
    \begin{tabular}{@{}l@{}}
\xpinyin*{\xpinyin{迢}{tiáo}}迢牵牛星,皎皎河汉女。\\
\xpinyin*{\xpinyin{纤}{xiān}}纤\xpinyin*{\xpinyin{擢}{zhuó}}素手,\xpinyin*{\xpinyin{札}{zhá}}札弄机\xpinyin*{\xpinyin{杼}{zhù}}。\\
终日不成章,泣涕零如雨。\\
河汉清且浅,相去复几许!\\
盈盈一水间,脉脉不得语。
    \end{tabular}
  \end{table}
\end{minipage}
\vspace{1cm}


\ptitle{秋风辞}\nopagebreak%
\addcontentsline{toc}{section}{\texorpdfstring{\makebox[10cm]{秋风辞\dotfill{} 【汉】刘彻}}{秋风辞\ 【汉】刘彻}}\nopagebreak%
\noindent\begin{minipage}{\linewidth}
  \pauthor{【汉】刘彻}
  \vskip-3pt\begin{table}[H]
    \centering
    \begin{tabular}{@{}l@{}}
秋风起兮白云飞,草木黄落兮雁南归。\\
兰有秀兮菊有芳,怀佳人兮不能忘。\\
泛楼船兮济\xpinyin*{\xpinyin{汾}{fén}}河,横中流兮扬素波。\\
箫鼓鸣兮发\xpinyin*{\xpinyin{棹}{zhào}}歌,欢乐极兮哀情多。\\
少壮几时兮奈老何!
    \end{tabular}
  \end{table}
\end{minipage}
\vspace{1cm}


\ptitle{李延年歌}\nopagebreak%
\addcontentsline{toc}{section}{\texorpdfstring{\makebox[10cm]{李延年歌\dotfill{} 【汉】李延年}}{李延年歌\ 【汉】李延年}}\nopagebreak%
\noindent\begin{minipage}{\linewidth}
  \pauthor{【汉】李延年}
  \vskip-3pt\begin{table}[H]
    \centering
    \begin{tabular}{@{}l@{}}
   北方有佳人,绝世而独立。\\
   一顾倾人城,再顾倾人国。\\
宁不知倾城与倾国?佳人难再得。
    \end{tabular}
  \end{table}
\end{minipage}
\vspace{1cm}


\ptitle{怨歌行}\nopagebreak%
\addcontentsline{toc}{section}{\texorpdfstring{\makebox[10cm]{怨歌行\dotfill{} 【汉】班婕妤}}{怨歌行\ 【汉】班婕妤}}\nopagebreak%
\noindent\begin{minipage}{\linewidth}
  \pauthor{【汉】班\xpinyin*{\xpinyin{婕}{jié}}\xpinyin*{\xpinyin{妤}{yú}}}
  \vskip-3pt\begin{table}[H]
    \centering
    \begin{tabular}{@{}l@{}}
新裂齐纨素,鲜洁如霜雪。裁为合欢扇,团团似明月。\\
出入君怀袖,动摇微风发。常恐秋节至,凉飚夺炎热。\\
弃捐\xpinyin*{\xpinyin{箧}{qiè}}\xpinyin*{\xpinyin{笥}{sì}}中,恩情中道绝。
    \end{tabular}
  \end{table}
\end{minipage}
\vspace{1cm}


\ptitle{涉江采芙蓉}\nopagebreak%
\addcontentsline{toc}{section}{\texorpdfstring{\makebox[10cm]{涉江采芙蓉\dotfill{} 《古诗十九首》}}{涉江采芙蓉\ 《古诗十九首》}}\nopagebreak%
\noindent\begin{minipage}{\linewidth}
  \pauthor{《古诗十九首》}
  \vskip-3pt\begin{table}[H]
    \centering
    \begin{tabular}{@{}l@{}}
涉江采芙蓉,兰泽多芳草。采之欲遗谁,所思在远道。\\
还顾望旧乡,长路漫浩浩。同心而离居,忧伤以终老。
    \end{tabular}
  \end{table}
\end{minipage}
\vspace{1cm}


\chapter{魏晋南北朝}
\ptitle{短歌行}\nopagebreak%
\addcontentsline{toc}{section}{\texorpdfstring{\makebox[10cm]{短歌行\dotfill{} 【三国】曹操}}{短歌行\ 【三国】曹操}}\nopagebreak%
\noindent\begin{minipage}{\linewidth}
  \pauthor{【三国】曹操}
  \vskip-3pt\begin{table}[H]
    \centering
    \begin{tabular}{@{}l@{}}
对酒当歌,人生几何?譬如朝露,去日苦多。\\
慨当以慷,忧思难忘。何以解忧?唯有杜康。\\
青青子\xpinyin*{\xpinyin{衿}{jīn}},悠悠我心。但为君故,沉吟至今。\\
\xpinyin*{\xpinyin{呦}{yōu}}呦鹿鸣,食野之苹。我有嘉宾,鼓瑟吹笙。\\
\\
明明如月,何时可\xpinyin*{\xpinyin{掇}{duō}}?忧从中来,不可断绝。\\
越陌度阡,枉用相存。契阔谈宴,心念旧恩。\\
月明星稀,乌鹊南飞。绕树三匝,何枝可依?\\
山不厌高,海不厌深。周公吐\xpinyin*{\xpinyin{哺}{bǔ}},天下归心。
    \end{tabular}
  \end{table}
\end{minipage}
\vspace{1cm}


\ptitle{观沧海}\nopagebreak%
\addcontentsline{toc}{section}{\texorpdfstring{\makebox[10cm]{观沧海\dotfill{} 【三国】曹操}}{观沧海\ 【三国】曹操}}\nopagebreak%
\noindent\begin{minipage}{\linewidth}
  \pauthor{【三国】曹操}
  \vskip-3pt\begin{table}[H]
    \centering
    \begin{tabular}{@{}l@{}}
东临碣石,以观沧海。水何\xpinyin*{\xpinyin{澹}{dàn}}澹,山岛\xpinyin*{\xpinyin{竦}{sǒng}}\xpinyin*{\xpinyin{峙}{zhì}}。\\
树木丛生,百草丰茂。秋风萧瑟,洪波涌起。\\
日月之行,若出其中;星汉灿烂,若出其里。\\
幸甚至哉,歌以咏志。
    \end{tabular}
  \end{table}
\end{minipage}
\vspace{1cm}


\ptitle{龟虽寿}\nopagebreak%
\addcontentsline{toc}{section}{\texorpdfstring{\makebox[10cm]{龟虽寿\dotfill{} 【三国】曹操}}{龟虽寿\ 【三国】曹操}}\nopagebreak%
\noindent\begin{minipage}{\linewidth}
  \pauthor{【三国】曹操}
  \vskip-3pt\begin{table}[H]
    \centering
    \begin{tabular}{@{}l@{}}
神龟虽寿,犹有竟时。腾蛇乘雾,终为土灰。\\
老\xpinyin*{\xpinyin{骥}{jì}}伏\xpinyin*{\xpinyin{枥}{lì}},志在千里。烈士暮年,壮心不已。\\
盈缩之期,不但在天;养怡之福,可得永年。\\
幸甚至哉,歌以咏志。
    \end{tabular}
  \end{table}
\end{minipage}
\vspace{1cm}


\ptitle{七步诗}\nopagebreak%
\addcontentsline{toc}{section}{\texorpdfstring{\makebox[10cm]{七步诗\dotfill{} 【三国】曹植}}{七步诗\ 【三国】曹植}}\nopagebreak%
\noindent\begin{minipage}{\linewidth}
  \pauthor{【三国】曹植}
  \vskip-3pt\begin{table}[H]
    \centering
    \begin{tabular}{@{}l@{}}
煮豆燃豆\xpinyin*{\xpinyin{萁}{qí}},豆在\xpinyin*{\xpinyin{釜}{fǔ}}中泣。\\
本是同根生,相煎何太急?
    \end{tabular}
  \end{table}
\end{minipage}
\vspace{1cm}


\ptitle{杂诗}\nopagebreak%
\addcontentsline{toc}{section}{\texorpdfstring{\makebox[10cm]{杂诗\dotfill{} 【三国】曹植}}{杂诗\ 【三国】曹植}}\nopagebreak%
\noindent\begin{minipage}{\linewidth}
  \pauthor{【三国】曹植}
  \vskip-3pt\begin{table}[H]
    \centering
    \begin{tabular}{@{}l@{}}
南国有佳人,容华若桃李。\\
朝游江北岸,夕宿潇湘\xpinyin*{\xpinyin{沚}{zhǐ}}。\\
时俗薄朱颜,谁为发皓齿?\\
俯仰岁将暮,荣耀难久恃。
    \end{tabular}
  \end{table}
\end{minipage}
\vspace{1cm}


\ptitle{燕歌行}\nopagebreak%
\addcontentsline{toc}{section}{\texorpdfstring{\makebox[10cm]{燕歌行\dotfill{} 【三国】曹丕}}{燕歌行\ 【三国】曹丕}}\nopagebreak%
\noindent\begin{minipage}{\linewidth}
  \pauthor{【三国】曹丕}
  \vskip-3pt\begin{table}[H]
    \centering
    \begin{tabular}{@{}l@{}}
秋风萧瑟天气凉,草木摇落露为霜,群燕辞归鹄南翔。\\
念君客游思断肠,\xpinyin*{\xpinyin{慊}{qiè}}慊思归恋故乡,君何淹留寄他方?\\
贱妾\xpinyin*{\xpinyin{茕}{qióng}}茕守空房,忧来思君不敢忘,不觉泪下沾衣裳。\\
援琴鸣弦发清商,短歌微吟不能长。\\
明月皎皎照我床,星汉西流夜未央。\\
牵牛织女遥相望,尔独何辜限河梁。
    \end{tabular}
  \end{table}
\end{minipage}
\vspace{1cm}


\ptitle{思吴江歌}\nopagebreak%
\addcontentsline{toc}{section}{\texorpdfstring{\makebox[10cm]{思吴江歌\dotfill{} 【魏晋】张翰}}{思吴江歌\ 【魏晋】张翰}}\nopagebreak%
\noindent\begin{minipage}{\linewidth}
  \pauthor{【魏晋】张翰}
  \vskip-3pt\begin{table}[H]
    \centering
    \begin{tabular}{@{}l@{}}
秋风起兮木叶飞,吴江水兮鲈正肥。\\
三千里兮家未归,恨难禁兮仰天悲。
    \end{tabular}
  \end{table}
\end{minipage}
\vspace{1cm}


\ptitle{归园田居}\nopagebreak%
\addcontentsline{toc}{section}{\texorpdfstring{\makebox[10cm]{归园田居\dotfill{} 【东晋】陶渊明}}{归园田居\ 【东晋】陶渊明}}\nopagebreak%
\noindent\begin{minipage}{\linewidth}
  \pauthor{【东晋】陶渊明}
  \vskip-3pt\begin{table}[H]
    \centering
    \begin{tabular}{@{}l@{}}
种豆南山下,草盛豆苗稀。\\
晨兴理荒\xpinyin*{\xpinyin{秽}{huì}},带月荷锄归。\\
道狭草木长,夕露沾我衣。\\
衣沾不足惜,但使愿无违。
    \end{tabular}
  \end{table}
\end{minipage}
\vspace{1cm}


\ptitle{归园田居·其一}\nopagebreak%
\addcontentsline{toc}{section}{\texorpdfstring{\makebox[10cm]{归园田居·其一\dotfill{} 【东晋】陶渊明}}{归园田居·其一\ 【东晋】陶渊明}}\nopagebreak%
\noindent\begin{minipage}{\linewidth}
  \pauthor{【东晋】陶渊明}
  \vskip-3pt\begin{table}[H]
    \centering
    \begin{tabular}{@{}l@{}}
少无适俗韵,性本爱丘山。误落尘网中,一去三十年。\\
羁鸟恋旧林,池鱼思故渊。开荒南野际,守拙归园田。\\
方宅十余亩,草屋八九间。榆柳荫后檐,桃李罗堂前。\\
暧暧远人村,依依墟里烟。狗吠深巷中,鸡鸣桑树颠。\\
户庭无尘杂,虚室有余闲。久在樊笼里,复得返自然。
    \end{tabular}
  \end{table}
\end{minipage}
\vspace{1cm}


\ptitle{杂诗}\nopagebreak%
\addcontentsline{toc}{section}{\texorpdfstring{\makebox[10cm]{杂诗\dotfill{} 【东晋】陶渊明}}{杂诗\ 【东晋】陶渊明}}\nopagebreak%
\noindent\begin{minipage}{\linewidth}
  \pauthor{【东晋】陶渊明}
  \vskip-3pt\begin{table}[H]
    \centering
    \begin{tabular}{@{}l@{}}
人生无根蒂,飘如陌上尘。\\
分散逐风转,此已非常身。\\
落地为兄弟,何必骨肉亲!\\
得欢当作乐,斗酒聚比邻。\\
盛年不重来,一日难再晨。\\
及时当勉励,岁月不待人。
    \end{tabular}
  \end{table}
\end{minipage}
\vspace{1cm}


\ptitle{饮酒}\nopagebreak%
\addcontentsline{toc}{section}{\texorpdfstring{\makebox[10cm]{饮酒\dotfill{} 【东晋】陶渊明}}{饮酒\ 【东晋】陶渊明}}\nopagebreak%
\noindent\begin{minipage}{\linewidth}
  \pauthor{【东晋】陶渊明}
  \vskip-3pt\begin{table}[H]
    \centering
    \begin{tabular}{@{}l@{}}
结庐在人境,而无车马喧。\\
问君何能尔?心远地自偏。\\
采菊东篱下,悠然见南山。\\
山气日夕佳,飞鸟相与还。\\
此中有真意,欲辨已忘言。
    \end{tabular}
  \end{table}
\end{minipage}
\vspace{1cm}


\ptitle{\xpinyin*{\xpinyin{敕}{chì}}\xpinyin*{\xpinyin{勒}{lè}}歌}\nopagebreak%
\addcontentsline{toc}{section}{\texorpdfstring{\makebox[10cm]{敕勒歌\dotfill{} 【北朝】乐府}}{敕勒歌\ 【北朝】乐府}}\nopagebreak%
\noindent\begin{minipage}{\linewidth}
  \pauthor{【北朝】乐府}
  \vskip-3pt\begin{table}[H]
    \centering
    \begin{tabular}{@{}l@{}}
敕勒川,阴山下,\\
天似穹庐,笼盖四野。\\
天苍苍,野茫茫,\\
风吹草低\xpinyin*{\xpinyin{见}{xiàn}}牛羊。
    \end{tabular}
  \end{table}
\end{minipage}
\vspace{1cm}


\ptitle{木兰辞}\nopagebreak%
\addcontentsline{toc}{section}{\texorpdfstring{\makebox[10cm]{木兰辞\dotfill{} 【北朝】乐府}}{木兰辞\ 【北朝】乐府}}\nopagebreak%
\noindent\begin{minipage}{\linewidth}
  \pauthor{【北朝】乐府}
  \vskip-3pt\begin{table}[H]
    \centering
    \begin{tabular}{@{}l@{}}
唧唧复唧唧,木兰当户织。不闻机\xpinyin*{\xpinyin{杼}{zhù}}声,唯闻女叹息。\\
问女何所思,问女何所忆。女亦无所思,女亦无所忆。\\
昨夜见军帖,\xpinyin*{\xpinyin{可}{kè}}\xpinyin*{\xpinyin{汗}{hán}}大点兵,军书十二卷,卷卷有爷名。\\
阿爷无大儿,木兰无长兄,愿为市\xpinyin*{\xpinyin{鞍}{ān}}马,从此替爷征。\\
\\
东市买骏马,西市买\xpinyin*{\xpinyin{鞍}{ān}}\xpinyin*{\xpinyin{鞯}{jiān}},南市买\xpinyin*{\xpinyin{辔}{pèi}}头,北市买长鞭。\\
旦辞爷娘去,暮宿黄河边。不闻爷娘唤女声,但闻黄河流水鸣\xpinyin*{\xpinyin{溅}{jiān}}溅。\\
旦辞黄河去,暮至黑山头。不闻爷娘唤女声,但闻燕山胡骑鸣\xpinyin*{\xpinyin{啾}{jiū}}啾。\\
\\
万里赴戎机,关山度若飞。\xpinyin*{\xpinyin{朔}{shuò}}气传金\xpinyin*{\xpinyin{柝}{tuò}},寒光照铁衣。\\
将军百战死,壮士十年归。归来见天子,天子坐明堂。\\
策勋十二转,赏赐百千强。可汗问所欲,木兰不用尚书郎,\\
愿驰千里足,送儿还故乡。\\
\\
爷娘闻女来,出郭相扶将;阿姊闻妹来,当户理红妆;\\
小弟闻姊来,磨刀霍霍向猪羊。开我东阁门,坐我西阁床。\\
脱我战时袍,著我旧时裳。当窗理云鬓,对镜帖花黄。\\
出门看火伴,火伴皆惊忙。同行十二年,不知木兰是女郎。\\
\\
雄兔脚\xpinyin*{\xpinyin{扑}{pū}}\xpinyin*{\xpinyin{朔}{shuò}},雌兔眼迷离;双兔傍地走,安能辨我是雄雌?
    \end{tabular}
  \end{table}
\end{minipage}
\vspace{1cm}


\ptitle{赠范\xpinyin*{\xpinyin{晔}{yè}}诗}\nopagebreak%
\addcontentsline{toc}{section}{\texorpdfstring{\makebox[10cm]{赠范晔诗\dotfill{} 【南北朝】陆凯}}{赠范晔诗\ 【南北朝】陆凯}}\nopagebreak%
\noindent\begin{minipage}{\linewidth}
  \pauthor{【南北朝】陆凯}
  \vskip-3pt\begin{table}[H]
    \centering
    \begin{tabular}{@{}l@{}}
折花逢驿使,寄与\xpinyin*{\xpinyin{陇}{lǒng}}头人。\\
江南无所有,聊赠一枝春。
    \end{tabular}
  \end{table}
\end{minipage}
\vspace{1cm}


\ptitle{山中杂诗}\nopagebreak%
\addcontentsline{toc}{section}{\texorpdfstring{\makebox[10cm]{山中杂诗\dotfill{} 【南朝】吴均}}{山中杂诗\ 【南朝】吴均}}\nopagebreak%
\noindent\begin{minipage}{\linewidth}
  \pauthor{【南朝】吴均}
  \vskip-3pt\begin{table}[H]
    \centering
    \begin{tabular}{@{}l@{}}
山际见来烟,竹中窥落日。\\
鸟向檐上飞,云从窗里出。
    \end{tabular}
  \end{table}
\end{minipage}
\vspace{1cm}


\chapter{唐}
\ptitle{菊花}\nopagebreak%
\addcontentsline{toc}{section}{\texorpdfstring{\makebox[10cm]{菊花\dotfill{} 【唐】元稹}}{菊花\ 【唐】元稹}}\nopagebreak%
\noindent\begin{minipage}{\linewidth}
  \pauthor{【唐】元\xpinyin*{\xpinyin{稹}{zhěn}}}
  \vskip-3pt\begin{table}[H]
    \centering
    \begin{tabular}{@{}l@{}}
秋丛绕舍似陶家,遍绕篱边日渐斜。\\
不是花中偏爱菊,此花开尽更无花。
    \end{tabular}
  \end{table}
\end{minipage}
\vspace{1cm}


\ptitle{渡汉江}\nopagebreak%
\addcontentsline{toc}{section}{\texorpdfstring{\makebox[10cm]{渡汉江\dotfill{} 【唐】宋之问}}{渡汉江\ 【唐】宋之问}}\nopagebreak%
\noindent\begin{minipage}{\linewidth}
  \pauthor{【唐】宋之问}
  \vskip-3pt\begin{table}[H]
    \centering
    \begin{tabular}{@{}l@{}}
岭外音书断,经冬复历春。\\
近乡情更怯,不敢问来人。
    \end{tabular}
  \end{table}
\end{minipage}
\vspace{1cm}


\ptitle{乐游原}\nopagebreak%
\addcontentsline{toc}{section}{\texorpdfstring{\makebox[10cm]{乐游原\dotfill{} 【唐】李商隐}}{乐游原\ 【唐】李商隐}}\nopagebreak%
\noindent\begin{minipage}{\linewidth}
  \pauthor{【唐】李商隐}
  \vskip-3pt\begin{table}[H]
    \centering
    \begin{tabular}{@{}l@{}}
向晚意不适, 驱车登古原。\\
夕阳无限好, 只是近黄昏。
    \end{tabular}
  \end{table}
\end{minipage}
\vspace{1cm}


\ptitle{峨眉山月歌}\nopagebreak%
\addcontentsline{toc}{section}{\texorpdfstring{\makebox[10cm]{峨眉山月歌\dotfill{} 【唐】李白}}{峨眉山月歌\ 【唐】李白}}\nopagebreak%
\noindent\begin{minipage}{\linewidth}
  \pauthor{【唐】李白}
  \vskip-3pt\begin{table}[H]
    \centering
    \begin{tabular}{@{}l@{}}
峨眉山月半轮秋,影入平羌江水流。\\
夜发清溪向三峡,思君不见下渝州。
    \end{tabular}
  \end{table}
\end{minipage}
\vspace{1cm}


\ptitle{月下独\xpinyin*{\xpinyin{酌}{zhuó}}}\nopagebreak%
\addcontentsline{toc}{section}{\texorpdfstring{\makebox[10cm]{月下独酌\dotfill{} 【唐】李白}}{月下独酌\ 【唐】李白}}\nopagebreak%
\noindent\begin{minipage}{\linewidth}
  \pauthor{【唐】李白}
  \vskip-3pt\begin{table}[H]
    \centering
    \begin{tabular}{@{}l@{}}
花间一壶酒,独酌无相亲。举杯邀明月,对影成三人。\\
月既不解饮,影徒随我身。暂伴月将影,行乐须及春。\\
我歌月徘徊,我舞影零乱。醒时相交欢,醉后各分散。\\
永结无情游,相期\xpinyin*{\xpinyin{邈}{miǎo}}云汉。
    \end{tabular}
  \end{table}
\end{minipage}
\vspace{1cm}


\ptitle{八阵图}\nopagebreak%
\addcontentsline{toc}{section}{\texorpdfstring{\makebox[10cm]{八阵图\dotfill{} 【唐】杜甫}}{八阵图\ 【唐】杜甫}}\nopagebreak%
\noindent\begin{minipage}{\linewidth}
  \pauthor{【唐】杜甫}
  \vskip-3pt\begin{table}[H]
    \centering
    \begin{tabular}{@{}l@{}}
功盖三分国,名成八阵图。\\
江流石不转,遗恨失吞吴。
    \end{tabular}
  \end{table}
\end{minipage}
\vspace{1cm}


\ptitle{绝句}\nopagebreak%
\addcontentsline{toc}{section}{\texorpdfstring{\makebox[10cm]{绝句\dotfill{} 【唐】杜甫}}{绝句\ 【唐】杜甫}}\nopagebreak%
\noindent\begin{minipage}{\linewidth}
  \pauthor{【唐】杜甫}
  \vskip-3pt\begin{table}[H]
    \centering
    \begin{tabular}{@{}l@{}}
江碧鸟逾白,山青花欲燃。\\
今春看又过,何日是归年。
    \end{tabular}
  \end{table}
\end{minipage}
\vspace{1cm}


\ptitle{田园乐(其六)}\nopagebreak%
\addcontentsline{toc}{section}{\texorpdfstring{\makebox[10cm]{田园乐(其六)\dotfill{} 【唐】王维}}{田园乐(其六)\ 【唐】王维}}\nopagebreak%
\noindent\begin{minipage}{\linewidth}
  \pauthor{【唐】王维}
  \vskip-3pt\begin{table}[H]
    \centering
    \begin{tabular}{@{}l@{}}
桃红复含宿雨,柳绿更带朝烟。\\
花落家童未扫,莺啼山客犹眠。
    \end{tabular}
  \end{table}
\end{minipage}
\vspace{1cm}


\ptitle{相思}\nopagebreak%
\addcontentsline{toc}{section}{\texorpdfstring{\makebox[10cm]{相思\dotfill{} 【唐】王维}}{相思\ 【唐】王维}}\nopagebreak%
\noindent\begin{minipage}{\linewidth}
  \pauthor{【唐】王维}
  \vskip-3pt\begin{table}[H]
    \centering
    \begin{tabular}{@{}l@{}}
红豆生南国,春来发几枝。\\
愿君多采撷,此物最相思。
    \end{tabular}
  \end{table}
\end{minipage}
\vspace{1cm}


\ptitle{送别}\nopagebreak%
\addcontentsline{toc}{section}{\texorpdfstring{\makebox[10cm]{送别\dotfill{} 【唐】王维}}{送别\ 【唐】王维}}\nopagebreak%
\noindent\begin{minipage}{\linewidth}
  \pauthor{【唐】王维}
  \vskip-3pt\begin{table}[H]
    \centering
    \begin{tabular}{@{}l@{}}
山中相送罢,日暮掩柴扉。\\
春草年年绿,王孙归不归。
    \end{tabular}
  \end{table}
\end{minipage}
\vspace{1cm}


\ptitle{题诗后}\nopagebreak%
\addcontentsline{toc}{section}{\texorpdfstring{\makebox[10cm]{题诗后\dotfill{} 【唐】贾岛}}{题诗后\ 【唐】贾岛}}\nopagebreak%
\noindent\begin{minipage}{\linewidth}
  \pauthor{【唐】贾岛}
  \vskip-3pt\begin{table}[H]
    \centering
    \begin{tabular}{@{}l@{}}
两句三年得,一吟双泪流。\\
知音如不赏,归卧故山秋。
    \end{tabular}
  \end{table}
\end{minipage}
\vspace{1cm}


\ptitle{山亭夏日}\nopagebreak%
\addcontentsline{toc}{section}{\texorpdfstring{\makebox[10cm]{山亭夏日\dotfill{} 【唐】高骈}}{山亭夏日\ 【唐】高骈}}\nopagebreak%
\noindent\begin{minipage}{\linewidth}
  \pauthor{【唐】高骈}
  \vskip-3pt\begin{table}[H]
    \centering
    \begin{tabular}{@{}l@{}}
绿树阴浓夏日长,楼台倒影入池塘。\\
水晶帘动微风起,满架蔷薇一院香。
    \end{tabular}
  \end{table}
\end{minipage}
\vspace{1cm}


\ptitle{行宫}\nopagebreak%
\addcontentsline{toc}{section}{\texorpdfstring{\makebox[10cm]{行宫\dotfill{} 【唐】元稹}}{行宫\ 【唐】元稹}}\nopagebreak%
\noindent\begin{minipage}{\linewidth}
  \pauthor{【唐】元\xpinyin*{\xpinyin{稹}{zhěn}}}
  \vskip-3pt\begin{table}[H]
    \centering
    \begin{tabular}{@{}l@{}}
寥落古行宫,宫花寂寞红。\\
白头宫女在,闲坐说玄宗。
    \end{tabular}
  \end{table}
\end{minipage}
\vspace{1cm}


\ptitle{乌衣巷}\nopagebreak%
\addcontentsline{toc}{section}{\texorpdfstring{\makebox[10cm]{乌衣巷\dotfill{} 【唐】刘禹锡}}{乌衣巷\ 【唐】刘禹锡}}\nopagebreak%
\noindent\begin{minipage}{\linewidth}
  \pauthor{【唐】刘禹锡}
  \vskip-3pt\begin{table}[H]
    \centering
    \begin{tabular}{@{}l@{}}
朱雀桥边野草花,乌衣巷口夕阳斜。\\
旧时王谢堂前燕,飞入寻常百姓家。
    \end{tabular}
  \end{table}
\end{minipage}
\vspace{1cm}


\ptitle{望洞庭}\nopagebreak%
\addcontentsline{toc}{section}{\texorpdfstring{\makebox[10cm]{望洞庭\dotfill{} 【唐】刘禹锡}}{望洞庭\ 【唐】刘禹锡}}\nopagebreak%
\noindent\begin{minipage}{\linewidth}
  \pauthor{【唐】刘禹锡}
  \vskip-3pt\begin{table}[H]
    \centering
    \begin{tabular}{@{}l@{}}
湖光秋月两相和,潭面无风镜未磨。\\
遥望洞庭山水翠,白银盘里一青螺。
    \end{tabular}
  \end{table}
\end{minipage}
\vspace{1cm}


\ptitle{浪淘沙}\nopagebreak%
\addcontentsline{toc}{section}{\texorpdfstring{\makebox[10cm]{浪淘沙\dotfill{} 【唐】刘禹锡}}{浪淘沙\ 【唐】刘禹锡}}\nopagebreak%
\noindent\begin{minipage}{\linewidth}
  \pauthor{【唐】刘\xpinyin*{\xpinyin{禹}{yǔ}}锡}
  \vskip-3pt\begin{table}[H]
    \centering
    \begin{tabular}{@{}l@{}}
九曲黄河万里沙,浪淘风\xpinyin*{\xpinyin{簸}{bǒ}}自天涯。\\
如今直上银河去,同到牵牛织女家。
    \end{tabular}
  \end{table}
\end{minipage}
\vspace{1cm}


\ptitle{秋词}\nopagebreak%
\addcontentsline{toc}{section}{\texorpdfstring{\makebox[10cm]{秋词\dotfill{} 【唐】刘禹锡}}{秋词\ 【唐】刘禹锡}}\nopagebreak%
\noindent\begin{minipage}{\linewidth}
  \pauthor{【唐】刘禹锡}
  \vskip-3pt\begin{table}[H]
    \centering
    \begin{tabular}{@{}l@{}}
自古逢秋悲寂寥,我言秋日胜春朝。\\
晴空一鹤排云上,便引诗情到碧宵。
    \end{tabular}
  \end{table}
\end{minipage}
\vspace{1cm}


\ptitle{秋风引}\nopagebreak%
\addcontentsline{toc}{section}{\texorpdfstring{\makebox[10cm]{秋风引\dotfill{} 【唐】刘禹锡}}{秋风引\ 【唐】刘禹锡}}\nopagebreak%
\noindent\begin{minipage}{\linewidth}
  \pauthor{【唐】刘禹锡}
  \vskip-3pt\begin{table}[H]
    \centering
    \begin{tabular}{@{}l@{}}
何处秋风至?萧萧送雁群。\\
朝来入庭树,孤客最先闻。
    \end{tabular}
  \end{table}
\end{minipage}
\vspace{1cm}


\ptitle{竹枝词 其一}\nopagebreak%
\addcontentsline{toc}{section}{\texorpdfstring{\makebox[10cm]{竹枝词 其一\dotfill{} 【唐】刘禹锡}}{竹枝词 其一\ 【唐】刘禹锡}}\nopagebreak%
\noindent\begin{minipage}{\linewidth}
  \pauthor{【唐】刘禹锡}
  \vskip-3pt\begin{table}[H]
    \centering
    \begin{tabular}{@{}l@{}}
杨柳青青江水平,闻郎江上踏歌声 。\\
东边日出西边雨,道是无晴却有晴。
    \end{tabular}
  \end{table}
\end{minipage}
\vspace{1cm}


\ptitle{竹枝词 其二}\nopagebreak%
\addcontentsline{toc}{section}{\texorpdfstring{\makebox[10cm]{竹枝词 其二\dotfill{} 【唐】刘禹锡}}{竹枝词 其二\ 【唐】刘禹锡}}\nopagebreak%
\noindent\begin{minipage}{\linewidth}
  \pauthor{【唐】刘禹锡}
  \vskip-3pt\begin{table}[H]
    \centering
    \begin{tabular}{@{}l@{}}
楚水巴山江雨多,巴人能唱本乡歌。\\
今朝北客思归去,回入纥那披绿罗。
    \end{tabular}
  \end{table}
\end{minipage}
\vspace{1cm}


\ptitle{酬乐天扬州初逢席上见赠}\nopagebreak%
\addcontentsline{toc}{section}{\texorpdfstring{\makebox[10cm]{酬乐天扬州初逢席上见赠\dotfill{} 【唐】刘禹锡}}{酬乐天扬州初逢席上见赠\ 【唐】刘禹锡}}\nopagebreak%
\noindent\begin{minipage}{\linewidth}
  \pauthor{【唐】刘禹锡}
  \vskip-3pt\begin{table}[H]
    \centering
    \begin{tabular}{@{}l@{}}
巴山楚水凄凉地,二十三年弃置身。\\
怀旧空吟闻笛赋,到乡翻似烂柯人。\\
沉舟侧畔千帆过,病树前头万木春。\\
今日听君歌一曲,暂凭杯酒长精神。
    \end{tabular}
  \end{table}
\end{minipage}
\vspace{1cm}


\ptitle{弹琴}\nopagebreak%
\addcontentsline{toc}{section}{\texorpdfstring{\makebox[10cm]{弹琴\dotfill{} 【唐】刘长卿}}{弹琴\ 【唐】刘长卿}}\nopagebreak%
\noindent\begin{minipage}{\linewidth}
  \pauthor{【唐】刘长卿}
  \vskip-3pt\begin{table}[H]
    \centering
    \begin{tabular}{@{}l@{}}
泠泠七弦上,静听松风寒。\\
古调虽自爱,今人多不弹。
    \end{tabular}
  \end{table}
\end{minipage}
\vspace{1cm}


\ptitle{逢雪宿芙蓉山主人}\nopagebreak%
\addcontentsline{toc}{section}{\texorpdfstring{\makebox[10cm]{逢雪宿芙蓉山主人\dotfill{} 【唐】刘长卿}}{逢雪宿芙蓉山主人\ 【唐】刘长卿}}\nopagebreak%
\noindent\begin{minipage}{\linewidth}
  \pauthor{【唐】刘长卿}
  \vskip-3pt\begin{table}[H]
    \centering
    \begin{tabular}{@{}l@{}}
日暮苍山远,天寒白屋贫。\\
柴门闻犬吠,风雪夜归人。
    \end{tabular}
  \end{table}
\end{minipage}
\vspace{1cm}


\ptitle{塞下曲}\nopagebreak%
\addcontentsline{toc}{section}{\texorpdfstring{\makebox[10cm]{塞下曲\dotfill{} 【唐】卢纶}}{塞下曲\ 【唐】卢纶}}\nopagebreak%
\noindent\begin{minipage}{\linewidth}
  \pauthor{【唐】卢\xpinyin*{\xpinyin{纶}{lún}}}
  \vskip-3pt\begin{table}[H]
    \centering
    \begin{tabular}{@{}l@{}}
月黑雁飞高,单于夜\xpinyin*{\xpinyin{遁}{dùn}}逃。\\
欲将轻\xpinyin*{\xpinyin{骑}{jì}}逐,大雪满弓刀。
    \end{tabular}
  \end{table}
\end{minipage}
\vspace{1cm}


\ptitle{牧童}\nopagebreak%
\addcontentsline{toc}{section}{\texorpdfstring{\makebox[10cm]{牧童\dotfill{} 【唐】吕岩}}{牧童\ 【唐】吕岩}}\nopagebreak%
\noindent\begin{minipage}{\linewidth}
  \pauthor{【唐】吕岩}
  \vskip-3pt\begin{table}[H]
    \centering
    \begin{tabular}{@{}l@{}}
草铺横野六七里,笛弄晚风三四声。\\
归来饱饭黄昏后,不脱蓑衣卧月明。
    \end{tabular}
  \end{table}
\end{minipage}
\vspace{1cm}


\ptitle{少年行}\nopagebreak%
\addcontentsline{toc}{section}{\texorpdfstring{\makebox[10cm]{少年行\dotfill{} 【唐】吴象之}}{少年行\ 【唐】吴象之}}\nopagebreak%
\noindent\begin{minipage}{\linewidth}
  \pauthor{【唐】吴象之}
  \vskip-3pt\begin{table}[H]
    \centering
    \begin{tabular}{@{}l@{}}
承恩借猎小平津,使气常游中贵人。\\
一掷千金浑是胆,家无四壁不知贫。
    \end{tabular}
  \end{table}
\end{minipage}
\vspace{1cm}


\ptitle{临洞庭湖赠张丞相}\nopagebreak%
\addcontentsline{toc}{section}{\texorpdfstring{\makebox[10cm]{临洞庭湖赠张丞相\dotfill{} 【唐】孟浩然}}{临洞庭湖赠张丞相\ 【唐】孟浩然}}\nopagebreak%
\noindent\begin{minipage}{\linewidth}
  \pauthor{【唐】孟浩然}
  \vskip-3pt\begin{table}[H]
    \centering
    \begin{tabular}{@{}l@{}}
八月湖水平,涵虚混太清。\\
气蒸云梦泽,波撼岳阳城。\\
欲济无舟楫,端居耻圣明。\\
坐观垂钓者,徒有羡鱼情。
    \end{tabular}
  \end{table}
\end{minipage}
\vspace{1cm}


\ptitle{宿建德江}\nopagebreak%
\addcontentsline{toc}{section}{\texorpdfstring{\makebox[10cm]{宿建德江\dotfill{} 【唐】孟浩然}}{宿建德江\ 【唐】孟浩然}}\nopagebreak%
\noindent\begin{minipage}{\linewidth}
  \pauthor{【唐】孟浩然}
  \vskip-3pt\begin{table}[H]
    \centering
    \begin{tabular}{@{}l@{}}
移舟泊烟\xpinyin*{\xpinyin{渚}{zhǔ}},日暮客愁新。\\
野旷天低树,江清月近人。
    \end{tabular}
  \end{table}
\end{minipage}
\vspace{1cm}


\ptitle{春晓}\nopagebreak%
\addcontentsline{toc}{section}{\texorpdfstring{\makebox[10cm]{春晓\dotfill{} 【唐】孟浩然}}{春晓\ 【唐】孟浩然}}\nopagebreak%
\noindent\begin{minipage}{\linewidth}
  \pauthor{【唐】孟浩然}
  \vskip-3pt\begin{table}[H]
    \centering
    \begin{tabular}{@{}l@{}}
春眠不觉晓,处处闻啼鸟。\\
夜来风雨声,花落知多少。
    \end{tabular}
  \end{table}
\end{minipage}
\vspace{1cm}


\ptitle{游子吟}\nopagebreak%
\addcontentsline{toc}{section}{\texorpdfstring{\makebox[10cm]{游子吟\dotfill{} 【唐】孟郊}}{游子吟\ 【唐】孟郊}}\nopagebreak%
\noindent\begin{minipage}{\linewidth}
  \pauthor{【唐】孟郊}
  \vskip-3pt\begin{table}[H]
    \centering
    \begin{tabular}{@{}l@{}}
慈母手中线,游子身上衣。\\
临行密密缝,意恐迟迟归。\\
谁言寸草心,报得三春\xpinyin*{\xpinyin{晖}{huī}}!
    \end{tabular}
  \end{table}
\end{minipage}
\vspace{1cm}


\ptitle{登科后}\nopagebreak%
\addcontentsline{toc}{section}{\texorpdfstring{\makebox[10cm]{登科后\dotfill{} 【唐】孟郊}}{登科后\ 【唐】孟郊}}\nopagebreak%
\noindent\begin{minipage}{\linewidth}
  \pauthor{【唐】孟郊}
  \vskip-3pt\begin{table}[H]
    \centering
    \begin{tabular}{@{}l@{}}
昔日\xpinyin*{\xpinyin{龌}{wò}}\xpinyin*{\xpinyin{龊}{chuò}}不足夸,今朝放荡思无涯。\\
春风得意马蹄疾,一日看尽长安花。
    \end{tabular}
  \end{table}
\end{minipage}
\vspace{1cm}


\ptitle{白雪歌送武判官归京}\nopagebreak%
\addcontentsline{toc}{section}{\texorpdfstring{\makebox[10cm]{白雪歌送武判官归京\dotfill{} 【唐】岑参}}{白雪歌送武判官归京\ 【唐】岑参}}\nopagebreak%
\noindent\begin{minipage}{\linewidth}
  \pauthor{【唐】岑参}
  \vskip-3pt\begin{table}[H]
    \centering
    \begin{tabular}{@{}l@{}}
北风卷地白草折,胡天八月即飞雪。忽如一夜春风来,千树万树梨花开。\\
散入珠帘湿罗幕,狐裘不暖锦衾薄。将军角弓不得控,都护铁衣冷难着。\\
瀚海阑干百丈冰,愁云惨淡万里凝。中军置酒饮归客,胡琴琵琶与羌笛。\\
纷纷暮雪下辕门,风\xpinyin*{\xpinyin{掣}{chè}}红旗冻不翻。轮台东门送君去,去时雪满天山路。\\
山回路转不见君,雪上空留马行处。
    \end{tabular}
  \end{table}
\end{minipage}
\vspace{1cm}


\ptitle{黄鹤楼}\nopagebreak%
\addcontentsline{toc}{section}{\texorpdfstring{\makebox[10cm]{黄鹤楼\dotfill{} 【唐】崔颢}}{黄鹤楼\ 【唐】崔颢}}\nopagebreak%
\noindent\begin{minipage}{\linewidth}
  \pauthor{【唐】崔\xpinyin*{\xpinyin{颢}{hào}}}
  \vskip-3pt\begin{table}[H]
    \centering
    \begin{tabular}{@{}l@{}}
昔人已乘黄鹤去,此地空余黄鹤楼。黄鹤一去不复返,白云千载空悠悠。\\
晴川历历汉阳树,芳草萋萋鹦鹉洲。日暮乡关何处是?烟波江上使人愁。
    \end{tabular}
  \end{table}
\end{minipage}
\vspace{1cm}


\ptitle{渔歌子}\nopagebreak%
\addcontentsline{toc}{section}{\texorpdfstring{\makebox[10cm]{渔歌子\dotfill{} 【唐】张志和}}{渔歌子\ 【唐】张志和}}\nopagebreak%
\noindent\begin{minipage}{\linewidth}
  \pauthor{【唐】张志和}
  \vskip-3pt\begin{table}[H]
    \centering
    \begin{tabular}{@{}l@{}}
西塞山前白\xpinyin*{\xpinyin{鹭}{lù}}飞,桃花流水\xpinyin*{\xpinyin{鳜}{guì}}鱼肥。\\
青\xpinyin*{\xpinyin{箬}{ruò}}\xpinyin*{\xpinyin{笠}{lì}},绿\xpinyin*{\xpinyin{蓑}{suō}}衣,斜风细雨不须归。
    \end{tabular}
  \end{table}
\end{minipage}
\vspace{1cm}


\ptitle{寄人}\nopagebreak%
\addcontentsline{toc}{section}{\texorpdfstring{\makebox[10cm]{寄人\dotfill{} 【唐】张泌}}{寄人\ 【唐】张泌}}\nopagebreak%
\noindent\begin{minipage}{\linewidth}
  \pauthor{【唐】张泌}
  \vskip-3pt\begin{table}[H]
    \centering
    \begin{tabular}{@{}l@{}}
别梦依依到谢家,小廊回合曲阑斜。\\
多情只有春庭月,犹为离人照落花。
    \end{tabular}
  \end{table}
\end{minipage}
\vspace{1cm}


\ptitle{宫词}\nopagebreak%
\addcontentsline{toc}{section}{\texorpdfstring{\makebox[10cm]{宫词\dotfill{} 【唐】张祜}}{宫词\ 【唐】张祜}}\nopagebreak%
\noindent\begin{minipage}{\linewidth}
  \pauthor{【唐】张\xpinyin*{\xpinyin{祜}{hù}}}
  \vskip-3pt\begin{table}[H]
    \centering
    \begin{tabular}{@{}l@{}}
故国三千里,深宫二十年。\\
一声何满子,双泪落君前。
    \end{tabular}
  \end{table}
\end{minipage}
\vspace{1cm}


\ptitle{秋思}\nopagebreak%
\addcontentsline{toc}{section}{\texorpdfstring{\makebox[10cm]{秋思\dotfill{} 【唐】张籍}}{秋思\ 【唐】张籍}}\nopagebreak%
\noindent\begin{minipage}{\linewidth}
  \pauthor{【唐】张籍}
  \vskip-3pt\begin{table}[H]
    \centering
    \begin{tabular}{@{}l@{}}
洛阳城里见秋风,欲作家书意万重。\\
复恐匆匆说不尽,行人临发又开封。
    \end{tabular}
  \end{table}
\end{minipage}
\vspace{1cm}


\ptitle{枫桥夜泊}\nopagebreak%
\addcontentsline{toc}{section}{\texorpdfstring{\makebox[10cm]{枫桥夜泊\dotfill{} 【唐】张继}}{枫桥夜泊\ 【唐】张继}}\nopagebreak%
\noindent\begin{minipage}{\linewidth}
  \pauthor{【唐】张继}
  \vskip-3pt\begin{table}[H]
    \centering
    \begin{tabular}{@{}l@{}}
月落乌啼霜满天,江枫渔火对愁眠。\\
姑苏城外寒山寺,夜半钟声到客船。
    \end{tabular}
  \end{table}
\end{minipage}
\vspace{1cm}


\ptitle{春江花月夜}\nopagebreak%
\addcontentsline{toc}{section}{\texorpdfstring{\makebox[10cm]{春江花月夜\dotfill{} 【唐】张若虚}}{春江花月夜\ 【唐】张若虚}}\nopagebreak%
\noindent\begin{minipage}{\linewidth}
  \pauthor{【唐】张若虚}
  \vskip-3pt\begin{table}[H]
    \centering
    \begin{tabular}{@{}l@{}}
春江潮水连海平,海上明月共潮生。滟滟随波千万里,何处春江无月明。\\
江流宛转绕芳甸,月照花林皆似\xpinyin*{\xpinyin{霰}{xiàn}}。空里流霜不觉飞,\xpinyin*{\xpinyin{汀}{tīng}}上白沙看不见。\\
江天一色无纤尘,皎皎空中孤月轮。江畔何人初见月?江月何年初照人?\\
人生代代无穷已,江月年年只相似。不知江月待何人,但见长江送流水。\\
白云一片去悠悠,青枫浦上不胜愁。谁家今夜扁舟子?何处相思明月楼?\\
可怜楼上月徘徊,应照离人妆镜台。玉户帘中卷不去,捣衣\xpinyin*{\xpinyin{砧}{zhēn}}上拂还来。\\
此时相望不相闻,愿逐月华流照君。鸿雁长飞光不度,鱼龙潜跃水成文。\\
昨夜闲潭梦落花,可怜春半不还家。江水流春去欲尽,江潭落月复西斜。\\
斜月沉沉藏海雾,\xpinyin*{\xpinyin{碣}{jié}}石潇湘无限路。不知乘月几人归,落月摇情满江树。
    \end{tabular}
  \end{table}
\end{minipage}
\vspace{1cm}


\ptitle{早梅}\nopagebreak%
\addcontentsline{toc}{section}{\texorpdfstring{\makebox[10cm]{早梅\dotfill{} 【唐】张谓}}{早梅\ 【唐】张谓}}\nopagebreak%
\noindent\begin{minipage}{\linewidth}
  \pauthor{【唐】张谓}
  \vskip-3pt\begin{table}[H]
    \centering
    \begin{tabular}{@{}l@{}}
一树寒梅白玉条,迥临林村傍溪桥。\\
不知近水花先发,疑是经冬雪未销。
    \end{tabular}
  \end{table}
\end{minipage}
\vspace{1cm}


\ptitle{宫词}\nopagebreak%
\addcontentsline{toc}{section}{\texorpdfstring{\makebox[10cm]{宫词\dotfill{} 【唐】朱庆馀}}{宫词\ 【唐】朱庆馀}}\nopagebreak%
\noindent\begin{minipage}{\linewidth}
  \pauthor{【唐】朱庆馀}
  \vskip-3pt\begin{table}[H]
    \centering
    \begin{tabular}{@{}l@{}}
寂寂花时闭院门,美人相并立琼轩。\\
含情欲说宫中事,鹦鹉前头不敢言。
    \end{tabular}
  \end{table}
\end{minipage}
\vspace{1cm}


\ptitle{山中}\nopagebreak%
\addcontentsline{toc}{section}{\texorpdfstring{\makebox[10cm]{山中\dotfill{} 【唐】李咸用}}{山中\ 【唐】李咸用}}\nopagebreak%
\noindent\begin{minipage}{\linewidth}
  \pauthor{【唐】李咸用}
  \vskip-3pt\begin{table}[H]
    \centering
    \begin{tabular}{@{}l@{}}
一簇烟霞荣辱外,秋山留得傍檐楹。朝钟暮鼓不到耳,明月孤云长挂情。\\
世上路岐何缭绕,水边蓑笠称平生。寻思阮籍当时意,岂是途穷泣利名。
    \end{tabular}
  \end{table}
\end{minipage}
\vspace{1cm}


\ptitle{乐游原}\nopagebreak%
\addcontentsline{toc}{section}{\texorpdfstring{\makebox[10cm]{乐游原\dotfill{} 【唐】李商隐}}{乐游原\ 【唐】李商隐}}\nopagebreak%
\noindent\begin{minipage}{\linewidth}
  \pauthor{【唐】李商隐}
  \vskip-3pt\begin{table}[H]
    \centering
    \begin{tabular}{@{}l@{}}
向晚意不适,驱车登古原。\\
夕阳无限好,只是近黄昏。
    \end{tabular}
  \end{table}
\end{minipage}
\vspace{1cm}


\ptitle{夜雨寄北}\nopagebreak%
\addcontentsline{toc}{section}{\texorpdfstring{\makebox[10cm]{夜雨寄北\dotfill{} 【唐】李商隐}}{夜雨寄北\ 【唐】李商隐}}\nopagebreak%
\noindent\begin{minipage}{\linewidth}
  \pauthor{【唐】李商隐}
  \vskip-3pt\begin{table}[H]
    \centering
    \begin{tabular}{@{}l@{}}
君问归期未有期,巴山夜雨涨秋池。\\
何当共剪西窗烛,却话巴山夜雨时。
    \end{tabular}
  \end{table}
\end{minipage}
\vspace{1cm}


\ptitle{嫦娥}\nopagebreak%
\addcontentsline{toc}{section}{\texorpdfstring{\makebox[10cm]{嫦娥\dotfill{} 【唐】李商隐}}{嫦娥\ 【唐】李商隐}}\nopagebreak%
\noindent\begin{minipage}{\linewidth}
  \pauthor{【唐】李商隐}
  \vskip-3pt\begin{table}[H]
    \centering
    \begin{tabular}{@{}l@{}}
云母屏风烛影深,长河渐落晓星沉。\\
嫦娥应悔偷灵药,碧海青天夜夜心。
    \end{tabular}
  \end{table}
\end{minipage}
\vspace{1cm}


\ptitle{无题}\nopagebreak%
\addcontentsline{toc}{section}{\texorpdfstring{\makebox[10cm]{无题\dotfill{} 【唐】李商隐}}{无题\ 【唐】李商隐}}\nopagebreak%
\noindent\begin{minipage}{\linewidth}
  \pauthor{【唐】李商隐}
  \vskip-3pt\begin{table}[H]
    \centering
    \begin{tabular}{@{}l@{}}
昨夜星辰昨夜风,画楼西畔桂堂东。身无彩凤双飞翼,心有灵犀一点通。\\
隔座送钩春酒暖,分曹射覆蜡灯红。嗟余听鼓应官去,走马兰台类转蓬。
    \end{tabular}
  \end{table}
\end{minipage}
\vspace{1cm}


\ptitle{无题}\nopagebreak%
\addcontentsline{toc}{section}{\texorpdfstring{\makebox[10cm]{无题\dotfill{} 【唐】李商隐}}{无题\ 【唐】李商隐}}\nopagebreak%
\noindent\begin{minipage}{\linewidth}
  \pauthor{【唐】李商隐}
  \vskip-3pt\begin{table}[H]
    \centering
    \begin{tabular}{@{}l@{}}
相见时难别亦难,东风无力百花残。春蚕到死丝方尽,蜡炬成灰泪始干。\\
晓镜但愁云鬓改,夜吟应觉月光寒。蓬山此去无多路,青鸟殷勤为探看。
    \end{tabular}
  \end{table}
\end{minipage}
\vspace{1cm}


\ptitle{锦瑟}\nopagebreak%
\addcontentsline{toc}{section}{\texorpdfstring{\makebox[10cm]{锦瑟\dotfill{} 【唐】李商隐}}{锦瑟\ 【唐】李商隐}}\nopagebreak%
\noindent\begin{minipage}{\linewidth}
  \pauthor{【唐】李商隐}
  \vskip-3pt\begin{table}[H]
    \centering
    \begin{tabular}{@{}l@{}}
锦瑟无端五十弦,一弦一柱思华年。庄生晓梦迷蝴蝶,望帝春心托杜鹃。\\
沧海月明珠有泪,蓝田日暖玉生烟。此情可待成追忆,只是当时已惘然。
    \end{tabular}
  \end{table}
\end{minipage}
\vspace{1cm}


\ptitle{霜月}\nopagebreak%
\addcontentsline{toc}{section}{\texorpdfstring{\makebox[10cm]{霜月\dotfill{} 【唐】李商隐}}{霜月\ 【唐】李商隐}}\nopagebreak%
\noindent\begin{minipage}{\linewidth}
  \pauthor{【唐】李商隐}
  \vskip-3pt\begin{table}[H]
    \centering
    \begin{tabular}{@{}l@{}}
初闻征雁已无蝉,百尺楼高水接天。\\
青女素娥俱耐冷,月中霜里斗婵娟。
    \end{tabular}
  \end{table}
\end{minipage}
\vspace{1cm}


\ptitle{风}\nopagebreak%
\addcontentsline{toc}{section}{\texorpdfstring{\makebox[10cm]{风\dotfill{} 【唐】李峤}}{风\ 【唐】李峤}}\nopagebreak%
\noindent\begin{minipage}{\linewidth}
  \pauthor{【唐】李\xpinyin*{\xpinyin{峤}{qiáo}}}
  \vskip-3pt\begin{table}[H]
    \centering
    \begin{tabular}{@{}l@{}}
解落三秋叶,能开二月花。\\
过江千尺浪,入竹万竿斜。
    \end{tabular}
  \end{table}
\end{minipage}
\vspace{1cm}


\ptitle{古朗月行}\nopagebreak%
\addcontentsline{toc}{section}{\texorpdfstring{\makebox[10cm]{古朗月行\dotfill{} 【唐】李白}}{古朗月行\ 【唐】李白}}\nopagebreak%
\noindent\begin{minipage}{\linewidth}
  \pauthor{【唐】李白}
  \vskip-3pt\begin{table}[H]
    \centering
    \begin{tabular}{@{}l@{}}
小时不识月,呼作白玉盘。又疑瑶台镜,飞在青云端。\\
仙人垂两足,桂树何团团。白兔捣药成,问言与谁餐?\\
\xpinyin*{\xpinyin{蟾}{chán}}\xpinyin*{\xpinyin{蜍}{chú}}蚀圆影,大明夜已残。\xpinyin*{\xpinyin{羿}{yì}}昔落九乌,天人清且安。\\
阴精此沦惑,去去不足观。忧来其如何?凄\xpinyin*{\xpinyin{怆}{chuàng}}摧心肝。
    \end{tabular}
  \end{table}
\end{minipage}
\vspace{1cm}


\ptitle{将进酒}\nopagebreak%
\addcontentsline{toc}{section}{\texorpdfstring{\makebox[10cm]{将进酒\dotfill{} 【唐】李白}}{将进酒\ 【唐】李白}}\nopagebreak%
\noindent\begin{minipage}{\linewidth}
  \pauthor{【唐】李白}
  \vskip-3pt\begin{table}[H]
    \centering
    \begin{tabular}{@{}l@{}}
君不见黄河之水天上来,奔流到海不复回。\\
君不见高堂明镜悲白发,朝如青丝暮成雪。\\
   人生得意须尽欢,莫使金樽空对月。\\
   天生我材必有用,千金散尽还复来。\\
   烹羊宰牛且为乐,会须一饮三百杯。\\
   岑夫子,丹丘生,将进酒,杯莫停。\\
     与君歌一曲,请君为我倾耳听。\\
   钟鼓馔玉不足贵,但愿长醉不复醒。\\
   古来圣贤皆寂寞,惟有饮者留其名。\\
   陈王昔时宴平乐,斗酒十千恣欢谑。\\
   主人何为言少钱,径须沽取对君酌。\\
   五花马,千金裘,\\
   呼儿将出换美酒,与尔同销万古愁。
    \end{tabular}
  \end{table}
\end{minipage}
\vspace{1cm}


\ptitle{忆秦娥·箫声咽}\nopagebreak%
\addcontentsline{toc}{section}{\texorpdfstring{\makebox[10cm]{忆秦娥·箫声咽\dotfill{} 【唐】李白}}{忆秦娥·箫声咽\ 【唐】李白}}\nopagebreak%
\noindent\begin{minipage}{\linewidth}
  \pauthor{【唐】李白}
  \vskip-3pt\begin{table}[H]
    \centering
    \begin{tabular}{@{}l@{}}
箫声咽,秦娥梦断秦楼月。\\
秦楼月,年年柳色,\xpinyin*{\xpinyin{灞}{bà}}\xpinyin*{\xpinyin{陵}{líng}}伤别。\\
\\
乐游原上清秋节,咸阳古道音尘绝。\\
音尘绝,西风残照,汉家陵\xpinyin*{\xpinyin{阙}{què}}。
    \end{tabular}
  \end{table}
\end{minipage}
\vspace{1cm}


\ptitle{怨情}\nopagebreak%
\addcontentsline{toc}{section}{\texorpdfstring{\makebox[10cm]{怨情\dotfill{} 【唐】李白}}{怨情\ 【唐】李白}}\nopagebreak%
\noindent\begin{minipage}{\linewidth}
  \pauthor{【唐】李白}
  \vskip-3pt\begin{table}[H]
    \centering
    \begin{tabular}{@{}l@{}}
美人卷珠帘,深坐\xpinyin*{\xpinyin{颦}{pín}}蛾眉。\\
但见泪痕湿,不知心恨谁。
    \end{tabular}
  \end{table}
\end{minipage}
\vspace{1cm}


\ptitle{早发白帝城}\nopagebreak%
\addcontentsline{toc}{section}{\texorpdfstring{\makebox[10cm]{早发白帝城\dotfill{} 【唐】李白}}{早发白帝城\ 【唐】李白}}\nopagebreak%
\noindent\begin{minipage}{\linewidth}
  \pauthor{【唐】李白}
  \vskip-3pt\begin{table}[H]
    \centering
    \begin{tabular}{@{}l@{}}
朝辞白帝彩云间,千里江陵一日还。\\
两岸猿声啼不住,轻舟已过万重山。
    \end{tabular}
  \end{table}
\end{minipage}
\vspace{1cm}


\ptitle{望天门山}\nopagebreak%
\addcontentsline{toc}{section}{\texorpdfstring{\makebox[10cm]{望天门山\dotfill{} 【唐】李白}}{望天门山\ 【唐】李白}}\nopagebreak%
\noindent\begin{minipage}{\linewidth}
  \pauthor{【唐】李白}
  \vskip-3pt\begin{table}[H]
    \centering
    \begin{tabular}{@{}l@{}}
天门中断楚江开,碧水东流至此回。\\
两岸青山相对出,孤帆一片日边来。
    \end{tabular}
  \end{table}
\end{minipage}
\vspace{1cm}


\ptitle{望庐山瀑布}\nopagebreak%
\addcontentsline{toc}{section}{\texorpdfstring{\makebox[10cm]{望庐山瀑布\dotfill{} 【唐】李白}}{望庐山瀑布\ 【唐】李白}}\nopagebreak%
\noindent\begin{minipage}{\linewidth}
  \pauthor{【唐】李白}
  \vskip-3pt\begin{table}[H]
    \centering
    \begin{tabular}{@{}l@{}}
日照香炉生紫烟,遥看瀑布挂前川。\\
飞流直下三千尺,疑是银河落九天。
    \end{tabular}
  \end{table}
\end{minipage}
\vspace{1cm}


\ptitle{梦游天\xpinyin*{\xpinyin{姥}{mǔ}}吟留别}\nopagebreak%
\addcontentsline{toc}{section}{\texorpdfstring{\makebox[10cm]{梦游天姥吟留别\dotfill{} 【唐】李白}}{梦游天姥吟留别\ 【唐】李白}}\nopagebreak%
\noindent\begin{minipage}{\linewidth}
  \pauthor{【唐】李白}
  \vskip-3pt\begin{table}[H]
    \centering
    \begin{tabular}{@{}l@{}}
海客谈\xpinyin*{\xpinyin{瀛}{yíng}}洲,烟涛微茫信难求。\\
越人语天姥,云霓明灭或可睹。\\
天姥连天向天横,势拔五岳掩赤城。\\
天台四万八千丈,对此欲倒东南倾。\\
我欲因之梦吴越,一夜飞度镜湖月。\\
湖月照我影,送我至\xpinyin*{\xpinyin{剡}{shàn}}溪。\\
谢公宿处今尚在,\xpinyin*{\xpinyin{渌}{lù}}水荡漾清猿啼。\\
脚著谢公屐,身登青云梯。\\
半壁见海日,空中闻天鸡。\\
千岩万转路不定,迷花倚石忽已暝。\\
熊咆龙吟殷岩泉,栗深林兮惊层巅。\\
云青青兮欲雨,水澹\xpinyin*{\xpinyin{澹}{dàn}}兮生烟。\\
列缺霹雳,丘峦崩摧。洞天石扉,訇然中开。\\
青冥浩荡不见底,日月照耀金银台。\\
霓为衣兮风为马,云之君兮纷纷而来下。\\
虎鼓瑟兮鸾回车,仙之人兮列如麻。\\
忽魂悸以魄动,恍惊起而长嗟。\\
惟觉时之枕席,失向来之烟霞。\\
世间行乐亦如此,古来万事东流水。\\
别君去兮何时还,且放白鹿青崖间,须行即骑访名山。
    \end{tabular}
  \end{table}
\end{minipage}
\vspace{1cm}


\ptitle{登金陵凤凰台}\nopagebreak%
\addcontentsline{toc}{section}{\texorpdfstring{\makebox[10cm]{登金陵凤凰台\dotfill{} 【唐】李白}}{登金陵凤凰台\ 【唐】李白}}\nopagebreak%
\noindent\begin{minipage}{\linewidth}
  \pauthor{【唐】李白}
  \vskip-3pt\begin{table}[H]
    \centering
    \begin{tabular}{@{}l@{}}
凤凰台上凤凰游,凤去台空江自流。吴宫花草埋幽径,晋代衣冠成古丘。\\
三山半落青天外,二水中分白鹭洲。总为浮云能蔽日,长安不见使人愁。
    \end{tabular}
  \end{table}
\end{minipage}
\vspace{1cm}


\ptitle{秋浦歌}\nopagebreak%
\addcontentsline{toc}{section}{\texorpdfstring{\makebox[10cm]{秋浦歌\dotfill{} 【唐】李白}}{秋浦歌\ 【唐】李白}}\nopagebreak%
\noindent\begin{minipage}{\linewidth}
  \pauthor{【唐】李白}
  \vskip-3pt\begin{table}[H]
    \centering
    \begin{tabular}{@{}l@{}}
白发三千丈,缘愁似个长?\\
不知明镜里,何处得秋霜!\\
\\
雨过山村\\
唐  王建\\
雨里鸡鸣一两家,竹溪村路板桥斜。\\
妇姑相唤浴蚕去,闲看中庭栀子花。
    \end{tabular}
  \end{table}
\end{minipage}
\vspace{1cm}


\ptitle{草书歌行}\nopagebreak%
\addcontentsline{toc}{section}{\texorpdfstring{\makebox[10cm]{草书歌行\dotfill{} 【唐】李白}}{草书歌行\ 【唐】李白}}\nopagebreak%
\noindent\begin{minipage}{\linewidth}
  \pauthor{【唐】李白}
  \vskip-3pt\begin{table}[H]
    \centering
    \begin{tabular}{@{}l@{}}
少年上人号怀素,草书天下称独步。墨池飞出北溟鱼,笔锋杀尽中山兔。\\
八月九月天气凉,酒徒词客满高堂。笺麻素绢排数厢,宣州石砚墨色光。\\
吾师醉后倚绳床,须臾扫尽数千张。飘风骤雨惊飒飒,落花飞雪何茫茫。\\
起来向壁不停手,一行数字大如斗。怳怳如闻神鬼惊,时时只见龙蛇走。\\
左盘右蹙如惊电,状同楚汉相攻战。湖南七郡凡几家,家家屏障书题遍。\\
王逸少,张伯英,古来几许浪得名。张颠老死不足数,我师此义不师古。\\
古来万事贵天生,何必要公孙大娘浑脱舞。
    \end{tabular}
  \end{table}
\end{minipage}
\vspace{1cm}


\ptitle{蜀道难}\nopagebreak%
\addcontentsline{toc}{section}{\texorpdfstring{\makebox[10cm]{蜀道难\dotfill{} 【唐】李白}}{蜀道难\ 【唐】李白}}\nopagebreak%
\noindent\begin{minipage}{\linewidth}
  \pauthor{【唐】李白}
  \vskip-3pt\begin{table}[H]
    \centering
    \begin{tabular}{@{}l@{}}
噫吁嚱,危乎高哉!蜀道之难,难于上青天!\\
蚕丛及鱼凫,开国何茫然!\\
尔来四万八千岁,不与秦塞通人烟。\\
西当太白有鸟道,可以横绝峨眉巅。\\
地崩山摧壮士死,然后天梯石栈相钩连。\\
上有六龙回日之高标,下有冲波逆折之回川。\\
黄鹤之飞尚不得过,猿猱欲度愁攀援。\\
青泥何盘盘,百步九折萦岩峦。\\
扪参历井仰胁息,以手抚膺坐长叹。\\
问君西游何时还?畏途巉岩不可攀。\\
但见悲鸟号古木,雄飞雌从绕林间。\\
又闻子规啼夜月,愁空山。\\
蜀道之难,难于上青天,使人听此凋朱颜!\\
连峰去天不盈尺,枯松倒挂倚绝壁。\\
飞湍瀑流争喧豗,砯崖转石万壑雷。\\
其险也如此,嗟尔远道之人胡为乎来哉!\\
剑阁峥嵘而崔嵬,一夫当关,万夫莫开。\\
所守或匪亲,化为狼与豺。\\
朝避猛虎,夕避长蛇,\\
磨牙吮血,杀人如麻。\\
锦城虽云乐,不如早还家。\\
蜀道之难,难于上青天,侧身西望长咨嗟!
    \end{tabular}
  \end{table}
\end{minipage}
\vspace{1cm}


\ptitle{行路难}\nopagebreak%
\addcontentsline{toc}{section}{\texorpdfstring{\makebox[10cm]{行路难\dotfill{} 【唐】李白}}{行路难\ 【唐】李白}}\nopagebreak%
\noindent\begin{minipage}{\linewidth}
  \pauthor{【唐】李白}
  \vskip-3pt\begin{table}[H]
    \centering
    \begin{tabular}{@{}l@{}}
金\xpinyin*{\xpinyin{樽}{zūn}}清酒斗十千,玉盘珍\xpinyin*{\xpinyin{馐}{xiū}}直万钱。停杯投\xpinyin*{\xpinyin{箸}{zhù}}不能食,拔剑四顾心茫然。\\
欲渡黄河冰塞川,将登太行雪满山。闲来垂钓碧溪上,忽复乘舟梦日边。\\
行路难,行路难,多歧路,今安在?长风破浪会有时,直挂云帆济沧海。
    \end{tabular}
  \end{table}
\end{minipage}
\vspace{1cm}


\ptitle{赠汪伦}\nopagebreak%
\addcontentsline{toc}{section}{\texorpdfstring{\makebox[10cm]{赠汪伦\dotfill{} 【唐】李白}}{赠汪伦\ 【唐】李白}}\nopagebreak%
\noindent\begin{minipage}{\linewidth}
  \pauthor{【唐】李白}
  \vskip-3pt\begin{table}[H]
    \centering
    \begin{tabular}{@{}l@{}}
李白乘舟将欲行,忽闻岸上踏歌声。\\
桃花潭水深千尺,不及汪伦送我情。
    \end{tabular}
  \end{table}
\end{minipage}
\vspace{1cm}


\ptitle{送友人}\nopagebreak%
\addcontentsline{toc}{section}{\texorpdfstring{\makebox[10cm]{送友人\dotfill{} 【唐】李白}}{送友人\ 【唐】李白}}\nopagebreak%
\noindent\begin{minipage}{\linewidth}
  \pauthor{【唐】李白}
  \vskip-3pt\begin{table}[H]
    \centering
    \begin{tabular}{@{}l@{}}
青山横北郭,白水绕东城。\\
此地一为别,孤蓬万里征。\\
浮云游子意,落日故人情。\\
挥手自兹去,萧萧班马鸣。
    \end{tabular}
  \end{table}
\end{minipage}
\vspace{1cm}


\ptitle{长干行·其一}\nopagebreak%
\addcontentsline{toc}{section}{\texorpdfstring{\makebox[10cm]{长干行·其一\dotfill{} 【唐】李白}}{长干行·其一\ 【唐】李白}}\nopagebreak%
\noindent\begin{minipage}{\linewidth}
  \pauthor{【唐】李白}
  \vskip-3pt\begin{table}[H]
    \centering
    \begin{tabular}{@{}l@{}}
妾发初覆额,折花门前剧。郎骑竹马来,绕床弄青梅。\\
同居长干里,两小无嫌猜,十四为君妇,羞颜未尝开。\\
低头向暗壁,千唤不一回。十五始展眉,愿同尘与灰。\\
常存抱柱信,岂上望夫台。十六君远行,瞿塘滟滪堆。\\
五月不可触,猿声天上哀。门前迟行迹,一一生绿苔。\\
苔深不能扫,落叶秋风早。八月蝴蝶来,双飞西园草。\\
感此伤妾心,坐愁红颜老。早晚下三巴,预将书报家。\\
相迎不道远,直至长风沙。
    \end{tabular}
  \end{table}
\end{minipage}
\vspace{1cm}


\ptitle{闻王昌龄左迁龙标遥有此寄}\nopagebreak%
\addcontentsline{toc}{section}{\texorpdfstring{\makebox[10cm]{闻王昌龄左迁龙标遥有此寄\dotfill{} 【唐】李白}}{闻王昌龄左迁龙标遥有此寄\ 【唐】李白}}\nopagebreak%
\noindent\begin{minipage}{\linewidth}
  \pauthor{【唐】李白}
  \vskip-3pt\begin{table}[H]
    \centering
    \begin{tabular}{@{}l@{}}
杨花落尽子规啼,闻道龙标过五溪。\\
我寄愁心与明月,随风直到夜郎西。
    \end{tabular}
  \end{table}
\end{minipage}
\vspace{1cm}


\ptitle{静夜思}\nopagebreak%
\addcontentsline{toc}{section}{\texorpdfstring{\makebox[10cm]{静夜思\dotfill{} 【唐】李白}}{静夜思\ 【唐】李白}}\nopagebreak%
\noindent\begin{minipage}{\linewidth}
  \pauthor{【唐】李白}
  \vskip-3pt\begin{table}[H]
    \centering
    \begin{tabular}{@{}l@{}}
床前明月光,疑是地上霜。\\
举头望明月,低头思故乡。
    \end{tabular}
  \end{table}
\end{minipage}
\vspace{1cm}


\ptitle{驾去温泉后赠杨山人}\nopagebreak%
\addcontentsline{toc}{section}{\texorpdfstring{\makebox[10cm]{驾去温泉后赠杨山人\dotfill{} 【唐】李白}}{驾去温泉后赠杨山人\ 【唐】李白}}\nopagebreak%
\noindent\begin{minipage}{\linewidth}
  \pauthor{【唐】李白}
  \vskip-3pt\begin{table}[H]
    \centering
    \begin{tabular}{@{}l@{}}
少年落魄楚汉间,风尘萧瑟多苦颜。自言管葛竟谁许,长吁莫错还闭关。\\
一朝君王垂拂拭,剖心输丹雪胸臆。忽蒙白日回景光,直上青云生羽翼。\\
幸陪鸾辇出鸿都,身骑飞龙天马驹。王公大人借颜色,金璋紫绶来相趋。\\
当时结交何纷纷,片言道合惟有君。待吾尽节报明主,然后相携卧白云。
    \end{tabular}
  \end{table}
\end{minipage}
\vspace{1cm}


\ptitle{黄鹤楼送孟浩然之广陵}\nopagebreak%
\addcontentsline{toc}{section}{\texorpdfstring{\makebox[10cm]{黄鹤楼送孟浩然之广陵\dotfill{} 【唐】李白}}{黄鹤楼送孟浩然之广陵\ 【唐】李白}}\nopagebreak%
\noindent\begin{minipage}{\linewidth}
  \pauthor{【唐】李白}
  \vskip-3pt\begin{table}[H]
    \centering
    \begin{tabular}{@{}l@{}}
故人西辞黄鹤楼,烟花三月下扬州。\\
孤帆远影碧空尽,唯见长江天际流。
    \end{tabular}
  \end{table}
\end{minipage}
\vspace{1cm}


\ptitle{悯农 其一}\nopagebreak%
\addcontentsline{toc}{section}{\texorpdfstring{\makebox[10cm]{悯农 其一\dotfill{} 【唐】李绅}}{悯农 其一\ 【唐】李绅}}\nopagebreak%
\noindent\begin{minipage}{\linewidth}
  \pauthor{【唐】李绅}
  \vskip-3pt\begin{table}[H]
    \centering
    \begin{tabular}{@{}l@{}}
春种一粒\xpinyin*{\xpinyin{粟}{sù}},秋收万颗子。\\
四海无闲田,农夫犹饿死。
    \end{tabular}
  \end{table}
\end{minipage}
\vspace{1cm}


\ptitle{悯农 其二}\nopagebreak%
\addcontentsline{toc}{section}{\texorpdfstring{\makebox[10cm]{悯农 其二\dotfill{} 【唐】李绅}}{悯农 其二\ 【唐】李绅}}\nopagebreak%
\noindent\begin{minipage}{\linewidth}
  \pauthor{【唐】李绅}
  \vskip-3pt\begin{table}[H]
    \centering
    \begin{tabular}{@{}l@{}}
锄禾日当午,汗滴禾下土。\\
谁知盘中餐,粒粒皆辛苦。
    \end{tabular}
  \end{table}
\end{minipage}
\vspace{1cm}


\ptitle{李凭\xpinyin*{\xpinyin{箜}{kōng}}\xpinyin*{\xpinyin{篌}{hóu}}引}\nopagebreak%
\addcontentsline{toc}{section}{\texorpdfstring{\makebox[10cm]{李凭箜篌引\dotfill{} 【唐】李贺}}{李凭箜篌引\ 【唐】李贺}}\nopagebreak%
\noindent\begin{minipage}{\linewidth}
  \pauthor{【唐】李贺}
  \vskip-3pt\begin{table}[H]
    \centering
    \begin{tabular}{@{}l@{}}
吴丝蜀桐张高秋,空山凝云颓不流。江娥啼竹素女愁,李凭中国弹箜篌。\\
昆山玉碎凤凰叫,芙蓉泣露香兰笑。十二门前融冷光,二十三丝动紫皇。\\
女娲炼石补天处,石破天惊逗秋雨。梦入神山教神妪,老鱼跳波瘦蛟舞。\\
吴质不眠倚桂树,露脚斜飞湿寒兔。
    \end{tabular}
  \end{table}
\end{minipage}
\vspace{1cm}


\ptitle{雁门太守行}\nopagebreak%
\addcontentsline{toc}{section}{\texorpdfstring{\makebox[10cm]{雁门太守行\dotfill{} 【唐】李贺}}{雁门太守行\ 【唐】李贺}}\nopagebreak%
\noindent\begin{minipage}{\linewidth}
  \pauthor{【唐】李贺}
  \vskip-3pt\begin{table}[H]
    \centering
    \begin{tabular}{@{}l@{}}
黑云压城城欲摧,甲光向日金鳞开。角声满天秋色里,塞上燕脂凝夜紫。\\
半卷红旗临易水,霜重鼓寒声不起。报君黄金台上意,提携玉龙为君死。
    \end{tabular}
  \end{table}
\end{minipage}
\vspace{1cm}


\ptitle{寄扬州韩\xpinyin*{\xpinyin{绰}{chuò}}判官}\nopagebreak%
\addcontentsline{toc}{section}{\texorpdfstring{\makebox[10cm]{寄扬州韩绰判官\dotfill{} 【唐】杜牧}}{寄扬州韩绰判官\ 【唐】杜牧}}\nopagebreak%
\noindent\begin{minipage}{\linewidth}
  \pauthor{【唐】杜牧}
  \vskip-3pt\begin{table}[H]
    \centering
    \begin{tabular}{@{}l@{}}
青山隐隐水迢迢,秋尽江南草未凋。\\
二十四桥明月夜,玉人何处教吹箫?
    \end{tabular}
  \end{table}
\end{minipage}
\vspace{1cm}


\ptitle{山行}\nopagebreak%
\addcontentsline{toc}{section}{\texorpdfstring{\makebox[10cm]{山行\dotfill{} 【唐】杜牧}}{山行\ 【唐】杜牧}}\nopagebreak%
\noindent\begin{minipage}{\linewidth}
  \pauthor{【唐】杜牧}
  \vskip-3pt\begin{table}[H]
    \centering
    \begin{tabular}{@{}l@{}}
远上寒山石径斜,白云生处有人家。\\
停车坐爱枫林晚,霜叶红于二月花。
    \end{tabular}
  \end{table}
\end{minipage}
\vspace{1cm}


\ptitle{江南春}\nopagebreak%
\addcontentsline{toc}{section}{\texorpdfstring{\makebox[10cm]{江南春\dotfill{} 【唐】杜牧}}{江南春\ 【唐】杜牧}}\nopagebreak%
\noindent\begin{minipage}{\linewidth}
  \pauthor{【唐】杜牧}
  \vskip-3pt\begin{table}[H]
    \centering
    \begin{tabular}{@{}l@{}}
千里莺啼绿映红,水村山郭酒旗风。\\
南朝四百八十寺,多少楼台烟雨中。
    \end{tabular}
  \end{table}
\end{minipage}
\vspace{1cm}


\ptitle{泊秦淮}\nopagebreak%
\addcontentsline{toc}{section}{\texorpdfstring{\makebox[10cm]{泊秦淮\dotfill{} 【唐】杜牧}}{泊秦淮\ 【唐】杜牧}}\nopagebreak%
\noindent\begin{minipage}{\linewidth}
  \pauthor{【唐】杜牧}
  \vskip-3pt\begin{table}[H]
    \centering
    \begin{tabular}{@{}l@{}}
烟笼寒水月笼沙,夜泊秦淮近酒家。\\
商女不知亡国恨,隔江犹唱后庭花。
    \end{tabular}
  \end{table}
\end{minipage}
\vspace{1cm}


\ptitle{清明}\nopagebreak%
\addcontentsline{toc}{section}{\texorpdfstring{\makebox[10cm]{清明\dotfill{} 【唐】杜牧}}{清明\ 【唐】杜牧}}\nopagebreak%
\noindent\begin{minipage}{\linewidth}
  \pauthor{【唐】杜牧}
  \vskip-3pt\begin{table}[H]
    \centering
    \begin{tabular}{@{}l@{}}
清明时节雨纷纷,路上行人欲断魂。\\
借问酒家何处有?牧童遥指杏花村。
    \end{tabular}
  \end{table}
\end{minipage}
\vspace{1cm}


\ptitle{秋夕}\nopagebreak%
\addcontentsline{toc}{section}{\texorpdfstring{\makebox[10cm]{秋夕\dotfill{} 【唐】杜牧}}{秋夕\ 【唐】杜牧}}\nopagebreak%
\noindent\begin{minipage}{\linewidth}
  \pauthor{【唐】杜牧}
  \vskip-3pt\begin{table}[H]
    \centering
    \begin{tabular}{@{}l@{}}
银烛秋光冷画屏,轻罗小扇扑流萤。\\
天阶夜色凉如水,卧看牵牛织女星。
    \end{tabular}
  \end{table}
\end{minipage}
\vspace{1cm}


\ptitle{赠别}\nopagebreak%
\addcontentsline{toc}{section}{\texorpdfstring{\makebox[10cm]{赠别\dotfill{} 【唐】杜牧}}{赠别\ 【唐】杜牧}}\nopagebreak%
\noindent\begin{minipage}{\linewidth}
  \pauthor{【唐】杜牧}
  \vskip-3pt\begin{table}[H]
    \centering
    \begin{tabular}{@{}l@{}}
多情却似总无情,唯觉\xpinyin*{\xpinyin{樽}{zūn}}前笑不成。\\
蜡烛有心还惜别,替人垂泪到天明。
    \end{tabular}
  \end{table}
\end{minipage}
\vspace{1cm}


\ptitle{赤壁}\nopagebreak%
\addcontentsline{toc}{section}{\texorpdfstring{\makebox[10cm]{赤壁\dotfill{} 【唐】杜牧}}{赤壁\ 【唐】杜牧}}\nopagebreak%
\noindent\begin{minipage}{\linewidth}
  \pauthor{【唐】杜牧}
  \vskip-3pt\begin{table}[H]
    \centering
    \begin{tabular}{@{}l@{}}
折戟沉沙铁未销,自将磨洗认前朝。\\
东风不与周郎便,铜雀春深锁二乔。
    \end{tabular}
  \end{table}
\end{minipage}
\vspace{1cm}


\ptitle{遣怀}\nopagebreak%
\addcontentsline{toc}{section}{\texorpdfstring{\makebox[10cm]{遣怀\dotfill{} 【唐】杜牧}}{遣怀\ 【唐】杜牧}}\nopagebreak%
\noindent\begin{minipage}{\linewidth}
  \pauthor{【唐】杜牧}
  \vskip-3pt\begin{table}[H]
    \centering
    \begin{tabular}{@{}l@{}}
落魄江南载酒行,楚腰纤细掌中轻。\\
十年一觉扬州梦,赢得青楼薄幸名。
    \end{tabular}
  \end{table}
\end{minipage}
\vspace{1cm}


\ptitle{严中丞枉驾见过}\nopagebreak%
\addcontentsline{toc}{section}{\texorpdfstring{\makebox[10cm]{严中丞枉驾见过\dotfill{} 【唐】杜甫}}{严中丞枉驾见过\ 【唐】杜甫}}\nopagebreak%
\noindent\begin{minipage}{\linewidth}
  \pauthor{【唐】杜甫}
  \vskip-3pt\begin{table}[H]
    \centering
    \begin{tabular}{@{}l@{}}
元戎小队出郊坰,问柳寻花到野亭。\\
川合东西瞻使节,地分南北任流萍。\\
扁舟不独如张翰,白帽还应似管宁。\\
寂寞江天云雾里,何人道有少微星。
    \end{tabular}
  \end{table}
\end{minipage}
\vspace{1cm}


\ptitle{八阵图}\nopagebreak%
\addcontentsline{toc}{section}{\texorpdfstring{\makebox[10cm]{八阵图\dotfill{} 【唐】杜甫}}{八阵图\ 【唐】杜甫}}\nopagebreak%
\noindent\begin{minipage}{\linewidth}
  \pauthor{【唐】杜甫}
  \vskip-3pt\begin{table}[H]
    \centering
    \begin{tabular}{@{}l@{}}
功盖三分国,名成八阵图。\\
江流石不转,遗恨失吞吴。
    \end{tabular}
  \end{table}
\end{minipage}
\vspace{1cm}


\ptitle{可叹}\nopagebreak%
\addcontentsline{toc}{section}{\texorpdfstring{\makebox[10cm]{可叹\dotfill{} 【唐】杜甫}}{可叹\ 【唐】杜甫}}\nopagebreak%
\noindent\begin{minipage}{\linewidth}
  \pauthor{【唐】杜甫}
  \vskip-3pt\begin{table}[H]
    \centering
    \begin{tabular}{@{}l@{}}
天上浮云如白衣,斯须改变如苍狗。古往今来共一时,\\
人生万事无不有。近者抉眼去其夫,河东女儿身姓柳。\\
丈夫正色动引经,酆城客子王季友。群书万卷常暗诵,\\
孝经一通看在手。贫穷老瘦家卖屐,好事就之为携酒。\\
豫章太守高帝孙,引为宾客敬颇久。闻道三年未曾语,\\
小心恐惧闭其口。太守得之更不疑,人生反覆看亦丑。\\
明月无瑕岂容易,紫气郁郁犹冲斗。时危可仗真豪俊,\\
二人得置君侧否。太守顷者领山南,邦人思之比父母。\\
王生早曾拜颜色,高山之外皆培塿,用为羲和天为成,\\
用平水土地为厚。王也论道阻江湖,李也丞疑旷前后。\\
死为星辰终不灭,致君尧舜焉肯朽。吾辈碌碌饱饭行,\\
风后力牧长回首。
    \end{tabular}
  \end{table}
\end{minipage}
\vspace{1cm}


\ptitle{客至}\nopagebreak%
\addcontentsline{toc}{section}{\texorpdfstring{\makebox[10cm]{客至\dotfill{} 【唐】杜甫}}{客至\ 【唐】杜甫}}\nopagebreak%
\noindent\begin{minipage}{\linewidth}
  \pauthor{【唐】杜甫}
  \vskip-3pt\begin{table}[H]
    \centering
    \begin{tabular}{@{}l@{}}
舍南舍北皆春水,但见群鸥日日来。花径不曾缘客扫,蓬门今始为君开。\\
盘\xpinyin*{\xpinyin{飧}{sūn}}市远无兼味,樽酒家贫只旧\xpinyin*{\xpinyin{醅}{pēi}}。肯与邻翁相对饮,隔篱呼取尽余杯。
    \end{tabular}
  \end{table}
\end{minipage}
\vspace{1cm}


\ptitle{小寒食舟中作}\nopagebreak%
\addcontentsline{toc}{section}{\texorpdfstring{\makebox[10cm]{小寒食舟中作\dotfill{} 【唐】杜甫}}{小寒食舟中作\ 【唐】杜甫}}\nopagebreak%
\noindent\begin{minipage}{\linewidth}
  \pauthor{【唐】杜甫}
  \vskip-3pt\begin{table}[H]
    \centering
    \begin{tabular}{@{}l@{}}
佳辰强饮食犹寒,隐几萧条戴鹖冠。\\
春水船如天上坐,老年花似雾中看。\\
娟娟戏蝶过闲幔,片片轻鸥下急湍。\\
云白山青万余里,愁看直北是长安。
    \end{tabular}
  \end{table}
\end{minipage}
\vspace{1cm}


\ptitle{春夜喜雨}\nopagebreak%
\addcontentsline{toc}{section}{\texorpdfstring{\makebox[10cm]{春夜喜雨\dotfill{} 【唐】杜甫}}{春夜喜雨\ 【唐】杜甫}}\nopagebreak%
\noindent\begin{minipage}{\linewidth}
  \pauthor{【唐】杜甫}
  \vskip-3pt\begin{table}[H]
    \centering
    \begin{tabular}{@{}l@{}}
好雨知时节,当春乃发生。随风潜入夜,润物细无声。\\
野径云俱黑,江船火独明。晓看红湿处,花重锦官城。
    \end{tabular}
  \end{table}
\end{minipage}
\vspace{1cm}


\ptitle{春望}\nopagebreak%
\addcontentsline{toc}{section}{\texorpdfstring{\makebox[10cm]{春望\dotfill{} 【唐】杜甫}}{春望\ 【唐】杜甫}}\nopagebreak%
\noindent\begin{minipage}{\linewidth}
  \pauthor{【唐】杜甫}
  \vskip-3pt\begin{table}[H]
    \centering
    \begin{tabular}{@{}l@{}}
国破山河在,城春草木深。感时花溅泪,恨别鸟惊心。\\
烽火连三月,家书抵万金。白头搔更短,浑欲不胜\xpinyin*{\xpinyin{簪}{zān}}。
    \end{tabular}
  \end{table}
\end{minipage}
\vspace{1cm}


\ptitle{望岳}\nopagebreak%
\addcontentsline{toc}{section}{\texorpdfstring{\makebox[10cm]{望岳\dotfill{} 【唐】杜甫}}{望岳\ 【唐】杜甫}}\nopagebreak%
\noindent\begin{minipage}{\linewidth}
  \pauthor{【唐】杜甫}
  \vskip-3pt\begin{table}[H]
    \centering
    \begin{tabular}{@{}l@{}}
\xpinyin*{\xpinyin{岱}{dài}}宗夫如何?齐鲁青未了。造化钟神秀,阴阳割昏晓。\\
荡胸生曾云,决\xpinyin*{\xpinyin{眦}{zì}}入归鸟。会当凌绝顶,一览众山小。
    \end{tabular}
  \end{table}
\end{minipage}
\vspace{1cm}


\ptitle{江南逢李龟年}\nopagebreak%
\addcontentsline{toc}{section}{\texorpdfstring{\makebox[10cm]{江南逢李龟年\dotfill{} 【唐】杜甫}}{江南逢李龟年\ 【唐】杜甫}}\nopagebreak%
\noindent\begin{minipage}{\linewidth}
  \pauthor{【唐】杜甫}
  \vskip-3pt\begin{table}[H]
    \centering
    \begin{tabular}{@{}l@{}}
岐王宅里寻常见,崔九堂前几度闻。\\
正是江南好风景,落花时节又逢君。
    \end{tabular}
  \end{table}
\end{minipage}
\vspace{1cm}


\ptitle{江畔独步寻花}\nopagebreak%
\addcontentsline{toc}{section}{\texorpdfstring{\makebox[10cm]{江畔独步寻花\dotfill{} 【唐】杜甫}}{江畔独步寻花\ 【唐】杜甫}}\nopagebreak%
\noindent\begin{minipage}{\linewidth}
  \pauthor{【唐】杜甫}
  \vskip-3pt\begin{table}[H]
    \centering
    \begin{tabular}{@{}l@{}}
黄师塔前江水东,春光懒困倚微风。\\
桃花一簇开无主,可爱深红爱浅红?
    \end{tabular}
  \end{table}
\end{minipage}
\vspace{1cm}


\ptitle{江雨有怀郑典设}\nopagebreak%
\addcontentsline{toc}{section}{\texorpdfstring{\makebox[10cm]{江雨有怀郑典设\dotfill{} 【唐】杜甫}}{江雨有怀郑典设\ 【唐】杜甫}}\nopagebreak%
\noindent\begin{minipage}{\linewidth}
  \pauthor{【唐】杜甫}
  \vskip-3pt\begin{table}[H]
    \centering
    \begin{tabular}{@{}l@{}}
春雨暗暗塞峡中,早晚来自楚王宫。\\
乱波分披已打岸,弱云狼藉不禁风。\\
宠光蕙叶与多碧,点注桃花舒小红。\\
谷口子真正忆汝,岸高瀼滑限西东。
    \end{tabular}
  \end{table}
\end{minipage}
\vspace{1cm}


\ptitle{登岳阳楼}\nopagebreak%
\addcontentsline{toc}{section}{\texorpdfstring{\makebox[10cm]{登岳阳楼\dotfill{} 【唐】杜甫}}{登岳阳楼\ 【唐】杜甫}}\nopagebreak%
\noindent\begin{minipage}{\linewidth}
  \pauthor{【唐】杜甫}
  \vskip-3pt\begin{table}[H]
    \centering
    \begin{tabular}{@{}l@{}}
昔闻洞庭水,今上岳阳楼。吴楚东南\xpinyin*{\xpinyin{坼}{chè}},乾坤日夜浮。\\
亲朋无一字,老病有孤舟。戎马关山北,凭轩涕泗流。
    \end{tabular}
  \end{table}
\end{minipage}
\vspace{1cm}


\ptitle{登高}\nopagebreak%
\addcontentsline{toc}{section}{\texorpdfstring{\makebox[10cm]{登高\dotfill{} 【唐】杜甫}}{登高\ 【唐】杜甫}}\nopagebreak%
\noindent\begin{minipage}{\linewidth}
  \pauthor{【唐】杜甫}
  \vskip-3pt\begin{table}[H]
    \centering
    \begin{tabular}{@{}l@{}}
风急天高猿啸哀,\xpinyin*{\xpinyin{渚}{zhǔ}}清沙白鸟飞回。无边落木萧萧下,不尽长江滚滚来。\\
万里悲秋常作客,百年多病独登台。艰难苦恨繁霜鬓,潦倒新停浊酒杯。
    \end{tabular}
  \end{table}
\end{minipage}
\vspace{1cm}


\ptitle{绝句}\nopagebreak%
\addcontentsline{toc}{section}{\texorpdfstring{\makebox[10cm]{绝句\dotfill{} 【唐】杜甫}}{绝句\ 【唐】杜甫}}\nopagebreak%
\noindent\begin{minipage}{\linewidth}
  \pauthor{【唐】杜甫}
  \vskip-3pt\begin{table}[H]
    \centering
    \begin{tabular}{@{}l@{}}
两个黄鹂鸣翠柳,一行白鹭上青天。\\
窗含西岭千秋雪,门泊东吴万里船。
    \end{tabular}
  \end{table}
\end{minipage}
\vspace{1cm}


\ptitle{绝句}\nopagebreak%
\addcontentsline{toc}{section}{\texorpdfstring{\makebox[10cm]{绝句\dotfill{} 【唐】杜甫}}{绝句\ 【唐】杜甫}}\nopagebreak%
\noindent\begin{minipage}{\linewidth}
  \pauthor{【唐】杜甫}
  \vskip-3pt\begin{table}[H]
    \centering
    \begin{tabular}{@{}l@{}}
迟日江山丽,春风花草香。\\
泥融飞燕子,沙暖睡鸳鸯。
    \end{tabular}
  \end{table}
\end{minipage}
\vspace{1cm}


\ptitle{茅屋为秋风所破歌}\nopagebreak%
\addcontentsline{toc}{section}{\texorpdfstring{\makebox[10cm]{茅屋为秋风所破歌\dotfill{} 【唐】杜甫}}{茅屋为秋风所破歌\ 【唐】杜甫}}\nopagebreak%
\noindent\begin{minipage}{\linewidth}
  \pauthor{【唐】杜甫}
  \vskip-3pt\begin{table}[H]
    \centering
    \begin{tabular}{@{}l@{}}
八月秋高风怒号,卷我屋上三重茅。\\
茅飞渡江洒江郊,高者挂罥长林梢,下者飘转沉塘\xpinyin*{\xpinyin{坳}{ào}}。\\
南村群童欺我老无力,忍能对面为盗贼。\\
公然抱茅入竹去,唇焦口燥呼不得,归来倚杖自叹息。\\
俄顷风定云墨色,秋天漠漠向昏黑。\\
布\xpinyin*{\xpinyin{衾}{qīn}}多年冷似铁,骄儿恶卧踏里裂。\\
床头屋漏无干处,雨脚如麻未断绝。\\
自经丧乱少睡眠,长夜沾湿何由彻!\\
安得广厦千万间,大庇天下寒士俱欢颜,风雨不动安如山!\\
呜呼!何时眼前突兀见此屋,吾庐独破受冻死亦足!
    \end{tabular}
  \end{table}
\end{minipage}
\vspace{1cm}


\ptitle{蜀相}\nopagebreak%
\addcontentsline{toc}{section}{\texorpdfstring{\makebox[10cm]{蜀相\dotfill{} 【唐】杜甫}}{蜀相\ 【唐】杜甫}}\nopagebreak%
\noindent\begin{minipage}{\linewidth}
  \pauthor{【唐】杜甫}
  \vskip-3pt\begin{table}[H]
    \centering
    \begin{tabular}{@{}l@{}}
丞相祠堂何处寻?锦官城外柏森森。映阶碧草自春色,隔叶黄鹂空好音。\\
三顾频烦天下计,两朝开济老臣心。出师未捷身先死,长使英雄泪满襟。
    \end{tabular}
  \end{table}
\end{minipage}
\vspace{1cm}


\ptitle{赠花卿}\nopagebreak%
\addcontentsline{toc}{section}{\texorpdfstring{\makebox[10cm]{赠花卿\dotfill{} 【唐】杜甫}}{赠花卿\ 【唐】杜甫}}\nopagebreak%
\noindent\begin{minipage}{\linewidth}
  \pauthor{【唐】杜甫}
  \vskip-3pt\begin{table}[H]
    \centering
    \begin{tabular}{@{}l@{}}
锦城丝管日纷纷,半入江风半入云。\\
此曲只应天上有,人间能得几回闻。
    \end{tabular}
  \end{table}
\end{minipage}
\vspace{1cm}


\ptitle{醉时歌·赠广文馆博士郑虔}\nopagebreak%
\addcontentsline{toc}{section}{\texorpdfstring{\makebox[10cm]{醉时歌·赠广文馆博士郑虔\dotfill{} 【唐】杜甫}}{醉时歌·赠广文馆博士郑虔\ 【唐】杜甫}}\nopagebreak%
\noindent\begin{minipage}{\linewidth}
  \pauthor{【唐】杜甫}
  \vskip-3pt\begin{table}[H]
    \centering
    \begin{tabular}{@{}l@{}}
诸公衮衮登台省,广文先生官独冷。甲第纷纷厌粱肉,\\
广文先生饭不足。先生有道出羲皇,先生有才过屈宋。\\
德尊一代常轗轲,名垂万古知何用。杜陵野客人更嗤,\\
被褐短窄鬓如丝。日籴太仓五升米,时赴郑老同襟期。\\
得钱即相觅,沽酒不复疑。忘形到尔汝,痛饮真吾师。\\
清夜沈沈动春酌,灯前细雨檐花落。但觉高歌有鬼神,\\
焉知饿死填沟壑。相如逸才亲涤器,子云识字终投阁。\\
先生早赋归去来,石田茅屋荒苍苔。儒术于我何有哉,\\
孔丘盗跖俱尘埃。不须闻此意惨怆,生前相遇且衔杯。
    \end{tabular}
  \end{table}
\end{minipage}
\vspace{1cm}


\ptitle{闻官军收河南河北}\nopagebreak%
\addcontentsline{toc}{section}{\texorpdfstring{\makebox[10cm]{闻官军收河南河北\dotfill{} 【唐】杜甫}}{闻官军收河南河北\ 【唐】杜甫}}\nopagebreak%
\noindent\begin{minipage}{\linewidth}
  \pauthor{【唐】杜甫}
  \vskip-3pt\begin{table}[H]
    \centering
    \begin{tabular}{@{}l@{}}
剑外忽传收蓟北,初闻涕泪满衣裳。\\
却看妻子愁何在,漫卷诗书喜欲狂。\\
白日放歌须纵酒,青春作伴好还乡。\\
即从巴峡穿巫峡,便下襄阳向洛阳。
    \end{tabular}
  \end{table}
\end{minipage}
\vspace{1cm}


\ptitle{城东早春}\nopagebreak%
\addcontentsline{toc}{section}{\texorpdfstring{\makebox[10cm]{城东早春\dotfill{} 【唐】杨巨源}}{城东早春\ 【唐】杨巨源}}\nopagebreak%
\noindent\begin{minipage}{\linewidth}
  \pauthor{【唐】杨巨源}
  \vskip-3pt\begin{table}[H]
    \centering
    \begin{tabular}{@{}l@{}}
诗家清景在新春,绿柳才黄半未匀。\\
若待上林花似锦,出门俱是看花人。
    \end{tabular}
  \end{table}
\end{minipage}
\vspace{1cm}


\ptitle{江雪}\nopagebreak%
\addcontentsline{toc}{section}{\texorpdfstring{\makebox[10cm]{江雪\dotfill{} 【唐】柳宗元}}{江雪\ 【唐】柳宗元}}\nopagebreak%
\noindent\begin{minipage}{\linewidth}
  \pauthor{【唐】柳宗元}
  \vskip-3pt\begin{table}[H]
    \centering
    \begin{tabular}{@{}l@{}}
千山鸟飞绝,万径人踪灭。\\
孤舟蓑笠翁,独钓寒江雪。
    \end{tabular}
  \end{table}
\end{minipage}
\vspace{1cm}


\ptitle{菩萨蛮}\nopagebreak%
\addcontentsline{toc}{section}{\texorpdfstring{\makebox[10cm]{菩萨蛮\dotfill{} 【唐】温庭筠}}{菩萨蛮\ 【唐】温庭筠}}\nopagebreak%
\noindent\begin{minipage}{\linewidth}
  \pauthor{【唐】温庭筠}
  \vskip-3pt\begin{table}[H]
    \centering
    \begin{tabular}{@{}l@{}}
小山重叠金明灭,鬓云欲度香腮雪。懒起画蛾眉,弄妆梳洗迟。\\
\\
照花前后镜,花面交相映。新帖绣罗襦,双双金鹧鸪。
    \end{tabular}
  \end{table}
\end{minipage}
\vspace{1cm}


\ptitle{凉州词}\nopagebreak%
\addcontentsline{toc}{section}{\texorpdfstring{\makebox[10cm]{凉州词\dotfill{} 【唐】王之涣}}{凉州词\ 【唐】王之涣}}\nopagebreak%
\noindent\begin{minipage}{\linewidth}
  \pauthor{【唐】王之涣}
  \vskip-3pt\begin{table}[H]
    \centering
    \begin{tabular}{@{}l@{}}
黄河远上白云间,一片孤城万\xpinyin*{\xpinyin{仞}{rèn}}山。\\
\xpinyin*{\xpinyin{羌}{qiāng}}笛何须怨杨柳,春风不度玉门关。
    \end{tabular}
  \end{table}
\end{minipage}
\vspace{1cm}


\ptitle{登\xpinyin*{\xpinyin{鹳}{guàn}}雀楼}\nopagebreak%
\addcontentsline{toc}{section}{\texorpdfstring{\makebox[10cm]{登鹳雀楼\dotfill{} 【唐】王之涣}}{登鹳雀楼\ 【唐】王之涣}}\nopagebreak%
\noindent\begin{minipage}{\linewidth}
  \pauthor{【唐】王之涣}
  \vskip-3pt\begin{table}[H]
    \centering
    \begin{tabular}{@{}l@{}}
白日依山尽,黄河入海流。\\
欲穷千里目,更上一层楼。
    \end{tabular}
  \end{table}
\end{minipage}
\vspace{1cm}


\ptitle{送杜少府之任蜀州}\nopagebreak%
\addcontentsline{toc}{section}{\texorpdfstring{\makebox[10cm]{送杜少府之任蜀州\dotfill{} 【唐】王勃}}{送杜少府之任蜀州\ 【唐】王勃}}\nopagebreak%
\noindent\begin{minipage}{\linewidth}
  \pauthor{【唐】王勃}
  \vskip-3pt\begin{table}[H]
    \centering
    \begin{tabular}{@{}l@{}}
城阙辅三秦,风烟望五津。与君离别意,同是宦游人。\\
海内存知己,天涯若比邻。无为在歧路,儿女共沾巾。
    \end{tabular}
  \end{table}
\end{minipage}
\vspace{1cm}


\ptitle{出塞(其一)}\nopagebreak%
\addcontentsline{toc}{section}{\texorpdfstring{\makebox[10cm]{出塞(其一)\dotfill{} 【唐】王昌龄}}{出塞(其一)\ 【唐】王昌龄}}\nopagebreak%
\noindent\begin{minipage}{\linewidth}
  \pauthor{【唐】王昌龄}
  \vskip-3pt\begin{table}[H]
    \centering
    \begin{tabular}{@{}l@{}}
秦时明月汉时关,万里长征人未还。\\
但使龙城飞将在,不教胡马度阴山。
    \end{tabular}
  \end{table}
\end{minipage}
\vspace{1cm}


\ptitle{芙蓉楼送辛渐}\nopagebreak%
\addcontentsline{toc}{section}{\texorpdfstring{\makebox[10cm]{芙蓉楼送辛渐\dotfill{} 【唐】王昌龄}}{芙蓉楼送辛渐\ 【唐】王昌龄}}\nopagebreak%
\noindent\begin{minipage}{\linewidth}
  \pauthor{【唐】王昌龄}
  \vskip-3pt\begin{table}[H]
    \centering
    \begin{tabular}{@{}l@{}}
寒雨连江夜入吴,平明送客楚山孤。\\
洛阳亲友如相问,一片冰心在玉壶。
    \end{tabular}
  \end{table}
\end{minipage}
\vspace{1cm}


\ptitle{闺怨}\nopagebreak%
\addcontentsline{toc}{section}{\texorpdfstring{\makebox[10cm]{闺怨\dotfill{} 【唐】王昌龄}}{闺怨\ 【唐】王昌龄}}\nopagebreak%
\noindent\begin{minipage}{\linewidth}
  \pauthor{【唐】王昌龄}
  \vskip-3pt\begin{table}[H]
    \centering
    \begin{tabular}{@{}l@{}}
闺中少妇不知愁,春日凝妆上翠楼。\\
忽见陌头杨柳色,悔教夫婿觅封侯。
    \end{tabular}
  \end{table}
\end{minipage}
\vspace{1cm}


\ptitle{次北固山下}\nopagebreak%
\addcontentsline{toc}{section}{\texorpdfstring{\makebox[10cm]{次北固山下\dotfill{} 【唐】王湾}}{次北固山下\ 【唐】王湾}}\nopagebreak%
\noindent\begin{minipage}{\linewidth}
  \pauthor{【唐】王湾}
  \vskip-3pt\begin{table}[H]
    \centering
    \begin{tabular}{@{}l@{}}
客路青山外,行舟绿水前。潮平两岸阔,风正一帆悬。\\
海日生残夜,江春入旧年。乡书何处达?归雁洛阳边。
    \end{tabular}
  \end{table}
\end{minipage}
\vspace{1cm}


\ptitle{九月九日忆山东兄弟}\nopagebreak%
\addcontentsline{toc}{section}{\texorpdfstring{\makebox[10cm]{九月九日忆山东兄弟\dotfill{} 【唐】王维}}{九月九日忆山东兄弟\ 【唐】王维}}\nopagebreak%
\noindent\begin{minipage}{\linewidth}
  \pauthor{【唐】王维}
  \vskip-3pt\begin{table}[H]
    \centering
    \begin{tabular}{@{}l@{}}
独在异乡为异客,每逢佳节倍思亲。\\
遥知兄弟登高处,遍插\xpinyin*{\xpinyin{茱}{zhū}}\xpinyin*{\xpinyin{萸}{yú}}少一人。
    \end{tabular}
  \end{table}
\end{minipage}
\vspace{1cm}


\ptitle{使至塞上}\nopagebreak%
\addcontentsline{toc}{section}{\texorpdfstring{\makebox[10cm]{使至塞上\dotfill{} 【唐】王维}}{使至塞上\ 【唐】王维}}\nopagebreak%
\noindent\begin{minipage}{\linewidth}
  \pauthor{【唐】王维}
  \vskip-3pt\begin{table}[H]
    \centering
    \begin{tabular}{@{}l@{}}
单车欲问边,属国过居延。征蓬出汉塞,归雁入胡天。\\
大漠孤烟直,长河落日圆。萧关逢候骑,都护在燕然。
    \end{tabular}
  \end{table}
\end{minipage}
\vspace{1cm}


\ptitle{山居秋暝}\nopagebreak%
\addcontentsline{toc}{section}{\texorpdfstring{\makebox[10cm]{山居秋暝\dotfill{} 【唐】王维}}{山居秋暝\ 【唐】王维}}\nopagebreak%
\noindent\begin{minipage}{\linewidth}
  \pauthor{【唐】王维}
  \vskip-3pt\begin{table}[H]
    \centering
    \begin{tabular}{@{}l@{}}
空山新雨后,天气晚来秋。明月松间照,清泉石上流。\\
竹喧归\xpinyin*{\xpinyin{浣}{huàn}}女,莲动下渔舟。随意春芳歇,王孙自可留。
    \end{tabular}
  \end{table}
\end{minipage}
\vspace{1cm}


\ptitle{杂诗}\nopagebreak%
\addcontentsline{toc}{section}{\texorpdfstring{\makebox[10cm]{杂诗\dotfill{} 【唐】王维}}{杂诗\ 【唐】王维}}\nopagebreak%
\noindent\begin{minipage}{\linewidth}
  \pauthor{【唐】王维}
  \vskip-3pt\begin{table}[H]
    \centering
    \begin{tabular}{@{}l@{}}
君自故乡来,应知故乡事。\\
来日\xpinyin*{\xpinyin{绮}{qǐ}}窗前,寒梅著花未?
    \end{tabular}
  \end{table}
\end{minipage}
\vspace{1cm}


\ptitle{杂诗(其二)}\nopagebreak%
\addcontentsline{toc}{section}{\texorpdfstring{\makebox[10cm]{杂诗(其二)\dotfill{} 【唐】王维}}{杂诗(其二)\ 【唐】王维}}\nopagebreak%
\noindent\begin{minipage}{\linewidth}
  \pauthor{【唐】王维}
  \vskip-3pt\begin{table}[H]
    \centering
    \begin{tabular}{@{}l@{}}
君自故乡来,应知故乡事。\\
来日绮窗前,寒梅著花未?
    \end{tabular}
  \end{table}
\end{minipage}
\vspace{1cm}


\ptitle{相思}\nopagebreak%
\addcontentsline{toc}{section}{\texorpdfstring{\makebox[10cm]{相思\dotfill{} 【唐】王维}}{相思\ 【唐】王维}}\nopagebreak%
\noindent\begin{minipage}{\linewidth}
  \pauthor{【唐】王维}
  \vskip-3pt\begin{table}[H]
    \centering
    \begin{tabular}{@{}l@{}}
红豆生南国,春来发几枝。\\
愿君多采\xpinyin*{\xpinyin{撷}{xié}},此物最相思。
    \end{tabular}
  \end{table}
\end{minipage}
\vspace{1cm}


\ptitle{竹里馆}\nopagebreak%
\addcontentsline{toc}{section}{\texorpdfstring{\makebox[10cm]{竹里馆\dotfill{} 【唐】王维}}{竹里馆\ 【唐】王维}}\nopagebreak%
\noindent\begin{minipage}{\linewidth}
  \pauthor{【唐】王维}
  \vskip-3pt\begin{table}[H]
    \centering
    \begin{tabular}{@{}l@{}}
独坐幽篁里,弹琴复长啸。\\
深林人不知,明月来相照。
    \end{tabular}
  \end{table}
\end{minipage}
\vspace{1cm}


\ptitle{送元二使安西}\nopagebreak%
\addcontentsline{toc}{section}{\texorpdfstring{\makebox[10cm]{送元二使安西\dotfill{} 【唐】王维}}{送元二使安西\ 【唐】王维}}\nopagebreak%
\noindent\begin{minipage}{\linewidth}
  \pauthor{【唐】王维}
  \vskip-3pt\begin{table}[H]
    \centering
    \begin{tabular}{@{}l@{}}
渭城朝雨\xpinyin*{\xpinyin{浥}{yì}}轻尘,客舍青青柳色新。\\
劝君更尽一杯酒,西出阳关无故人。
    \end{tabular}
  \end{table}
\end{minipage}
\vspace{1cm}


\ptitle{鹿\xpinyin*{\xpinyin{柴}{zhài}}}\nopagebreak%
\addcontentsline{toc}{section}{\texorpdfstring{\makebox[10cm]{鹿柴\dotfill{} 【唐】王维}}{鹿柴\ 【唐】王维}}\nopagebreak%
\noindent\begin{minipage}{\linewidth}
  \pauthor{【唐】王维}
  \vskip-3pt\begin{table}[H]
    \centering
    \begin{tabular}{@{}l@{}}
空山不见人,但闻人语响。\\
返景入深林,复照青苔上。
    \end{tabular}
  \end{table}
\end{minipage}
\vspace{1cm}


\ptitle{凉州词}\nopagebreak%
\addcontentsline{toc}{section}{\texorpdfstring{\makebox[10cm]{凉州词\dotfill{} 【唐】王翰}}{凉州词\ 【唐】王翰}}\nopagebreak%
\noindent\begin{minipage}{\linewidth}
  \pauthor{【唐】王翰}
  \vskip-3pt\begin{table}[H]
    \centering
    \begin{tabular}{@{}l@{}}
葡萄美酒夜光杯,欲饮琵琶马上催。\\
醉卧沙场君莫笑,古来征战几人回?
    \end{tabular}
  \end{table}
\end{minipage}
\vspace{1cm}


\ptitle{二月二日}\nopagebreak%
\addcontentsline{toc}{section}{\texorpdfstring{\makebox[10cm]{二月二日\dotfill{} 【唐】白居易}}{二月二日\ 【唐】白居易}}\nopagebreak%
\noindent\begin{minipage}{\linewidth}
  \pauthor{【唐】白居易}
  \vskip-3pt\begin{table}[H]
    \centering
    \begin{tabular}{@{}l@{}}
二月二日新雨晴,草芽菜甲一时生。\\
轻衫细马春年少,十字津头一字行。
    \end{tabular}
  \end{table}
\end{minipage}
\vspace{1cm}


\ptitle{卖炭翁}\nopagebreak%
\addcontentsline{toc}{section}{\texorpdfstring{\makebox[10cm]{卖炭翁\dotfill{} 【唐】白居易}}{卖炭翁\ 【唐】白居易}}\nopagebreak%
\noindent\begin{minipage}{\linewidth}
  \pauthor{【唐】白居易}
  \vskip-3pt\begin{table}[H]
    \centering
    \begin{tabular}{@{}l@{}}
    卖炭翁,伐薪烧炭南山中。满面尘灰烟火色,两鬓苍苍十指黑。\\
卖炭得钱何所营?身上衣裳口中食。可怜身上衣正单,心忧炭贱愿天寒。\\
夜来城外一尺雪,晓驾炭车辗冰辙。牛困人饥日已高,市南门外泥中歇。\\
\\
翩翩两骑来是谁?黄衣使者白衫儿。手把文书口称\xpinyin*{\xpinyin{敕}{chì}},回车叱牛牵向北。\\
一车炭,千余斤,宫使驱将惜不得。半匹红绡一丈绫,系向牛头充炭直。
    \end{tabular}
  \end{table}
\end{minipage}
\vspace{1cm}


\ptitle{忆江南}\nopagebreak%
\addcontentsline{toc}{section}{\texorpdfstring{\makebox[10cm]{忆江南\dotfill{} 【唐】白居易}}{忆江南\ 【唐】白居易}}\nopagebreak%
\noindent\begin{minipage}{\linewidth}
  \pauthor{【唐】白居易}
  \vskip-3pt\begin{table}[H]
    \centering
    \begin{tabular}{@{}l@{}}
江南好,风景旧曾\xpinyin*{\xpinyin{谙}{ān}}。\\
日出江花红胜火,春来江水绿如蓝。\\
能不忆江南?
    \end{tabular}
  \end{table}
\end{minipage}
\vspace{1cm}


\ptitle{李白墓}\nopagebreak%
\addcontentsline{toc}{section}{\texorpdfstring{\makebox[10cm]{李白墓\dotfill{} 【唐】白居易}}{李白墓\ 【唐】白居易}}\nopagebreak%
\noindent\begin{minipage}{\linewidth}
  \pauthor{【唐】白居易}
  \vskip-3pt\begin{table}[H]
    \centering
    \begin{tabular}{@{}l@{}}
采石江边李白坟,绕田无限草连云。\\
可怜荒垄穷泉骨,曾有惊天动地文。\\
但是诗人多薄命,就中沦落不过君。
    \end{tabular}
  \end{table}
\end{minipage}
\vspace{1cm}


\ptitle{池上}\nopagebreak%
\addcontentsline{toc}{section}{\texorpdfstring{\makebox[10cm]{池上\dotfill{} 【唐】白居易}}{池上\ 【唐】白居易}}\nopagebreak%
\noindent\begin{minipage}{\linewidth}
  \pauthor{【唐】白居易}
  \vskip-3pt\begin{table}[H]
    \centering
    \begin{tabular}{@{}l@{}}
小娃撑小艇,偷采白莲回。\\
不解藏踪迹,浮萍一道开。
    \end{tabular}
  \end{table}
\end{minipage}
\vspace{1cm}


\ptitle{琵琶行(并序)}\nopagebreak%
\addcontentsline{toc}{section}{\texorpdfstring{\makebox[10cm]{琵琶行(并序)\dotfill{} 【唐】白居易}}{琵琶行(并序)\ 【唐】白居易}}\nopagebreak%
\noindent\begin{minipage}{\linewidth}
  \pauthor{【唐】白居易}
  \vskip-3pt\begin{table}[H]
    \centering
    \begin{tabular}{@{}l@{}}
浔阳江头夜送客,枫叶荻花秋瑟瑟。主人下马客在船,举酒欲饮无管弦。\\
醉不成欢惨将别,别时茫茫江浸月。忽闻水上琵琶声,主人忘归客不发。\\
寻声暗问弹者谁?琵琶声停欲语迟。移船相近邀相见,添酒回灯重开宴。\\
千呼万唤始出来,犹抱琵琶半遮面。转轴拨弦三两声,未成曲调先有情。\\
弦弦掩抑声声思,似诉平生不得志。低眉信手续续弹,说尽心中无限事。\\
轻拢慢捻抹复挑,初为霓裳后六幺。大弦嘈嘈如急雨,小弦切切如私语。\\
嘈嘈切切错杂弹,大珠小珠落玉盘。间关莺语花底滑,幽咽泉流冰下难。\\
冰泉冷涩弦凝绝,凝绝不通声渐歇。别有幽愁暗恨生,此时无声胜有声。\\
银瓶乍破水浆迸,铁骑突出刀枪鸣。曲终收拨当心画,四弦一声如裂帛。\\
东船西舫悄无言,唯见江心秋月白。沉吟放拨插弦中,整顿衣裳起敛容。\\
自言本是京城女,家在虾蟆陵下住。十三学得琵琶成,名属教坊第一部。\\
曲罢曾教善才服,妆成每被秋娘妒。五陵年少争缠头,一曲红绡不知数。\\
钿头银篦击节碎,血色罗裙翻酒污。今年欢笑复明年,秋月春风等闲度。\\
弟走从军阿姨死,暮去朝来颜色故。门前冷落鞍马稀,老大嫁作商人妇。\\
商人重利轻别离,前月浮梁买茶去。去来江口守空船,绕船月明江水寒。\\
夜深忽梦少年事,梦啼妆泪红阑干。我闻琵琶已叹息,又闻此语重唧唧。\\
同是天涯沦落人,相逢何必曾相识!我从去年辞帝京,谪居卧病浔阳城。\\
浔阳地僻无音乐,终岁不闻丝竹声。住近湓江地低湿,黄芦苦竹绕宅生。\\
其间旦暮闻何物?杜鹃啼血猿哀鸣。春江花朝秋月夜,往往取酒还独倾。\\
岂无山歌与村笛?呕哑嘲哳难为听。今夜闻君琵琶语,如听仙乐耳暂明。\\
莫辞更坐弹一曲,为君翻作琵琶行。感我此言良久立,却坐促弦弦转急。\\
凄凄不似向前声,满座重闻皆掩泣。座中泣下谁最多?江州司马青衫湿。
    \end{tabular}
  \end{table}
\end{minipage}
\vspace{1cm}


\ptitle{菩提寺上方晚眺}\nopagebreak%
\addcontentsline{toc}{section}{\texorpdfstring{\makebox[10cm]{菩提寺上方晚眺\dotfill{} 【唐】白居易}}{菩提寺上方晚眺\ 【唐】白居易}}\nopagebreak%
\noindent\begin{minipage}{\linewidth}
  \pauthor{【唐】白居易}
  \vskip-3pt\begin{table}[H]
    \centering
    \begin{tabular}{@{}l@{}}
楼阁高低树浅深,山光水色暝沉沉。嵩烟半卷青绡幕,伊浪平铺绿绮衾。\\
飞鸟灭时宜极目,远风来处好开襟。谁知不离簪缨内,长得逍遥自在心。
    \end{tabular}
  \end{table}
\end{minipage}
\vspace{1cm}


\ptitle{赋得古原草送別}\nopagebreak%
\addcontentsline{toc}{section}{\texorpdfstring{\makebox[10cm]{赋得古原草送別\dotfill{} 【唐】白居易}}{赋得古原草送別\ 【唐】白居易}}\nopagebreak%
\noindent\begin{minipage}{\linewidth}
  \pauthor{【唐】白居易}
  \vskip-3pt\begin{table}[H]
    \centering
    \begin{tabular}{@{}l@{}}
离离原上草,一岁一枯荣。野火烧不尽,春风吹又生。\\
远芳侵古道,晴翠接荒城。又送王孙去,\xpinyin*{\xpinyin{萋}{qī}}萋满别情。
    \end{tabular}
  \end{table}
\end{minipage}
\vspace{1cm}


\ptitle{钱塘湖春行}\nopagebreak%
\addcontentsline{toc}{section}{\texorpdfstring{\makebox[10cm]{钱塘湖春行\dotfill{} 【唐】白居易}}{钱塘湖春行\ 【唐】白居易}}\nopagebreak%
\noindent\begin{minipage}{\linewidth}
  \pauthor{【唐】白居易}
  \vskip-3pt\begin{table}[H]
    \centering
    \begin{tabular}{@{}l@{}}
孤山寺北贾亭西,水面初平云脚低。几处早莺争暖树,谁家新燕啄春泥。\\
乱花渐欲迷人眼,浅草才能没马蹄。最爱湖东行不足,绿杨阴里白沙堤。
    \end{tabular}
  \end{table}
\end{minipage}
\vspace{1cm}


\ptitle{长恨歌}\nopagebreak%
\addcontentsline{toc}{section}{\texorpdfstring{\makebox[10cm]{长恨歌\dotfill{} 【唐】白居易}}{长恨歌\ 【唐】白居易}}\nopagebreak%
\noindent\begin{minipage}{\linewidth}
  \pauthor{【唐】白居易}
  \vskip-3pt\begin{table}[H]
    \centering
    \begin{tabular}{@{}l@{}}
汉皇重色思倾国,御宇多年求不得。杨家有女初长成,养在深闺人未识。\\
天生丽质难自弃,一朝选在君王侧。回眸一笑百媚生,六宫粉黛无颜色。\\
春寒赐浴华清池,温泉水滑洗凝脂。侍儿扶起娇无力,始是新承恩泽时。\\
云鬓花颜金步摇,芙蓉帐暖度春宵。春宵苦短日高起,从此君王不早朝。\\
承欢侍宴无闲暇,春从春游夜专夜。后宫佳丽三千人,三千宠爱在一身。\\
金屋妆成娇侍夜,玉楼宴罢醉和春。姊妹弟兄皆列土,可怜光彩生门户。\\
遂令天下父母心,不重生男重生女。骊宫高处入青云,仙乐风飘处处闻。\\
缓歌慢舞凝丝竹,尽日君王看不足。渔阳鼙鼓动地来,惊破霓裳羽衣曲。\\
九重城阙烟尘生,千乘万骑西南行。翠华摇摇行复止,西出都门百余里。\\
六军不发无奈何,宛转蛾眉马前死。花钿委地无人收,翠翘金雀玉搔头。\\
君王掩面救不得,回看血泪相和流。黄埃散漫风萧索,云栈萦纡登剑阁。\\
峨嵋山下少人行,旌旗无光日色薄。蜀江水碧蜀山青,圣主朝朝暮暮情。\\
行宫见月伤心色,夜雨闻铃肠断声。天旋地转回龙驭,到此踌躇不能去。\\
马嵬坡下泥土中,不见玉颜空死处。君臣相顾尽沾衣,东望都门信马归。\\
归来池苑皆依旧,太液芙蓉未央柳。芙蓉如面柳如眉,对此如何不泪垂。\\
春风桃李花开夜,秋雨梧桐叶落时。西宫南苑多秋草,落叶满阶红不扫。\\
梨园弟子白发新,椒房阿监青娥老。夕殿萤飞思悄然,孤灯挑尽未成眠。\\
迟迟钟鼓初长夜,耿耿星河欲曙天。鸳鸯瓦冷霜华重,翡翠衾寒谁与共。\\
悠悠生死别经年,魂魄不曾来入梦。临邛道士鸿都客,能以精诚致魂魄。\\
为感君王辗转思,遂教方士殷勤觅。排空驭气奔如电,升天入地求之遍。\\
上穷碧落下黄泉,两处茫茫皆不见。忽闻海上有仙山,山在虚无缥渺间。\\
楼阁玲珑五云起,其中绰约多仙子。中有一人字太真,雪肤花貌参差是。\\
金阙西厢叩玉扃,转教小玉报双成。闻道汉家天子使,九华帐里梦魂惊。\\
揽衣推枕起徘徊,珠箔银屏迤逦开。云鬓半偏新睡觉,花冠不整下堂来。\\
风吹仙袂飘飖举,犹似霓裳羽衣舞。玉容寂寞泪阑干,梨花一枝春带雨。\\
含情凝睇谢君王,一别音容两渺茫。昭阳殿里恩爱绝,蓬莱宫中日月长。\\
回头下望人寰处,不见长安见尘雾。惟将旧物表深情,钿合金钗寄将去。\\
钗留一股合一扇,钗擘黄金合分钿。但令心似金钿坚,天上人间会相见。\\
临别殷勤重寄词,词中有誓两心知。七月七日长生殿,夜半无人私语时。\\
在天愿作比翼鸟,在地愿为连理枝。天长地久有时尽,此恨绵绵无绝期。
    \end{tabular}
  \end{table}
\end{minipage}
\vspace{1cm}


\ptitle{问刘十九}\nopagebreak%
\addcontentsline{toc}{section}{\texorpdfstring{\makebox[10cm]{问刘十九\dotfill{} 【唐】白居易}}{问刘十九\ 【唐】白居易}}\nopagebreak%
\noindent\begin{minipage}{\linewidth}
  \pauthor{【唐】白居易}
  \vskip-3pt\begin{table}[H]
    \centering
    \begin{tabular}{@{}l@{}}
绿蚁新\xpinyin*{\xpinyin{醅}{pēi}}酒,红泥小火炉。\\
晚来天欲雪,能饮一杯无?
    \end{tabular}
  \end{table}
\end{minipage}
\vspace{1cm}


\ptitle{终南望余雪}\nopagebreak%
\addcontentsline{toc}{section}{\texorpdfstring{\makebox[10cm]{终南望余雪\dotfill{} 【唐】祖咏}}{终南望余雪\ 【唐】祖咏}}\nopagebreak%
\noindent\begin{minipage}{\linewidth}
  \pauthor{【唐】祖咏}
  \vskip-3pt\begin{table}[H]
    \centering
    \begin{tabular}{@{}l@{}}
终南阴岭秀,积雪浮云端。\\
林表明霁色,城中增暮寒。
    \end{tabular}
  \end{table}
\end{minipage}
\vspace{1cm}


\ptitle{蜂}\nopagebreak%
\addcontentsline{toc}{section}{\texorpdfstring{\makebox[10cm]{蜂\dotfill{} 【唐】罗隐}}{蜂\ 【唐】罗隐}}\nopagebreak%
\noindent\begin{minipage}{\linewidth}
  \pauthor{【唐】罗隐}
  \vskip-3pt\begin{table}[H]
    \centering
    \begin{tabular}{@{}l@{}}
不论平地与山尖,无限风光尽被占。\\
采得百花成蜜后,为谁辛苦为谁甜。
    \end{tabular}
  \end{table}
\end{minipage}
\vspace{1cm}


\ptitle{小儿垂钓}\nopagebreak%
\addcontentsline{toc}{section}{\texorpdfstring{\makebox[10cm]{小儿垂钓\dotfill{} 【唐】胡令能}}{小儿垂钓\ 【唐】胡令能}}\nopagebreak%
\noindent\begin{minipage}{\linewidth}
  \pauthor{【唐】胡令能}
  \vskip-3pt\begin{table}[H]
    \centering
    \begin{tabular}{@{}l@{}}
\xpinyin*{\xpinyin{蓬}{péng}}头\xpinyin*{\xpinyin{稚}{zhì}}子学垂\xpinyin*{\xpinyin{纶}{lún}},侧坐莓苔草映身。\\
路人借问遥招手,怕得鱼惊不应人。
    \end{tabular}
  \end{table}
\end{minipage}
\vspace{1cm}


\ptitle{蝉}\nopagebreak%
\addcontentsline{toc}{section}{\texorpdfstring{\makebox[10cm]{蝉\dotfill{} 【唐】虞世南}}{蝉\ 【唐】虞世南}}\nopagebreak%
\noindent\begin{minipage}{\linewidth}
  \pauthor{【唐】虞世南}
  \vskip-3pt\begin{table}[H]
    \centering
    \begin{tabular}{@{}l@{}}
垂緌饮清露,流响出疏桐。\\
居高声自远,非是藉秋风。
    \end{tabular}
  \end{table}
\end{minipage}
\vspace{1cm}


\ptitle{哥舒歌}\nopagebreak%
\addcontentsline{toc}{section}{\texorpdfstring{\makebox[10cm]{哥舒歌\dotfill{} 【唐】西鄙人}}{哥舒歌\ 【唐】西鄙人}}\nopagebreak%
\noindent\begin{minipage}{\linewidth}
  \pauthor{【唐】西鄙人}
  \vskip-3pt\begin{table}[H]
    \centering
    \begin{tabular}{@{}l@{}}
北斗七星高,哥舒夜带刀。\\
至今窥牧马,不敢过临\xpinyin*{\xpinyin{洮}{táo}}。
    \end{tabular}
  \end{table}
\end{minipage}
\vspace{1cm}


\ptitle{咏柳}\nopagebreak%
\addcontentsline{toc}{section}{\texorpdfstring{\makebox[10cm]{咏柳\dotfill{} 【唐】贺知章}}{咏柳\ 【唐】贺知章}}\nopagebreak%
\noindent\begin{minipage}{\linewidth}
  \pauthor{【唐】贺知章}
  \vskip-3pt\begin{table}[H]
    \centering
    \begin{tabular}{@{}l@{}}
碧玉妆成一树高,万条垂下绿丝\xpinyin*{\xpinyin{绦}{tāo}}。\\
不知细叶谁裁出,二月春风似剪刀。
    \end{tabular}
  \end{table}
\end{minipage}
\vspace{1cm}


\ptitle{回乡偶书}\nopagebreak%
\addcontentsline{toc}{section}{\texorpdfstring{\makebox[10cm]{回乡偶书\dotfill{} 【唐】贺知章}}{回乡偶书\ 【唐】贺知章}}\nopagebreak%
\noindent\begin{minipage}{\linewidth}
  \pauthor{【唐】贺知章}
  \vskip-3pt\begin{table}[H]
    \centering
    \begin{tabular}{@{}l@{}}
少小离家老大回,乡音无改\xpinyin*{\xpinyin{鬓}{bìn}}毛\xpinyin*{\xpinyin{衰}{cuī}}。\\
儿童相见不相识,笑问客从何处来。
    \end{tabular}
  \end{table}
\end{minipage}
\vspace{1cm}


\ptitle{寻隐者不遇}\nopagebreak%
\addcontentsline{toc}{section}{\texorpdfstring{\makebox[10cm]{寻隐者不遇\dotfill{} 【唐】贾岛}}{寻隐者不遇\ 【唐】贾岛}}\nopagebreak%
\noindent\begin{minipage}{\linewidth}
  \pauthor{【唐】贾岛}
  \vskip-3pt\begin{table}[H]
    \centering
    \begin{tabular}{@{}l@{}}
松下问童子,言师采药去。\\
只在此山中,云深不知处。
    \end{tabular}
  \end{table}
\end{minipage}
\vspace{1cm}


\ptitle{登幽州台歌}\nopagebreak%
\addcontentsline{toc}{section}{\texorpdfstring{\makebox[10cm]{登幽州台歌\dotfill{} 【唐】陈子昂}}{登幽州台歌\ 【唐】陈子昂}}\nopagebreak%
\noindent\begin{minipage}{\linewidth}
  \pauthor{【唐】陈子昂}
  \vskip-3pt\begin{table}[H]
    \centering
    \begin{tabular}{@{}l@{}}
前不见古人,后不见来者。\\
念天地之悠悠,独怆然而涕下。
    \end{tabular}
  \end{table}
\end{minipage}
\vspace{1cm}


\ptitle{陇西行}\nopagebreak%
\addcontentsline{toc}{section}{\texorpdfstring{\makebox[10cm]{陇西行\dotfill{} 【唐】陈陶}}{陇西行\ 【唐】陈陶}}\nopagebreak%
\noindent\begin{minipage}{\linewidth}
  \pauthor{【唐】陈陶}
  \vskip-3pt\begin{table}[H]
    \centering
    \begin{tabular}{@{}l@{}}
誓扫匈奴不顾身,五千貂锦丧胡尘。\\
可怜无定河边骨,犹是春闺梦里人!
    \end{tabular}
  \end{table}
\end{minipage}
\vspace{1cm}


\ptitle{\xpinyin*{\xpinyin{滁}{chú}}州西\xpinyin*{\xpinyin{涧}{jiàn}}}\nopagebreak%
\addcontentsline{toc}{section}{\texorpdfstring{\makebox[10cm]{滁州西涧\dotfill{} 【唐】韦应物}}{滁州西涧\ 【唐】韦应物}}\nopagebreak%
\noindent\begin{minipage}{\linewidth}
  \pauthor{【唐】韦应物}
  \vskip-3pt\begin{table}[H]
    \centering
    \begin{tabular}{@{}l@{}}
独怜幽草\xpinyin*{\xpinyin{涧}{jiàn}}边生,上有黄鹂深树鸣。\\
春潮带雨晚来急,野渡无人舟自横。
    \end{tabular}
  \end{table}
\end{minipage}
\vspace{1cm}


\ptitle{早春呈水部张十八员外 其一}\nopagebreak%
\addcontentsline{toc}{section}{\texorpdfstring{\makebox[10cm]{早春呈水部张十八员外 其一\dotfill{} 【唐】韩愈}}{早春呈水部张十八员外 其一\ 【唐】韩愈}}\nopagebreak%
\noindent\begin{minipage}{\linewidth}
  \pauthor{【唐】韩愈}
  \vskip-3pt\begin{table}[H]
    \centering
    \begin{tabular}{@{}l@{}}
天街小雨润如酥,草色遥看近却无。\\
最是一年春好处,绝胜烟柳满皇都。
    \end{tabular}
  \end{table}
\end{minipage}
\vspace{1cm}


\ptitle{晚春}\nopagebreak%
\addcontentsline{toc}{section}{\texorpdfstring{\makebox[10cm]{晚春\dotfill{} 【唐】韩愈}}{晚春\ 【唐】韩愈}}\nopagebreak%
\noindent\begin{minipage}{\linewidth}
  \pauthor{【唐】韩愈}
  \vskip-3pt\begin{table}[H]
    \centering
    \begin{tabular}{@{}l@{}}
草木知春不久归,百般红紫斗芳菲。\\
杨花榆荚无才思,惟解漫天作雪飞。
    \end{tabular}
  \end{table}
\end{minipage}
\vspace{1cm}


\ptitle{劝学}\nopagebreak%
\addcontentsline{toc}{section}{\texorpdfstring{\makebox[10cm]{劝学\dotfill{} 【唐】颜真卿}}{劝学\ 【唐】颜真卿}}\nopagebreak%
\noindent\begin{minipage}{\linewidth}
  \pauthor{【唐】颜真卿}
  \vskip-3pt\begin{table}[H]
    \centering
    \begin{tabular}{@{}l@{}}
三更灯火五更鸡,正是男儿读书时。\\
黑发不知勤学早,白首方悔读书迟。
    \end{tabular}
  \end{table}
\end{minipage}
\vspace{1cm}


\ptitle{咏鹅}\nopagebreak%
\addcontentsline{toc}{section}{\texorpdfstring{\makebox[10cm]{咏鹅\dotfill{} 【唐】骆宾王}}{咏鹅\ 【唐】骆宾王}}\nopagebreak%
\noindent\begin{minipage}{\linewidth}
  \pauthor{【唐】骆宾王}
  \vskip-3pt\begin{table}[H]
    \centering
    \begin{tabular}{@{}l@{}}
鹅,鹅,鹅,曲项向天歌。\\
白毛浮绿水,红掌拨清波。
    \end{tabular}
  \end{table}
\end{minipage}
\vspace{1cm}


\ptitle{别董大 其一}\nopagebreak%
\addcontentsline{toc}{section}{\texorpdfstring{\makebox[10cm]{别董大 其一\dotfill{} 【唐】高适}}{别董大 其一\ 【唐】高适}}\nopagebreak%
\noindent\begin{minipage}{\linewidth}
  \pauthor{【唐】高适}
  \vskip-3pt\begin{table}[H]
    \centering
    \begin{tabular}{@{}l@{}}
千里黄云白日\xpinyin*{\xpinyin{曛}{xūn}},北风吹雁雪纷纷。\\
莫愁前路无知己,天下谁人不识君。
    \end{tabular}
  \end{table}
\end{minipage}
\vspace{1cm}


\ptitle{燕歌行}\nopagebreak%
\addcontentsline{toc}{section}{\texorpdfstring{\makebox[10cm]{燕歌行\dotfill{} 【唐】高适}}{燕歌行\ 【唐】高适}}\nopagebreak%
\noindent\begin{minipage}{\linewidth}
  \pauthor{【唐】高适}
  \vskip-3pt\begin{table}[H]
    \centering
    \begin{tabular}{@{}l@{}}
 汉家烟尘在东北,汉将辞家破残贼。\\
 男儿本自重横行,天子非常赐颜色。\\
 \xpinyin*{\xpinyin{摐}{chuāng}}金伐鼓下榆关,旌旗逶迤碣石间。\\
 校尉羽书飞瀚海,单于猎火照狼山。\\
 山川萧条极边土,胡骑凭陵杂风雨。\\
 战士军前半死生,美人帐下犹歌舞。\\
 大漠穷秋塞草衰,孤城落日斗兵稀。\\
 身当恩遇常轻敌,力尽关山未解围。\\
 铁衣远戍辛勤久,玉箸应啼别离后。\\
 少妇城南欲断肠,征人蓟北空回首。\\
 边风飘飘那可度,绝域苍茫更何有。\\
 杀气三时作阵云,寒声一夜传刁斗。\\
 相看白刃血纷纷,死节从来岂顾勋。\\
君不见沙场征战苦,至今犹忆李将军。
    \end{tabular}
  \end{table}
\end{minipage}
\vspace{1cm}


\ptitle{菊花}\nopagebreak%
\addcontentsline{toc}{section}{\texorpdfstring{\makebox[10cm]{菊花\dotfill{} 【唐】黄巢}}{菊花\ 【唐】黄巢}}\nopagebreak%
\noindent\begin{minipage}{\linewidth}
  \pauthor{【唐】黄巢}
  \vskip-3pt\begin{table}[H]
    \centering
    \begin{tabular}{@{}l@{}}
待到秋来九月八, 我花开后百花杀。\\
冲天香阵透长安, 满城尽带黄金甲。
    \end{tabular}
  \end{table}
\end{minipage}
\vspace{1cm}


\chapter{五代十国}
\ptitle{浪淘沙令}\nopagebreak%
\addcontentsline{toc}{section}{\texorpdfstring{\makebox[10cm]{浪淘沙令\dotfill{} 【南唐】李煜}}{浪淘沙令\ 【南唐】李煜}}\nopagebreak%
\noindent\begin{minipage}{\linewidth}
  \pauthor{【南唐】李煜}
  \vskip-3pt\begin{table}[H]
    \centering
    \begin{tabular}{@{}l@{}}
帘外雨\xpinyin*{\xpinyin{潺}{chán}}潺,春意阑珊。罗\xpinyin*{\xpinyin{衾}{qīn}}不耐五更寒。\\
梦里不知身是客,一晌贪欢。\\
\\
独自莫凭栏,无限江山,别时容易见时难。\\
流水落花春去也,天上人间。
    \end{tabular}
  \end{table}
\end{minipage}
\vspace{1cm}


\ptitle{相见欢}\nopagebreak%
\addcontentsline{toc}{section}{\texorpdfstring{\makebox[10cm]{相见欢\dotfill{} 【南唐】李煜}}{相见欢\ 【南唐】李煜}}\nopagebreak%
\noindent\begin{minipage}{\linewidth}
  \pauthor{【南唐】李\xpinyin*{\xpinyin{煜}{yù}}}
  \vskip-3pt\begin{table}[H]
    \centering
    \begin{tabular}{@{}l@{}}
无言独上西楼,月如钩。 寂寞梧桐深院锁清秋。\\
\\
剪不断,理还乱,是离愁,别是一般滋味在心头。
    \end{tabular}
  \end{table}
\end{minipage}
\vspace{1cm}


\ptitle{虞美人}\nopagebreak%
\addcontentsline{toc}{section}{\texorpdfstring{\makebox[10cm]{虞美人\dotfill{} 【南唐】李煜}}{虞美人\ 【南唐】李煜}}\nopagebreak%
\noindent\begin{minipage}{\linewidth}
  \pauthor{【南唐】李煜}
  \vskip-3pt\begin{table}[H]
    \centering
    \begin{tabular}{@{}l@{}}
春花秋月何时了?往事知多少。\\
小楼昨夜又东风,故国不堪回首月明中。\\
\\
雕栏玉\xpinyin*{\xpinyin{砌}{qì}}应犹在,只是朱颜改。\\
问君能有几多愁?恰似一江春水向东流。
    \end{tabular}
  \end{table}
\end{minipage}
\vspace{1cm}


\chapter{宋}
\ptitle{题榴花}\nopagebreak%
\addcontentsline{toc}{section}{\texorpdfstring{\makebox[10cm]{题榴花\dotfill{} 【宋】朱熹}}{题榴花\ 【宋】朱熹}}\nopagebreak%
\noindent\begin{minipage}{\linewidth}
  \pauthor{【宋】朱熹}
  \vskip-3pt\begin{table}[H]
    \centering
    \begin{tabular}{@{}l@{}}
五月榴花照眼明,枝间时见子初成。\\
可怜此地无车马,颠倒苍苔落绛英。
    \end{tabular}
  \end{table}
\end{minipage}
\vspace{1cm}


\ptitle{登飞来峰}\nopagebreak%
\addcontentsline{toc}{section}{\texorpdfstring{\makebox[10cm]{登飞来峰\dotfill{} 【宋】王安石}}{登飞来峰\ 【宋】王安石}}\nopagebreak%
\noindent\begin{minipage}{\linewidth}
  \pauthor{【宋】王安石}
  \vskip-3pt\begin{table}[H]
    \centering
    \begin{tabular}{@{}l@{}}
飞来山上千寻塔,闻说鸡鸣见日升。\\
不畏浮云遮望眼,自缘身在最高层。
    \end{tabular}
  \end{table}
\end{minipage}
\vspace{1cm}


\ptitle{秋月}\nopagebreak%
\addcontentsline{toc}{section}{\texorpdfstring{\makebox[10cm]{秋月\dotfill{} 【宋】程颢}}{秋月\ 【宋】程颢}}\nopagebreak%
\noindent\begin{minipage}{\linewidth}
  \pauthor{【宋】程颢}
  \vskip-3pt\begin{table}[H]
    \centering
    \begin{tabular}{@{}l@{}}
清溪流过碧山头,空水澄鲜一色秋。\\
隔断红尘三十里,白云红叶两悠悠。
    \end{tabular}
  \end{table}
\end{minipage}
\vspace{1cm}


\ptitle{花影}\nopagebreak%
\addcontentsline{toc}{section}{\texorpdfstring{\makebox[10cm]{花影\dotfill{} 【宋】苏轼}}{花影\ 【宋】苏轼}}\nopagebreak%
\noindent\begin{minipage}{\linewidth}
  \pauthor{【宋】苏轼}
  \vskip-3pt\begin{table}[H]
    \centering
    \begin{tabular}{@{}l@{}}
重重叠叠上瑶台,几度呼童扫不开。\\
刚被太阳收拾去,又叫明月送将来。
    \end{tabular}
  \end{table}
\end{minipage}
\vspace{1cm}


\ptitle{水亭二首(其一)}\nopagebreak%
\addcontentsline{toc}{section}{\texorpdfstring{\makebox[10cm]{水亭二首(其一)\dotfill{} 【宋】陆 游}}{水亭二首(其一)\ 【宋】陆 游}}\nopagebreak%
\noindent\begin{minipage}{\linewidth}
  \pauthor{【宋】陆 游}
  \vskip-3pt\begin{table}[H]
    \centering
    \begin{tabular}{@{}l@{}}
水亭不受俗尘侵,葛帐筠床弄素琴。\\
一片风光谁画得:红蜻蜓点绿荷心。
    \end{tabular}
  \end{table}
\end{minipage}
\vspace{1cm}


\ptitle{冬夜读书示子\xpinyin*{\xpinyin{聿}{yù}}}\nopagebreak%
\addcontentsline{toc}{section}{\texorpdfstring{\makebox[10cm]{冬夜读书示子聿\dotfill{} 【宋】陆游}}{冬夜读书示子聿\ 【宋】陆游}}\nopagebreak%
\noindent\begin{minipage}{\linewidth}
  \pauthor{【宋】陆游}
  \vskip-3pt\begin{table}[H]
    \centering
    \begin{tabular}{@{}l@{}}
古人学问无遗力,少壮工夫老始成。\\
纸上得来终觉浅,绝知此事要躬行。
    \end{tabular}
  \end{table}
\end{minipage}
\vspace{1cm}


\ptitle{十一月四日风雨大作}\nopagebreak%
\addcontentsline{toc}{section}{\texorpdfstring{\makebox[10cm]{十一月四日风雨大作\dotfill{} 【宋】陆游}}{十一月四日风雨大作\ 【宋】陆游}}\nopagebreak%
\noindent\begin{minipage}{\linewidth}
  \pauthor{【宋】陆游}
  \vskip-3pt\begin{table}[H]
    \centering
    \begin{tabular}{@{}l@{}}
僵卧孤村不自哀,尚思为国戍轮台。\\
夜阑卧听风吹雨,铁马冰河入梦来。
    \end{tabular}
  \end{table}
\end{minipage}
\vspace{1cm}


\ptitle{贺新郎}\nopagebreak%
\addcontentsline{toc}{section}{\texorpdfstring{\makebox[10cm]{贺新郎\dotfill{} 【宋】刘克庄}}{贺新郎\ 【宋】刘克庄}}\nopagebreak%
\noindent\begin{minipage}{\linewidth}
  \pauthor{【宋】刘克庄}
  \vskip-3pt\begin{table}[H]
    \centering
    \begin{tabular}{@{}l@{}}
国脉微如缕。问长缨何时入手,缚将戎主?\\
未必人间无好汉,谁与宽些尺度?\\
试看取当年韩五。岂有谷城公付授,也不干曾遇骊山母。\\
谈笑起,两河路。\\
\\
少时棋柝曾联句。叹而今登楼揽镜,事机频误。\\
闻说北风吹面急,边上冲梯屡舞。\\
君莫道投鞭虚语,自古一贤能制难,有金汤便可无张许?\\
快投笔,莫题柱。
    \end{tabular}
  \end{table}
\end{minipage}
\vspace{1cm}


\ptitle{四字令}\nopagebreak%
\addcontentsline{toc}{section}{\texorpdfstring{\makebox[10cm]{四字令\dotfill{} 【宋】刘过}}{四字令\ 【宋】刘过}}\nopagebreak%
\noindent\begin{minipage}{\linewidth}
  \pauthor{【宋】刘过}
  \vskip-3pt\begin{table}[H]
    \centering
    \begin{tabular}{@{}l@{}}
情深意真。眉长鬓青。小楼明月调筝。写春风数声。\\
\\
思君忆君。魂牵梦萦。翠销香暖云屏。更那堪酒醒。
    \end{tabular}
  \end{table}
\end{minipage}
\vspace{1cm}


\ptitle{雪梅}\nopagebreak%
\addcontentsline{toc}{section}{\texorpdfstring{\makebox[10cm]{雪梅\dotfill{} 【宋】卢梅坡}}{雪梅\ 【宋】卢梅坡}}\nopagebreak%
\noindent\begin{minipage}{\linewidth}
  \pauthor{【宋】卢梅坡}
  \vskip-3pt\begin{table}[H]
    \centering
    \begin{tabular}{@{}l@{}}
梅雪争春未肯降,骚人搁笔费评章。\\
梅须逊雪三分白,雪却输梅一段香。
    \end{tabular}
  \end{table}
\end{minipage}
\vspace{1cm}


\ptitle{游园不值}\nopagebreak%
\addcontentsline{toc}{section}{\texorpdfstring{\makebox[10cm]{游园不值\dotfill{} 【宋】叶绍翁}}{游园不值\ 【宋】叶绍翁}}\nopagebreak%
\noindent\begin{minipage}{\linewidth}
  \pauthor{【宋】叶绍翁}
  \vskip-3pt\begin{table}[H]
    \centering
    \begin{tabular}{@{}l@{}}
应怜\xpinyin*{\xpinyin{屐}{jī}}齿印苍苔,小扣柴\xpinyin*{\xpinyin{扉}{fēi}}久不开。\\
春色满园关不住,一枝红杏出墙来。
    \end{tabular}
  \end{table}
\end{minipage}
\vspace{1cm}


\ptitle{庵居}\nopagebreak%
\addcontentsline{toc}{section}{\texorpdfstring{\makebox[10cm]{庵居\dotfill{} 【宋】吕本中}}{庵居\ 【宋】吕本中}}\nopagebreak%
\noindent\begin{minipage}{\linewidth}
  \pauthor{【宋】吕本中}
  \vskip-3pt\begin{table}[H]
    \centering
    \begin{tabular}{@{}l@{}}
鸟语花香变夕阴,稍闲复恐病相寻。正应独有江山分,素自都无廊庙心。\\
堂上老亲双白发,门前稚子旧青衿。儿曹不会庵居意,古涧寒泉疑至今。
    \end{tabular}
  \end{table}
\end{minipage}
\vspace{1cm}


\ptitle{苏幕遮}\nopagebreak%
\addcontentsline{toc}{section}{\texorpdfstring{\makebox[10cm]{苏幕遮\dotfill{} 【宋】周邦彦}}{苏幕遮\ 【宋】周邦彦}}\nopagebreak%
\noindent\begin{minipage}{\linewidth}
  \pauthor{【宋】周邦彦}
  \vskip-3pt\begin{table}[H]
    \centering
    \begin{tabular}{@{}l@{}}
燎沉香,消溽暑。鸟雀呼晴,侵晓窥檐语。\\
叶上初阳干宿雨。水面清圆,一一风荷举。\\
\\
故乡遥,何日去?家住吴门,久作长安旅。\\
五月渔郎相忆否?小楫轻舟,梦入芙蓉浦。
    \end{tabular}
  \end{table}
\end{minipage}
\vspace{1cm}


\ptitle{扬州慢}\nopagebreak%
\addcontentsline{toc}{section}{\texorpdfstring{\makebox[10cm]{扬州慢\dotfill{} 【宋】姜夔}}{扬州慢\ 【宋】姜夔}}\nopagebreak%
\noindent\begin{minipage}{\linewidth}
  \pauthor{【宋】姜\xpinyin*{\xpinyin{夔}{kuí}}}
  \vskip-3pt\begin{table}[H]
    \centering
    \begin{tabular}{@{}l@{}}
淮左名都,竹西佳处,解鞍少驻初程。\\
过春风十里,尽荠麦青青。\\
自胡马窥江去后,废池乔木,犹厌言兵。\\
渐黄昏,清角吹寒,都在空城。\\
\\
杜郎俊赏,算而今重到须惊。\\
纵豆蔻词工,青楼梦好,难赋深情。\\
二十四桥仍在,波心荡,冷月无声。\\
念桥边红药,年年知为谁生?
    \end{tabular}
  \end{table}
\end{minipage}
\vspace{1cm}


\ptitle{柳}\nopagebreak%
\addcontentsline{toc}{section}{\texorpdfstring{\makebox[10cm]{柳\dotfill{} 【宋】寇准}}{柳\ 【宋】寇准}}\nopagebreak%
\noindent\begin{minipage}{\linewidth}
  \pauthor{【宋】\xpinyin*{\xpinyin{寇}{kòu}}准}
  \vskip-3pt\begin{table}[H]
    \centering
    \begin{tabular}{@{}l@{}}
晓带轻烟间杏花,晚凝深翠拂平沙。\\
长条别有风流处,密映钱塘苏小家。
    \end{tabular}
  \end{table}
\end{minipage}
\vspace{1cm}


\ptitle{小重山}\nopagebreak%
\addcontentsline{toc}{section}{\texorpdfstring{\makebox[10cm]{小重山\dotfill{} 【宋】岳飞}}{小重山\ 【宋】岳飞}}\nopagebreak%
\noindent\begin{minipage}{\linewidth}
  \pauthor{【宋】岳飞}
  \vskip-3pt\begin{table}[H]
    \centering
    \begin{tabular}{@{}l@{}}
昨夜寒蛩不住鸣。惊回千里梦,已三更。\\
起来独自绕阶行。人悄悄,帘外月胧明。\\
\\
白首为功名。旧山松竹老,阻归程。\\
欲将心事付瑶琴。知音少,弦断有谁听。
    \end{tabular}
  \end{table}
\end{minipage}
\vspace{1cm}


\ptitle{满江红·怒发冲冠}\nopagebreak%
\addcontentsline{toc}{section}{\texorpdfstring{\makebox[10cm]{满江红·怒发冲冠\dotfill{} 【宋】岳飞}}{满江红·怒发冲冠\ 【宋】岳飞}}\nopagebreak%
\noindent\begin{minipage}{\linewidth}
  \pauthor{【宋】岳飞}
  \vskip-3pt\begin{table}[H]
    \centering
    \begin{tabular}{@{}l@{}}
怒发冲冠,凭阑处、潇潇雨歇。\\
抬望眼,仰天长啸,壮怀激烈。\\
三十功名尘与土,八千里路云和月。\\
莫等闲,白了少年头,空悲切。\\
\\
靖康耻,犹未雪;臣子恨,何时灭?\\
驾长车,踏破贺兰山缺。\\
壮志饥餐胡虏肉,笑谈渴饮匈奴血。\\
待从头,收拾旧山河,朝天阙。
    \end{tabular}
  \end{table}
\end{minipage}
\vspace{1cm}


\ptitle{念奴娇·过洞庭}\nopagebreak%
\addcontentsline{toc}{section}{\texorpdfstring{\makebox[10cm]{念奴娇·过洞庭\dotfill{} 【宋】张孝祥}}{念奴娇·过洞庭\ 【宋】张孝祥}}\nopagebreak%
\noindent\begin{minipage}{\linewidth}
  \pauthor{【宋】张孝祥}
  \vskip-3pt\begin{table}[H]
    \centering
    \begin{tabular}{@{}l@{}}
洞庭青草,近中秋、更无一点风色。\\
玉鉴琼田三万顷,着我扁 舟一叶。\\
素月分辉,明河共影,表里俱澄澈。\\
悠然心会,妙处难与君说。\\
\\
应念岭海经年,孤光自照,肝肺皆冰雪。\\
短发萧骚襟袖冷,稳泛沧浪空阔。\\
尽吸西江,细斟北斗,万象为宾客。\\
扣舷独啸,不知今夕何夕。
    \end{tabular}
  \end{table}
\end{minipage}
\vspace{1cm}


\ptitle{绝句·古木阴中系短篷}\nopagebreak%
\addcontentsline{toc}{section}{\texorpdfstring{\makebox[10cm]{绝句·古木阴中系短篷\dotfill{} 【宋】志南}}{绝句·古木阴中系短篷\ 【宋】志南}}\nopagebreak%
\noindent\begin{minipage}{\linewidth}
  \pauthor{【宋】志南}
  \vskip-3pt\begin{table}[H]
    \centering
    \begin{tabular}{@{}l@{}}
古木阴中系短篷,杖藜扶我过桥东。\\
沾衣欲湿杏花雨,吹面不寒杨柳风。
    \end{tabular}
  \end{table}
\end{minipage}
\vspace{1cm}


\ptitle{过零丁洋}\nopagebreak%
\addcontentsline{toc}{section}{\texorpdfstring{\makebox[10cm]{过零丁洋\dotfill{} 【宋】文天祥}}{过零丁洋\ 【宋】文天祥}}\nopagebreak%
\noindent\begin{minipage}{\linewidth}
  \pauthor{【宋】文天祥}
  \vskip-3pt\begin{table}[H]
    \centering
    \begin{tabular}{@{}l@{}}
辛苦遭逢起一经,干戈寥落四周星。山河破碎风飘絮,身世浮沉雨打萍。\\
惶恐滩头说惶恐,零丁洋里叹零丁。人生自古谁无死,留取丹心照汗青。
    \end{tabular}
  \end{table}
\end{minipage}
\vspace{1cm}


\ptitle{浣溪沙}\nopagebreak%
\addcontentsline{toc}{section}{\texorpdfstring{\makebox[10cm]{浣溪沙\dotfill{} 【宋】晏殊}}{浣溪沙\ 【宋】晏殊}}\nopagebreak%
\noindent\begin{minipage}{\linewidth}
  \pauthor{【宋】\xpinyin*{\xpinyin{晏}{yàn}}殊}
  \vskip-3pt\begin{table}[H]
    \centering
    \begin{tabular}{@{}l@{}}
一曲新词酒一杯,去年天气旧亭台。夕阳西下几时回。\\
\\
无可奈何花落去,似曾相识燕归来。小园香径独徘徊。
    \end{tabular}
  \end{table}
\end{minipage}
\vspace{1cm}


\ptitle{踏莎行}\nopagebreak%
\addcontentsline{toc}{section}{\texorpdfstring{\makebox[10cm]{踏莎行\dotfill{} 【宋】晏殊}}{踏莎行\ 【宋】晏殊}}\nopagebreak%
\noindent\begin{minipage}{\linewidth}
  \pauthor{【宋】晏殊}
  \vskip-3pt\begin{table}[H]
    \centering
    \begin{tabular}{@{}l@{}}
祖席离歌,长亭别宴。香尘已隔犹回面。居人匹马映林嘶,行人去棹依波转。\\
画阁魂消,高楼目断。斜阳只送平波远。无穷无尽是离愁,天涯地角寻思遍。
    \end{tabular}
  \end{table}
\end{minipage}
\vspace{1cm}


\ptitle{三\xpinyin*{\xpinyin{衢}{qú}}道中}\nopagebreak%
\addcontentsline{toc}{section}{\texorpdfstring{\makebox[10cm]{三衢道中\dotfill{} 【宋】曾几}}{三衢道中\ 【宋】曾几}}\nopagebreak%
\noindent\begin{minipage}{\linewidth}
  \pauthor{【宋】曾几}
  \vskip-3pt\begin{table}[H]
    \centering
    \begin{tabular}{@{}l@{}}
梅子黄时日日晴,小溪泛尽却山行。\\
绿阴不减来时路,添得黄鹂四五声。
    \end{tabular}
  \end{table}
\end{minipage}
\vspace{1cm}


\ptitle{春日}\nopagebreak%
\addcontentsline{toc}{section}{\texorpdfstring{\makebox[10cm]{春日\dotfill{} 【宋】朱熹}}{春日\ 【宋】朱熹}}\nopagebreak%
\noindent\begin{minipage}{\linewidth}
  \pauthor{【宋】朱\xpinyin*{\xpinyin{熹}{xī}}}
  \vskip-3pt\begin{table}[H]
    \centering
    \begin{tabular}{@{}l@{}}
胜日寻芳\xpinyin*{\xpinyin{泗}{sì}}水滨,无边光景一时新。\\
等闲识得东风面,万紫千红总是春。
    \end{tabular}
  \end{table}
\end{minipage}
\vspace{1cm}


\ptitle{观书有感}\nopagebreak%
\addcontentsline{toc}{section}{\texorpdfstring{\makebox[10cm]{观书有感\dotfill{} 【宋】朱熹}}{观书有感\ 【宋】朱熹}}\nopagebreak%
\noindent\begin{minipage}{\linewidth}
  \pauthor{【宋】朱\xpinyin*{\xpinyin{熹}{xī}}}
  \vskip-3pt\begin{table}[H]
    \centering
    \begin{tabular}{@{}l@{}}
半亩方塘一鉴开,天光云影共徘徊。\\
问渠那得清如许,为有源头活水来。
    \end{tabular}
  \end{table}
\end{minipage}
\vspace{1cm}


\ptitle{声声慢}\nopagebreak%
\addcontentsline{toc}{section}{\texorpdfstring{\makebox[10cm]{声声慢\dotfill{} 【宋】李清照}}{声声慢\ 【宋】李清照}}\nopagebreak%
\noindent\begin{minipage}{\linewidth}
  \pauthor{【宋】李清照}
  \vskip-3pt\begin{table}[H]
    \centering
    \begin{tabular}{@{}l@{}}
寻寻觅觅,冷冷清清,凄凄惨惨戚戚。\\
乍暖还寒时候,最难将息。\\
三杯两盏淡酒,怎敌他、晚来风急!\\
雁过也,正伤心,却是旧时相识。\\
\\
满地黄花堆积,憔悴损,如今有谁堪摘?\\
守着窗儿,独自怎生得黑!\\
梧桐更兼细雨,到黄昏、点点滴滴。\\
这次第,怎一个愁字了得!
    \end{tabular}
  \end{table}
\end{minipage}
\vspace{1cm}


\ptitle{夏日绝句}\nopagebreak%
\addcontentsline{toc}{section}{\texorpdfstring{\makebox[10cm]{夏日绝句\dotfill{} 【宋】李清照}}{夏日绝句\ 【宋】李清照}}\nopagebreak%
\noindent\begin{minipage}{\linewidth}
  \pauthor{【宋】李清照}
  \vskip-3pt\begin{table}[H]
    \centering
    \begin{tabular}{@{}l@{}}
生当做人杰,死亦为鬼雄。\\
至今思项羽,不肯过江东。
    \end{tabular}
  \end{table}
\end{minipage}
\vspace{1cm}


\ptitle{渔家傲}\nopagebreak%
\addcontentsline{toc}{section}{\texorpdfstring{\makebox[10cm]{渔家傲\dotfill{} 【宋】李清照}}{渔家傲\ 【宋】李清照}}\nopagebreak%
\noindent\begin{minipage}{\linewidth}
  \pauthor{【宋】李清照}
  \vskip-3pt\begin{table}[H]
    \centering
    \begin{tabular}{@{}l@{}}
天接云涛连晓雾,星河欲转千帆舞。\\
仿佛梦魂归帝所,闻天语,殷勤问我归何处。\\
\\
我报路长\xpinyin*{\xpinyin{嗟}{jiē}}日暮,学诗谩有惊人句。\\
九万里风鹏正举。风休住,蓬舟吹取三山去!
    \end{tabular}
  \end{table}
\end{minipage}
\vspace{1cm}


\ptitle{如梦令}\nopagebreak%
\addcontentsline{toc}{section}{\texorpdfstring{\makebox[10cm]{如梦令\dotfill{} 【宋】李清照}}{如梦令\ 【宋】李清照}}\nopagebreak%
\noindent\begin{minipage}{\linewidth}
  \pauthor{【宋】李清照}
  \vskip-3pt\begin{table}[H]
    \centering
    \begin{tabular}{@{}l@{}}
常记溪亭日暮,沉醉不知归路。\\
兴尽晚回舟,误入藕花深处。\\
争渡,争渡,惊起一滩鸥鹭。
    \end{tabular}
  \end{table}
\end{minipage}
\vspace{1cm}


\ptitle{小池}\nopagebreak%
\addcontentsline{toc}{section}{\texorpdfstring{\makebox[10cm]{小池\dotfill{} 【宋】杨万里}}{小池\ 【宋】杨万里}}\nopagebreak%
\noindent\begin{minipage}{\linewidth}
  \pauthor{【宋】杨万里}
  \vskip-3pt\begin{table}[H]
    \centering
    \begin{tabular}{@{}l@{}}
泉眼无声惜细流,树阴照水爱晴柔。\\
小荷才露尖尖角,早有蜻蜓立上头。
    \end{tabular}
  \end{table}
\end{minipage}
\vspace{1cm}


\ptitle{晓出净慈寺送林子方}\nopagebreak%
\addcontentsline{toc}{section}{\texorpdfstring{\makebox[10cm]{晓出净慈寺送林子方\dotfill{} 【宋】杨万里}}{晓出净慈寺送林子方\ 【宋】杨万里}}\nopagebreak%
\noindent\begin{minipage}{\linewidth}
  \pauthor{【宋】杨万里}
  \vskip-3pt\begin{table}[H]
    \centering
    \begin{tabular}{@{}l@{}}
毕竟西湖六月中,风光不与四时同。\\
接天莲叶无穷碧,映日荷花别样红。
    \end{tabular}
  \end{table}
\end{minipage}
\vspace{1cm}


\ptitle{舟过安仁}\nopagebreak%
\addcontentsline{toc}{section}{\texorpdfstring{\makebox[10cm]{舟过安仁\dotfill{} 【宋】杨万里}}{舟过安仁\ 【宋】杨万里}}\nopagebreak%
\noindent\begin{minipage}{\linewidth}
  \pauthor{【宋】杨万里}
  \vskip-3pt\begin{table}[H]
    \centering
    \begin{tabular}{@{}l@{}}
一叶渔船两小童,收篙停棹坐船中。\\
怪生无雨都张伞,不是遮头是使风。
    \end{tabular}
  \end{table}
\end{minipage}
\vspace{1cm}


\ptitle{题临安\xpinyin*{\xpinyin{邸}{dǐ}}}\nopagebreak%
\addcontentsline{toc}{section}{\texorpdfstring{\makebox[10cm]{题临安邸\dotfill{} 【宋】林升}}{题临安邸\ 【宋】林升}}\nopagebreak%
\noindent\begin{minipage}{\linewidth}
  \pauthor{【宋】林升}
  \vskip-3pt\begin{table}[H]
    \centering
    \begin{tabular}{@{}l@{}}
山外青山楼外楼,西湖歌舞几时休?\\
暖风\xpinyin*{\xpinyin{熏}{xūn}}得游人醉,直把杭州作\xpinyin*{\xpinyin{汴}{biàn}}州。
    \end{tabular}
  \end{table}
\end{minipage}
\vspace{1cm}


\ptitle{望海潮}\nopagebreak%
\addcontentsline{toc}{section}{\texorpdfstring{\makebox[10cm]{望海潮\dotfill{} 【宋】柳永}}{望海潮\ 【宋】柳永}}\nopagebreak%
\noindent\begin{minipage}{\linewidth}
  \pauthor{【宋】柳永}
  \vskip-3pt\begin{table}[H]
    \centering
    \begin{tabular}{@{}l@{}}
东南形胜,三吴都会,钱塘自古繁华。\\
烟柳画桥,风帘翠幕, 参差十万人家。\\
云树绕堤沙。怒涛卷霜雪,天堑无涯。\\
市列珠玑,户盈罗绮,竞豪奢。\\
\\
重湖叠巘清嘉,有三秋桂子,十里荷花。\\
羌管弄晴,菱歌泛夜,嬉嬉钓叟莲娃。\\
千骑拥高牙。乘醉听箫鼓,吟赏烟霞。\\
异日图将好景,归去凤池夸。
    \end{tabular}
  \end{table}
\end{minipage}
\vspace{1cm}


\ptitle{书湖阴先生壁}\nopagebreak%
\addcontentsline{toc}{section}{\texorpdfstring{\makebox[10cm]{书湖阴先生壁\dotfill{} 【宋】王安石}}{书湖阴先生壁\ 【宋】王安石}}\nopagebreak%
\noindent\begin{minipage}{\linewidth}
  \pauthor{【宋】王安石}
  \vskip-3pt\begin{table}[H]
    \centering
    \begin{tabular}{@{}l@{}}
茅\xpinyin*{\xpinyin{檐}{yán}}长扫净无苔,花木成\xpinyin*{\xpinyin{畦}{qí}}手自栽。\\
一水护田将绿绕,两山排\xpinyin*{\xpinyin{闼}{tà}}送青来。
    \end{tabular}
  \end{table}
\end{minipage}
\vspace{1cm}


\ptitle{元日}\nopagebreak%
\addcontentsline{toc}{section}{\texorpdfstring{\makebox[10cm]{元日\dotfill{} 【宋】王安石}}{元日\ 【宋】王安石}}\nopagebreak%
\noindent\begin{minipage}{\linewidth}
  \pauthor{【宋】王安石}
  \vskip-3pt\begin{table}[H]
    \centering
    \begin{tabular}{@{}l@{}}
爆竹声中一岁除,春风送暖入\xpinyin*{\xpinyin{屠}{tú}}苏。\\
千门万户\xpinyin*{\xpinyin{曈}{tóng}}曈日,总把新桃换旧符。
    \end{tabular}
  \end{table}
\end{minipage}
\vspace{1cm}


\ptitle{桂枝香·金陵怀古}\nopagebreak%
\addcontentsline{toc}{section}{\texorpdfstring{\makebox[10cm]{桂枝香·金陵怀古\dotfill{} 【宋】王安石}}{桂枝香·金陵怀古\ 【宋】王安石}}\nopagebreak%
\noindent\begin{minipage}{\linewidth}
  \pauthor{【宋】王安石}
  \vskip-3pt\begin{table}[H]
    \centering
    \begin{tabular}{@{}l@{}}
登临送目。正故国晚秋,天气初肃。\\
千里澄江似练,翠峰如簇。\\
归帆去棹残阳里,背西风、酒旗斜矗。\\
彩舟云淡,星河鹭起,画图难足。\\
\\
念往昔、\\
繁华竞逐。叹门外楼头,悲恨相续。\\
千古凭高对此, 谩嗟荣辱。\\
六朝旧事随流水,但寒烟、芳草凝绿。\\
至今商女,时时犹唱,后庭遗曲。
    \end{tabular}
  \end{table}
\end{minipage}
\vspace{1cm}


\ptitle{江上}\nopagebreak%
\addcontentsline{toc}{section}{\texorpdfstring{\makebox[10cm]{江上\dotfill{} 【宋】王安石}}{江上\ 【宋】王安石}}\nopagebreak%
\noindent\begin{minipage}{\linewidth}
  \pauthor{【宋】王安石}
  \vskip-3pt\begin{table}[H]
    \centering
    \begin{tabular}{@{}l@{}}
江北秋阴一半开,晚云含雨却低回。\\
青山缭绕疑无路,忽见千帆隐映来。
    \end{tabular}
  \end{table}
\end{minipage}
\vspace{1cm}


\ptitle{泊船瓜洲}\nopagebreak%
\addcontentsline{toc}{section}{\texorpdfstring{\makebox[10cm]{泊船瓜洲\dotfill{} 【宋】王安石}}{泊船瓜洲\ 【宋】王安石}}\nopagebreak%
\noindent\begin{minipage}{\linewidth}
  \pauthor{【宋】王安石}
  \vskip-3pt\begin{table}[H]
    \centering
    \begin{tabular}{@{}l@{}}
京口瓜洲一水间,钟山只隔数重山。\\
春风又绿江南岸,明月何时照我还。
    \end{tabular}
  \end{table}
\end{minipage}
\vspace{1cm}


\ptitle{登飞来峰}\nopagebreak%
\addcontentsline{toc}{section}{\texorpdfstring{\makebox[10cm]{登飞来峰\dotfill{} 【宋】王安石}}{登飞来峰\ 【宋】王安石}}\nopagebreak%
\noindent\begin{minipage}{\linewidth}
  \pauthor{【宋】王安石}
  \vskip-3pt\begin{table}[H]
    \centering
    \begin{tabular}{@{}l@{}}
飞来山上千寻塔,闻说鸡鸣见日升。\\
不畏浮云遮望眼,自缘身在最高层。
    \end{tabular}
  \end{table}
\end{minipage}
\vspace{1cm}


\ptitle{春日}\nopagebreak%
\addcontentsline{toc}{section}{\texorpdfstring{\makebox[10cm]{春日\dotfill{} 【宋】秦观}}{春日\ 【宋】秦观}}\nopagebreak%
\noindent\begin{minipage}{\linewidth}
  \pauthor{【宋】秦观}
  \vskip-3pt\begin{table}[H]
    \centering
    \begin{tabular}{@{}l@{}}
一夕轻雷落万丝,\xpinyin*{\xpinyin{霁}{jì}}光浮瓦碧参差。\\
有情芍药含春泪,无力蔷薇卧晓枝。
    \end{tabular}
  \end{table}
\end{minipage}
\vspace{1cm}


\ptitle{鹊桥仙}\nopagebreak%
\addcontentsline{toc}{section}{\texorpdfstring{\makebox[10cm]{鹊桥仙\dotfill{} 【宋】秦观}}{鹊桥仙\ 【宋】秦观}}\nopagebreak%
\noindent\begin{minipage}{\linewidth}
  \pauthor{【宋】秦观}
  \vskip-3pt\begin{table}[H]
    \centering
    \begin{tabular}{@{}l@{}}
纤云弄巧,飞星传恨,银汉迢迢暗度。金风玉露一相逢,便胜却人间无数。\\
\\
柔情似水,佳期如梦,忍顾鹊桥归路。两情若是久长时,又岂在朝朝暮暮。
    \end{tabular}
  \end{table}
\end{minipage}
\vspace{1cm}


\ptitle{乡村四月}\nopagebreak%
\addcontentsline{toc}{section}{\texorpdfstring{\makebox[10cm]{乡村四月\dotfill{} 【宋】翁卷}}{乡村四月\ 【宋】翁卷}}\nopagebreak%
\noindent\begin{minipage}{\linewidth}
  \pauthor{【宋】翁卷}
  \vskip-3pt\begin{table}[H]
    \centering
    \begin{tabular}{@{}l@{}}
绿遍山原白满川,子规声里雨如烟。\\
乡村四月闲人少,才了蚕桑又插田。
    \end{tabular}
  \end{table}
\end{minipage}
\vspace{1cm}


\ptitle{六月二十七日望湖楼醉书}\nopagebreak%
\addcontentsline{toc}{section}{\texorpdfstring{\makebox[10cm]{六月二十七日望湖楼醉书\dotfill{} 【宋】苏轼}}{六月二十七日望湖楼醉书\ 【宋】苏轼}}\nopagebreak%
\noindent\begin{minipage}{\linewidth}
  \pauthor{【宋】苏轼}
  \vskip-3pt\begin{table}[H]
    \centering
    \begin{tabular}{@{}l@{}}
黑云翻墨未遮山,白雨跳珠乱入船。\\
卷地风来忽吹散,望湖楼下水如天。
    \end{tabular}
  \end{table}
\end{minipage}
\vspace{1cm}


\ptitle{六月二十七日望湖楼醉书(二)}\nopagebreak%
\addcontentsline{toc}{section}{\texorpdfstring{\makebox[10cm]{六月二十七日望湖楼醉书(二)\dotfill{} 【宋】苏轼}}{六月二十七日望湖楼醉书(二)\ 【宋】苏轼}}\nopagebreak%
\noindent\begin{minipage}{\linewidth}
  \pauthor{【宋】苏轼}
  \vskip-3pt\begin{table}[H]
    \centering
    \begin{tabular}{@{}l@{}}
放生鱼鳖逐人来,无主荷花到处开。\\
水枕能令山俯仰,风船解与月徘徊。
    \end{tabular}
  \end{table}
\end{minipage}
\vspace{1cm}


\ptitle{和子由渑池怀旧}\nopagebreak%
\addcontentsline{toc}{section}{\texorpdfstring{\makebox[10cm]{和子由渑池怀旧\dotfill{} 【宋】苏轼}}{和子由渑池怀旧\ 【宋】苏轼}}\nopagebreak%
\noindent\begin{minipage}{\linewidth}
  \pauthor{【宋】苏轼}
  \vskip-3pt\begin{table}[H]
    \centering
    \begin{tabular}{@{}l@{}}
人生到处知何似,应似飞鸿踏雪泥:\\
泥上偶然留指爪,鸿飞那复计东西。\\
老僧已死成新塔,坏壁无由见旧题。\\
往日崎岖还记否,路长人困蹇驴嘶。
    \end{tabular}
  \end{table}
\end{minipage}
\vspace{1cm}


\ptitle{念奴娇·赤壁怀古}\nopagebreak%
\addcontentsline{toc}{section}{\texorpdfstring{\makebox[10cm]{念奴娇·赤壁怀古\dotfill{} 【宋】苏轼}}{念奴娇·赤壁怀古\ 【宋】苏轼}}\nopagebreak%
\noindent\begin{minipage}{\linewidth}
  \pauthor{【宋】苏轼}
  \vskip-3pt\begin{table}[H]
    \centering
    \begin{tabular}{@{}l@{}}
大江东去,浪淘尽,千古风流人物。\\
故垒西边,人道是,三国周郎赤壁。\\
乱石穿空,惊涛拍岸,卷起千堆雪。\\
江山如画,一时多少豪杰。\\
\\
遥想公瑾当年,小乔初嫁了,雄姿英发。\\
羽扇纶巾,谈笑间,樯橹灰飞烟灭。\\
故国神游,多情应笑我,早生华发。\\
人生如梦,一尊还酹江月。
    \end{tabular}
  \end{table}
\end{minipage}
\vspace{1cm}


\ptitle{惠崇春江晓景}\nopagebreak%
\addcontentsline{toc}{section}{\texorpdfstring{\makebox[10cm]{惠崇春江晓景\dotfill{} 【宋】苏轼}}{惠崇春江晓景\ 【宋】苏轼}}\nopagebreak%
\noindent\begin{minipage}{\linewidth}
  \pauthor{【宋】苏轼}
  \vskip-3pt\begin{table}[H]
    \centering
    \begin{tabular}{@{}l@{}}
竹外桃花三两枝,春江水暖鸭先知。\\
\xpinyin*{\xpinyin{蒌}{lóu}}\xpinyin*{\xpinyin{蒿}{hāo}}满地芦芽短,正是河\xpinyin*{\xpinyin{豚}{tún}}欲上时。
    \end{tabular}
  \end{table}
\end{minipage}
\vspace{1cm}


\ptitle{春宵}\nopagebreak%
\addcontentsline{toc}{section}{\texorpdfstring{\makebox[10cm]{春宵\dotfill{} 【宋】苏轼}}{春宵\ 【宋】苏轼}}\nopagebreak%
\noindent\begin{minipage}{\linewidth}
  \pauthor{【宋】苏轼}
  \vskip-3pt\begin{table}[H]
    \centering
    \begin{tabular}{@{}l@{}}
春宵一刻值千金,花有清香月有阴。\\
歌管楼台声细细,秋千院落夜沉沉。
    \end{tabular}
  \end{table}
\end{minipage}
\vspace{1cm}


\ptitle{水调歌头·明月几时有}\nopagebreak%
\addcontentsline{toc}{section}{\texorpdfstring{\makebox[10cm]{水调歌头·明月几时有\dotfill{} 【宋】苏轼}}{水调歌头·明月几时有\ 【宋】苏轼}}\nopagebreak%
\noindent\begin{minipage}{\linewidth}
  \pauthor{【宋】苏轼}
  \vskip-3pt\begin{table}[H]
    \centering
    \begin{tabular}{@{}l@{}}
明月几时有,把酒问青天。\\
不知天上宫阙,今夕是何年。\\
我欲乘风归去,又恐琼楼玉宇,高处不胜寒。\\
起舞弄清影,何似在人间。\\
\\
转朱阁,低\xpinyin*{\xpinyin{绮}{qǐ}}户,照无眠。\\
不应有恨,何事长向别时圆。\\
人有悲欢离合,月有阴晴圆缺,此事古难全。\\
但愿人长久,千里共婵娟。
    \end{tabular}
  \end{table}
\end{minipage}
\vspace{1cm}


\ptitle{江城子·乙卯正月二十日夜记梦}\nopagebreak%
\addcontentsline{toc}{section}{\texorpdfstring{\makebox[10cm]{江城子·乙卯正月二十日夜记梦\dotfill{} 【宋】苏轼}}{江城子·乙卯正月二十日夜记梦\ 【宋】苏轼}}\nopagebreak%
\noindent\begin{minipage}{\linewidth}
  \pauthor{【宋】苏轼}
  \vskip-3pt\begin{table}[H]
    \centering
    \begin{tabular}{@{}l@{}}
十年生死两茫茫。不思量,自难忘。\\
千里孤坟,无处话凄凉。\\
纵使相逢应不识,尘满面,鬓如霜。\\
\\
夜来幽梦忽还乡。小轩窗,正梳妆。\\
相顾无言,惟有泪千行。\\
料得年年肠断处,明月夜,短松冈。
    \end{tabular}
  \end{table}
\end{minipage}
\vspace{1cm}


\ptitle{江城子·密州出猎}\nopagebreak%
\addcontentsline{toc}{section}{\texorpdfstring{\makebox[10cm]{江城子·密州出猎\dotfill{} 【宋】苏轼}}{江城子·密州出猎\ 【宋】苏轼}}\nopagebreak%
\noindent\begin{minipage}{\linewidth}
  \pauthor{【宋】苏轼}
  \vskip-3pt\begin{table}[H]
    \centering
    \begin{tabular}{@{}l@{}}
老夫聊发少年狂,左牵黄,右擎苍。锦帽貂裘,千骑卷平冈。\\
为报倾城随太守,亲射虎,看孙郎。\\
\\
酒酣胸胆尚开张,鬓微霜,又何妨!持节云中,何日遣冯唐。\\
会挽雕弓如满月,西北望,射天狼。
    \end{tabular}
  \end{table}
\end{minipage}
\vspace{1cm}


\ptitle{题西林壁}\nopagebreak%
\addcontentsline{toc}{section}{\texorpdfstring{\makebox[10cm]{题西林壁\dotfill{} 【宋】苏轼}}{题西林壁\ 【宋】苏轼}}\nopagebreak%
\noindent\begin{minipage}{\linewidth}
  \pauthor{【宋】苏轼}
  \vskip-3pt\begin{table}[H]
    \centering
    \begin{tabular}{@{}l@{}}
横看成岭侧成峰,远近高低各不同。\\
不识庐山真面目,只缘身在此山中。
    \end{tabular}
  \end{table}
\end{minipage}
\vspace{1cm}


\ptitle{饮湖上初晴后雨}\nopagebreak%
\addcontentsline{toc}{section}{\texorpdfstring{\makebox[10cm]{饮湖上初晴后雨\dotfill{} 【宋】苏轼}}{饮湖上初晴后雨\ 【宋】苏轼}}\nopagebreak%
\noindent\begin{minipage}{\linewidth}
  \pauthor{【宋】苏轼}
  \vskip-3pt\begin{table}[H]
    \centering
    \begin{tabular}{@{}l@{}}
水光\xpinyin*{\xpinyin{潋}{liàn}}\xpinyin*{\xpinyin{滟}{yàn}}晴方好,山色空蒙雨亦奇。\\
欲把西湖比西子,淡妆浓抹总相宜。
    \end{tabular}
  \end{table}
\end{minipage}
\vspace{1cm}


\ptitle{江上渔者}\nopagebreak%
\addcontentsline{toc}{section}{\texorpdfstring{\makebox[10cm]{江上渔者\dotfill{} 【宋】范仲淹}}{江上渔者\ 【宋】范仲淹}}\nopagebreak%
\noindent\begin{minipage}{\linewidth}
  \pauthor{【宋】范仲淹}
  \vskip-3pt\begin{table}[H]
    \centering
    \begin{tabular}{@{}l@{}}
江上往来人,但爱\xpinyin*{\xpinyin{鲈}{lú}}鱼美。\\
君看一叶舟,出没风波里。
    \end{tabular}
  \end{table}
\end{minipage}
\vspace{1cm}


\ptitle{渔家傲}\nopagebreak%
\addcontentsline{toc}{section}{\texorpdfstring{\makebox[10cm]{渔家傲\dotfill{} 【宋】范仲淹}}{渔家傲\ 【宋】范仲淹}}\nopagebreak%
\noindent\begin{minipage}{\linewidth}
  \pauthor{【宋】范仲淹}
  \vskip-3pt\begin{table}[H]
    \centering
    \begin{tabular}{@{}l@{}}
塞下秋来风景异,衡阳雁去无留意。四面边声连角起。\\
千嶂里,长烟落日孤城闭。\\
\\
浊酒一杯家万里,燕然未勒归无计。羌管悠悠霜满地。\\
人不寐,将军白发征夫泪。
    \end{tabular}
  \end{table}
\end{minipage}
\vspace{1cm}


\ptitle{四时田园杂兴}\nopagebreak%
\addcontentsline{toc}{section}{\texorpdfstring{\makebox[10cm]{四时田园杂兴\dotfill{} 【宋】范成大}}{四时田园杂兴\ 【宋】范成大}}\nopagebreak%
\noindent\begin{minipage}{\linewidth}
  \pauthor{【宋】范成大}
  \vskip-3pt\begin{table}[H]
    \centering
    \begin{tabular}{@{}l@{}}
昼出耘田夜绩麻,村庄儿女各当家。\\
童孙未解供耕织,也傍桑阴学种瓜。
    \end{tabular}
  \end{table}
\end{minipage}
\vspace{1cm}


\ptitle{四时田园杂兴}\nopagebreak%
\addcontentsline{toc}{section}{\texorpdfstring{\makebox[10cm]{四时田园杂兴\dotfill{} 【宋】范成大}}{四时田园杂兴\ 【宋】范成大}}\nopagebreak%
\noindent\begin{minipage}{\linewidth}
  \pauthor{【宋】范成大}
  \vskip-3pt\begin{table}[H]
    \centering
    \begin{tabular}{@{}l@{}}
梅子金黄杏子肥,麦花雪白菜花稀。\\
日长篱落无人过,唯有蜻蜓蛱蝶飞。
    \end{tabular}
  \end{table}
\end{minipage}
\vspace{1cm}


\ptitle{有约}\nopagebreak%
\addcontentsline{toc}{section}{\texorpdfstring{\makebox[10cm]{有约\dotfill{} 【宋】赵师秀}}{有约\ 【宋】赵师秀}}\nopagebreak%
\noindent\begin{minipage}{\linewidth}
  \pauthor{【宋】赵师秀}
  \vskip-3pt\begin{table}[H]
    \centering
    \begin{tabular}{@{}l@{}}
黄梅时节家家雨,青草池塘处处蛙。\\
有约不来过夜半,闲敲棋子落灯花。
    \end{tabular}
  \end{table}
\end{minipage}
\vspace{1cm}


\ptitle{南乡子}\nopagebreak%
\addcontentsline{toc}{section}{\texorpdfstring{\makebox[10cm]{南乡子\dotfill{} 【宋】辛弃疾}}{南乡子\ 【宋】辛弃疾}}\nopagebreak%
\noindent\begin{minipage}{\linewidth}
  \pauthor{【宋】辛弃疾}
  \vskip-3pt\begin{table}[H]
    \centering
    \begin{tabular}{@{}l@{}}
何处望神州?满眼风光北固楼。\\
千古兴亡多少事?悠悠。不尽长江滚滚流。\\
\\
年少万\xpinyin*{\xpinyin{兜}{dōu}}\xpinyin*{\xpinyin{鍪}{móu}},坐断东南战未休。\\
天下英雄谁敌手?曹刘。生子当如孙仲谋。
    \end{tabular}
  \end{table}
\end{minipage}
\vspace{1cm}


\ptitle{永遇乐·京口北固亭怀古}\nopagebreak%
\addcontentsline{toc}{section}{\texorpdfstring{\makebox[10cm]{永遇乐·京口北固亭怀古\dotfill{} 【宋】辛弃疾}}{永遇乐·京口北固亭怀古\ 【宋】辛弃疾}}\nopagebreak%
\noindent\begin{minipage}{\linewidth}
  \pauthor{【宋】辛弃疾}
  \vskip-3pt\begin{table}[H]
    \centering
    \begin{tabular}{@{}l@{}}
千古江山,英雄无觅孙仲谋处。\\
舞榭歌台,风流总被雨打风吹去。\\
斜阳草树,寻常巷陌,人道寄奴曾住。\\
 想当年,金戈铁马,气吞万里如虎。\\
\\
元嘉草草,封狼居胥,赢得仓皇北顾。\\
四十三年,望中犹记,烽火扬州路。\\
可堪回首,佛狸祠下,一片神鸦社鼓。\\
 凭谁问,廉颇老矣,尚能饭否?
    \end{tabular}
  \end{table}
\end{minipage}
\vspace{1cm}


\ptitle{破阵子·为陈同甫赋壮语以寄之}\nopagebreak%
\addcontentsline{toc}{section}{\texorpdfstring{\makebox[10cm]{破阵子·为陈同甫赋壮语以寄之\dotfill{} 【宋】辛弃疾}}{破阵子·为陈同甫赋壮语以寄之\ 【宋】辛弃疾}}\nopagebreak%
\noindent\begin{minipage}{\linewidth}
  \pauthor{【宋】辛弃疾}
  \vskip-3pt\begin{table}[H]
    \centering
    \begin{tabular}{@{}l@{}}
醉里挑灯看剑,梦回吹角连营。\\
八百里分麾下炙,五十弦翻塞外声。沙场秋点兵。\\
\\
马作\xpinyin*{\xpinyin{的}{dì}}\xpinyin*{\xpinyin{卢}{lú}}飞快,弓如霹雳弦惊。\\
了却君王天下事,赢得生前身后名。可怜白发生。
    \end{tabular}
  \end{table}
\end{minipage}
\vspace{1cm}


\ptitle{菩萨蛮·书江西造口壁}\nopagebreak%
\addcontentsline{toc}{section}{\texorpdfstring{\makebox[10cm]{菩萨蛮·书江西造口壁\dotfill{} 【宋】辛弃疾}}{菩萨蛮·书江西造口壁\ 【宋】辛弃疾}}\nopagebreak%
\noindent\begin{minipage}{\linewidth}
  \pauthor{【宋】辛弃疾}
  \vskip-3pt\begin{table}[H]
    \centering
    \begin{tabular}{@{}l@{}}
郁孤台下清江水,中间多少行人泪。\\
西北望长安,可怜无数山。\\
\\
青山遮不住,毕竟东流去。\\
江晚正愁余,山深闻鹧鸪。
    \end{tabular}
  \end{table}
\end{minipage}
\vspace{1cm}


\ptitle{青玉案·元夕}\nopagebreak%
\addcontentsline{toc}{section}{\texorpdfstring{\makebox[10cm]{青玉案·元夕\dotfill{} 【宋】辛弃疾}}{青玉案·元夕\ 【宋】辛弃疾}}\nopagebreak%
\noindent\begin{minipage}{\linewidth}
  \pauthor{【宋】辛弃疾}
  \vskip-3pt\begin{table}[H]
    \centering
    \begin{tabular}{@{}l@{}}
东风夜放花千树。更吹落、星如雨。\\
宝马雕车香满路。凤箫声动,玉壶光转,一夜鱼龙舞。\\
\\
蛾儿雪柳黄金缕。笑语盈盈暗香去。\\
众里寻他千百度。蓦然回首,那人却在,灯火阑珊处。
    \end{tabular}
  \end{table}
\end{minipage}
\vspace{1cm}


\ptitle{寒菊}\nopagebreak%
\addcontentsline{toc}{section}{\texorpdfstring{\makebox[10cm]{寒菊\dotfill{} 【宋】郑思肖}}{寒菊\ 【宋】郑思肖}}\nopagebreak%
\noindent\begin{minipage}{\linewidth}
  \pauthor{【宋】郑思肖}
  \vskip-3pt\begin{table}[H]
    \centering
    \begin{tabular}{@{}l@{}}
花开不并百花丛,独立疏篱趣未穷。\\
宁可枝头抱香死,何曾吹落北风中。
    \end{tabular}
  \end{table}
\end{minipage}
\vspace{1cm}


\ptitle{临安春雨初\xpinyin*{\xpinyin{霁}{jì}}}\nopagebreak%
\addcontentsline{toc}{section}{\texorpdfstring{\makebox[10cm]{临安春雨初霁\dotfill{} 【宋】陆游}}{临安春雨初霁\ 【宋】陆游}}\nopagebreak%
\noindent\begin{minipage}{\linewidth}
  \pauthor{【宋】陆游}
  \vskip-3pt\begin{table}[H]
    \centering
    \begin{tabular}{@{}l@{}}
世味年来薄似纱,谁令骑马客京华。小楼一夜听春雨,深巷明朝卖杏花。\\
矮纸斜行闲作草,晴窗细乳戏分茶。素衣莫起风尘叹,犹及清明可到家。
    \end{tabular}
  \end{table}
\end{minipage}
\vspace{1cm}


\ptitle{书愤}\nopagebreak%
\addcontentsline{toc}{section}{\texorpdfstring{\makebox[10cm]{书愤\dotfill{} 【宋】陆游}}{书愤\ 【宋】陆游}}\nopagebreak%
\noindent\begin{minipage}{\linewidth}
  \pauthor{【宋】陆游}
  \vskip-3pt\begin{table}[H]
    \centering
    \begin{tabular}{@{}l@{}}
早岁那知世事艰,中原北望气如山。\\
楼船夜雪瓜洲渡,铁马秋风大散关。\\
塞上长城空自许,镜中衰鬓已先斑。\\
出师一表真名世,千载谁堪伯仲间!
    \end{tabular}
  \end{table}
\end{minipage}
\vspace{1cm}


\ptitle{书愤}\nopagebreak%
\addcontentsline{toc}{section}{\texorpdfstring{\makebox[10cm]{书愤\dotfill{} 【宋】陆游}}{书愤\ 【宋】陆游}}\nopagebreak%
\noindent\begin{minipage}{\linewidth}
  \pauthor{【宋】陆游}
  \vskip-3pt\begin{table}[H]
    \centering
    \begin{tabular}{@{}l@{}}
早岁那知世事艰,中原北望气如山。楼船夜雪瓜洲渡,铁马秋风大散关。\\
塞上长城空自许,镜中衰鬓已先斑。出师一表真名世,千载谁堪伯仲间。
    \end{tabular}
  \end{table}
\end{minipage}
\vspace{1cm}


\ptitle{剑门道中遇微雨}\nopagebreak%
\addcontentsline{toc}{section}{\texorpdfstring{\makebox[10cm]{剑门道中遇微雨\dotfill{} 【宋】陆游}}{剑门道中遇微雨\ 【宋】陆游}}\nopagebreak%
\noindent\begin{minipage}{\linewidth}
  \pauthor{【宋】陆游}
  \vskip-3pt\begin{table}[H]
    \centering
    \begin{tabular}{@{}l@{}}
衣上征尘杂酒痕,远游无处不消魂。\\
此身合是诗人未?细雨骑驴入剑门。
    \end{tabular}
  \end{table}
\end{minipage}
\vspace{1cm}


\ptitle{梅花绝句}\nopagebreak%
\addcontentsline{toc}{section}{\texorpdfstring{\makebox[10cm]{梅花绝句\dotfill{} 【宋】陆游}}{梅花绝句\ 【宋】陆游}}\nopagebreak%
\noindent\begin{minipage}{\linewidth}
  \pauthor{【宋】陆游}
  \vskip-3pt\begin{table}[H]
    \centering
    \begin{tabular}{@{}l@{}}
闻道梅花坼晓风,雪堆遍满四山中。\\
何方可化身千亿,一树梅花一放翁。
    \end{tabular}
  \end{table}
\end{minipage}
\vspace{1cm}


\ptitle{游山西村}\nopagebreak%
\addcontentsline{toc}{section}{\texorpdfstring{\makebox[10cm]{游山西村\dotfill{} 【宋】陆游}}{游山西村\ 【宋】陆游}}\nopagebreak%
\noindent\begin{minipage}{\linewidth}
  \pauthor{【宋】陆游}
  \vskip-3pt\begin{table}[H]
    \centering
    \begin{tabular}{@{}l@{}}
莫笑农家腊酒浑,丰年留客足鸡豚。山重水复疑无路,柳暗花明又一村。\\
箫鼓追随春社近,衣冠简朴古风存。从今若许闲乘月,拄杖无时夜叩门。
    \end{tabular}
  \end{table}
\end{minipage}
\vspace{1cm}


\ptitle{示儿}\nopagebreak%
\addcontentsline{toc}{section}{\texorpdfstring{\makebox[10cm]{示儿\dotfill{} 【宋】陆游}}{示儿\ 【宋】陆游}}\nopagebreak%
\noindent\begin{minipage}{\linewidth}
  \pauthor{【宋】陆游}
  \vskip-3pt\begin{table}[H]
    \centering
    \begin{tabular}{@{}l@{}}
死去元知万事空,但悲不见九州同。\\
王师北定中原日,家祭无忘告乃翁。
    \end{tabular}
  \end{table}
\end{minipage}
\vspace{1cm}


\ptitle{秋夜将晓出篱门迎凉有感}\nopagebreak%
\addcontentsline{toc}{section}{\texorpdfstring{\makebox[10cm]{秋夜将晓出篱门迎凉有感\dotfill{} 【宋】陆游}}{秋夜将晓出篱门迎凉有感\ 【宋】陆游}}\nopagebreak%
\noindent\begin{minipage}{\linewidth}
  \pauthor{【宋】陆游}
  \vskip-3pt\begin{table}[H]
    \centering
    \begin{tabular}{@{}l@{}}
三万里河东入海,五千仞岳上摩天。\\
遗民泪尽胡尘里,南望王师又一年。
    \end{tabular}
  \end{table}
\end{minipage}
\vspace{1cm}


\ptitle{拟行路难·其四}\nopagebreak%
\addcontentsline{toc}{section}{\texorpdfstring{\makebox[10cm]{拟行路难·其四\dotfill{} 【宋】鲍照}}{拟行路难·其四\ 【宋】鲍照}}\nopagebreak%
\noindent\begin{minipage}{\linewidth}
  \pauthor{【宋】鲍照}
  \vskip-3pt\begin{table}[H]
    \centering
    \begin{tabular}{@{}l@{}}
泻水置平地,各自东西南北流。人生亦有命,安能行叹复坐愁?\\
酌酒以自宽,举杯断绝歌路难。心非木石岂无感?吞声\xpinyin*{\xpinyin{踯}{zhí}}\xpinyin*{\xpinyin{躅}{zhú}}不敢言。
    \end{tabular}
  \end{table}
\end{minipage}
\vspace{1cm}


\ptitle{登快阁}\nopagebreak%
\addcontentsline{toc}{section}{\texorpdfstring{\makebox[10cm]{登快阁\dotfill{} 【宋】黄庭坚}}{登快阁\ 【宋】黄庭坚}}\nopagebreak%
\noindent\begin{minipage}{\linewidth}
  \pauthor{【宋】黄庭坚}
  \vskip-3pt\begin{table}[H]
    \centering
    \begin{tabular}{@{}l@{}}
痴儿了却公家事,快阁东西倚晚晴。落木千山天远大,澄江一道月分明。\\
朱弦已为佳人绝,青眼聊因美酒横。万里归船弄长笛,此心吾与白鸥盟。
    \end{tabular}
  \end{table}
\end{minipage}
\vspace{1cm}


\ptitle{宋 辛弃疾}\nopagebreak%
\addcontentsline{toc}{section}{\texorpdfstring{\makebox[10cm]{宋 辛弃疾\dotfill{} 少年不识愁滋味,爱上层楼。}}{宋 辛弃疾\ 少年不识愁滋味,爱上层楼。}}\nopagebreak%
\noindent\begin{minipage}{\linewidth}
  \pauthor{少年不识愁滋味,爱上层楼。}
  \vskip-3pt\begin{table}[H]
    \centering
    \begin{tabular}{@{}l@{}}
爱上层楼。为赋新词强说愁。\\
而今识尽愁滋味,欲说还休。\\
欲说还休。却道天凉好个秋。
    \end{tabular}
  \end{table}
\end{minipage}
\vspace{1cm}


\chapter{元}
\ptitle{墨梅}\nopagebreak%
\addcontentsline{toc}{section}{\texorpdfstring{\makebox[10cm]{墨梅\dotfill{} 【元】王冕}}{墨梅\ 【元】王冕}}\nopagebreak%
\noindent\begin{minipage}{\linewidth}
  \pauthor{【元】王冕}
  \vskip-3pt\begin{table}[H]
    \centering
    \begin{tabular}{@{}l@{}}
吾家洗砚池头树, 个个花开淡墨痕。\\
不要人夸好颜色, 只留清气满乾坤。
    \end{tabular}
  \end{table}
\end{minipage}
\vspace{1cm}


\ptitle{梅花}\nopagebreak%
\addcontentsline{toc}{section}{\texorpdfstring{\makebox[10cm]{梅花\dotfill{} 【元】王冕}}{梅花\ 【元】王冕}}\nopagebreak%
\noindent\begin{minipage}{\linewidth}
  \pauthor{【元】王冕}
  \vskip-3pt\begin{table}[H]
    \centering
    \begin{tabular}{@{}l@{}}
三月东风吹雪消,湖南山色翠如浇。\\
一声羌管无人见,无数梅花落野桥。
    \end{tabular}
  \end{table}
\end{minipage}
\vspace{1cm}


\ptitle{摸鱼儿}\nopagebreak%
\addcontentsline{toc}{section}{\texorpdfstring{\makebox[10cm]{摸鱼儿\dotfill{} 【元】元好问}}{摸鱼儿\ 【元】元好问}}\nopagebreak%
\noindent\begin{minipage}{\linewidth}
  \pauthor{【元】元好问}
  \vskip-3pt\begin{table}[H]
    \centering
    \begin{tabular}{@{}l@{}}
问莲根、有丝多少,莲心知为谁苦?\\
双花脉脉娇相向,只是旧家儿女。\\
天已许。甚不教、白头生死鸳鸯浦?\\
夕阳无语。\\
算谢客烟中,湘妃江上,未是断肠处。\\
\\
香奁梦,好在灵芝瑞露。人间俯仰今古。\\
海枯石烂情缘在,幽恨不埋黄土。\\
相思树,流年度,无端又被西风误。\\
兰舟少住。\\
怕载酒重来,红衣半落,狼藉卧风雨。
    \end{tabular}
  \end{table}
\end{minipage}
\vspace{1cm}


\ptitle{山坡羊·潼关怀古}\nopagebreak%
\addcontentsline{toc}{section}{\texorpdfstring{\makebox[10cm]{山坡羊·潼关怀古\dotfill{} 【元】张养浩}}{山坡羊·潼关怀古\ 【元】张养浩}}\nopagebreak%
\noindent\begin{minipage}{\linewidth}
  \pauthor{【元】张养浩}
  \vskip-3pt\begin{table}[H]
    \centering
    \begin{tabular}{@{}l@{}}
峰峦如聚,波涛如怒,山河表里潼关路。\\
望西都,意踌躇,伤心秦汉经行处,宫阙万间都做了土。\\
兴,百姓苦;亡,百姓苦!
    \end{tabular}
  \end{table}
\end{minipage}
\vspace{1cm}


\ptitle{墨梅}\nopagebreak%
\addcontentsline{toc}{section}{\texorpdfstring{\makebox[10cm]{墨梅\dotfill{} 【元】王冕}}{墨梅\ 【元】王冕}}\nopagebreak%
\noindent\begin{minipage}{\linewidth}
  \pauthor{【元】王\xpinyin*{\xpinyin{冕}{miǎn}}}
  \vskip-3pt\begin{table}[H]
    \centering
    \begin{tabular}{@{}l@{}}
我家洗\xpinyin*{\xpinyin{砚}{yàn}}池头树,朵朵花开淡墨痕。\\
不要人夸颜色好,只留清气满乾坤。
    \end{tabular}
  \end{table}
\end{minipage}
\vspace{1cm}


\ptitle{长亭送别}\nopagebreak%
\addcontentsline{toc}{section}{\texorpdfstring{\makebox[10cm]{长亭送别\dotfill{} 【元】王实甫}}{长亭送别\ 【元】王实甫}}\nopagebreak%
\noindent\begin{minipage}{\linewidth}
  \pauthor{【元】王实甫}
  \vskip-3pt\begin{table}[H]
    \centering
    \begin{tabular}{@{}l@{}}
碧云天,黄花地,西风紧。北雁南飞。\\
晓来谁染霜林醉?总是离人泪。
    \end{tabular}
  \end{table}
\end{minipage}
\vspace{1cm}


\ptitle{天净沙·秋思}\nopagebreak%
\addcontentsline{toc}{section}{\texorpdfstring{\makebox[10cm]{天净沙·秋思\dotfill{} 【元】马致远}}{天净沙·秋思\ 【元】马致远}}\nopagebreak%
\noindent\begin{minipage}{\linewidth}
  \pauthor{【元】马致远}
  \vskip-3pt\begin{table}[H]
    \centering
    \begin{tabular}{@{}l@{}}
枯藤老树昏鸦,小桥流水人家,古道西风瘦马。\\
夕阳西下,断肠人在天涯。
    \end{tabular}
  \end{table}
\end{minipage}
\vspace{1cm}


\chapter{明}
\ptitle{石灰吟}\nopagebreak%
\addcontentsline{toc}{section}{\texorpdfstring{\makebox[10cm]{石灰吟\dotfill{} 【明】于谦}}{石灰吟\ 【明】于谦}}\nopagebreak%
\noindent\begin{minipage}{\linewidth}
  \pauthor{【明】于谦}
  \vskip-3pt\begin{table}[H]
    \centering
    \begin{tabular}{@{}l@{}}
千锤万凿出深山,烈火焚烧若等闲。\\
粉骨碎身全不怕,要留清白在人间。
    \end{tabular}
  \end{table}
\end{minipage}
\vspace{1cm}


\ptitle{桃花庵歌}\nopagebreak%
\addcontentsline{toc}{section}{\texorpdfstring{\makebox[10cm]{桃花庵歌\dotfill{} 【明】唐寅}}{桃花庵歌\ 【明】唐寅}}\nopagebreak%
\noindent\begin{minipage}{\linewidth}
  \pauthor{【明】唐寅}
  \vskip-3pt\begin{table}[H]
    \centering
    \begin{tabular}{@{}l@{}}
桃花坞里桃花庵,桃花庵下桃花仙。\\
桃花仙人种桃树,又折花枝当酒钱。\\
酒醒只在花前坐,酒醉还须花下眠。\\
花前花后日复日,酒醉酒醒年复年。\\
不愿鞠躬车马前,但愿老死花酒间。\\
车尘马足贵者趣,酒盏花枝贫者缘。\\
若将富贵比贫贱,一在平地一在天。\\
若将贫贱比车马,他得驱驰我得闲。\\
世人笑我\xpinyin*{\xpinyin{忒}{tēi}}疯癫,我笑世人看不穿。\\
记得五陵豪杰墓,无酒无花锄作田。
    \end{tabular}
  \end{table}
\end{minipage}
\vspace{1cm}


\ptitle{临江仙}\nopagebreak%
\addcontentsline{toc}{section}{\texorpdfstring{\makebox[10cm]{临江仙\dotfill{} 【明】杨慎}}{临江仙\ 【明】杨慎}}\nopagebreak%
\noindent\begin{minipage}{\linewidth}
  \pauthor{【明】杨慎}
  \vskip-3pt\begin{table}[H]
    \centering
    \begin{tabular}{@{}l@{}}
滚滚长江东逝水,浪花淘尽英雄。\\
是非成败转头空。青山依旧在,几度夕阳红。\\
\\
白发渔樵江渚上,惯看秋月春风。\\
一壶浊酒喜相逢。古今多少事,都付笑谈中。
    \end{tabular}
  \end{table}
\end{minipage}
\vspace{1cm}


\ptitle{朝天子·咏喇叭}\nopagebreak%
\addcontentsline{toc}{section}{\texorpdfstring{\makebox[10cm]{朝天子·咏喇叭\dotfill{} 【明】王磐}}{朝天子·咏喇叭\ 【明】王磐}}\nopagebreak%
\noindent\begin{minipage}{\linewidth}
  \pauthor{【明】王磐}
  \vskip-3pt\begin{table}[H]
    \centering
    \begin{tabular}{@{}l@{}}
喇叭,锁呐,曲儿小腔儿大。\\
官船来往乱如麻,全仗你抬声价。\\
军听了军愁,民听了民怕。\\
那里去辨甚么真共假?\\
眼见的吹翻了这家,吹伤了那家,只吹的水尽鹅飞罢!
    \end{tabular}
  \end{table}
\end{minipage}
\vspace{1cm}


\ptitle{明日歌}\nopagebreak%
\addcontentsline{toc}{section}{\texorpdfstring{\makebox[10cm]{明日歌\dotfill{} 【明】钱福}}{明日歌\ 【明】钱福}}\nopagebreak%
\noindent\begin{minipage}{\linewidth}
  \pauthor{【明】钱福}
  \vskip-3pt\begin{table}[H]
    \centering
    \begin{tabular}{@{}l@{}}
明日复明日,明日何其多。\\
我生待明日,万事成蹉跎。\\
世人若被明日累,春去秋来老将至。\\
朝看水东流,暮看日西坠。\\
百年明日能几何?请君听我明日歌。
    \end{tabular}
  \end{table}
\end{minipage}
\vspace{1cm}


\chapter{清}
\ptitle{题秋江独钓图}\nopagebreak%
\addcontentsline{toc}{section}{\texorpdfstring{\makebox[10cm]{题秋江独钓图\dotfill{} 【清】王士祯}}{题秋江独钓图\ 【清】王士祯}}\nopagebreak%
\noindent\begin{minipage}{\linewidth}
  \pauthor{【清】王士祯}
  \vskip-3pt\begin{table}[H]
    \centering
    \begin{tabular}{@{}l@{}}
一蓑一笠一扁舟,一丈丝纶一寸钩。\\
一曲高歌一樽酒,一人独钓一江秋。
    \end{tabular}
  \end{table}
\end{minipage}
\vspace{1cm}


\ptitle{出塞}\nopagebreak%
\addcontentsline{toc}{section}{\texorpdfstring{\makebox[10cm]{出塞\dotfill{} 【清】徐锡麟}}{出塞\ 【清】徐锡麟}}\nopagebreak%
\noindent\begin{minipage}{\linewidth}
  \pauthor{【清】徐锡麟}
  \vskip-3pt\begin{table}[H]
    \centering
    \begin{tabular}{@{}l@{}}
军歌应唱大刀环,誓灭胡奴出玉关。\\
只解沙场为国死,何须马革裹尸还。
    \end{tabular}
  \end{table}
\end{minipage}
\vspace{1cm}


\ptitle{五美吟·明妃}\nopagebreak%
\addcontentsline{toc}{section}{\texorpdfstring{\makebox[10cm]{五美吟·明妃\dotfill{} 【清】曹雪芹}}{五美吟·明妃\ 【清】曹雪芹}}\nopagebreak%
\noindent\begin{minipage}{\linewidth}
  \pauthor{【清】曹雪芹}
  \vskip-3pt\begin{table}[H]
    \centering
    \begin{tabular}{@{}l@{}}
绝艳惊人出汉宫,红颜命薄古今同。\\
君王纵使轻颜色,予夺权何\xpinyin*{\xpinyin{畀}{bì}}画工?
    \end{tabular}
  \end{table}
\end{minipage}
\vspace{1cm}


\ptitle{五美吟·西施}\nopagebreak%
\addcontentsline{toc}{section}{\texorpdfstring{\makebox[10cm]{五美吟·西施\dotfill{} 【清】曹雪芹}}{五美吟·西施\ 【清】曹雪芹}}\nopagebreak%
\noindent\begin{minipage}{\linewidth}
  \pauthor{【清】曹雪芹}
  \vskip-3pt\begin{table}[H]
    \centering
    \begin{tabular}{@{}l@{}}
一代倾城逐浪花,吴宫空自忆儿家。\\
效颦莫笑东村女,头白西边上浣纱。
    \end{tabular}
  \end{table}
\end{minipage}
\vspace{1cm}


\ptitle{本事诗十首·选二}\nopagebreak%
\addcontentsline{toc}{section}{\texorpdfstring{\makebox[10cm]{本事诗十首·选二\dotfill{} 【清】苏曼殊}}{本事诗十首·选二\ 【清】苏曼殊}}\nopagebreak%
\noindent\begin{minipage}{\linewidth}
  \pauthor{【清】苏曼殊}
  \vskip-3pt\begin{table}[H]
    \centering
    \begin{tabular}{@{}l@{}}
乌舍凌波肌似雪,亲持红叶索题诗。\\
还卿一\xpinyin*{\xpinyin{钵}{bō}}无情泪,恨不相逢未剃时。
    \end{tabular}
  \end{table}
\end{minipage}
\vspace{1cm}


\ptitle{偶作}\nopagebreak%
\addcontentsline{toc}{section}{\texorpdfstring{\makebox[10cm]{偶作\dotfill{} 【清】袁枚}}{偶作\ 【清】袁枚}}\nopagebreak%
\noindent\begin{minipage}{\linewidth}
  \pauthor{【清】袁枚}
  \vskip-3pt\begin{table}[H]
    \centering
    \begin{tabular}{@{}l@{}}
偶寻半开梅,闲倚一竿竹。\\
儿童不知春,问草何故绿。
    \end{tabular}
  \end{table}
\end{minipage}
\vspace{1cm}


\ptitle{所见}\nopagebreak%
\addcontentsline{toc}{section}{\texorpdfstring{\makebox[10cm]{所见\dotfill{} 【清】袁枚}}{所见\ 【清】袁枚}}\nopagebreak%
\noindent\begin{minipage}{\linewidth}
  \pauthor{【清】袁枚}
  \vskip-3pt\begin{table}[H]
    \centering
    \begin{tabular}{@{}l@{}}
牧童骑黄牛,歌声振林\xpinyin*{\xpinyin{樾}{yuè}}。\\
意欲捕鸣蝉,忽然闭口立。
    \end{tabular}
  \end{table}
\end{minipage}
\vspace{1cm}


\ptitle{苔}\nopagebreak%
\addcontentsline{toc}{section}{\texorpdfstring{\makebox[10cm]{苔\dotfill{} 【清】袁枚}}{苔\ 【清】袁枚}}\nopagebreak%
\noindent\begin{minipage}{\linewidth}
  \pauthor{【清】袁枚}
  \vskip-3pt\begin{table}[H]
    \centering
    \begin{tabular}{@{}l@{}}
白日不到处,青春恰自来。\\
苔花如米小,也学牡丹开。
    \end{tabular}
  \end{table}
\end{minipage}
\vspace{1cm}


\ptitle{论诗五首}\nopagebreak%
\addcontentsline{toc}{section}{\texorpdfstring{\makebox[10cm]{论诗五首\dotfill{} 【清】赵翼}}{论诗五首\ 【清】赵翼}}\nopagebreak%
\noindent\begin{minipage}{\linewidth}
  \pauthor{【清】赵翼}
  \vskip-3pt\begin{table}[H]
    \centering
    \begin{tabular}{@{}l@{}}
李杜诗篇万口传,至今已觉不新鲜。\\
江山代有才人出,各领风骚数百年。
    \end{tabular}
  \end{table}
\end{minipage}
\vspace{1cm}


\ptitle{竹石}\nopagebreak%
\addcontentsline{toc}{section}{\texorpdfstring{\makebox[10cm]{竹石\dotfill{} 【清】郑燮}}{竹石\ 【清】郑燮}}\nopagebreak%
\noindent\begin{minipage}{\linewidth}
  \pauthor{【清】郑\xpinyin*{\xpinyin{燮}{xiè}}}
  \vskip-3pt\begin{table}[H]
    \centering
    \begin{tabular}{@{}l@{}}
咬定青山不放松,立根原在破岩中。\\
千磨万击还坚劲,任尔东西南北风。
    \end{tabular}
  \end{table}
\end{minipage}
\vspace{1cm}


\ptitle{竹石}\nopagebreak%
\addcontentsline{toc}{section}{\texorpdfstring{\makebox[10cm]{竹石\dotfill{} 【清】郑燮}}{竹石\ 【清】郑燮}}\nopagebreak%
\noindent\begin{minipage}{\linewidth}
  \pauthor{【清】郑\xpinyin*{\xpinyin{燮}{xiè}}}
  \vskip-3pt\begin{table}[H]
    \centering
    \begin{tabular}{@{}l@{}}
咬定青山不放松,立根原在破岩中。\\
千磨万击还坚劲,任尔东西南北风!
    \end{tabular}
  \end{table}
\end{minipage}
\vspace{1cm}


\ptitle{一字诗}\nopagebreak%
\addcontentsline{toc}{section}{\texorpdfstring{\makebox[10cm]{一字诗\dotfill{} 【清】陈沆}}{一字诗\ 【清】陈沆}}\nopagebreak%
\noindent\begin{minipage}{\linewidth}
  \pauthor{【清】陈\xpinyin*{\xpinyin{沆}{hàng}}}
  \vskip-3pt\begin{table}[H]
    \centering
    \begin{tabular}{@{}l@{}}
一帆一桨一渔舟,一个渔翁一钓钩。\\
一俯一仰一场笑,一江明月一江秋。
    \end{tabular}
  \end{table}
\end{minipage}
\vspace{1cm}


\ptitle{村居}\nopagebreak%
\addcontentsline{toc}{section}{\texorpdfstring{\makebox[10cm]{村居\dotfill{} 【清】高鼎}}{村居\ 【清】高鼎}}\nopagebreak%
\noindent\begin{minipage}{\linewidth}
  \pauthor{【清】高鼎}
  \vskip-3pt\begin{table}[H]
    \centering
    \begin{tabular}{@{}l@{}}
草长莺飞二月天,拂堤杨柳醉春烟。\\
儿童散学归来早,忙趁东风放纸\xpinyin*{\xpinyin{鸢}{yuān}}。
    \end{tabular}
  \end{table}
\end{minipage}
\vspace{1cm}


\ptitle{绮怀十六首·其十五}\nopagebreak%
\addcontentsline{toc}{section}{\texorpdfstring{\makebox[10cm]{绮怀十六首·其十五\dotfill{} 【清】黄景仁}}{绮怀十六首·其十五\ 【清】黄景仁}}\nopagebreak%
\noindent\begin{minipage}{\linewidth}
  \pauthor{【清】黄景仁}
  \vskip-3pt\begin{table}[H]
    \centering
    \begin{tabular}{@{}l@{}}
几回花下坐吹箫,银汉红墙入望遥。\\
似此星辰非昨夜,为谁风露立中宵。\\
缠绵思尽抽残茧,宛转心伤剥后蕉。\\
三五年时三五月,可怜杯酒不曾消。
    \end{tabular}
  \end{table}
\end{minipage}
\vspace{1cm}


\ptitle{己亥杂诗}\nopagebreak%
\addcontentsline{toc}{section}{\texorpdfstring{\makebox[10cm]{己亥杂诗\dotfill{} 【清】龚自珍}}{己亥杂诗\ 【清】龚自珍}}\nopagebreak%
\noindent\begin{minipage}{\linewidth}
  \pauthor{【清】龚自珍}
  \vskip-3pt\begin{table}[H]
    \centering
    \begin{tabular}{@{}l@{}}
九州生气恃风雷,万马齐\xpinyin*{\xpinyin{喑}{yīn}}究可哀。\\
我劝天公重抖擞,不拘一格降人才。
    \end{tabular}
  \end{table}
\end{minipage}
\vspace{1cm}


\ptitle{己亥杂诗·其五}\nopagebreak%
\addcontentsline{toc}{section}{\texorpdfstring{\makebox[10cm]{己亥杂诗·其五\dotfill{} 【清】龚自珍}}{己亥杂诗·其五\ 【清】龚自珍}}\nopagebreak%
\noindent\begin{minipage}{\linewidth}
  \pauthor{【清】\xpinyin*{\xpinyin{龚}{gōng}}自珍}
  \vskip-3pt\begin{table}[H]
    \centering
    \begin{tabular}{@{}l@{}}
浩荡离愁白日斜,吟鞭东指即天涯。\\
落红不是无情物,化作春泥更护花。
    \end{tabular}
  \end{table}
\end{minipage}
\vspace{1cm}


\chapter{近代}
\ptitle{七律·人民解放军占领南京}\nopagebreak%
\addcontentsline{toc}{section}{\texorpdfstring{\makebox[10cm]{七律·人民解放军占领南京\dotfill{} 毛泽东}}{七律·人民解放军占领南京\ 毛泽东}}\nopagebreak%
\noindent\begin{minipage}{\linewidth}
  \pauthor{毛泽东}
  \vskip-3pt\begin{table}[H]
    \centering
    \begin{tabular}{@{}l@{}}
钟山风雨起苍黄,百万雄师过大江。\\
虎\xpinyin*{\xpinyin{踞}{jù}}龙盘今胜昔,天翻地\xpinyin*{\xpinyin{覆}{fù}}\xpinyin*{\xpinyin{慨}{kǎi}}而\xpinyin*{\xpinyin{慷}{kāng}} 。\\
宜将\xpinyin*{\xpinyin{剩}{shèng}}勇追穷\xpinyin*{\xpinyin{寇}{kòu}} ,不可\xpinyin*{\xpinyin{沽}{gū}}名学霸王。\\
天若有情天亦老,人间正道是\xpinyin*{\xpinyin{沧}{cāng}}桑。
    \end{tabular}
  \end{table}
\end{minipage}
\vspace{1cm}


\ptitle{七律·到韶山}\nopagebreak%
\addcontentsline{toc}{section}{\texorpdfstring{\makebox[10cm]{七律·到韶山\dotfill{} 毛泽东}}{七律·到韶山\ 毛泽东}}\nopagebreak%
\noindent\begin{minipage}{\linewidth}
  \pauthor{毛泽东}
  \vskip-3pt\begin{table}[H]
    \centering
    \begin{tabular}{@{}l@{}}
别梦依稀\xpinyin*{\xpinyin{咒}{zhòu}}逝川,故园三十二年前。\\
红旗卷起农奴\xpinyin*{\xpinyin{戟}{jǐ}},黑手高悬霸主鞭。\\
为有牺牲多壮志,敢叫日月换新天。\\
喜看稻\xpinyin*{\xpinyin{菽}{shū}}千重浪,遍地英雄下夕烟。
    \end{tabular}
  \end{table}
\end{minipage}
\vspace{1cm}


\ptitle{七律·长征}\nopagebreak%
\addcontentsline{toc}{section}{\texorpdfstring{\makebox[10cm]{七律·长征\dotfill{} 毛泽东}}{七律·长征\ 毛泽东}}\nopagebreak%
\noindent\begin{minipage}{\linewidth}
  \pauthor{毛泽东}
  \vskip-3pt\begin{table}[H]
    \centering
    \begin{tabular}{@{}l@{}}
红军不怕远征难,万水千山只等闲。\\
五岭\xpinyin*{\xpinyin{逶}{wēi}}\xpinyin*{\xpinyin{迤}{yí}}腾细浪,乌蒙\xpinyin*{\xpinyin{磅}{páng}}\xpinyin*{\xpinyin{礴}{bó}}走泥丸。\\
金沙水拍云崖暖,大渡桥横铁索寒。\\
更喜\xpinyin*{\xpinyin{岷}{mín}}山千里雪,三军过后尽开颜。
    \end{tabular}
  \end{table}
\end{minipage}
\vspace{1cm}


\ptitle{水调歌头·游泳}\nopagebreak%
\addcontentsline{toc}{section}{\texorpdfstring{\makebox[10cm]{水调歌头·游泳\dotfill{} 毛泽东}}{水调歌头·游泳\ 毛泽东}}\nopagebreak%
\noindent\begin{minipage}{\linewidth}
  \pauthor{毛泽东}
  \vskip-3pt\begin{table}[H]
    \centering
    \begin{tabular}{@{}l@{}}
才饮长沙水,又食武昌鱼。\\
万里长江横渡,极目楚天舒。\\
不管风吹浪打,胜似闲庭信步,今日得宽\xpinyin*{\xpinyin{馀}{yú}} 。\\
子在川上曰:逝者如斯夫!\\
\\
风\xpinyin*{\xpinyin{樯}{qiáng}}动,龟蛇静,起宏图。\\
一桥飞架南北,天堑变通途。\\
更立西江石壁,截断巫山云雨,高峡出平湖。\\
神女应无\xpinyin*{\xpinyin{恙}{yàng}} ,当惊世界殊。
    \end{tabular}
  \end{table}
\end{minipage}
\vspace{1cm}


\ptitle{清平乐·六盘山}\nopagebreak%
\addcontentsline{toc}{section}{\texorpdfstring{\makebox[10cm]{清平乐·六盘山\dotfill{} 毛泽东}}{清平乐·六盘山\ 毛泽东}}\nopagebreak%
\noindent\begin{minipage}{\linewidth}
  \pauthor{毛泽东}
  \vskip-3pt\begin{table}[H]
    \centering
    \begin{tabular}{@{}l@{}}
天高云淡,望断南飞雁。\\
不到长城非好汉,屈指行程二万。\\
六盘山上高峰,红旗漫卷西风。\\
今日长\xpinyin*{\xpinyin{缨}{yīng}}在手,何时\xpinyin*{\xpinyin{缚}{fù}}住苍龙?
    \end{tabular}
  \end{table}
\end{minipage}
\vspace{1cm}


\ptitle{采桑子·重阳}\nopagebreak%
\addcontentsline{toc}{section}{\texorpdfstring{\makebox[10cm]{采桑子·重阳\dotfill{} 毛泽东}}{采桑子·重阳\ 毛泽东}}\nopagebreak%
\noindent\begin{minipage}{\linewidth}
  \pauthor{毛泽东}
  \vskip-3pt\begin{table}[H]
    \centering
    \begin{tabular}{@{}l@{}}
人生易老天难老,岁岁重阳。\\
今又重阳,战地黄花分外香。\\
一年一度秋风劲,不似春光。\\
胜似春光,\xpinyin*{\xpinyin{廖}{liào}}\xpinyin*{\xpinyin{廓}{kuò}}江天万里霜。
    \end{tabular}
  \end{table}
\end{minipage}
\vspace{1cm}


\ptitle{七绝·赠父诗}\nopagebreak%
\addcontentsline{toc}{section}{\texorpdfstring{\makebox[10cm]{七绝·赠父诗\dotfill{} 【现代】毛泽东}}{七绝·赠父诗\ 【现代】毛泽东}}\nopagebreak%
\noindent\begin{minipage}{\linewidth}
  \pauthor{【现代】毛泽东}
  \vskip-3pt\begin{table}[H]
    \centering
    \begin{tabular}{@{}l@{}}
孩儿立志出乡关,学不成名誓不还。\\
埋骨何须桑梓地,人生无处不青山。
    \end{tabular}
  \end{table}
\end{minipage}
\vspace{1cm}


\ptitle{忆秦娥·娄山关}\nopagebreak%
\addcontentsline{toc}{section}{\texorpdfstring{\makebox[10cm]{忆秦娥·娄山关\dotfill{} 【现代】毛泽东}}{忆秦娥·娄山关\ 【现代】毛泽东}}\nopagebreak%
\noindent\begin{minipage}{\linewidth}
  \pauthor{【现代】毛泽东}
  \vskip-3pt\begin{table}[H]
    \centering
    \begin{tabular}{@{}l@{}}
西风烈,长空雁叫霜晨月。\\
霜晨月,马蹄声碎,喇叭声咽。\\
\\
雄关漫道真如铁,而今迈步从头越。\\
从头越,苍山如海,残阳如血。
    \end{tabular}
  \end{table}
\end{minipage}
\vspace{1cm}


\ptitle{沁园春·长沙}\nopagebreak%
\addcontentsline{toc}{section}{\texorpdfstring{\makebox[10cm]{沁园春·长沙\dotfill{} 【现代】毛泽东}}{沁园春·长沙\ 【现代】毛泽东}}\nopagebreak%
\noindent\begin{minipage}{\linewidth}
  \pauthor{【现代】毛泽东}
  \vskip-3pt\begin{table}[H]
    \centering
    \begin{tabular}{@{}l@{}}
独立寒秋,湘江北去,\xpinyin*{\xpinyin{橘}{jú}}子洲头。\\
看万山红遍,层林尽染;\\
漫江碧透,百\xpinyin*{\xpinyin{舸}{gě}}争流。\\
鹰击长空,鱼翔浅底,万类霜天竞自由。\\
\xpinyin*{\xpinyin{怅}{chàng}}\xpinyin*{\xpinyin{寥}{liào}}\xpinyin*{\xpinyin{廓}{kuò}} ,问苍茫大地,谁主沉浮?\\
\\
\xpinyin*{\xpinyin{携}{xié}}来百侣曾游,忆往昔\xpinyin*{\xpinyin{峥}{zhēng}}\xpinyin*{\xpinyin{嵘}{róng}}岁月\xpinyin*{\xpinyin{稠}{chóu}} 。\\
恰同学少年,风华正茂;\\
书生意气,挥斥方\xpinyin*{\xpinyin{遒}{qiú}} 。\\
指点江山,激扬文字,粪土当年万户侯。\\
曾记否,到中流击水,浪\xpinyin*{\xpinyin{遏}{è}}飞舟?
    \end{tabular}
  \end{table}
\end{minipage}
\vspace{1cm}


\ptitle{沁园春·雪}\nopagebreak%
\addcontentsline{toc}{section}{\texorpdfstring{\makebox[10cm]{沁园春·雪\dotfill{} 【现代】毛泽东}}{沁园春·雪\ 【现代】毛泽东}}\nopagebreak%
\noindent\begin{minipage}{\linewidth}
  \pauthor{【现代】毛泽东}
  \vskip-3pt\begin{table}[H]
    \centering
    \begin{tabular}{@{}l@{}}
北国风光,千里冰封,万里雪飘。\\
望长城内外,惟余莽莽;大河上下,顿失滔滔。\\
山舞银蛇,原驰蜡象,欲与天公试比高。\\
\\
须晴日,看红装素裹,分外妖娆。\\
江山如此多娇,引无数英雄竞折腰。\\
惜秦皇汉武,略输文采;唐宗宋祖,稍逊风骚。\\
一代天骄,成吉思汗,只识弯弓射大雕。\\
俱往矣,数风流人物,还看今朝。
    \end{tabular}
  \end{table}
\end{minipage}
\vspace{1cm}


\ptitle{咏松}\nopagebreak%
\addcontentsline{toc}{section}{\texorpdfstring{\makebox[10cm]{咏松\dotfill{} 【现代】陈毅}}{咏松\ 【现代】陈毅}}\nopagebreak%
\noindent\begin{minipage}{\linewidth}
  \pauthor{【现代】陈毅}
  \vskip-3pt\begin{table}[H]
    \centering
    \begin{tabular}{@{}l@{}}
大雪压青松,青松挺且直。\\
欲知松高洁,待到雪化时。
    \end{tabular}
  \end{table}
\end{minipage}
\vspace{1cm}


\ptitle{卜算子·咏梅}\nopagebreak%
\addcontentsline{toc}{section}{\texorpdfstring{\makebox[10cm]{卜算子·咏梅\dotfill{} 【近代】毛泽东}}{卜算子·咏梅\ 【近代】毛泽东}}\nopagebreak%
\noindent\begin{minipage}{\linewidth}
  \pauthor{【近代】毛泽东}
  \vskip-3pt\begin{table}[H]
    \centering
    \begin{tabular}{@{}l@{}}
风雨送春归,飞雪迎春到。\\
已是悬崖百丈冰,犹有花枝俏。\\
\\
俏也不争春,只把春来报。\\
待到山花烂漫时,她在丛中笑。
    \end{tabular}
  \end{table}
\end{minipage}
\vspace{1cm}


\ptitle{沁园春·雪}\nopagebreak%
\addcontentsline{toc}{section}{\texorpdfstring{\makebox[10cm]{沁园春·雪\dotfill{} 【近代】毛泽东}}{沁园春·雪\ 【近代】毛泽东}}\nopagebreak%
\noindent\begin{minipage}{\linewidth}
  \pauthor{【近代】毛泽东}
  \vskip-3pt\begin{table}[H]
    \centering
    \begin{tabular}{@{}l@{}}
北国风光,千里冰封,万里雪飘。\\
望长城内外,惟余莽\xpinyin*{\xpinyin{莽}{mǎng}} ;\\
大河上下,顿失滔滔。\\
山舞银蛇,原\xpinyin*{\xpinyin{驰}{chí}}蜡象,欲与天公试比高。\\
须晴日,看红装素裹,分外妖\xpinyin*{\xpinyin{娆}{ráo}} 。\\
\\
江山如此多娇,引无数英雄竞折腰。\\
惜秦皇汉武,略输文采;\\
唐宗宋祖,稍\xpinyin*{\xpinyin{逊}{xùn}}风\xpinyin*{\xpinyin{骚}{sāo}} 。\\
一代天骄,成吉思\xpinyin*{\xpinyin{汗}{hán}} ,只识弯弓射大\xpinyin*{\xpinyin{雕}{diāo}} 。\\
俱往矣,数风流人物,还看今朝。
    \end{tabular}
  \end{table}
\end{minipage}
\vspace{1cm}


\ptitle{忆秦娥·娄山关}\nopagebreak%
\addcontentsline{toc}{section}{\texorpdfstring{\makebox[10cm]{忆秦娥·娄山关\dotfill{} 【近代】毛泽东}}{忆秦娥·娄山关\ 【近代】毛泽东}}\nopagebreak%
\noindent\begin{minipage}{\linewidth}
  \pauthor{【近代】毛泽东}
  \vskip-3pt\begin{table}[H]
    \centering
    \begin{tabular}{@{}l@{}}
西风烈,长空雁叫霜晨月。\\
霜晨月,马蹄声碎,喇叭声咽。\\
\\
雄关漫道真如铁,而今迈步从头越。\\
从头越,苍山如海,残阳如血。
    \end{tabular}
  \end{table}
\end{minipage}
\vspace{1cm}


\ptitle{满江红}\nopagebreak%
\addcontentsline{toc}{section}{\texorpdfstring{\makebox[10cm]{满江红\dotfill{} 【近代】秋瑾}}{满江红\ 【近代】秋瑾}}\nopagebreak%
\noindent\begin{minipage}{\linewidth}
  \pauthor{【近代】秋瑾}
  \vskip-3pt\begin{table}[H]
    \centering
    \begin{tabular}{@{}l@{}}
小住京华,早又是、中秋佳节。\\
为篱下、黄花开遍,秋容如拭。\\
四面歌残终破楚,八年风味徒思浙。\\
苦将侬、强派作娥眉,殊未屑!\\
\\
身不得,男儿列,心却比,男儿烈。\\
算平生肝胆,因人常热。\\
俗子胸襟谁识我?英雄末路当磨折。\\
莽红尘,何处觅知音?青衫湿!
    \end{tabular}
  \end{table}
\end{minipage}
\vspace{1cm}


ERROR!!!


\end{document}

