\noindent\begin{minipage}{\linewidth}
  \ptitle{卜算子·咏梅}
  \addcontentsline{toc}{section}{\makebox[9cm]{卜算子·咏梅\dotfill{} 毛泽东}}
  \pauthor{【近代 】毛泽东}
  \vskip-3pt\begin{table}[H]
    \centering
    \begin{tabular}{@{}l@{}}
风雨送春归,飞雪迎春到。\\
已是悬崖百丈冰,犹有花枝俏。\\
\\
俏也不争春,只把春来报。\\
待到山花烂漫时,她在丛中笑。
    \end{tabular}
  \end{table}
\end{minipage}
\vspace{1cm}


\noindent\begin{minipage}{\linewidth}
  \ptitle{采桑子·重阳}
  \addcontentsline{toc}{section}{\makebox[9cm]{采桑子·重阳\dotfill{} 毛泽东}}
  \pauthor{毛泽东}
  \vskip-3pt\begin{table}[H]
    \centering
    \begin{tabular}{@{}l@{}}
人生易老天难老,岁岁重阳。\\
今又重阳,战地黄花分外香。\\
一年一度秋风劲,不似春光。\\
胜似春光,\xpinyin*{\xpinyin{廖}{liào}}\xpinyin*{\xpinyin{廓}{kuò}}江天万里霜。
    \end{tabular}
  \end{table}
\end{minipage}
\vspace{1cm}


\noindent\begin{minipage}{\linewidth}
  \ptitle{沁园春·雪}
  \addcontentsline{toc}{section}{\makebox[9cm]{沁园春·雪\dotfill{} 毛泽东}}
  \pauthor{【近代 】毛泽东}
  \vskip-3pt\begin{table}[H]
    \centering
    \begin{tabular}{@{}l@{}}
北国风光,千里冰封,万里雪飘。\\
望长城内外,惟余莽\xpinyin*{\xpinyin{莽}{mǎng}} ;\\
大河上下,顿失滔滔。\\
山舞银蛇,原\xpinyin*{\xpinyin{驰}{chí}}蜡象,欲与天公试比高。\\
须晴日,看红装素裹,分外妖\xpinyin*{\xpinyin{娆}{ráo}} 。\\
\\
江山如此多娇,引无数英雄竞折腰。\\
惜秦皇汉武,略输文采;\\
唐宗宋祖,稍\xpinyin*{\xpinyin{逊}{xùn}}风\xpinyin*{\xpinyin{骚}{sāo}} 。\\
一代天骄,成吉思\xpinyin*{\xpinyin{汗}{hán}} ,只识弯弓射大\xpinyin*{\xpinyin{雕}{diāo}} 。\\
俱往矣,数风流人物,还看今朝。
    \end{tabular}
  \end{table}
\end{minipage}
\vspace{1cm}


\noindent\begin{minipage}{\linewidth}
  \ptitle{沁园春·长沙}
  \addcontentsline{toc}{section}{\makebox[9cm]{沁园春·长沙\dotfill{} 毛泽东}}
  \pauthor{【现代】毛泽东}
  \vskip-3pt\begin{table}[H]
    \centering
    \begin{tabular}{@{}l@{}}
独立寒秋,湘江北去,\xpinyin*{\xpinyin{橘}{jú}}子洲头。\\
看万山红遍,层林尽染;\\
漫江碧透,百\xpinyin*{\xpinyin{舸}{gě}}争流。\\
鹰击长空,鱼翔浅底,万类霜天竞自由。\\
\xpinyin*{\xpinyin{怅}{chàng}}\xpinyin*{\xpinyin{寥}{liào}}\xpinyin*{\xpinyin{廓}{kuò}} ,问苍茫大地,谁主沉浮?\\
\\
\xpinyin*{\xpinyin{携}{xié}}来百侣曾游,忆往昔\xpinyin*{\xpinyin{峥}{zhēng}}\xpinyin*{\xpinyin{嵘}{róng}}岁月\xpinyin*{\xpinyin{稠}{chóu}} 。\\
恰同学少年,风华正茂;\\
书生意气,挥斥方\xpinyin*{\xpinyin{遒}{qiú}} 。\\
指点江山,激扬文字,粪土当年万户侯。\\
曾记否,到中流击水,浪\xpinyin*{\xpinyin{遏}{è}}飞舟?
    \end{tabular}
  \end{table}
\end{minipage}
\vspace{1cm}


\noindent\begin{minipage}{\linewidth}
  \ptitle{忆秦娥·娄山关}
  \addcontentsline{toc}{section}{\makebox[9cm]{忆秦娥·娄山关\dotfill{} 毛泽东}}
  \pauthor{【近代】毛泽东}
  \vskip-3pt\begin{table}[H]
    \centering
    \begin{tabular}{@{}l@{}}
西风烈,长空雁叫霜晨月。\\
霜晨月,马蹄声碎,喇叭声咽。\\
\\
雄关漫道真如铁,而今迈步从头越。\\
从头越,苍山如海,残阳如血。
    \end{tabular}
  \end{table}
\end{minipage}
\vspace{1cm}


\noindent\begin{minipage}{\linewidth}
  \ptitle{忆秦娥·箫声咽}
  \addcontentsline{toc}{section}{\makebox[9cm]{忆秦娥·箫声咽\dotfill{} 李白}}
  \pauthor{【唐】李白}
  \vskip-3pt\begin{table}[H]
    \centering
    \begin{tabular}{@{}l@{}}
箫声咽,秦娥梦断秦楼月。\\
秦楼月,年年柳色,\xpinyin*{\xpinyin{灞}{bà}}\xpinyin*{\xpinyin{陵}{líng}}伤别。\\
\\
乐游原上清秋节,咸阳古道音尘绝。\\
音尘绝,西风残照,汉家陵\xpinyin*{\xpinyin{阙}{què}}。
    \end{tabular}
  \end{table}
\end{minipage}
\vspace{1cm}


\noindent\begin{minipage}{\linewidth}
  \ptitle{水调歌头·游泳}
  \addcontentsline{toc}{section}{\makebox[9cm]{水调歌头·游泳\dotfill{} 毛泽东}}
  \pauthor{毛泽东}
  \vskip-3pt\begin{table}[H]
    \centering
    \begin{tabular}{@{}l@{}}
才饮长沙水,又食武昌鱼。\\
万里长江横渡,极目楚天舒。\\
不管风吹浪打,胜似闲庭信步,今日得宽\xpinyin*{\xpinyin{馀}{yú}} 。\\
子在川上曰:逝者如斯夫!\\
\\
风\xpinyin*{\xpinyin{樯}{qiáng}}动,龟蛇静,起宏图。\\
一桥飞架南北,天堑变通途。\\
更立西江石壁,截断巫山云雨,高峡出平湖。\\
神女应无\xpinyin*{\xpinyin{恙}{yàng}} ,当惊世界殊。
    \end{tabular}
  \end{table}
\end{minipage}
\vspace{1cm}


\noindent\begin{minipage}{\linewidth}
  \ptitle{水调歌头·明月几时有}
  \addcontentsline{toc}{section}{\makebox[9cm]{水调歌头·明月几时有\dotfill{} 苏轼}}
  \pauthor{【宋】苏轼}
  \vskip-3pt\begin{table}[H]
    \centering
    \begin{tabular}{@{}l@{}}
明月几时有,把酒问青天。\\
不知天上宫阙,今夕是何年。\\
我欲乘风归去,又恐琼楼玉宇,高处不胜寒。\\
起舞弄清影,何似在人间。\\
\\
转朱阁,低\xpinyin*{\xpinyin{绮}{qǐ}}户,照无眠。\\
不应有恨,何事长向别时圆。\\
人有悲欢离合,月有阴晴圆缺,此事古难全。\\
但愿人长久,千里共婵娟。
    \end{tabular}
  \end{table}
\end{minipage}
\vspace{1cm}


\noindent\begin{minipage}{\linewidth}
  \ptitle{七律·人民解放军占领南京}
  \addcontentsline{toc}{section}{\makebox[9cm]{七律·人民解放军占领南京\dotfill{} 毛泽东}}
  \pauthor{毛泽东}
  \vskip-3pt\begin{table}[H]
    \centering
    \begin{tabular}{@{}l@{}}
钟山风雨起苍黄,百万雄师过大江。\\
虎\xpinyin*{\xpinyin{踞}{jù}}龙盘今胜昔,天翻地\xpinyin*{\xpinyin{覆}{fù}}\xpinyin*{\xpinyin{慨}{kǎi}}而\xpinyin*{\xpinyin{慷}{kāng}} 。\\
宜将\xpinyin*{\xpinyin{剩}{shèng}}勇追穷\xpinyin*{\xpinyin{寇}{kòu}} ,不可\xpinyin*{\xpinyin{沽}{gū}}名学霸王。\\
天若有情天亦老,人间正道是\xpinyin*{\xpinyin{沧}{cāng}}桑。
    \end{tabular}
  \end{table}
\end{minipage}
\vspace{1cm}


\noindent\begin{minipage}{\linewidth}
  \ptitle{七律·长征}
  \addcontentsline{toc}{section}{\makebox[9cm]{七律·长征\dotfill{} 毛泽东}}
  \pauthor{毛泽东}
  \vskip-3pt\begin{table}[H]
    \centering
    \begin{tabular}{@{}l@{}}
红军不怕远征难,万水千山只等闲。\\
五岭\xpinyin*{\xpinyin{逶}{wēi}}\xpinyin*{\xpinyin{迤}{yí}}腾细浪,乌蒙\xpinyin*{\xpinyin{磅}{páng}}\xpinyin*{\xpinyin{礴}{bó}}走泥丸。\\
金沙水拍云崖暖,大渡桥横铁索寒。\\
更喜\xpinyin*{\xpinyin{岷}{mín}}山千里雪,三军过后尽开颜。
    \end{tabular}
  \end{table}
\end{minipage}
\vspace{1cm}


\noindent\begin{minipage}{\linewidth}
  \ptitle{清平乐·六盘山}
  \addcontentsline{toc}{section}{\makebox[9cm]{清平乐·六盘山\dotfill{} 毛泽东}}
  \pauthor{毛泽东}
  \vskip-3pt\begin{table}[H]
    \centering
    \begin{tabular}{@{}l@{}}
天高云淡,望断南飞雁。\\
不到长城非好汉,屈指行程二万。\\
六盘山上高峰,红旗漫卷西风。\\
今日长\xpinyin*{\xpinyin{缨}{yīng}}在手,何时\xpinyin*{\xpinyin{缚}{fù}}住苍龙?
    \end{tabular}
  \end{table}
\end{minipage}
\vspace{1cm}


\noindent\begin{minipage}{\linewidth}
  \ptitle{七律·到韶山}
  \addcontentsline{toc}{section}{\makebox[9cm]{七律·到韶山\dotfill{} 毛泽东}}
  \pauthor{毛泽东}
  \vskip-3pt\begin{table}[H]
    \centering
    \begin{tabular}{@{}l@{}}
别梦依稀\xpinyin*{\xpinyin{咒}{zhòu}}逝川,故园三十二年前。\\
红旗卷起农奴\xpinyin*{\xpinyin{戟}{jǐ}},黑手高悬霸主鞭。\\
为有牺牲多壮志,敢叫日月换新天。\\
喜看稻\xpinyin*{\xpinyin{菽}{shū}}千重浪,遍地英雄下夕烟。
    \end{tabular}
  \end{table}
\end{minipage}
\vspace{1cm}


\noindent\begin{minipage}{\linewidth}
  \ptitle{江南}
  \addcontentsline{toc}{section}{\makebox[9cm]{江南\dotfill{} 乐府}}
  \pauthor{【汉】乐府}
  \vskip-3pt\begin{table}[H]
    \centering
    \begin{tabular}{@{}l@{}}
江南可采莲,莲叶何田田,鱼戏莲叶间。\\
鱼戏莲叶东,鱼戏莲叶西。鱼戏莲叶南,鱼戏莲叶北。
    \end{tabular}
  \end{table}
\end{minipage}
\vspace{1cm}


\noindent\begin{minipage}{\linewidth}
  \ptitle{长歌行}
  \addcontentsline{toc}{section}{\makebox[9cm]{长歌行\dotfill{} 乐府}}
  \pauthor{【汉】乐府}
  \vskip-3pt\begin{table}[H]
    \centering
    \begin{tabular}{@{}l@{}}
青青园中葵,朝露待日\xpinyin*{\xpinyin{晞}{xī}}。\\
阳春布德泽,万物生光辉。\\
常恐秋节至,\xpinyin*{\xpinyin{焜}{kūn}}黄华叶衰。\\
百川东到海,何时复西归?\\
少壮不努力,老大徒伤悲!
    \end{tabular}
  \end{table}
\end{minipage}
\vspace{1cm}


\noindent\begin{minipage}{\linewidth}
  \ptitle{敕勒歌}
  \addcontentsline{toc}{section}{\makebox[9cm]{敕勒歌\dotfill{} 乐府}}
  \pauthor{【北朝】乐府}
  \vskip-3pt\begin{table}[H]
    \centering
    \begin{tabular}{@{}l@{}}
敕勒川,阴山下,\\
天似穹庐,笼盖四野。\\
天苍苍,野茫茫,\\
风吹草低见牛羊。
    \end{tabular}
  \end{table}
\end{minipage}
\vspace{1cm}


\noindent\begin{minipage}{\linewidth}
  \ptitle{咏鹅}
  \addcontentsline{toc}{section}{\makebox[9cm]{咏鹅\dotfill{} 骆宾王}}
  \pauthor{【唐】骆宾王}
  \vskip-3pt\begin{table}[H]
    \centering
    \begin{tabular}{@{}l@{}}
鹅,鹅,鹅,曲项向天歌。\\
白毛浮绿水,红掌拨清波。
    \end{tabular}
  \end{table}
\end{minipage}
\vspace{1cm}


\noindent\begin{minipage}{\linewidth}
  \ptitle{风}
  \addcontentsline{toc}{section}{\makebox[9cm]{风\dotfill{} 李峤}}
  \pauthor{【唐】李\xpinyin*{\xpinyin{峤}{qiáo}}}
  \vskip-3pt\begin{table}[H]
    \centering
    \begin{tabular}{@{}l@{}}
解落三秋叶,能开二月花。\\
过江千尺浪,入竹万竿斜。
    \end{tabular}
  \end{table}
\end{minipage}
\vspace{1cm}


\noindent\begin{minipage}{\linewidth}
  \ptitle{咏柳}
  \addcontentsline{toc}{section}{\makebox[9cm]{咏柳\dotfill{} 贺知章}}
  \pauthor{【唐】贺知章}
  \vskip-3pt\begin{table}[H]
    \centering
    \begin{tabular}{@{}l@{}}
碧玉妆成一树高,万条垂下绿丝\xpinyin*{\xpinyin{绦}{tāo}}。\\
不知细叶谁裁出,二月春风似剪刀。
    \end{tabular}
  \end{table}
\end{minipage}
\vspace{1cm}


\noindent\begin{minipage}{\linewidth}
  \ptitle{回乡偶书}
  \addcontentsline{toc}{section}{\makebox[9cm]{回乡偶书\dotfill{} 贺知章}}
  \pauthor{【唐】贺知章}
  \vskip-3pt\begin{table}[H]
    \centering
    \begin{tabular}{@{}l@{}}
少小离家老大回,乡音无改\xpinyin*{\xpinyin{鬓}{bìn}}毛\xpinyin*{\xpinyin{衰}{cuī}}。\\
儿童相见不相识,笑问客从何处来。
    \end{tabular}
  \end{table}
\end{minipage}
\vspace{1cm}


\noindent\begin{minipage}{\linewidth}
  \ptitle{凉州词}
  \addcontentsline{toc}{section}{\makebox[9cm]{凉州词\dotfill{} 王之涣}}
  \pauthor{【唐】王之涣}
  \vskip-3pt\begin{table}[H]
    \centering
    \begin{tabular}{@{}l@{}}
黄河远上白云间,一片孤城万\xpinyin*{\xpinyin{仞}{rèn}}山。\\
\xpinyin*{\xpinyin{羌}{qiāng}}笛何须怨杨柳,春风不度玉门关。
    \end{tabular}
  \end{table}
\end{minipage}
\vspace{1cm}


\noindent\begin{minipage}{\linewidth}
  \ptitle{登\xpinyin*{\xpinyin{鹳}{guàn}}雀楼}
  \addcontentsline{toc}{section}{\makebox[9cm]{登鹳雀楼\dotfill{} 王之涣}}
  \pauthor{【唐】王之涣}
  \vskip-3pt\begin{table}[H]
    \centering
    \begin{tabular}{@{}l@{}}
白日依山尽,黄河入海流。\\
欲穷千里目,更上一层楼。
    \end{tabular}
  \end{table}
\end{minipage}
\vspace{1cm}


\noindent\begin{minipage}{\linewidth}
  \ptitle{春晓}
  \addcontentsline{toc}{section}{\makebox[9cm]{春晓\dotfill{} 孟浩然}}
  \pauthor{【唐】孟浩然}
  \vskip-3pt\begin{table}[H]
    \centering
    \begin{tabular}{@{}l@{}}
春眠不觉晓,处处闻啼鸟。\\
夜来风雨声,花落知多少。
    \end{tabular}
  \end{table}
\end{minipage}
\vspace{1cm}


\noindent\begin{minipage}{\linewidth}
  \ptitle{凉州词}
  \addcontentsline{toc}{section}{\makebox[9cm]{凉州词\dotfill{} 王翰}}
  \pauthor{【唐】王翰}
  \vskip-3pt\begin{table}[H]
    \centering
    \begin{tabular}{@{}l@{}}
葡萄美酒夜光杯,欲饮琵琶马上催。\\
醉卧沙场君莫笑,古来征战几人回?
    \end{tabular}
  \end{table}
\end{minipage}
\vspace{1cm}


\noindent\begin{minipage}{\linewidth}
  \ptitle{出塞(其一)}
  \addcontentsline{toc}{section}{\makebox[9cm]{出塞(其一)\dotfill{} 王昌龄}}
  \pauthor{【唐】王昌龄}
  \vskip-3pt\begin{table}[H]
    \centering
    \begin{tabular}{@{}l@{}}
秦时明月汉时关,万里长征人未还。\\
但使龙城飞将在,不教胡马度阴山。
    \end{tabular}
  \end{table}
\end{minipage}
\vspace{1cm}


\noindent\begin{minipage}{\linewidth}
  \ptitle{芙蓉楼送辛渐}
  \addcontentsline{toc}{section}{\makebox[9cm]{芙蓉楼送辛渐\dotfill{} 王昌龄}}
  \pauthor{【唐】王昌龄}
  \vskip-3pt\begin{table}[H]
    \centering
    \begin{tabular}{@{}l@{}}
寒雨连江夜入吴,平明送客楚山孤。\\
洛阳亲友如相问,一片冰心在玉壶。
    \end{tabular}
  \end{table}
\end{minipage}
\vspace{1cm}


\noindent\begin{minipage}{\linewidth}
  \ptitle{鹿\xpinyin*{\xpinyin{柴}{zhài}}}
  \addcontentsline{toc}{section}{\makebox[9cm]{鹿柴\dotfill{} 王维}}
  \pauthor{【唐】王维}
  \vskip-3pt\begin{table}[H]
    \centering
    \begin{tabular}{@{}l@{}}
空山不见人,但闻人语响。\\
返景入深林,复照青苔上。
    \end{tabular}
  \end{table}
\end{minipage}
\vspace{1cm}


\noindent\begin{minipage}{\linewidth}
  \ptitle{送元二使安西}
  \addcontentsline{toc}{section}{\makebox[9cm]{送元二使安西\dotfill{} 王维}}
  \pauthor{【唐】王维}
  \vskip-3pt\begin{table}[H]
    \centering
    \begin{tabular}{@{}l@{}}
渭城朝雨\xpinyin*{\xpinyin{浥}{yì}}轻尘,客舍青青柳色新。\\
劝君更尽一杯酒,西出阳关无故人。
    \end{tabular}
  \end{table}
\end{minipage}
\vspace{1cm}


\noindent\begin{minipage}{\linewidth}
  \ptitle{九月九日忆山东兄弟}
  \addcontentsline{toc}{section}{\makebox[9cm]{九月九日忆山东兄弟\dotfill{} 王维}}
  \pauthor{【唐】王维}
  \vskip-3pt\begin{table}[H]
    \centering
    \begin{tabular}{@{}l@{}}
独在异乡为异客,每逢佳节倍思亲。\\
遥知兄弟登高处,遍插\xpinyin*{\xpinyin{茱}{zhū}}\xpinyin*{\xpinyin{萸}{yú}}少一人。
    \end{tabular}
  \end{table}
\end{minipage}
\vspace{1cm}


\noindent\begin{minipage}{\linewidth}
  \ptitle{静夜思}
  \addcontentsline{toc}{section}{\makebox[9cm]{静夜思\dotfill{} 李白}}
  \pauthor{【唐】李白}
  \vskip-3pt\begin{table}[H]
    \centering
    \begin{tabular}{@{}l@{}}
床前明月光,疑是地上霜。\\
举头望明月,低头思故乡。
    \end{tabular}
  \end{table}
\end{minipage}
\vspace{1cm}


\noindent\begin{minipage}{\linewidth}
  \ptitle{古朗月行}
  \addcontentsline{toc}{section}{\makebox[9cm]{古朗月行\dotfill{} 李白}}
  \pauthor{【唐】李白}
  \vskip-3pt\begin{table}[H]
    \centering
    \begin{tabular}{@{}l@{}}
小时不识月,呼作白玉盘。又疑瑶台镜,飞在青云端。\\
仙人垂两足,桂树何团团。白兔捣药成,问言与谁餐?\\
\xpinyin*{\xpinyin{蟾}{chán}}\xpinyin*{\xpinyin{蜍}{chú}}蚀圆影,大明夜已残。\xpinyin*{\xpinyin{羿}{yì}}昔落九乌,天人清且安。\\
阴精此沦惑,去去不足观。忧来其如何?凄\xpinyin*{\xpinyin{怆}{chuàng}}摧心肝。
    \end{tabular}
  \end{table}
\end{minipage}
\vspace{1cm}


\noindent\begin{minipage}{\linewidth}
  \ptitle{望庐山瀑布}
  \addcontentsline{toc}{section}{\makebox[9cm]{望庐山瀑布\dotfill{} 李白}}
  \pauthor{【唐】李白}
  \vskip-3pt\begin{table}[H]
    \centering
    \begin{tabular}{@{}l@{}}
日照香炉生紫烟,遥看瀑布挂前川。\\
飞流直下三千尺,疑是银河落九天。
    \end{tabular}
  \end{table}
\end{minipage}
\vspace{1cm}


\noindent\begin{minipage}{\linewidth}
  \ptitle{赠汪伦}
  \addcontentsline{toc}{section}{\makebox[9cm]{赠汪伦\dotfill{} 李白}}
  \pauthor{【唐】李白}
  \vskip-3pt\begin{table}[H]
    \centering
    \begin{tabular}{@{}l@{}}
李白乘舟将欲行,忽闻岸上踏歌声。\\
桃花潭水深千尺,不及汪伦送我情。
    \end{tabular}
  \end{table}
\end{minipage}
\vspace{1cm}


\noindent\begin{minipage}{\linewidth}
  \ptitle{黄鹤楼送孟浩然之广陵}
  \addcontentsline{toc}{section}{\makebox[9cm]{黄鹤楼送孟浩然之广陵\dotfill{} 李白}}
  \pauthor{【唐】李白}
  \vskip-3pt\begin{table}[H]
    \centering
    \begin{tabular}{@{}l@{}}
故人西辞黄鹤楼,烟花三月下扬州。\\
孤帆远影碧空尽,唯见长江天际流。
    \end{tabular}
  \end{table}
\end{minipage}
\vspace{1cm}


\noindent\begin{minipage}{\linewidth}
  \ptitle{早发白帝城}
  \addcontentsline{toc}{section}{\makebox[9cm]{早发白帝城\dotfill{} 李白}}
  \pauthor{【唐】李白}
  \vskip-3pt\begin{table}[H]
    \centering
    \begin{tabular}{@{}l@{}}
朝辞白帝彩云间,千里江陵一日还。\\
两岸猿声啼不住,轻舟已过万重山。
    \end{tabular}
  \end{table}
\end{minipage}
\vspace{1cm}


\noindent\begin{minipage}{\linewidth}
  \ptitle{望天门山}
  \addcontentsline{toc}{section}{\makebox[9cm]{望天门山\dotfill{} 李白}}
  \pauthor{【唐】李白}
  \vskip-3pt\begin{table}[H]
    \centering
    \begin{tabular}{@{}l@{}}
天门中断楚江开,碧水东流至此回。\\
两岸青山相对出,孤帆一片日边来。
    \end{tabular}
  \end{table}
\end{minipage}
\vspace{1cm}


\noindent\begin{minipage}{\linewidth}
  \ptitle{别董大 其一}
  \addcontentsline{toc}{section}{\makebox[9cm]{别董大 其一\dotfill{} 高适}}
  \pauthor{【唐】高适}
  \vskip-3pt\begin{table}[H]
    \centering
    \begin{tabular}{@{}l@{}}
千里黄云白日曛,北风吹雁雪纷纷。\\
莫愁前路无知己,天下谁人不识君。
    \end{tabular}
  \end{table}
\end{minipage}
\vspace{1cm}


\noindent\begin{minipage}{\linewidth}
  \ptitle{绝句}
  \addcontentsline{toc}{section}{\makebox[9cm]{绝句\dotfill{} 杜甫}}
  \pauthor{【唐】杜甫}
  \vskip-3pt\begin{table}[H]
    \centering
    \begin{tabular}{@{}l@{}}
两个黄鹂鸣翠柳,一行白鹭上青天。\\
窗含西岭千秋雪,门泊东吴万里船。
    \end{tabular}
  \end{table}
\end{minipage}
\vspace{1cm}


\noindent\begin{minipage}{\linewidth}
  \ptitle{绝句}
  \addcontentsline{toc}{section}{\makebox[9cm]{绝句\dotfill{} 杜甫}}
  \pauthor{【唐】杜甫}
  \vskip-3pt\begin{table}[H]
    \centering
    \begin{tabular}{@{}l@{}}
迟日江山丽,春风花草香。\\
泥融飞燕子,沙暖睡鸳鸯。
    \end{tabular}
  \end{table}
\end{minipage}
\vspace{1cm}


\noindent\begin{minipage}{\linewidth}
  \ptitle{春夜喜雨}
  \addcontentsline{toc}{section}{\makebox[9cm]{春夜喜雨\dotfill{} 杜甫}}
  \pauthor{【唐】杜甫}
  \vskip-3pt\begin{table}[H]
    \centering
    \begin{tabular}{@{}l@{}}
好雨知时节,当春乃发生。随风潜入夜,润物细无声。\\
野径云俱黑,江船火独明。晓看红湿处,花重锦官城。
    \end{tabular}
  \end{table}
\end{minipage}
\vspace{1cm}


\noindent\begin{minipage}{\linewidth}
  \ptitle{江畔独步寻花}
  \addcontentsline{toc}{section}{\makebox[9cm]{江畔独步寻花\dotfill{} 杜甫}}
  \pauthor{【唐】杜甫}
  \vskip-3pt\begin{table}[H]
    \centering
    \begin{tabular}{@{}l@{}}
黄师塔前江水东,春光懒困倚微风。\\
桃花一簇开无主,可爱深红爱浅红?
    \end{tabular}
  \end{table}
\end{minipage}
\vspace{1cm}


\noindent\begin{minipage}{\linewidth}
  \ptitle{枫桥夜泊}
  \addcontentsline{toc}{section}{\makebox[9cm]{枫桥夜泊\dotfill{} 张继}}
  \pauthor{【唐】张继}
  \vskip-3pt\begin{table}[H]
    \centering
    \begin{tabular}{@{}l@{}}
月落乌啼霜满天,江枫渔火对愁眠。\\
姑苏城外寒山寺,夜半钟声到客船。
    \end{tabular}
  \end{table}
\end{minipage}
\vspace{1cm}


\noindent\begin{minipage}{\linewidth}
  \ptitle{\xpinyin*{\xpinyin{滁}{chú}}州西\xpinyin*{\xpinyin{涧}{jiàn}}}
  \addcontentsline{toc}{section}{\makebox[9cm]{滁州西涧\dotfill{} 韦应物}}
  \pauthor{【唐】韦应物}
  \vskip-3pt\begin{table}[H]
    \centering
    \begin{tabular}{@{}l@{}}
独怜幽草\xpinyin*{\xpinyin{涧}{jiàn}}边生,上有黄鹂深树鸣。\\
春潮带雨晚来急,野渡无人舟自横。
    \end{tabular}
  \end{table}
\end{minipage}
\vspace{1cm}


\noindent\begin{minipage}{\linewidth}
  \ptitle{游子吟}
  \addcontentsline{toc}{section}{\makebox[9cm]{游子吟\dotfill{} 孟郊}}
  \pauthor{【唐】孟郊}
  \vskip-3pt\begin{table}[H]
    \centering
    \begin{tabular}{@{}l@{}}
慈母手中线,游子身上衣。\\
临行密密缝,意恐迟迟归。\\
谁言寸草心,报得三春\xpinyin*{\xpinyin{晖}{huī}}!
    \end{tabular}
  \end{table}
\end{minipage}
\vspace{1cm}


\noindent\begin{minipage}{\linewidth}
  \ptitle{早春呈水部张十八员外 其一}
  \addcontentsline{toc}{section}{\makebox[9cm]{早春呈水部张十八员外 其一\dotfill{} 韩愈}}
  \pauthor{【唐】韩愈}
  \vskip-3pt\begin{table}[H]
    \centering
    \begin{tabular}{@{}l@{}}
天街小雨润如酥,草色遥看近却无。\\
最是一年春好处,绝胜烟柳满皇都。
    \end{tabular}
  \end{table}
\end{minipage}
\vspace{1cm}


\noindent\begin{minipage}{\linewidth}
  \ptitle{渔歌子}
  \addcontentsline{toc}{section}{\makebox[9cm]{渔歌子\dotfill{} 张志和}}
  \pauthor{【唐】张志和}
  \vskip-3pt\begin{table}[H]
    \centering
    \begin{tabular}{@{}l@{}}
西塞山前白\xpinyin*{\xpinyin{鹭}{lù}}飞,桃花流水\xpinyin*{\xpinyin{鳜}{guì}}鱼肥。\\
青\xpinyin*{\xpinyin{箬}{ruò}}\xpinyin*{\xpinyin{笠}{lì}},绿\xpinyin*{\xpinyin{蓑}{suō}}衣,斜风细雨不须归。
    \end{tabular}
  \end{table}
\end{minipage}
\vspace{1cm}


\noindent\begin{minipage}{\linewidth}
  \ptitle{塞下曲}
  \addcontentsline{toc}{section}{\makebox[9cm]{塞下曲\dotfill{} 卢纶}}
  \pauthor{【唐】卢\xpinyin*{\xpinyin{纶}{lún}}}
  \vskip-3pt\begin{table}[H]
    \centering
    \begin{tabular}{@{}l@{}}
月黑雁飞高,单于夜\xpinyin*{\xpinyin{遁}{dùn}}逃。\\
欲将轻\xpinyin*{\xpinyin{骑}{jì}}逐,大雪满弓刀。
    \end{tabular}
  \end{table}
\end{minipage}
\vspace{1cm}


\noindent\begin{minipage}{\linewidth}
  \ptitle{望洞庭}
  \addcontentsline{toc}{section}{\makebox[9cm]{望洞庭\dotfill{} 刘禹锡}}
  \pauthor{【唐】刘禹锡}
  \vskip-3pt\begin{table}[H]
    \centering
    \begin{tabular}{@{}l@{}}
湖光秋月两相和,潭面无风镜未磨。\\
遥望洞庭山水翠,白银盘里一青螺。
    \end{tabular}
  \end{table}
\end{minipage}
\vspace{1cm}


\noindent\begin{minipage}{\linewidth}
  \ptitle{浪淘沙}
  \addcontentsline{toc}{section}{\makebox[9cm]{浪淘沙\dotfill{} 刘禹锡}}
  \pauthor{【唐】刘\xpinyin*{\xpinyin{禹}{yǔ}}锡}
  \vskip-3pt\begin{table}[H]
    \centering
    \begin{tabular}{@{}l@{}}
九曲黄河万里沙,浪淘风\xpinyin*{\xpinyin{簸}{bǒ}}自天涯。\\
如今直上银河去,同到牵牛织女家。
    \end{tabular}
  \end{table}
\end{minipage}
\vspace{1cm}


\noindent\begin{minipage}{\linewidth}
  \ptitle{赋得古原草送別}
  \addcontentsline{toc}{section}{\makebox[9cm]{赋得古原草送別\dotfill{} 白居易}}
  \pauthor{【唐】白居易}
  \vskip-3pt\begin{table}[H]
    \centering
    \begin{tabular}{@{}l@{}}
离离原上草,一岁一枯荣。野火烧不尽,春风吹又生。\\
远芳侵古道,晴翠接荒城。又送王孙去,\xpinyin*{\xpinyin{萋}{qī}}萋满别情。
    \end{tabular}
  \end{table}
\end{minipage}
\vspace{1cm}


\noindent\begin{minipage}{\linewidth}
  \ptitle{池上}
  \addcontentsline{toc}{section}{\makebox[9cm]{池上\dotfill{} 白居易}}
  \pauthor{【唐】白居易}
  \vskip-3pt\begin{table}[H]
    \centering
    \begin{tabular}{@{}l@{}}
小娃撑小艇,偷采白莲回。\\
不解藏踪迹,浮萍一道开。
    \end{tabular}
  \end{table}
\end{minipage}
\vspace{1cm}


\noindent\begin{minipage}{\linewidth}
  \ptitle{忆江南}
  \addcontentsline{toc}{section}{\makebox[9cm]{忆江南\dotfill{} 白居易}}
  \pauthor{【唐】白居易}
  \vskip-3pt\begin{table}[H]
    \centering
    \begin{tabular}{@{}l@{}}
江南好,风景旧曾\xpinyin*{\xpinyin{谙}{ān}}。\\
日出江花红胜火,春来江水绿如蓝。\\
能不忆江南?
    \end{tabular}
  \end{table}
\end{minipage}
\vspace{1cm}


\noindent\begin{minipage}{\linewidth}
  \ptitle{小儿垂钓}
  \addcontentsline{toc}{section}{\makebox[9cm]{小儿垂钓\dotfill{} 胡令能}}
  \pauthor{【唐】胡令能}
  \vskip-3pt\begin{table}[H]
    \centering
    \begin{tabular}{@{}l@{}}
\xpinyin*{\xpinyin{蓬}{péng}}头\xpinyin*{\xpinyin{稚}{zhì}}子学垂\xpinyin*{\xpinyin{纶}{lún}},侧坐莓苔草映身。\\
路人借问遥招手,怕得鱼惊不应人。
    \end{tabular}
  \end{table}
\end{minipage}
\vspace{1cm}


\noindent\begin{minipage}{\linewidth}
  \ptitle{悯农 其一}
  \addcontentsline{toc}{section}{\makebox[9cm]{悯农 其一\dotfill{} 李绅}}
  \pauthor{【唐】李绅}
  \vskip-3pt\begin{table}[H]
    \centering
    \begin{tabular}{@{}l@{}}
春种一粒\xpinyin*{\xpinyin{粟}{sù}},秋收万颗子。\\
四海无闲田,农夫犹饿死。
    \end{tabular}
  \end{table}
\end{minipage}
\vspace{1cm}


\noindent\begin{minipage}{\linewidth}
  \ptitle{悯农 其二}
  \addcontentsline{toc}{section}{\makebox[9cm]{悯农 其二\dotfill{} 李绅}}
  \pauthor{【唐】李绅}
  \vskip-3pt\begin{table}[H]
    \centering
    \begin{tabular}{@{}l@{}}
锄禾日当午,汗滴禾下土。\\
谁知盘中餐,粒粒皆辛苦。
    \end{tabular}
  \end{table}
\end{minipage}
\vspace{1cm}


\noindent\begin{minipage}{\linewidth}
  \ptitle{江雪}
  \addcontentsline{toc}{section}{\makebox[9cm]{江雪\dotfill{} 柳宗元}}
  \pauthor{【唐】柳宗元}
  \vskip-3pt\begin{table}[H]
    \centering
    \begin{tabular}{@{}l@{}}
千山鸟飞绝,万径人踪灭。\\
孤舟蓑笠翁,独钓寒江雪。
    \end{tabular}
  \end{table}
\end{minipage}
\vspace{1cm}


\noindent\begin{minipage}{\linewidth}
  \ptitle{寻隐者不遇}
  \addcontentsline{toc}{section}{\makebox[9cm]{寻隐者不遇\dotfill{} 贾岛}}
  \pauthor{【唐】贾岛}
  \vskip-3pt\begin{table}[H]
    \centering
    \begin{tabular}{@{}l@{}}
松下问童子,言师采药去。\\
只在此山中,云深不知处。
    \end{tabular}
  \end{table}
\end{minipage}
\vspace{1cm}


\noindent\begin{minipage}{\linewidth}
  \ptitle{山行}
  \addcontentsline{toc}{section}{\makebox[9cm]{山行\dotfill{} 杜牧}}
  \pauthor{【唐】杜牧}
  \vskip-3pt\begin{table}[H]
    \centering
    \begin{tabular}{@{}l@{}}
远上寒山石径斜,白云生处有人家。\\
停车坐爱枫林晚,霜叶红于二月花。
    \end{tabular}
  \end{table}
\end{minipage}
\vspace{1cm}


\noindent\begin{minipage}{\linewidth}
  \ptitle{清明}
  \addcontentsline{toc}{section}{\makebox[9cm]{清明\dotfill{} 杜牧}}
  \pauthor{【唐】杜牧}
  \vskip-3pt\begin{table}[H]
    \centering
    \begin{tabular}{@{}l@{}}
清明时节雨纷纷,路上行人欲断魂。\\
借问酒家何处有?牧童遥指杏花村。
    \end{tabular}
  \end{table}
\end{minipage}
\vspace{1cm}


\noindent\begin{minipage}{\linewidth}
  \ptitle{江南春}
  \addcontentsline{toc}{section}{\makebox[9cm]{江南春\dotfill{} 杜牧}}
  \pauthor{【唐】杜牧}
  \vskip-3pt\begin{table}[H]
    \centering
    \begin{tabular}{@{}l@{}}
千里莺啼绿映红,水村山郭酒旗风。\\
南朝四百八十寺,多少楼台烟雨中。
    \end{tabular}
  \end{table}
\end{minipage}
\vspace{1cm}


\noindent\begin{minipage}{\linewidth}
  \ptitle{蜂}
  \addcontentsline{toc}{section}{\makebox[9cm]{蜂\dotfill{} 罗隐}}
  \pauthor{【唐】罗隐}
  \vskip-3pt\begin{table}[H]
    \centering
    \begin{tabular}{@{}l@{}}
不论平地与山尖,无限风光尽被占。\\
采得百花成蜜后,为谁辛苦为谁甜。
    \end{tabular}
  \end{table}
\end{minipage}
\vspace{1cm}


\noindent\begin{minipage}{\linewidth}
  \ptitle{江上渔者}
  \addcontentsline{toc}{section}{\makebox[9cm]{江上渔者\dotfill{} 范仲淹}}
  \pauthor{【宋】范仲淹}
  \vskip-3pt\begin{table}[H]
    \centering
    \begin{tabular}{@{}l@{}}
江上往来人,但爱\xpinyin*{\xpinyin{鲈}{lú}}鱼美。\\
君看一叶舟,出没风波里。
    \end{tabular}
  \end{table}
\end{minipage}
\vspace{1cm}


\noindent\begin{minipage}{\linewidth}
  \ptitle{元日}
  \addcontentsline{toc}{section}{\makebox[9cm]{元日\dotfill{} 王安石}}
  \pauthor{【宋】王安石}
  \vskip-3pt\begin{table}[H]
    \centering
    \begin{tabular}{@{}l@{}}
爆竹声中一岁除,春风送暖入\xpinyin*{\xpinyin{屠}{tú}}苏。\\
千门万户\xpinyin*{\xpinyin{曈}{tóng}}曈日,总把新桃换旧符。
    \end{tabular}
  \end{table}
\end{minipage}
\vspace{1cm}


\noindent\begin{minipage}{\linewidth}
  \ptitle{泊船瓜洲}
  \addcontentsline{toc}{section}{\makebox[9cm]{泊船瓜洲\dotfill{} 王安石}}
  \pauthor{【宋】王安石}
  \vskip-3pt\begin{table}[H]
    \centering
    \begin{tabular}{@{}l@{}}
京口瓜洲一水间,钟山只隔数重山。\\
春风又绿江南岸,明月何时照我还。
    \end{tabular}
  \end{table}
\end{minipage}
\vspace{1cm}


\noindent\begin{minipage}{\linewidth}
  \ptitle{书湖阴先生壁}
  \addcontentsline{toc}{section}{\makebox[9cm]{书湖阴先生壁\dotfill{} 王安石}}
  \pauthor{【宋】王安石}
  \vskip-3pt\begin{table}[H]
    \centering
    \begin{tabular}{@{}l@{}}
茅\xpinyin*{\xpinyin{檐}{yán}}长扫净无苔,花木成\xpinyin*{\xpinyin{畦}{qí}}手自栽。\\
一水护田将绿绕,两山排\xpinyin*{\xpinyin{闼}{tà}}送青来。
    \end{tabular}
  \end{table}
\end{minipage}
\vspace{1cm}


\noindent\begin{minipage}{\linewidth}
  \ptitle{六月二十七日望湖楼醉书}
  \addcontentsline{toc}{section}{\makebox[9cm]{六月二十七日望湖楼醉书\dotfill{} 苏轼}}
  \pauthor{【宋】苏轼}
  \vskip-3pt\begin{table}[H]
    \centering
    \begin{tabular}{@{}l@{}}
黑云翻墨未遮山,白雨跳珠乱入船。\\
卷地风来忽吹散,望湖楼下水如天。
    \end{tabular}
  \end{table}
\end{minipage}
\vspace{1cm}


\noindent\begin{minipage}{\linewidth}
  \ptitle{饮湖上初晴后雨}
  \addcontentsline{toc}{section}{\makebox[9cm]{饮湖上初晴后雨\dotfill{} 苏轼}}
  \pauthor{【宋】苏轼}
  \vskip-3pt\begin{table}[H]
    \centering
    \begin{tabular}{@{}l@{}}
水光\xpinyin*{\xpinyin{潋}{liàn}}\xpinyin*{\xpinyin{滟}{yàn}}晴方好,山色空蒙雨亦奇。\\
欲把西湖比西子,淡妆浓抹总相宜。
    \end{tabular}
  \end{table}
\end{minipage}
\vspace{1cm}


\noindent\begin{minipage}{\linewidth}
  \ptitle{惠崇春江晓景}
  \addcontentsline{toc}{section}{\makebox[9cm]{惠崇春江晓景\dotfill{} 苏轼}}
  \pauthor{【宋】苏轼}
  \vskip-3pt\begin{table}[H]
    \centering
    \begin{tabular}{@{}l@{}}
竹外桃花三两枝,春江水暖鸭先知。\\
\xpinyin*{\xpinyin{蒌}{lóu}}\xpinyin*{\xpinyin{蒿}{hāo}}满地芦芽短,正是河\xpinyin*{\xpinyin{豚}{tún}}欲上时。
    \end{tabular}
  \end{table}
\end{minipage}
\vspace{1cm}


\noindent\begin{minipage}{\linewidth}
  \ptitle{题西林壁}
  \addcontentsline{toc}{section}{\makebox[9cm]{题西林壁\dotfill{} 苏轼}}
  \pauthor{【宋】苏轼}
  \vskip-3pt\begin{table}[H]
    \centering
    \begin{tabular}{@{}l@{}}
横看成岭侧成峰,远近高低各不同。\\
不识庐山真面目,只缘身在此山中。
    \end{tabular}
  \end{table}
\end{minipage}
\vspace{1cm}


\noindent\begin{minipage}{\linewidth}
  \ptitle{夏日绝句}
  \addcontentsline{toc}{section}{\makebox[9cm]{夏日绝句\dotfill{} 李清照}}
  \pauthor{【宋】李清照}
  \vskip-3pt\begin{table}[H]
    \centering
    \begin{tabular}{@{}l@{}}
生当做人杰,死亦为鬼雄。\\
至今思项羽,不肯过江东。
    \end{tabular}
  \end{table}
\end{minipage}
\vspace{1cm}


\noindent\begin{minipage}{\linewidth}
  \ptitle{三\xpinyin*{\xpinyin{衢}{qú}}道中}
  \addcontentsline{toc}{section}{\makebox[9cm]{三衢道中\dotfill{} 曾几}}
  \pauthor{【宋】曾几}
  \vskip-3pt\begin{table}[H]
    \centering
    \begin{tabular}{@{}l@{}}
梅子黄时日日晴,小溪泛尽却山行。\\
绿阴不减来时路,添得黄鹂四五声。
    \end{tabular}
  \end{table}
\end{minipage}
\vspace{1cm}


\noindent\begin{minipage}{\linewidth}
  \ptitle{示儿}
  \addcontentsline{toc}{section}{\makebox[9cm]{示儿\dotfill{} 陆游}}
  \pauthor{【宋】陆游}
  \vskip-3pt\begin{table}[H]
    \centering
    \begin{tabular}{@{}l@{}}
死去元知万事空,但悲不见九州同。\\
王师北定中原日,家祭无忘告乃翁。
    \end{tabular}
  \end{table}
\end{minipage}
\vspace{1cm}


\noindent\begin{minipage}{\linewidth}
  \ptitle{秋夜将晓出篱门迎凉有感}
  \addcontentsline{toc}{section}{\makebox[9cm]{秋夜将晓出篱门迎凉有感\dotfill{} 陆游}}
  \pauthor{【宋】陆游}
  \vskip-3pt\begin{table}[H]
    \centering
    \begin{tabular}{@{}l@{}}
三万里河东入海,五千仞岳上摩天。\\
遗民泪尽胡尘里,南望王师又一年。
    \end{tabular}
  \end{table}
\end{minipage}
\vspace{1cm}


\noindent\begin{minipage}{\linewidth}
  \ptitle{四时田园杂兴}
  \addcontentsline{toc}{section}{\makebox[9cm]{四时田园杂兴\dotfill{} 范成大}}
  \pauthor{【宋】范成大}
  \vskip-3pt\begin{table}[H]
    \centering
    \begin{tabular}{@{}l@{}}
昼出耘田夜绩麻,村庄儿女各当家。\\
童孙未解供耕织,也傍桑阴学种瓜。
    \end{tabular}
  \end{table}
\end{minipage}
\vspace{1cm}


\noindent\begin{minipage}{\linewidth}
  \ptitle{四时田园杂兴}
  \addcontentsline{toc}{section}{\makebox[9cm]{四时田园杂兴\dotfill{} 范成大}}
  \pauthor{【宋】范成大}
  \vskip-3pt\begin{table}[H]
    \centering
    \begin{tabular}{@{}l@{}}
梅子金黄杏子肥,麦花雪白菜花稀。\\
日长篱落无人过,唯有蜻蜓蛱蝶飞。
    \end{tabular}
  \end{table}
\end{minipage}
\vspace{1cm}


\noindent\begin{minipage}{\linewidth}
  \ptitle{小池}
  \addcontentsline{toc}{section}{\makebox[9cm]{小池\dotfill{} 杨万里}}
  \pauthor{【宋】杨万里}
  \vskip-3pt\begin{table}[H]
    \centering
    \begin{tabular}{@{}l@{}}
泉眼无声惜细流,树阴照水爱晴柔。\\
小荷才露尖尖角,早有蜻蜓立上头。
    \end{tabular}
  \end{table}
\end{minipage}
\vspace{1cm}


\noindent\begin{minipage}{\linewidth}
  \ptitle{晓出净慈寺送林子方}
  \addcontentsline{toc}{section}{\makebox[9cm]{晓出净慈寺送林子方\dotfill{} 杨万里}}
  \pauthor{【宋】杨万里}
  \vskip-3pt\begin{table}[H]
    \centering
    \begin{tabular}{@{}l@{}}
毕竟西湖六月中,风光不与四时同。\\
接天莲叶无穷碧,映日荷花别样红。
    \end{tabular}
  \end{table}
\end{minipage}
\vspace{1cm}


\noindent\begin{minipage}{\linewidth}
  \ptitle{春日}
  \addcontentsline{toc}{section}{\makebox[9cm]{春日\dotfill{} 朱熹}}
  \pauthor{【宋】朱\xpinyin*{\xpinyin{熹}{xī}}}
  \vskip-3pt\begin{table}[H]
    \centering
    \begin{tabular}{@{}l@{}}
胜日寻芳\xpinyin*{\xpinyin{泗}{sì}}水滨,无边光景一时新。\\
等闲识得东风面,万紫千红总是春。
    \end{tabular}
  \end{table}
\end{minipage}
\vspace{1cm}


\noindent\begin{minipage}{\linewidth}
  \ptitle{观书有感}
  \addcontentsline{toc}{section}{\makebox[9cm]{观书有感\dotfill{} 朱熹}}
  \pauthor{【宋】朱\xpinyin*{\xpinyin{熹}{xī}}}
  \vskip-3pt\begin{table}[H]
    \centering
    \begin{tabular}{@{}l@{}}
半亩方塘一鉴开,天光云影共徘徊。\\
问渠那得清如许,为有源头活水来。
    \end{tabular}
  \end{table}
\end{minipage}
\vspace{1cm}


\noindent\begin{minipage}{\linewidth}
  \ptitle{题临安\xpinyin*{\xpinyin{邸}{dǐ}}}
  \addcontentsline{toc}{section}{\makebox[9cm]{题临安邸\dotfill{} 林升}}
  \pauthor{【宋】林升}
  \vskip-3pt\begin{table}[H]
    \centering
    \begin{tabular}{@{}l@{}}
山外青山楼外楼,西湖歌舞几时休?\\
暖风\xpinyin*{\xpinyin{熏}{xūn}}得游人醉,直把杭州作\xpinyin*{\xpinyin{汴}{biàn}}州。
    \end{tabular}
  \end{table}
\end{minipage}
\vspace{1cm}


\noindent\begin{minipage}{\linewidth}
  \ptitle{游园不值}
  \addcontentsline{toc}{section}{\makebox[9cm]{游园不值\dotfill{} 叶绍翁}}
  \pauthor{【宋】叶绍翁}
  \vskip-3pt\begin{table}[H]
    \centering
    \begin{tabular}{@{}l@{}}
应怜\xpinyin*{\xpinyin{屐}{jī}}齿印苍苔,小扣柴\xpinyin*{\xpinyin{扉}{fēi}}久不开。\\
春色满园关不住,一枝红杏出墙来。
    \end{tabular}
  \end{table}
\end{minipage}
\vspace{1cm}


\noindent\begin{minipage}{\linewidth}
  \ptitle{乡村四月}
  \addcontentsline{toc}{section}{\makebox[9cm]{乡村四月\dotfill{} 翁卷}}
  \pauthor{【宋】翁卷}
  \vskip-3pt\begin{table}[H]
    \centering
    \begin{tabular}{@{}l@{}}
绿遍山原白满川,子规声里雨如烟。\\
乡村四月闲人少,才了蚕桑又插田。
    \end{tabular}
  \end{table}
\end{minipage}
\vspace{1cm}


\noindent\begin{minipage}{\linewidth}
  \ptitle{墨梅}
  \addcontentsline{toc}{section}{\makebox[9cm]{墨梅\dotfill{} 王冕}}
  \pauthor{【元】王\xpinyin*{\xpinyin{冕}{miǎn}}}
  \vskip-3pt\begin{table}[H]
    \centering
    \begin{tabular}{@{}l@{}}
我家洗\xpinyin*{\xpinyin{砚}{yàn}}池头树,朵朵花开淡墨痕。\\
不要人夸颜色好,只留清气满乾坤。
    \end{tabular}
  \end{table}
\end{minipage}
\vspace{1cm}


\noindent\begin{minipage}{\linewidth}
  \ptitle{石灰吟}
  \addcontentsline{toc}{section}{\makebox[9cm]{石灰吟\dotfill{} 于谦}}
  \pauthor{【明】于谦}
  \vskip-3pt\begin{table}[H]
    \centering
    \begin{tabular}{@{}l@{}}
千锤万凿出深山,烈火焚烧若等闲。\\
粉骨碎身全不怕,要留清白在人间。
    \end{tabular}
  \end{table}
\end{minipage}
\vspace{1cm}


\noindent\begin{minipage}{\linewidth}
  \ptitle{竹石}
  \addcontentsline{toc}{section}{\makebox[9cm]{竹石\dotfill{} 郑燮}}
  \pauthor{【清】郑\xpinyin*{\xpinyin{燮}{xiè}}}
  \vskip-3pt\begin{table}[H]
    \centering
    \begin{tabular}{@{}l@{}}
咬定青山不放松,立根原在破岩中。\\
千磨万击还坚劲,任尔东西南北风!
    \end{tabular}
  \end{table}
\end{minipage}
\vspace{1cm}


\noindent\begin{minipage}{\linewidth}
  \ptitle{所见}
  \addcontentsline{toc}{section}{\makebox[9cm]{所见\dotfill{} 袁枚}}
  \pauthor{【清】袁枚}
  \vskip-3pt\begin{table}[H]
    \centering
    \begin{tabular}{@{}l@{}}
牧童骑黄牛,歌声振林\xpinyin*{\xpinyin{樾}{yuè}}。\\
意欲捕鸣蝉,忽然闭口立。
    \end{tabular}
  \end{table}
\end{minipage}
\vspace{1cm}


\noindent\begin{minipage}{\linewidth}
  \ptitle{村居}
  \addcontentsline{toc}{section}{\makebox[9cm]{村居\dotfill{} 高鼎}}
  \pauthor{【清】高鼎}
  \vskip-3pt\begin{table}[H]
    \centering
    \begin{tabular}{@{}l@{}}
草长莺飞二月天,拂堤杨柳醉春烟。\\
儿童散学归来早,忙趁东风放纸\xpinyin*{\xpinyin{鸢}{yuān}}。
    \end{tabular}
  \end{table}
\end{minipage}
\vspace{1cm}


\noindent\begin{minipage}{\linewidth}
  \ptitle{己亥杂诗}
  \addcontentsline{toc}{section}{\makebox[9cm]{己亥杂诗\dotfill{} 龚自珍}}
  \pauthor{【清】龚自珍}
  \vskip-3pt\begin{table}[H]
    \centering
    \begin{tabular}{@{}l@{}}
九州生气恃风雷,万马齐\xpinyin*{\xpinyin{喑}{yīn}}究可哀。\\
我劝天公重抖擞,不拘一格降人才。
    \end{tabular}
  \end{table}
\end{minipage}
\vspace{1cm}


\noindent\begin{minipage}{\linewidth}
  \ptitle{关雎}
  \addcontentsline{toc}{section}{\makebox[9cm]{关雎\dotfill{} 《诗经·周南》}}
  \pauthor{《诗经·周南》}
  \vskip-3pt\begin{table}[H]
    \centering
    \begin{tabular}{@{}l@{}}
关关\xpinyin*{\xpinyin{雎}{jū}}\xpinyin*{\xpinyin{鸠}{jiū}},在河之洲。\xpinyin*{\xpinyin{窈}{yǎo}}\xpinyin*{\xpinyin{窕}{tiǎo}}淑女,君子好\xpinyin*{\xpinyin{逑}{qiú}}。\\
\\
参差\xpinyin*{\xpinyin{荇}{xìng}}菜,左右流之。窈窕淑女,\xpinyin*{\xpinyin{寤}{wù}}\xpinyin*{\xpinyin{寐}{mèi}}求之。\\
\\
求之不得,寤寐思服。悠哉悠哉,\xpinyin*{\xpinyin{辗}{zhǎn}}转反侧。\\
\\
参差荇菜,左右采之。窈窕淑女,琴瑟友之。\\
\\
参差荇菜,左右\xpinyin*{\xpinyin{芼}{mào}}之。窈窕淑女,钟鼓乐之。
    \end{tabular}
  \end{table}
\end{minipage}
\vspace{1cm}


\noindent\begin{minipage}{\linewidth}
  \ptitle{\xpinyin*{\xpinyin{蒹}{jiān}}\xpinyin*{\xpinyin{葭}{jiā}}}
  \addcontentsline{toc}{section}{\makebox[9cm]{蒹葭\dotfill{} 《诗经·秦风》}}
  \pauthor{《诗经·秦风》}
  \vskip-3pt\begin{table}[H]
    \centering
    \begin{tabular}{@{}l@{}}
蒹葭苍苍,白露为霜。所谓伊人,在水一方。\\
\xpinyin*{\xpinyin{溯}{sù}}\xpinyin*{\xpinyin{洄}{huí}}从之,道阻且长。溯游从之,宛在水中央。\\
\\
蒹葭萋萋,白露未\xpinyin*{\xpinyin{晞}{xī}}。所谓伊人,在水之\xpinyin*{\xpinyin{湄}{méi}}。\\
溯洄从之,道阻且\xpinyin*{\xpinyin{跻}{jī}}。溯游从之,宛在水中\xpinyin*{\xpinyin{坻}{chí}}。\\
\\
蒹葭采采,白露未已。所谓伊人,在水之\xpinyin*{\xpinyin{涘}{sì}}。\\
溯洄从之,道阻且右。溯游从之,宛在水中\xpinyin*{\xpinyin{沚}{zhǐ}}。
    \end{tabular}
  \end{table}
\end{minipage}
\vspace{1cm}


\noindent\begin{minipage}{\linewidth}
  \ptitle{十五从军征}
  \addcontentsline{toc}{section}{\makebox[9cm]{十五从军征\dotfill{} 乐府}}
  \pauthor{【汉】乐府}
  \vskip-3pt\begin{table}[H]
    \centering
    \begin{tabular}{@{}l@{}}
十五从军征,八十始得归。道逢乡里人:家中有阿谁?\\
遥看是君家,松柏\xpinyin*{\xpinyin{冢}{zhǒng}}累累。兔从狗\xpinyin*{\xpinyin{窦}{dòu}}入,\xpinyin*{\xpinyin{雉}{zhì}}从梁上飞,\\
中庭生旅谷,井上生旅葵。\xpinyin*{\xpinyin{舂}{chōng}}谷持作饭,采葵持作\xpinyin*{\xpinyin{羹}{gēng}}。\\
羹饭一时熟,不知\xpinyin*{\xpinyin{贻}{yí}}阿谁。出门东向看,泪落沾我衣。
    \end{tabular}
  \end{table}
\end{minipage}
\vspace{1cm}


\noindent\begin{minipage}{\linewidth}
  \ptitle{观沧海}
  \addcontentsline{toc}{section}{\makebox[9cm]{观沧海\dotfill{} 曹操}}
  \pauthor{【三国】曹操}
  \vskip-3pt\begin{table}[H]
    \centering
    \begin{tabular}{@{}l@{}}
东临碣石,以观沧海。水何\xpinyin*{\xpinyin{澹}{dàn}}澹,山岛\xpinyin*{\xpinyin{竦}{sǒng}}\xpinyin*{\xpinyin{峙}{zhì}}。\\
树木丛生,百草丰茂。秋风萧瑟,洪波涌起。\\
日月之行,若出其中;星汉灿烂,若出其里。\\
幸甚至哉,歌以咏志。
    \end{tabular}
  \end{table}
\end{minipage}
\vspace{1cm}


\noindent\begin{minipage}{\linewidth}
  \ptitle{饮酒}
  \addcontentsline{toc}{section}{\makebox[9cm]{饮酒\dotfill{} 陶渊明}}
  \pauthor{【东晋】陶渊明}
  \vskip-3pt\begin{table}[H]
    \centering
    \begin{tabular}{@{}l@{}}
结庐在人境,而无车马喧。问君何能尔?心远地自偏。\\
采菊东篱下,悠然见南山。山气日夕佳,飞鸟相与还。\\
此中有真意,欲辨已忘言。
    \end{tabular}
  \end{table}
\end{minipage}
\vspace{1cm}


\noindent\begin{minipage}{\linewidth}
  \ptitle{木兰辞}
  \addcontentsline{toc}{section}{\makebox[9cm]{木兰辞\dotfill{} 乐府}}
  \pauthor{【北朝】乐府}
  \vskip-3pt\begin{table}[H]
    \centering
    \begin{tabular}{@{}l@{}}
唧唧复唧唧,木兰当户织。不闻机\xpinyin*{\xpinyin{杼}{zhù}}声,唯闻女叹息。\\
问女何所思,问女何所忆。女亦无所思,女亦无所忆。\\
昨夜见军帖,\xpinyin*{\xpinyin{可}{kè}}\xpinyin*{\xpinyin{汗}{hán}}大点兵,军书十二卷,卷卷有爷名。\\
阿爷无大儿,木兰无长兄,愿为市\xpinyin*{\xpinyin{鞍}{ān}}马,从此替爷征。\\
\\
东市买骏马,西市买\xpinyin*{\xpinyin{鞍}{ān}}\xpinyin*{\xpinyin{鞯}{jiān}},南市买\xpinyin*{\xpinyin{辔}{pèi}}头,北市买长鞭。\\
旦辞爷娘去,暮宿黄河边。不闻爷娘唤女声,但闻黄河流水鸣\xpinyin*{\xpinyin{溅}{jiān}}溅。\\
旦辞黄河去,暮至黑山头。不闻爷娘唤女声,但闻燕山胡骑鸣\xpinyin*{\xpinyin{啾}{jiū}}啾。\\
\\
万里赴戎机,关山度若飞。\xpinyin*{\xpinyin{朔}{shuò}}气传金\xpinyin*{\xpinyin{柝}{tuò}},寒光照铁衣。\\
将军百战死,壮士十年归。归来见天子,天子坐明堂。\\
策勋十二转,赏赐百千强。可汗问所欲,木兰不用尚书郎,\\
愿驰千里足,送儿还故乡。\\
\\
爷娘闻女来,出郭相扶将;阿姊闻妹来,当户理红妆;\\
小弟闻姊来,磨刀霍霍向猪羊。开我东阁门,坐我西阁床。\\
脱我战时袍,著我旧时裳。当窗理云鬓,对镜帖花黄。\\
出门看火伴,火伴皆惊忙。同行十二年,不知木兰是女郎。\\
\\
雄兔脚\xpinyin*{\xpinyin{扑}{pū}}\xpinyin*{\xpinyin{朔}{shuò}},雌兔眼迷离;双兔傍地走,安能辨我是雄雌?
    \end{tabular}
  \end{table}
\end{minipage}
\vspace{1cm}


\noindent\begin{minipage}{\linewidth}
  \ptitle{送杜少府之任蜀州}
  \addcontentsline{toc}{section}{\makebox[9cm]{送杜少府之任蜀州\dotfill{} 王勃}}
  \pauthor{【唐】王勃}
  \vskip-3pt\begin{table}[H]
    \centering
    \begin{tabular}{@{}l@{}}
城阙辅三秦,风烟望五津。与君离别意,同是宦游人。\\
海内存知己,天涯若比邻。无为在歧路,儿女共沾巾。
    \end{tabular}
  \end{table}
\end{minipage}
\vspace{1cm}


\noindent\begin{minipage}{\linewidth}
  \ptitle{登幽州台歌}
  \addcontentsline{toc}{section}{\makebox[9cm]{登幽州台歌\dotfill{} 陈子昂}}
  \pauthor{【唐】陈子昂}
  \vskip-3pt\begin{table}[H]
    \centering
    \begin{tabular}{@{}l@{}}
前不见古人,后不见来者。\\
念天地之悠悠,独怆然而涕下。
    \end{tabular}
  \end{table}
\end{minipage}
\vspace{1cm}


\noindent\begin{minipage}{\linewidth}
  \ptitle{次北固山下}
  \addcontentsline{toc}{section}{\makebox[9cm]{次北固山下\dotfill{} 王湾}}
  \pauthor{【唐】王湾}
  \vskip-3pt\begin{table}[H]
    \centering
    \begin{tabular}{@{}l@{}}
客路青山外,行舟绿水前。潮平两岸阔,风正一帆悬。\\
海日生残夜,江春入旧年。乡书何处达?归雁洛阳边。
    \end{tabular}
  \end{table}
\end{minipage}
\vspace{1cm}


\noindent\begin{minipage}{\linewidth}
  \ptitle{使至塞上}
  \addcontentsline{toc}{section}{\makebox[9cm]{使至塞上\dotfill{} 王维}}
  \pauthor{【唐】王维}
  \vskip-3pt\begin{table}[H]
    \centering
    \begin{tabular}{@{}l@{}}
单车欲问边,属国过居延。征蓬出汉塞,归雁入胡天。\\
大漠孤烟直,长河落日圆。萧关逢候骑,都护在燕然。
    \end{tabular}
  \end{table}
\end{minipage}
\vspace{1cm}


\noindent\begin{minipage}{\linewidth}
  \ptitle{闻王昌龄左迁龙标遥有此寄}
  \addcontentsline{toc}{section}{\makebox[9cm]{闻王昌龄左迁龙标遥有此寄\dotfill{} 李白}}
  \pauthor{【唐】李白}
  \vskip-3pt\begin{table}[H]
    \centering
    \begin{tabular}{@{}l@{}}
杨花落尽子规啼,闻道龙标过五溪。\\
我寄愁心与明月,随风直到夜郎西。
    \end{tabular}
  \end{table}
\end{minipage}
\vspace{1cm}


\noindent\begin{minipage}{\linewidth}
  \ptitle{行路难}
  \addcontentsline{toc}{section}{\makebox[9cm]{行路难\dotfill{} 李白}}
  \pauthor{【唐】李白}
  \vskip-3pt\begin{table}[H]
    \centering
    \begin{tabular}{@{}l@{}}
金\xpinyin*{\xpinyin{樽}{zūn}}清酒斗十千,玉盘珍\xpinyin*{\xpinyin{馐}{xiū}}直万钱。停杯投\xpinyin*{\xpinyin{箸}{zhù}}不能食,拔剑四顾心茫然。\\
欲渡黄河冰塞川,将登太行雪满山。闲来垂钓碧溪上,忽复乘舟梦日边。\\
行路难,行路难,多歧路,今安在?长风破浪会有时,直挂云帆济沧海。
    \end{tabular}
  \end{table}
\end{minipage}
\vspace{1cm}


\noindent\begin{minipage}{\linewidth}
  \ptitle{登金陵凤凰台}
  \addcontentsline{toc}{section}{\makebox[9cm]{登金陵凤凰台\dotfill{} 李白}}
  \pauthor{【唐】李白}
  \vskip-3pt\begin{table}[H]
    \centering
    \begin{tabular}{@{}l@{}}
凤凰台上凤凰游,凤去台空江自流。吴宫花草埋幽径,晋代衣冠成古丘。\\
三山半落青天外,二水中分白鹭洲。总为浮云能蔽日,长安不见使人愁。
    \end{tabular}
  \end{table}
\end{minipage}
\vspace{1cm}


\noindent\begin{minipage}{\linewidth}
  \ptitle{黄鹤楼}
  \addcontentsline{toc}{section}{\makebox[9cm]{黄鹤楼\dotfill{} 崔颢}}
  \pauthor{【唐】崔\xpinyin*{\xpinyin{颢}{hào}}}
  \vskip-3pt\begin{table}[H]
    \centering
    \begin{tabular}{@{}l@{}}
昔人已乘黄鹤去,此地空余黄鹤楼。黄鹤一去不复返,白云千载空悠悠。\\
晴川历历汉阳树,芳草萋萋鹦鹉洲。日暮乡关何处是?烟波江上使人愁。
    \end{tabular}
  \end{table}
\end{minipage}
\vspace{1cm}


\noindent\begin{minipage}{\linewidth}
  \ptitle{望岳}
  \addcontentsline{toc}{section}{\makebox[9cm]{望岳\dotfill{} 杜甫}}
  \pauthor{【唐】杜甫}
  \vskip-3pt\begin{table}[H]
    \centering
    \begin{tabular}{@{}l@{}}
\xpinyin*{\xpinyin{岱}{dài}}宗夫如何?齐鲁青未了。造化钟神秀,阴阳割昏晓。\\
荡胸生曾云,决\xpinyin*{\xpinyin{眦}{zì}}入归鸟。会当凌绝顶,一览众山小。
    \end{tabular}
  \end{table}
\end{minipage}
\vspace{1cm}


\noindent\begin{minipage}{\linewidth}
  \ptitle{春望}
  \addcontentsline{toc}{section}{\makebox[9cm]{春望\dotfill{} 杜甫}}
  \pauthor{【唐】杜甫}
  \vskip-3pt\begin{table}[H]
    \centering
    \begin{tabular}{@{}l@{}}
国破山河在,城春草木深。感时花溅泪,恨别鸟惊心。\\
烽火连三月,家书抵万金。白头搔更短,浑欲不胜\xpinyin*{\xpinyin{簪}{zān}}。
    \end{tabular}
  \end{table}
\end{minipage}
\vspace{1cm}


\noindent\begin{minipage}{\linewidth}
  \ptitle{茅屋为秋风所破歌}
  \addcontentsline{toc}{section}{\makebox[9cm]{茅屋为秋风所破歌\dotfill{} 杜甫}}
  \pauthor{【唐】杜甫}
  \vskip-3pt\begin{table}[H]
    \centering
    \begin{tabular}{@{}l@{}}
八月秋高风怒号,卷我屋上三重茅。\\
茅飞渡江洒江郊,高者挂罥长林梢,下者飘转沉塘\xpinyin*{\xpinyin{坳}{ào}}。\\
南村群童欺我老无力,忍能对面为盗贼。\\
公然抱茅入竹去,唇焦口燥呼不得,归来倚杖自叹息。\\
俄顷风定云墨色,秋天漠漠向昏黑。\\
布\xpinyin*{\xpinyin{衾}{qīn}}多年冷似铁,骄儿恶卧踏里裂。\\
床头屋漏无干处,雨脚如麻未断绝。\\
自经丧乱少睡眠,长夜沾湿何由彻!\\
安得广厦千万间,大庇天下寒士俱欢颜,风雨不动安如山!\\
呜呼!何时眼前突兀见此屋,吾庐独破受冻死亦足!
    \end{tabular}
  \end{table}
\end{minipage}
\vspace{1cm}


\noindent\begin{minipage}{\linewidth}
  \ptitle{白雪歌送武判官归京}
  \addcontentsline{toc}{section}{\makebox[9cm]{白雪歌送武判官归京\dotfill{} 岑参}}
  \pauthor{【唐】岑参}
  \vskip-3pt\begin{table}[H]
    \centering
    \begin{tabular}{@{}l@{}}
北风卷地白草折,胡天八月即飞雪。忽如一夜春风来,千树万树梨花开。\\
散入珠帘湿罗幕,狐裘不暖锦衾薄。将军角弓不得控,都护铁衣冷难着。\\
瀚海阑干百丈冰,愁云惨淡万里凝。中军置酒饮归客,胡琴琵琶与羌笛。\\
纷纷暮雪下辕门,风\xpinyin*{\xpinyin{掣}{chè}}红旗冻不翻。轮台东门送君去,去时雪满天山路。\\
山回路转不见君,雪上空留马行处。
    \end{tabular}
  \end{table}
\end{minipage}
\vspace{1cm}


\noindent\begin{minipage}{\linewidth}
  \ptitle{酬乐天扬州初逢席上见赠}
  \addcontentsline{toc}{section}{\makebox[9cm]{酬乐天扬州初逢席上见赠\dotfill{} 刘禹锡}}
  \pauthor{【唐】刘禹锡}
  \vskip-3pt\begin{table}[H]
    \centering
    \begin{tabular}{@{}l@{}}
巴山楚水凄凉地,二十三年弃置身。怀旧空吟闻笛赋,到乡翻似烂柯人。\\
沉舟侧畔千帆过,病树前头万木春。今日听君歌一曲,暂凭杯酒长精神。
    \end{tabular}
  \end{table}
\end{minipage}
\vspace{1cm}


\noindent\begin{minipage}{\linewidth}
  \ptitle{卖炭翁}
  \addcontentsline{toc}{section}{\makebox[9cm]{卖炭翁\dotfill{} 白居易}}
  \pauthor{【唐】白居易}
  \vskip-3pt\begin{table}[H]
    \centering
    \begin{tabular}{@{}l@{}}
    卖炭翁,伐薪烧炭南山中。满面尘灰烟火色,两鬓苍苍十指黑。\\
卖炭得钱何所营?身上衣裳口中食。可怜身上衣正单,心忧炭贱愿天寒。\\
夜来城外一尺雪,晓驾炭车辗冰辙。牛困人饥日已高,市南门外泥中歇。\\
\\
翩翩两骑来是谁?黄衣使者白衫儿。手把文书口称\xpinyin*{\xpinyin{敕}{chì}},回车叱牛牵向北。\\
一车炭,千余斤,宫使驱将惜不得。半匹红绡一丈绫,系向牛头充炭直。
    \end{tabular}
  \end{table}
\end{minipage}
\vspace{1cm}


\noindent\begin{minipage}{\linewidth}
  \ptitle{钱塘湖春行}
  \addcontentsline{toc}{section}{\makebox[9cm]{钱塘湖春行\dotfill{} 白居易}}
  \pauthor{【唐】白居易}
  \vskip-3pt\begin{table}[H]
    \centering
    \begin{tabular}{@{}l@{}}
孤山寺北贾亭西,水面初平云脚低。几处早莺争暖树,谁家新燕啄春泥。\\
乱花渐欲迷人眼,浅草才能没马蹄。最爱湖东行不足,绿杨阴里白沙堤。
    \end{tabular}
  \end{table}
\end{minipage}
\vspace{1cm}


\noindent\begin{minipage}{\linewidth}
  \ptitle{雁门太守行}
  \addcontentsline{toc}{section}{\makebox[9cm]{雁门太守行\dotfill{} 李贺}}
  \pauthor{【唐】李贺}
  \vskip-3pt\begin{table}[H]
    \centering
    \begin{tabular}{@{}l@{}}
黑云压城城欲摧,甲光向日金鳞开。角声满天秋色里,塞上燕脂凝夜紫。\\
半卷红旗临易水,霜重鼓寒声不起。报君黄金台上意,提携玉龙为君死。
    \end{tabular}
  \end{table}
\end{minipage}
\vspace{1cm}


\noindent\begin{minipage}{\linewidth}
  \ptitle{赤壁}
  \addcontentsline{toc}{section}{\makebox[9cm]{赤壁\dotfill{} 杜牧}}
  \pauthor{【唐】杜牧}
  \vskip-3pt\begin{table}[H]
    \centering
    \begin{tabular}{@{}l@{}}
折戟沉沙铁未销,自将磨洗认前朝。\\
东风不与周郎便,铜雀春深锁二乔。
    \end{tabular}
  \end{table}
\end{minipage}
\vspace{1cm}


\noindent\begin{minipage}{\linewidth}
  \ptitle{泊秦淮}
  \addcontentsline{toc}{section}{\makebox[9cm]{泊秦淮\dotfill{} 杜牧}}
  \pauthor{【唐】杜牧}
  \vskip-3pt\begin{table}[H]
    \centering
    \begin{tabular}{@{}l@{}}
烟笼寒水月笼沙,夜泊秦淮近酒家。\\
商女不知亡国恨,隔江犹唱后庭花。
    \end{tabular}
  \end{table}
\end{minipage}
\vspace{1cm}


\noindent\begin{minipage}{\linewidth}
  \ptitle{夜雨寄北}
  \addcontentsline{toc}{section}{\makebox[9cm]{夜雨寄北\dotfill{} 李商隐}}
  \pauthor{【唐】李商隐}
  \vskip-3pt\begin{table}[H]
    \centering
    \begin{tabular}{@{}l@{}}
君问归期未有期,巴山夜雨涨秋池。\\
何当共剪西窗烛,却话巴山夜雨时。
    \end{tabular}
  \end{table}
\end{minipage}
\vspace{1cm}


\noindent\begin{minipage}{\linewidth}
  \ptitle{无题}
  \addcontentsline{toc}{section}{\makebox[9cm]{无题\dotfill{} 李商隐}}
  \pauthor{【唐】李商隐}
  \vskip-3pt\begin{table}[H]
    \centering
    \begin{tabular}{@{}l@{}}
相见时难别亦难,东风无力百花残。春蚕到死丝方尽,蜡炬成灰泪始干。\\
晓镜但愁云鬓改,夜吟应觉月光寒。蓬山此去无多路,青鸟殷勤为探看。
    \end{tabular}
  \end{table}
\end{minipage}
\vspace{1cm}


\noindent\begin{minipage}{\linewidth}
  \ptitle{登飞来峰}
  \addcontentsline{toc}{section}{\makebox[9cm]{登飞来峰\dotfill{} 王安石}}
  \pauthor{【宋】王安石}
  \vskip-3pt\begin{table}[H]
    \centering
    \begin{tabular}{@{}l@{}}
飞来山上千寻塔,闻说鸡鸣见日升。\\
不畏浮云遮望眼,自缘身在最高层。
    \end{tabular}
  \end{table}
\end{minipage}
\vspace{1cm}


\noindent\begin{minipage}{\linewidth}
  \ptitle{游山西村}
  \addcontentsline{toc}{section}{\makebox[9cm]{游山西村\dotfill{} 陆游}}
  \pauthor{【宋】陆游}
  \vskip-3pt\begin{table}[H]
    \centering
    \begin{tabular}{@{}l@{}}
莫笑农家腊酒浑,丰年留客足鸡豚。山重水复疑无路,柳暗花明又一村。\\
箫鼓追随春社近,衣冠简朴古风存。从今若许闲乘月,拄杖无时夜叩门。
    \end{tabular}
  \end{table}
\end{minipage}
\vspace{1cm}


\noindent\begin{minipage}{\linewidth}
  \ptitle{过零丁洋}
  \addcontentsline{toc}{section}{\makebox[9cm]{过零丁洋\dotfill{} 文天祥}}
  \pauthor{【宋】文天祥}
  \vskip-3pt\begin{table}[H]
    \centering
    \begin{tabular}{@{}l@{}}
辛苦遭逢起一经,干戈寥落四周星。山河破碎风飘絮,身世浮沉雨打萍。\\
惶恐滩头说惶恐,零丁洋里叹零丁。人生自古谁无死,留取丹心照汗青。
    \end{tabular}
  \end{table}
\end{minipage}
\vspace{1cm}


\noindent\begin{minipage}{\linewidth}
  \ptitle{己亥杂诗 其五}
  \addcontentsline{toc}{section}{\makebox[9cm]{己亥杂诗 其五\dotfill{} 龚自珍}}
  \pauthor{【清】龚自珍}
  \vskip-3pt\begin{table}[H]
    \centering
    \begin{tabular}{@{}l@{}}
浩荡离愁白日斜,吟鞭东指即天涯。\\
落红不是无情物,化作春泥更护花。
    \end{tabular}
  \end{table}
\end{minipage}
\vspace{1cm}


\noindent\begin{minipage}{\linewidth}
  \ptitle{相见欢}
  \addcontentsline{toc}{section}{\makebox[9cm]{相见欢\dotfill{} 李煜}}
  \pauthor{【南唐】李\xpinyin*{\xpinyin{煜}{yù}}}
  \vskip-3pt\begin{table}[H]
    \centering
    \begin{tabular}{@{}l@{}}
无言独上西楼,月如钩。 寂寞梧桐深院锁清秋。\\
\\
剪不断,理还乱,是离愁,别是一般滋味在心头。
    \end{tabular}
  \end{table}
\end{minipage}
\vspace{1cm}


\noindent\begin{minipage}{\linewidth}
  \ptitle{渔家傲}
  \addcontentsline{toc}{section}{\makebox[9cm]{渔家傲\dotfill{} 范仲淹}}
  \pauthor{【宋】范仲淹}
  \vskip-3pt\begin{table}[H]
    \centering
    \begin{tabular}{@{}l@{}}
塞下秋来风景异,衡阳雁去无留意。四面边声连角起。\\
千嶂里,长烟落日孤城闭。\\
\\
浊酒一杯家万里,燕然未勒归无计。羌管悠悠霜满地。\\
人不寐,将军白发征夫泪。
    \end{tabular}
  \end{table}
\end{minipage}
\vspace{1cm}


\noindent\begin{minipage}{\linewidth}
  \ptitle{浣溪沙}
  \addcontentsline{toc}{section}{\makebox[9cm]{浣溪沙\dotfill{} 晏殊}}
  \pauthor{【宋】\xpinyin*{\xpinyin{晏}{yàn}}殊}
  \vskip-3pt\begin{table}[H]
    \centering
    \begin{tabular}{@{}l@{}}
一曲新词酒一杯,去年天气旧亭台。夕阳西下几时回。\\
\\
无可奈何花落去,似曾相识燕归来。小园香径独徘徊。
    \end{tabular}
  \end{table}
\end{minipage}
\vspace{1cm}


\noindent\begin{minipage}{\linewidth}
  \ptitle{江城子·密州出猎}
  \addcontentsline{toc}{section}{\makebox[9cm]{江城子·密州出猎\dotfill{} 苏轼}}
  \pauthor{【宋】苏轼}
  \vskip-3pt\begin{table}[H]
    \centering
    \begin{tabular}{@{}l@{}}
老夫聊发少年狂,左牵黄,右擎苍。锦帽貂裘,千骑卷平冈。\\
为报倾城随太守,亲射虎,看孙郎。\\
\\
酒酣胸胆尚开张,鬓微霜,又何妨!持节云中,何日遣冯唐。\\
会挽雕弓如满月,西北望,射天狼。
    \end{tabular}
  \end{table}
\end{minipage}
\vspace{1cm}


\noindent\begin{minipage}{\linewidth}
  \ptitle{渔家傲}
  \addcontentsline{toc}{section}{\makebox[9cm]{渔家傲\dotfill{} 李清照}}
  \pauthor{【宋】李清照}
  \vskip-3pt\begin{table}[H]
    \centering
    \begin{tabular}{@{}l@{}}
天接云涛连晓雾,星河欲转千帆舞。\\
仿佛梦魂归帝所,闻天语,殷勤问我归何处。\\
\\
我报路长\xpinyin*{\xpinyin{嗟}{jiē}}日暮,学诗谩有惊人句。\\
九万里风鹏正举。风休住,蓬舟吹取三山去!
    \end{tabular}
  \end{table}
\end{minipage}
\vspace{1cm}


\noindent\begin{minipage}{\linewidth}
  \ptitle{南乡子}
  \addcontentsline{toc}{section}{\makebox[9cm]{南乡子\dotfill{} 辛弃疾}}
  \pauthor{【宋】辛弃疾}
  \vskip-3pt\begin{table}[H]
    \centering
    \begin{tabular}{@{}l@{}}
何处望神州?满眼风光北固楼。\\
千古兴亡多少事?悠悠。不尽长江滚滚流。\\
\\
年少万\xpinyin*{\xpinyin{兜}{dōu}}\xpinyin*{\xpinyin{鍪}{móu}},坐断东南战未休。\\
天下英雄谁敌手?曹刘。生子当如孙仲谋。
    \end{tabular}
  \end{table}
\end{minipage}
\vspace{1cm}


\noindent\begin{minipage}{\linewidth}
  \ptitle{破阵子·为陈同甫赋壮语以寄之}
  \addcontentsline{toc}{section}{\makebox[9cm]{破阵子·为陈同甫赋壮语以寄之\dotfill{} 辛弃疾}}
  \pauthor{【宋】辛弃疾}
  \vskip-3pt\begin{table}[H]
    \centering
    \begin{tabular}{@{}l@{}}
醉里挑灯看剑,梦回吹角连营。\\
八百里分麾下炙,五十弦翻塞外声。沙场秋点兵。\\
\\
马作\xpinyin*{\xpinyin{的}{dì}}\xpinyin*{\xpinyin{卢}{lú}}飞快,弓如霹雳弦惊。\\
了却君王天下事,赢得生前身后名。可怜白发生。
    \end{tabular}
  \end{table}
\end{minipage}
\vspace{1cm}


\noindent\begin{minipage}{\linewidth}
  \ptitle{小重山}
  \addcontentsline{toc}{section}{\makebox[9cm]{小重山\dotfill{} 岳飞}}
  \pauthor{【宋】岳飞}
  \vskip-3pt\begin{table}[H]
    \centering
    \begin{tabular}{@{}l@{}}
昨夜寒蛩不住鸣。惊回千里梦,已三更。\\
起来独自绕阶行。人悄悄,帘外月胧明。\\
\\
白首为功名。旧山松竹老,阻归程。\\
欲将心事付瑶琴。知音少,弦断有谁听。
    \end{tabular}
  \end{table}
\end{minipage}
\vspace{1cm}


\noindent\begin{minipage}{\linewidth}
  \ptitle{满江红}
  \addcontentsline{toc}{section}{\makebox[9cm]{满江红\dotfill{} 岳飞}}
  \pauthor{【宋】岳飞}
  \vskip-3pt\begin{table}[H]
    \centering
    \begin{tabular}{@{}l@{}}
怒发冲冠,凭阑处、潇潇雨歇。\\
抬望眼,仰天长啸,壮怀激烈。\\
三十功名尘与土,八千里路云和月。\\
莫等闲、白了少年头,空悲切。\\
\\
靖康耻,犹未雪。臣子恨,何时灭。\\
驾长车,踏破贺兰山缺。\\
壮志饥餐胡虏肉,笑谈渴饮匈奴血。\\
待从头、收拾旧山河,朝天阙⒂。
    \end{tabular}
  \end{table}
\end{minipage}
\vspace{1cm}


\noindent\begin{minipage}{\linewidth}
  \ptitle{满江红}
  \addcontentsline{toc}{section}{\makebox[9cm]{满江红\dotfill{} 秋瑾}}
  \pauthor{【近代】秋瑾}
  \vskip-3pt\begin{table}[H]
    \centering
    \begin{tabular}{@{}l@{}}
小住京华,早又是、中秋佳节。\\
为篱下、黄花开遍,秋容如拭。\\
四面歌残终破楚,八年风味徒思浙。\\
苦将侬、强派作娥眉,殊未屑!\\
\\
身不得,男儿列,心却比,男儿烈。\\
算平生肝胆,因人常热。\\
俗子胸襟谁识我?英雄末路当磨折。\\
莽红尘,何处觅知音?青衫湿!
    \end{tabular}
  \end{table}
\end{minipage}
\vspace{1cm}


\noindent\begin{minipage}{\linewidth}
  \ptitle{天净沙·秋思}
  \addcontentsline{toc}{section}{\makebox[9cm]{天净沙·秋思\dotfill{} 马致远}}
  \pauthor{【元】马致远}
  \vskip-3pt\begin{table}[H]
    \centering
    \begin{tabular}{@{}l@{}}
枯藤老树昏鸦,小桥流水人家,古道西风瘦马。\\
夕阳西下,断肠人在天涯。
    \end{tabular}
  \end{table}
\end{minipage}
\vspace{1cm}


\noindent\begin{minipage}{\linewidth}
  \ptitle{山坡羊·潼关怀古}
  \addcontentsline{toc}{section}{\makebox[9cm]{山坡羊·潼关怀古\dotfill{} 张养浩}}
  \pauthor{【元】张养浩}
  \vskip-3pt\begin{table}[H]
    \centering
    \begin{tabular}{@{}l@{}}
峰峦如聚,波涛如怒,山河表里潼关路。\\
望西都,意踌躇,伤心秦汉经行处,宫阙万间都做了土。\\
兴,百姓苦;亡,百姓苦!
    \end{tabular}
  \end{table}
\end{minipage}
\vspace{1cm}


\noindent\begin{minipage}{\linewidth}
  \ptitle{静女}
  \addcontentsline{toc}{section}{\makebox[9cm]{静女\dotfill{} 《诗经·邶风》}}
  \pauthor{《诗经·邶风》}
  \vskip-3pt\begin{table}[H]
    \centering
    \begin{tabular}{@{}l@{}}
静女其姝,俟我于城隅。爱而不见,搔首踟蹰。\\
\\
静女其娈,贻我彤管。彤管有炜,说怿女美。\\
\\
自牧归荑,洵美且异。匪女之为美,美人之贻。
    \end{tabular}
  \end{table}
\end{minipage}
\vspace{1cm}


\noindent\begin{minipage}{\linewidth}
  \ptitle{无衣}
  \addcontentsline{toc}{section}{\makebox[9cm]{无衣\dotfill{} 《诗经·秦风》}}
  \pauthor{《诗经·秦风》}
  \vskip-3pt\begin{table}[H]
    \centering
    \begin{tabular}{@{}l@{}}
岂曰无衣?与子同袍。王于兴师,修我戈矛,与子同仇。\\
\\
岂曰无衣?与子同泽。王于兴师,修我矛戟,与子偕作。\\
\\
岂曰无衣?与子同裳。王于兴师,修我甲兵,与子偕行。
    \end{tabular}
  \end{table}
\end{minipage}
\vspace{1cm}


\noindent\begin{minipage}{\linewidth}
  \ptitle{离骚}
  \addcontentsline{toc}{section}{\makebox[9cm]{离骚\dotfill{} 屈原}}
  \pauthor{【战国】屈原}
  \vskip-3pt\begin{table}[H]
    \centering
    \begin{tabular}{@{}l@{}}
帝高阳之苗裔兮,朕皇考曰伯庸。\\
摄提贞于孟陬兮,惟庚寅吾以降。\\
皇览揆余初度兮,肇锡余以嘉名。\\
名余曰正则兮,字余曰灵均。\\
纷吾既有此内美兮,又重之以修能。\\
扈江离与辟芷兮,纫秋兰以为佩。\\
汩余若将不及兮,恐年岁之不吾与。\\
朝搴阰之木兰兮,夕揽洲之宿莽。\\
日月忽其不淹兮,春与秋其代序。\\
惟草木之零落兮,恐美人之迟暮。\\
不抚壮而弃秽兮,何不改乎此度?\\
乘骐骥以驰骋兮,来吾道夫先路。
    \end{tabular}
  \end{table}
\end{minipage}
\vspace{1cm}


\noindent\begin{minipage}{\linewidth}
  \ptitle{涉江采芙蓉}
  \addcontentsline{toc}{section}{\makebox[9cm]{涉江采芙蓉\dotfill{} 《古诗十九首》}}
  \pauthor{《古诗十九首》}
  \vskip-3pt\begin{table}[H]
    \centering
    \begin{tabular}{@{}l@{}}
涉江采芙蓉,兰泽多芳草。采之欲遗谁,所思在远道。\\
还顾望旧乡,长路漫浩浩。同心而离居,忧伤以终老。
    \end{tabular}
  \end{table}
\end{minipage}
\vspace{1cm}


\noindent\begin{minipage}{\linewidth}
  \ptitle{短歌行}
  \addcontentsline{toc}{section}{\makebox[9cm]{短歌行\dotfill{} 曹操}}
  \pauthor{【三国】曹操}
  \vskip-3pt\begin{table}[H]
    \centering
    \begin{tabular}{@{}l@{}}
对酒当歌,人生几何?譬如朝露,去日苦多。\\
慨当以慷,忧思难忘。何以解忧?唯有杜康。\\
青青子衿,悠悠我心。但为君故,沉吟至今。\\
呦呦鹿鸣,食野之苹。我有嘉宾,鼓瑟吹笙。\\
\\
明明如月,何时可\xpinyin*{\xpinyin{掇}{duō}}?忧从中来,不可断绝。\\
越陌度阡,枉用相存。契阔谈宴,心念旧恩。\\
月明星稀,乌鹊南飞。绕树三匝,何枝可依?\\
山不厌高,海不厌深。周公吐\xpinyin*{\xpinyin{哺}{bǔ}},天下归心。
    \end{tabular}
  \end{table}
\end{minipage}
\vspace{1cm}


\noindent\begin{minipage}{\linewidth}
  \ptitle{归园田居·其一}
  \addcontentsline{toc}{section}{\makebox[9cm]{归园田居·其一\dotfill{} 东晋陶渊明}}
  \pauthor{东晋陶渊明}
  \vskip-3pt\begin{table}[H]
    \centering
    \begin{tabular}{@{}l@{}}
少无适俗韵,性本爱丘山。误落尘网中,一去三十年。\\
羁鸟恋旧林,池鱼思故渊。开荒南野际,守拙归园田。\\
方宅十余亩,草屋八九间。榆柳荫后檐,桃李罗堂前。\\
暧暧远人村,依依墟里烟。狗吠深巷中,鸡鸣桑树颠。\\
户庭无尘杂,虚室有余闲。久在樊笼里,复得返自然。
    \end{tabular}
  \end{table}
\end{minipage}
\vspace{1cm}


\noindent\begin{minipage}{\linewidth}
  \ptitle{拟行路难·其四}
  \addcontentsline{toc}{section}{\makebox[9cm]{拟行路难·其四\dotfill{} 鲍照}}
  \pauthor{【宋】鲍照}
  \vskip-3pt\begin{table}[H]
    \centering
    \begin{tabular}{@{}l@{}}
泻水置平地,各自东西南北流。人生亦有命,安能行叹复坐愁?\\
酌酒以自宽,举杯断绝歌路难。心非木石岂无感?吞声\xpinyin*{\xpinyin{踯}{zhí}}\xpinyin*{\xpinyin{躅}{zhú}}不敢言。
    \end{tabular}
  \end{table}
\end{minipage}
\vspace{1cm}


\noindent\begin{minipage}{\linewidth}
  \ptitle{春江花月夜}
  \addcontentsline{toc}{section}{\makebox[9cm]{春江花月夜\dotfill{} 张若虚}}
  \pauthor{【唐】张若虚}
  \vskip-3pt\begin{table}[H]
    \centering
    \begin{tabular}{@{}l@{}}
春江潮水连海平,海上明月共潮生。滟滟随波千万里,何处春江无月明。\\
江流宛转绕芳甸,月照花林皆似\xpinyin*{\xpinyin{霰}{xiàn}}。空里流霜不觉飞,\xpinyin*{\xpinyin{汀}{tīng}}上白沙看不见。\\
江天一色无纤尘,皎皎空中孤月轮。江畔何人初见月?江月何年初照人?\\
人生代代无穷已,江月年年只相似。不知江月待何人,但见长江送流水。\\
白云一片去悠悠,青枫浦上不胜愁。谁家今夜扁舟子?何处相思明月楼?\\
可怜楼上月徘徊,应照离人妆镜台。玉户帘中卷不去,捣衣\xpinyin*{\xpinyin{砧}{zhēn}}上拂还来。\\
此时相望不相闻,愿逐月华流照君。鸿雁长飞光不度,鱼龙潜跃水成文。\\
昨夜闲潭梦落花,可怜春半不还家。江水流春去欲尽,江潭落月复西斜。\\
斜月沉沉藏海雾,\xpinyin*{\xpinyin{碣}{jié}}石潇湘无限路。不知乘月几人归,落月摇情满江树。
    \end{tabular}
  \end{table}
\end{minipage}
\vspace{1cm}


\noindent\begin{minipage}{\linewidth}
  \ptitle{山居秋暝}
  \addcontentsline{toc}{section}{\makebox[9cm]{山居秋暝\dotfill{} 王维}}
  \pauthor{【唐】王维}
  \vskip-3pt\begin{table}[H]
    \centering
    \begin{tabular}{@{}l@{}}
空山新雨后,天气晚来秋。明月松间照,清泉石上流。\\
竹喧归\xpinyin*{\xpinyin{浣}{huàn}}女,莲动下渔舟。随意春芳歇,王孙自可留。
    \end{tabular}
  \end{table}
\end{minipage}
\vspace{1cm}


\noindent\begin{minipage}{\linewidth}
  \ptitle{燕歌行}
  \addcontentsline{toc}{section}{\makebox[9cm]{燕歌行\dotfill{} 高适}}
  \pauthor{【唐】高适}
  \vskip-3pt\begin{table}[H]
    \centering
    \begin{tabular}{@{}l@{}}
 汉家烟尘在东北,汉将辞家破残贼。\\
 男儿本自重横行,天子非常赐颜色。\\
 \xpinyin*{\xpinyin{摐}{chuāng}}金伐鼓下榆关,旌旗逶迤碣石间。\\
 校尉羽书飞瀚海,单于猎火照狼山。\\
 山川萧条极边土,胡骑凭陵杂风雨。\\
 战士军前半死生,美人帐下犹歌舞。\\
 大漠穷秋塞草衰,孤城落日斗兵稀。\\
 身当恩遇常轻敌,力尽关山未解围。\\
 铁衣远戍辛勤久,玉箸应啼别离后。\\
 少妇城南欲断肠,征人蓟北空回首。\\
 边风飘飘那可度,绝域苍茫更何有。\\
 杀气三时作阵云,寒声一夜传刁斗。\\
 相看白刃血纷纷,死节从来岂顾勋。\\
君不见沙场征战苦,至今犹忆李将军。
    \end{tabular}
  \end{table}
\end{minipage}
\vspace{1cm}


\noindent\begin{minipage}{\linewidth}
  \ptitle{蜀相}
  \addcontentsline{toc}{section}{\makebox[9cm]{蜀相\dotfill{} 杜甫}}
  \pauthor{【唐】杜甫}
  \vskip-3pt\begin{table}[H]
    \centering
    \begin{tabular}{@{}l@{}}
丞相祠堂何处寻?锦官城外柏森森。映阶碧草自春色,隔叶黄鹂空好音。\\
三顾频烦天下计,两朝开济老臣心。出师未捷身先死,长使英雄泪满襟。
    \end{tabular}
  \end{table}
\end{minipage}
\vspace{1cm}


\noindent\begin{minipage}{\linewidth}
  \ptitle{客至}
  \addcontentsline{toc}{section}{\makebox[9cm]{客至\dotfill{} 杜甫}}
  \pauthor{【唐】杜甫}
  \vskip-3pt\begin{table}[H]
    \centering
    \begin{tabular}{@{}l@{}}
舍南舍北皆春水,但见群鸥日日来。花径不曾缘客扫,蓬门今始为君开。\\
盘\xpinyin*{\xpinyin{飧}{sūn}}市远无兼味,樽酒家贫只旧\xpinyin*{\xpinyin{醅}{pēi}}。肯与邻翁相对饮,隔篱呼取尽余杯。
    \end{tabular}
  \end{table}
\end{minipage}
\vspace{1cm}


\noindent\begin{minipage}{\linewidth}
  \ptitle{登高}
  \addcontentsline{toc}{section}{\makebox[9cm]{登高\dotfill{} 杜甫}}
  \pauthor{【唐】杜甫}
  \vskip-3pt\begin{table}[H]
    \centering
    \begin{tabular}{@{}l@{}}
风急天高猿啸哀,\xpinyin*{\xpinyin{渚}{zhǔ}}清沙白鸟飞回。无边落木萧萧下,不尽长江滚滚来。\\
万里悲秋常作客,百年多病独登台。艰难苦恨繁霜鬓,潦倒新停浊酒杯。
    \end{tabular}
  \end{table}
\end{minipage}
\vspace{1cm}


\noindent\begin{minipage}{\linewidth}
  \ptitle{登岳阳楼}
  \addcontentsline{toc}{section}{\makebox[9cm]{登岳阳楼\dotfill{} 杜甫}}
  \pauthor{【唐】杜甫}
  \vskip-3pt\begin{table}[H]
    \centering
    \begin{tabular}{@{}l@{}}
昔闻洞庭水,今上岳阳楼。吴楚东南\xpinyin*{\xpinyin{坼}{chè}},乾坤日夜浮。\\
亲朋无一字,老病有孤舟。戎马关山北,凭轩涕泗流。
    \end{tabular}
  \end{table}
\end{minipage}
\vspace{1cm}


\noindent\begin{minipage}{\linewidth}
  \ptitle{将进酒}
  \addcontentsline{toc}{section}{\makebox[9cm]{将进酒\dotfill{} 李白}}
  \pauthor{【唐】李白}
  \vskip-3pt\begin{table}[H]
    \centering
    \begin{tabular}{@{}l@{}}
君不见黄河之水天上来,奔流到海不复回。\\
君不见高堂明镜悲白发,朝如青丝暮成雪。\\
   人生得意须尽欢,莫使金樽空对月。\\
   天生我材必有用,千金散尽还复来。\\
   烹羊宰牛且为乐,会须一饮三百杯。\\
   岑夫子,丹丘生,将进酒,杯莫停。\\
     与君歌一曲,请君为我倾耳听。\\
   钟鼓馔玉不足贵,但愿长醉不复醒。\\
   古来圣贤皆寂寞,惟有饮者留其名。\\
   陈王昔时宴平乐,斗酒十千恣欢谑。\\
   主人何为言少钱,径须沽取对君酌。\\
   五花马,千金裘,\\
   呼儿将出换美酒,与尔同销万古愁。
    \end{tabular}
  \end{table}
\end{minipage}
\vspace{1cm}


\noindent\begin{minipage}{\linewidth}
  \ptitle{蜀道难}
  \addcontentsline{toc}{section}{\makebox[9cm]{蜀道难\dotfill{} 李白}}
  \pauthor{【唐】李白}
  \vskip-3pt\begin{table}[H]
    \centering
    \begin{tabular}{@{}l@{}}
噫吁嚱,危乎高哉!蜀道之难,难于上青天!\\
蚕丛及鱼凫,开国何茫然!\\
尔来四万八千岁,不与秦塞通人烟。\\
西当太白有鸟道,可以横绝峨眉巅。\\
地崩山摧壮士死,然后天梯石栈相钩连。\\
上有六龙回日之高标,下有冲波逆折之回川。\\
黄鹤之飞尚不得过,猿猱欲度愁攀援。\\
青泥何盘盘,百步九折萦岩峦。\\
扪参历井仰胁息,以手抚膺坐长叹。\\
问君西游何时还?畏途巉岩不可攀。\\
但见悲鸟号古木,雄飞雌从绕林间。\\
又闻子规啼夜月,愁空山。\\
蜀道之难,难于上青天,使人听此凋朱颜!\\
连峰去天不盈尺,枯松倒挂倚绝壁。\\
飞湍瀑流争喧豗,砯崖转石万壑雷。\\
其险也如此,嗟尔远道之人胡为乎来哉!\\
剑阁峥嵘而崔嵬,一夫当关,万夫莫开。\\
所守或匪亲,化为狼与豺。\\
朝避猛虎,夕避长蛇,\\
磨牙吮血,杀人如麻。\\
锦城虽云乐,不如早还家。\\
蜀道之难,难于上青天,侧身西望长咨嗟!
    \end{tabular}
  \end{table}
\end{minipage}
\vspace{1cm}


\noindent\begin{minipage}{\linewidth}
  \ptitle{梦游天\xpinyin*{\xpinyin{姥}{mǔ}}吟留别}
  \addcontentsline{toc}{section}{\makebox[9cm]{梦游天姥吟留别\dotfill{} 李白}}
  \pauthor{【唐】李白}
  \vskip-3pt\begin{table}[H]
    \centering
    \begin{tabular}{@{}l@{}}
海客谈\xpinyin*{\xpinyin{瀛}{yíng}}洲,烟涛微茫信难求。\\
越人语天姥,云霓明灭或可睹。\\
天姥连天向天横,势拔五岳掩赤城。\\
天台四万八千丈,对此欲倒东南倾。\\
我欲因之梦吴越,一夜飞度镜湖月。\\
湖月照我影,送我至\xpinyin*{\xpinyin{剡}{shàn}}溪。\\
谢公宿处今尚在,\xpinyin*{\xpinyin{渌}{lù}}水荡漾清猿啼。\\
脚著谢公屐,身登青云梯。\\
半壁见海日,空中闻天鸡。\\
千岩万转路不定,迷花倚石忽已暝。\\
熊咆龙吟殷岩泉,栗深林兮惊层巅。\\
云青青兮欲雨,水澹\xpinyin*{\xpinyin{澹}{dàn}}兮生烟。\\
列缺霹雳,丘峦崩摧。洞天石扉,訇然中开。\\
青冥浩荡不见底,日月照耀金银台。\\
霓为衣兮风为马,云之君兮纷纷而来下。\\
虎鼓瑟兮鸾回车,仙之人兮列如麻。\\
忽魂悸以魄动,恍惊起而长嗟。\\
惟觉时之枕席,失向来之烟霞。\\
世间行乐亦如此,古来万事东流水。\\
别君去兮何时还,且放白鹿青崖间,须行即骑访名山。
    \end{tabular}
  \end{table}
\end{minipage}
\vspace{1cm}


\noindent\begin{minipage}{\linewidth}
  \ptitle{琵琶行(并序)}
  \addcontentsline{toc}{section}{\makebox[9cm]{琵琶行(并序)\dotfill{} 白居易}}
  \pauthor{【唐】白居易}
  \vskip-3pt\begin{table}[H]
    \centering
    \begin{tabular}{@{}l@{}}
浔阳江头夜送客,枫叶荻花秋瑟瑟。主人下马客在船,举酒欲饮无管弦。\\
醉不成欢惨将别,别时茫茫江浸月。忽闻水上琵琶声,主人忘归客不发。\\
寻声暗问弹者谁?琵琶声停欲语迟。移船相近邀相见,添酒回灯重开宴。\\
千呼万唤始出来,犹抱琵琶半遮面。转轴拨弦三两声,未成曲调先有情。\\
弦弦掩抑声声思,似诉平生不得志。低眉信手续续弹,说尽心中无限事。\\
轻拢慢捻抹复挑,初为霓裳后六幺。大弦嘈嘈如急雨,小弦切切如私语。\\
嘈嘈切切错杂弹,大珠小珠落玉盘。间关莺语花底滑,幽咽泉流冰下难。\\
冰泉冷涩弦凝绝,凝绝不通声渐歇。别有幽愁暗恨生,此时无声胜有声。\\
银瓶乍破水浆迸,铁骑突出刀枪鸣。曲终收拨当心画,四弦一声如裂帛。\\
东船西舫悄无言,唯见江心秋月白。沉吟放拨插弦中,整顿衣裳起敛容。\\
自言本是京城女,家在虾蟆陵下住。十三学得琵琶成,名属教坊第一部。\\
曲罢曾教善才服,妆成每被秋娘妒。五陵年少争缠头,一曲红绡不知数。\\
钿头银篦击节碎,血色罗裙翻酒污。今年欢笑复明年,秋月春风等闲度。\\
弟走从军阿姨死,暮去朝来颜色故。门前冷落鞍马稀,老大嫁作商人妇。\\
商人重利轻别离,前月浮梁买茶去。去来江口守空船,绕船月明江水寒。\\
夜深忽梦少年事,梦啼妆泪红阑干。我闻琵琶已叹息,又闻此语重唧唧。\\
同是天涯沦落人,相逢何必曾相识!我从去年辞帝京,谪居卧病浔阳城。\\
浔阳地僻无音乐,终岁不闻丝竹声。住近湓江地低湿,黄芦苦竹绕宅生。\\
其间旦暮闻何物?杜鹃啼血猿哀鸣。春江花朝秋月夜,往往取酒还独倾。\\
岂无山歌与村笛?呕哑嘲哳难为听。今夜闻君琵琶语,如听仙乐耳暂明。\\
莫辞更坐弹一曲,为君翻作琵琶行。感我此言良久立,却坐促弦弦转急。\\
凄凄不似向前声,满座重闻皆掩泣。座中泣下谁最多?江州司马青衫湿。
    \end{tabular}
  \end{table}
\end{minipage}
\vspace{1cm}


\noindent\begin{minipage}{\linewidth}
  \ptitle{李凭\xpinyin*{\xpinyin{箜}{kōng}}\xpinyin*{\xpinyin{篌}{hóu}}引}
  \addcontentsline{toc}{section}{\makebox[9cm]{李凭箜篌引\dotfill{} 李贺}}
  \pauthor{【唐】李贺}
  \vskip-3pt\begin{table}[H]
    \centering
    \begin{tabular}{@{}l@{}}
吴丝蜀桐张高秋,空山凝云颓不流。江娥啼竹素女愁,李凭中国弹箜篌。\\
昆山玉碎凤凰叫,芙蓉泣露香兰笑。十二门前融冷光,二十三丝动紫皇。\\
女娲炼石补天处,石破天惊逗秋雨。梦入神山教神妪,老鱼跳波瘦蛟舞。\\
吴质不眠倚桂树,露脚斜飞湿寒兔。
    \end{tabular}
  \end{table}
\end{minipage}
\vspace{1cm}


\noindent\begin{minipage}{\linewidth}
  \ptitle{菩萨蛮}
  \addcontentsline{toc}{section}{\makebox[9cm]{菩萨蛮\dotfill{} 温庭筠}}
  \pauthor{【唐】温庭筠}
  \vskip-3pt\begin{table}[H]
    \centering
    \begin{tabular}{@{}l@{}}
小山重叠金明灭,鬓云欲度香腮雪。懒起画蛾眉,弄妆梳洗迟。\\
\\
照花前后镜,花面交相映。新帖绣罗襦,双双金鹧鸪。
    \end{tabular}
  \end{table}
\end{minipage}
\vspace{1cm}


\noindent\begin{minipage}{\linewidth}
  \ptitle{锦瑟}
  \addcontentsline{toc}{section}{\makebox[9cm]{锦瑟\dotfill{} 李商隐}}
  \pauthor{【唐】李商隐}
  \vskip-3pt\begin{table}[H]
    \centering
    \begin{tabular}{@{}l@{}}
锦瑟无端五十弦,一弦一柱思华年。庄生晓梦迷蝴蝶,望帝春心托杜鹃。\\
沧海月明珠有泪,蓝田日暖玉生烟。此情可待成追忆,只是当时已惘然。
    \end{tabular}
  \end{table}
\end{minipage}
\vspace{1cm}


\noindent\begin{minipage}{\linewidth}
  \ptitle{登快阁}
  \addcontentsline{toc}{section}{\makebox[9cm]{登快阁\dotfill{} 黄庭坚}}
  \pauthor{【宋】黄庭坚}
  \vskip-3pt\begin{table}[H]
    \centering
    \begin{tabular}{@{}l@{}}
痴儿了却公家事,快阁东西倚晚晴。落木千山天远大,澄江一道月分明。\\
朱弦已为佳人绝,青眼聊因美酒横。万里归船弄长笛,此心吾与白鸥盟。
    \end{tabular}
  \end{table}
\end{minipage}
\vspace{1cm}


\noindent\begin{minipage}{\linewidth}
  \ptitle{临安春雨初\xpinyin*{\xpinyin{霁}{jì}}}
  \addcontentsline{toc}{section}{\makebox[9cm]{临安春雨初霁\dotfill{} 陆游}}
  \pauthor{【宋】陆游}
  \vskip-3pt\begin{table}[H]
    \centering
    \begin{tabular}{@{}l@{}}
世味年来薄似纱,谁令骑马客京华。小楼一夜听春雨,深巷明朝卖杏花。\\
矮纸斜行闲作草,晴窗细乳戏分茶。素衣莫起风尘叹,犹及清明可到家。
    \end{tabular}
  \end{table}
\end{minipage}
\vspace{1cm}


\noindent\begin{minipage}{\linewidth}
  \ptitle{书愤}
  \addcontentsline{toc}{section}{\makebox[9cm]{书愤\dotfill{} 陆游}}
  \pauthor{【宋】陆游}
  \vskip-3pt\begin{table}[H]
    \centering
    \begin{tabular}{@{}l@{}}
早岁那知世事艰,中原北望气如山。楼船夜雪瓜洲渡,铁马秋风大散关。\\
塞上长城空自许,镜中衰鬓已先斑。出师一表真名世,千载谁堪伯仲间。
    \end{tabular}
  \end{table}
\end{minipage}
\vspace{1cm}


\noindent\begin{minipage}{\linewidth}
  \ptitle{虞美人}
  \addcontentsline{toc}{section}{\makebox[9cm]{虞美人\dotfill{} 李煜}}
  \pauthor{【南唐】李煜}
  \vskip-3pt\begin{table}[H]
    \centering
    \begin{tabular}{@{}l@{}}
春花秋月何时了?往事知多少。\\
小楼昨夜又东风,故国不堪回首月明中。\\
\\
雕栏玉砌应犹在,只是朱颜改。\\
问君能有几多愁?恰似一江春水向东流。
    \end{tabular}
  \end{table}
\end{minipage}
\vspace{1cm}


\noindent\begin{minipage}{\linewidth}
  \ptitle{望海潮}
  \addcontentsline{toc}{section}{\makebox[9cm]{望海潮\dotfill{} 柳永}}
  \pauthor{【宋】柳永}
  \vskip-3pt\begin{table}[H]
    \centering
    \begin{tabular}{@{}l@{}}
东南形胜,三吴都会,钱塘自古繁华。\\
烟柳画桥,风帘翠幕, 参差十万人家。\\
云树绕堤沙。怒涛卷霜雪,天堑无涯。\\
市列珠玑,户盈罗绮,竞豪奢。\\
\\
重湖叠巘清嘉,有三秋桂子,十里荷花。\\
羌管弄晴,菱歌泛夜,嬉嬉钓叟莲娃。\\
千骑拥高牙。乘醉听箫鼓,吟赏烟霞。\\
异日图将好景,归去凤池夸。
    \end{tabular}
  \end{table}
\end{minipage}
\vspace{1cm}


\noindent\begin{minipage}{\linewidth}
  \ptitle{桂枝香·金陵怀古}
  \addcontentsline{toc}{section}{\makebox[9cm]{桂枝香·金陵怀古\dotfill{} 王安石}}
  \pauthor{【宋】王安石}
  \vskip-3pt\begin{table}[H]
    \centering
    \begin{tabular}{@{}l@{}}
登临送目。正故国晚秋,天气初肃。\\
千里澄江似练,翠峰如簇。\\
归帆去棹残阳里,背西风、酒旗斜矗。\\
彩舟云淡,星河鹭起,画图难足。\\
\\
念往昔、\\
繁华竞逐。叹门外楼头,悲恨相续。\\
千古凭高对此, 谩嗟荣辱。\\
六朝旧事随流水,但寒烟、芳草凝绿。\\
至今商女,时时犹唱,后庭遗曲。
    \end{tabular}
  \end{table}
\end{minipage}
\vspace{1cm}


\noindent\begin{minipage}{\linewidth}
  \ptitle{江城子·乙卯正月二十日夜记梦}
  \addcontentsline{toc}{section}{\makebox[9cm]{江城子·乙卯正月二十日夜记梦\dotfill{} 苏轼}}
  \pauthor{【宋】苏轼}
  \vskip-3pt\begin{table}[H]
    \centering
    \begin{tabular}{@{}l@{}}
十年生死两茫茫。不思量,自难忘。\\
千里孤坟,无处话凄凉。\\
纵使相逢应不识,尘满面,鬓如霜。\\
\\
夜来幽梦忽还乡。小轩窗,正梳妆。\\
相顾无言,惟有泪千行。\\
料得年年肠断处,明月夜,短松冈。
    \end{tabular}
  \end{table}
\end{minipage}
\vspace{1cm}


\noindent\begin{minipage}{\linewidth}
  \ptitle{念奴娇·赤壁怀古}
  \addcontentsline{toc}{section}{\makebox[9cm]{念奴娇·赤壁怀古\dotfill{} 苏轼}}
  \pauthor{【宋】苏轼}
  \vskip-3pt\begin{table}[H]
    \centering
    \begin{tabular}{@{}l@{}}
大江东去,浪淘尽,千古风流人物。\\
故垒西边,人道是,三国周郎赤壁。\\
乱石穿空,惊涛拍岸,卷起千堆雪。\\
江山如画,一时多少豪杰。\\
\\
遥想公瑾当年,小乔初嫁了,雄姿英发。\\
羽扇纶巾,谈笑间,樯橹灰飞烟灭。\\
故国神游,多情应笑我,早生华发。\\
人生如梦,一尊还酹江月。
    \end{tabular}
  \end{table}
\end{minipage}
\vspace{1cm}


\noindent\begin{minipage}{\linewidth}
  \ptitle{鹊桥仙}
  \addcontentsline{toc}{section}{\makebox[9cm]{鹊桥仙\dotfill{} 秦观}}
  \pauthor{【宋】秦观}
  \vskip-3pt\begin{table}[H]
    \centering
    \begin{tabular}{@{}l@{}}
纤云弄巧,飞星传恨,银汉迢迢暗度。金风玉露一相逢,便胜却人间无数。\\
\\
柔情似水,佳期如梦,忍顾鹊桥归路。两情若是久长时,又岂在朝朝暮暮。
    \end{tabular}
  \end{table}
\end{minipage}
\vspace{1cm}


\noindent\begin{minipage}{\linewidth}
  \ptitle{苏幕遮}
  \addcontentsline{toc}{section}{\makebox[9cm]{苏幕遮\dotfill{} 周邦彦}}
  \pauthor{【宋】周邦彦}
  \vskip-3pt\begin{table}[H]
    \centering
    \begin{tabular}{@{}l@{}}
燎沉香,消溽暑。鸟雀呼晴,侵晓窥檐语。\\
叶上初阳干宿雨。水面清圆,一一风荷举。\\
\\
故乡遥,何日去?家住吴门,久作长安旅。\\
五月渔郎相忆否?小楫轻舟,梦入芙蓉浦。
    \end{tabular}
  \end{table}
\end{minipage}
\vspace{1cm}


\noindent\begin{minipage}{\linewidth}
  \ptitle{声声慢}
  \addcontentsline{toc}{section}{\makebox[9cm]{声声慢\dotfill{} 李清照}}
  \pauthor{【宋】李清照}
  \vskip-3pt\begin{table}[H]
    \centering
    \begin{tabular}{@{}l@{}}
寻寻觅觅,冷冷清清,凄凄惨惨戚戚。\\
乍暖还寒时候,最难将息。\\
三杯两盏淡酒,怎敌他、晚来风急!\\
雁过也,正伤心,却是旧时相识。\\
\\
满地黄花堆积,憔悴损,如今有谁堪摘?\\
守着窗儿,独自怎生得黑!\\
梧桐更兼细雨,到黄昏、点点滴滴。\\
这次第,怎一个愁字了得!
    \end{tabular}
  \end{table}
\end{minipage}
\vspace{1cm}


\noindent\begin{minipage}{\linewidth}
  \ptitle{念奴娇·过洞庭}
  \addcontentsline{toc}{section}{\makebox[9cm]{念奴娇·过洞庭\dotfill{} 张孝祥}}
  \pauthor{【宋】张孝祥}
  \vskip-3pt\begin{table}[H]
    \centering
    \begin{tabular}{@{}l@{}}
洞庭青草,近中秋、更无一点风色。\\
玉鉴琼田三万顷,着我扁 舟一叶。\\
素月分辉,明河共影,表里俱澄澈。\\
悠然心会,妙处难与君说。\\
\\
应念岭海经年,孤光自照,肝肺皆冰雪。\\
短发萧骚襟袖冷,稳泛沧浪空阔。\\
尽吸西江,细斟北斗,万象为宾客。\\
扣舷独啸,不知今夕何夕。
    \end{tabular}
  \end{table}
\end{minipage}
\vspace{1cm}


\noindent\begin{minipage}{\linewidth}
  \ptitle{永遇乐·京口北固亭怀古}
  \addcontentsline{toc}{section}{\makebox[9cm]{永遇乐·京口北固亭怀古\dotfill{} 辛弃疾}}
  \pauthor{【宋】辛弃疾}
  \vskip-3pt\begin{table}[H]
    \centering
    \begin{tabular}{@{}l@{}}
千古江山,英雄无觅孙仲谋处。\\
舞榭歌台,风流总被雨打风吹去。\\
斜阳草树,寻常巷陌,人道寄奴曾住。\\
 想当年,金戈铁马,气吞万里如虎。\\
\\
元嘉草草,封狼居胥,赢得仓皇北顾。\\
四十三年,望中犹记,烽火扬州路。\\
可堪回首,佛狸祠下,一片神鸦社鼓。\\
 凭谁问,廉颇老矣,尚能饭否?
    \end{tabular}
  \end{table}
\end{minipage}
\vspace{1cm}


\noindent\begin{minipage}{\linewidth}
  \ptitle{菩萨蛮·书江西造口壁}
  \addcontentsline{toc}{section}{\makebox[9cm]{菩萨蛮·书江西造口壁\dotfill{} 辛弃疾}}
  \pauthor{【宋】辛弃疾}
  \vskip-3pt\begin{table}[H]
    \centering
    \begin{tabular}{@{}l@{}}
郁孤台下清江水,中间多少行人泪。\\
西北望长安,可怜无数山。\\
\\
青山遮不住,毕竟东流去。\\
江晚正愁余,山深闻鹧鸪。
    \end{tabular}
  \end{table}
\end{minipage}
\vspace{1cm}


\noindent\begin{minipage}{\linewidth}
  \ptitle{青玉案·元夕}
  \addcontentsline{toc}{section}{\makebox[9cm]{青玉案·元夕\dotfill{} 辛弃疾}}
  \pauthor{【唐】辛弃疾}
  \vskip-3pt\begin{table}[H]
    \centering
    \begin{tabular}{@{}l@{}}
东风夜放花千树。更吹落、星如雨。\\
宝马雕车香满路。凤箫声动,玉壶光转,一夜鱼龙舞。\\
\\
蛾儿雪柳黄金缕。笑语盈盈暗香去。\\
众里寻他千百度。蓦然回首,那人却在,灯火阑珊处。
    \end{tabular}
  \end{table}
\end{minipage}
\vspace{1cm}


\noindent\begin{minipage}{\linewidth}
  \ptitle{贺新郎}
  \addcontentsline{toc}{section}{\makebox[9cm]{贺新郎\dotfill{} 刘克庄}}
  \pauthor{【宋】刘克庄}
  \vskip-3pt\begin{table}[H]
    \centering
    \begin{tabular}{@{}l@{}}
国脉微如缕。问长缨何时入手,缚将戎主?\\
未必人间无好汉,谁与宽些尺度?\\
试看取当年韩五。岂有谷城公付授,也不干曾遇骊山母。\\
谈笑起,两河路。\\
\\
少时棋柝曾联句。叹而今登楼揽镜,事机频误。\\
闻说北风吹面急,边上冲梯屡舞。\\
君莫道投鞭虚语,自古一贤能制难,有金汤便可无张许?\\
快投笔,莫题柱。
    \end{tabular}
  \end{table}
\end{minipage}
\vspace{1cm}


\noindent\begin{minipage}{\linewidth}
  \ptitle{扬州慢}
  \addcontentsline{toc}{section}{\makebox[9cm]{扬州慢\dotfill{} 姜夔}}
  \pauthor{【宋】姜\xpinyin*{\xpinyin{夔}{kuí}}}
  \vskip-3pt\begin{table}[H]
    \centering
    \begin{tabular}{@{}l@{}}
淮左名都,竹西佳处,解鞍少驻初程。\\
过春风十里,尽荠麦青青。\\
自胡马窥江去后,废池乔木,犹厌言兵。\\
渐黄昏,清角吹寒,都在空城。\\
\\
杜郎俊赏,算而今重到须惊。\\
纵豆蔻词工,青楼梦好,难赋深情。\\
二十四桥仍在,波心荡,冷月无声。\\
念桥边红药,年年知为谁生?
    \end{tabular}
  \end{table}
\end{minipage}
\vspace{1cm}


\noindent\begin{minipage}{\linewidth}
  \ptitle{长亭送别}
  \addcontentsline{toc}{section}{\makebox[9cm]{长亭送别\dotfill{} 王实甫}}
  \pauthor{【元】王实甫}
  \vskip-3pt\begin{table}[H]
    \centering
    \begin{tabular}{@{}l@{}}
碧云天,黄花地,西风紧。北雁南飞。\\
晓来谁染霜林醉?总是离人泪。
    \end{tabular}
  \end{table}
\end{minipage}
\vspace{1cm}


\noindent\begin{minipage}{\linewidth}
  \ptitle{朝天子·咏喇叭}
  \addcontentsline{toc}{section}{\makebox[9cm]{朝天子·咏喇叭\dotfill{} 王磐}}
  \pauthor{【明】王磐}
  \vskip-3pt\begin{table}[H]
    \centering
    \begin{tabular}{@{}l@{}}
喇叭,锁呐,曲儿小腔儿大。\\
官船来往乱如麻,全仗你抬声价。\\
军听了军愁,民听了民怕。\\
那里去辨甚么真共假?\\
眼见的吹翻了这家,吹伤了那家,只吹的水尽鹅飞罢!
    \end{tabular}
  \end{table}
\end{minipage}
\vspace{1cm}


\noindent\begin{minipage}{\linewidth}
  \ptitle{苔}
  \addcontentsline{toc}{section}{\makebox[9cm]{苔\dotfill{} 袁枚}}
  \pauthor{【清】袁枚}
  \vskip-3pt\begin{table}[H]
    \centering
    \begin{tabular}{@{}l@{}}
白日不到处,青春恰自来。\\
苔花如米小,也学牡丹开。
    \end{tabular}
  \end{table}
\end{minipage}
\vspace{1cm}


\noindent\begin{minipage}{\linewidth}
  \ptitle{山中杂诗}
  \addcontentsline{toc}{section}{\makebox[9cm]{山中杂诗\dotfill{} 吴均}}
  \pauthor{【南朝】吴均}
  \vskip-3pt\begin{table}[H]
    \centering
    \begin{tabular}{@{}l@{}}
山际见来烟,竹中窥落日。\\
鸟向檐上飞,云从窗里出。
    \end{tabular}
  \end{table}
\end{minipage}
\vspace{1cm}


\noindent\begin{minipage}{\linewidth}
  \ptitle{送别}
  \addcontentsline{toc}{section}{\makebox[9cm]{送别\dotfill{} 王维}}
  \pauthor{【唐 】王维}
  \vskip-3pt\begin{table}[H]
    \centering
    \begin{tabular}{@{}l@{}}
山中相送罢,日暮掩柴扉。\\
春草年年绿,王孙归不归。
    \end{tabular}
  \end{table}
\end{minipage}
\vspace{1cm}


\noindent\begin{minipage}{\linewidth}
  \ptitle{竹里馆}
  \addcontentsline{toc}{section}{\makebox[9cm]{竹里馆\dotfill{} 王维}}
  \pauthor{【唐】王维}
  \vskip-3pt\begin{table}[H]
    \centering
    \begin{tabular}{@{}l@{}}
独坐幽篁里,弹琴复长啸。\\
深林人不知,明月来相照。
    \end{tabular}
  \end{table}
\end{minipage}
\vspace{1cm}


\noindent\begin{minipage}{\linewidth}
  \ptitle{杂诗(其二)}
  \addcontentsline{toc}{section}{\makebox[9cm]{杂诗(其二)\dotfill{} 王维}}
  \pauthor{【唐】王维}
  \vskip-3pt\begin{table}[H]
    \centering
    \begin{tabular}{@{}l@{}}
君自故乡来,应知故乡事。\\
来日绮窗前,寒梅著花未?
    \end{tabular}
  \end{table}
\end{minipage}
\vspace{1cm}


\noindent\begin{minipage}{\linewidth}
  \ptitle{咏松}
  \addcontentsline{toc}{section}{\makebox[9cm]{咏松\dotfill{} 陈毅}}
  \pauthor{【现代】陈毅}
  \vskip-3pt\begin{table}[H]
    \centering
    \begin{tabular}{@{}l@{}}
大雪压青松,青松挺且直。\\
欲知松高洁,待到雪化时。
    \end{tabular}
  \end{table}
\end{minipage}
\vspace{1cm}


\noindent\begin{minipage}{\linewidth}
  \ptitle{八阵图}
  \addcontentsline{toc}{section}{\makebox[9cm]{八阵图\dotfill{} 杜甫}}
  \pauthor{【唐 】杜甫}
  \vskip-3pt\begin{table}[H]
    \centering
    \begin{tabular}{@{}l@{}}
功盖三分国,名成八阵图。\\
江流石不转,遗恨失吞吴。
    \end{tabular}
  \end{table}
\end{minipage}
\vspace{1cm}


\noindent\begin{minipage}{\linewidth}
  \ptitle{游园不值}
  \addcontentsline{toc}{section}{\makebox[9cm]{游园不值\dotfill{} 叶绍翁}}
  \pauthor{【宋】叶绍翁}
  \vskip-3pt\begin{table}[H]
    \centering
    \begin{tabular}{@{}l@{}}
应怜屐齿印苍苔,小扣柴扉久不开。\\
春色满园关不住,一枝红杏出墙来。
    \end{tabular}
  \end{table}
\end{minipage}
\vspace{1cm}


\noindent\begin{minipage}{\linewidth}
  \ptitle{题诗后}
  \addcontentsline{toc}{section}{\makebox[9cm]{题诗后\dotfill{} 贾岛}}
  \pauthor{【唐 】贾岛}
  \vskip-3pt\begin{table}[H]
    \centering
    \begin{tabular}{@{}l@{}}
两句三年得,一吟双泪流。\\
知音如不赏,归卧故山秋。
    \end{tabular}
  \end{table}
\end{minipage}
\vspace{1cm}


\noindent\begin{minipage}{\linewidth}
  \ptitle{花影}
  \addcontentsline{toc}{section}{\makebox[9cm]{花影\dotfill{} 苏轼}}
  \pauthor{【宋 】苏轼}
  \vskip-3pt\begin{table}[H]
    \centering
    \begin{tabular}{@{}l@{}}
重重叠叠上瑶台,几度呼童扫不开。\\
刚被太阳收拾去,又叫明月送将来。
    \end{tabular}
  \end{table}
\end{minipage}
\vspace{1cm}


\noindent\begin{minipage}{\linewidth}
  \ptitle{相思}
  \addcontentsline{toc}{section}{\makebox[9cm]{相思\dotfill{} 王维}}
  \pauthor{【唐 】王维}
  \vskip-3pt\begin{table}[H]
    \centering
    \begin{tabular}{@{}l@{}}
红豆生南国,春来发几枝。\\
愿君多采撷,此物最相思。
    \end{tabular}
  \end{table}
\end{minipage}
\vspace{1cm}


\noindent\begin{minipage}{\linewidth}
  \ptitle{偶作}
  \addcontentsline{toc}{section}{\makebox[9cm]{偶作\dotfill{} 袁枚}}
  \pauthor{【清】袁枚}
  \vskip-3pt\begin{table}[H]
    \centering
    \begin{tabular}{@{}l@{}}
偶寻半开梅,闲倚一竿竹。\\
儿童不知春,问草何故绿。
    \end{tabular}
  \end{table}
\end{minipage}
\vspace{1cm}


\noindent\begin{minipage}{\linewidth}
  \ptitle{渡汉江}
  \addcontentsline{toc}{section}{\makebox[9cm]{渡汉江\dotfill{} 宋之问}}
  \pauthor{【唐 】宋之问}
  \vskip-3pt\begin{table}[H]
    \centering
    \begin{tabular}{@{}l@{}}
岭外音书断,经冬复历春。\\
近乡情更怯,不敢问来人。
    \end{tabular}
  \end{table}
\end{minipage}
\vspace{1cm}


\noindent\begin{minipage}{\linewidth}
  \ptitle{梅花}
  \addcontentsline{toc}{section}{\makebox[9cm]{梅花\dotfill{} 王冕}}
  \pauthor{【元 】王冕}
  \vskip-3pt\begin{table}[H]
    \centering
    \begin{tabular}{@{}l@{}}
三月东风吹雪消,湖南山色翠如浇。\\
一声羌管无人见,无数梅花落野桥。
    \end{tabular}
  \end{table}
\end{minipage}
\vspace{1cm}


\noindent\begin{minipage}{\linewidth}
  \ptitle{绝句}
  \addcontentsline{toc}{section}{\makebox[9cm]{绝句\dotfill{} 杜甫}}
  \pauthor{【唐 】杜甫}
  \vskip-3pt\begin{table}[H]
    \centering
    \begin{tabular}{@{}l@{}}
江碧鸟逾白,山青花欲燃。\\
今春看又过,何日是归年。
    \end{tabular}
  \end{table}
\end{minipage}
\vspace{1cm}


\noindent\begin{minipage}{\linewidth}
  \ptitle{七步诗}
  \addcontentsline{toc}{section}{\makebox[9cm]{七步诗\dotfill{} 曹植}}
  \pauthor{【三国】曹植}
  \vskip-3pt\begin{table}[H]
    \centering
    \begin{tabular}{@{}l@{}}
煮豆燃豆萁,豆在釜中泣。\\
本自同根生,相煎何太急?
    \end{tabular}
  \end{table}
\end{minipage}
\vspace{1cm}


\noindent\begin{minipage}{\linewidth}
  \ptitle{蝉}
  \addcontentsline{toc}{section}{\makebox[9cm]{蝉\dotfill{} 虞世南}}
  \pauthor{【唐】虞世南}
  \vskip-3pt\begin{table}[H]
    \centering
    \begin{tabular}{@{}l@{}}
垂緌饮清露,流响出疏桐。\\
居高声自远,非是藉秋风。
    \end{tabular}
  \end{table}
\end{minipage}
\vspace{1cm}


\noindent\begin{minipage}{\linewidth}
  \ptitle{二月二日}
  \addcontentsline{toc}{section}{\makebox[9cm]{二月二日\dotfill{} 白居易}}
  \pauthor{【唐】白居易}
  \vskip-3pt\begin{table}[H]
    \centering
    \begin{tabular}{@{}l@{}}
二月二日新雨晴,草芽菜甲一时生。\\
轻衫细马春年少,十字津头一字行。
    \end{tabular}
  \end{table}
\end{minipage}
\vspace{1cm}


\noindent\begin{minipage}{\linewidth}
  \ptitle{赠花卿}
  \addcontentsline{toc}{section}{\makebox[9cm]{赠花卿\dotfill{} 杜甫}}
  \pauthor{【唐】杜甫}
  \vskip-3pt\begin{table}[H]
    \centering
    \begin{tabular}{@{}l@{}}
锦城丝管日纷纷,半入江风半入云。\\
此曲只应天上有,人间能得几回闻。
    \end{tabular}
  \end{table}
\end{minipage}
\vspace{1cm}


\noindent\begin{minipage}{\linewidth}
  \ptitle{墨梅}
  \addcontentsline{toc}{section}{\makebox[9cm]{墨梅\dotfill{} 王冕}}
  \pauthor{【元  】王冕}
  \vskip-3pt\begin{table}[H]
    \centering
    \begin{tabular}{@{}l@{}}
吾家洗砚池头树, 个个花开淡墨痕。\\
不要人夸好颜色, 只留清气满乾坤。
    \end{tabular}
  \end{table}
\end{minipage}
\vspace{1cm}


\noindent\begin{minipage}{\linewidth}
  \ptitle{题秋江独钓图}
  \addcontentsline{toc}{section}{\makebox[9cm]{题秋江独钓图\dotfill{} 王士祯}}
  \pauthor{【清 】王士祯}
  \vskip-3pt\begin{table}[H]
    \centering
    \begin{tabular}{@{}l@{}}
一蓑一笠一扁舟,一丈丝纶一寸钩。\\
一曲高歌一樽酒,一人独钓一江秋。
    \end{tabular}
  \end{table}
\end{minipage}
\vspace{1cm}


\noindent\begin{minipage}{\linewidth}
  \ptitle{乐游原}
  \addcontentsline{toc}{section}{\makebox[9cm]{乐游原\dotfill{} 李商隐}}
  \pauthor{【唐 】李商隐}
  \vskip-3pt\begin{table}[H]
    \centering
    \begin{tabular}{@{}l@{}}
向晚意不适, 驱车登古原。\\
夕阳无限好, 只是近黄昏。
    \end{tabular}
  \end{table}
\end{minipage}
\vspace{1cm}


\noindent\begin{minipage}{\linewidth}
  \ptitle{田园乐(其六)}
  \addcontentsline{toc}{section}{\makebox[9cm]{田园乐(其六)\dotfill{} 王维}}
  \pauthor{【唐 】王维}
  \vskip-3pt\begin{table}[H]
    \centering
    \begin{tabular}{@{}l@{}}
桃红复含宿雨,柳绿更带朝烟。\\
花落家童未扫,莺啼山客犹眠。
    \end{tabular}
  \end{table}
\end{minipage}
\vspace{1cm}


\noindent\begin{minipage}{\linewidth}
  \ptitle{弹琴}
  \addcontentsline{toc}{section}{\makebox[9cm]{弹琴\dotfill{} 刘长卿}}
  \pauthor{【唐】刘长卿}
  \vskip-3pt\begin{table}[H]
    \centering
    \begin{tabular}{@{}l@{}}
泠泠七弦上,静听松风寒。\\
古调虽自爱,今人多不弹。
    \end{tabular}
  \end{table}
\end{minipage}
\vspace{1cm}


\noindent\begin{minipage}{\linewidth}
  \ptitle{秋风引}
  \addcontentsline{toc}{section}{\makebox[9cm]{秋风引\dotfill{} 刘禹锡}}
  \pauthor{【唐 】刘禹锡}
  \vskip-3pt\begin{table}[H]
    \centering
    \begin{tabular}{@{}l@{}}
何处秋风至?萧萧送雁群。\\
朝来入庭树,孤客最先闻。
    \end{tabular}
  \end{table}
\end{minipage}
\vspace{1cm}


\noindent\begin{minipage}{\linewidth}
  \ptitle{秋浦歌}
  \addcontentsline{toc}{section}{\makebox[9cm]{秋浦歌\dotfill{} 李白}}
  \pauthor{【唐】李白}
  \vskip-3pt\begin{table}[H]
    \centering
    \begin{tabular}{@{}l@{}}
白发三千丈,缘愁似个长?\\
不知明镜里,何处得秋霜!\\
\\
雨过山村\\
唐  王建\\
雨里鸡鸣一两家,竹溪村路板桥斜。\\
妇姑相唤浴蚕去,闲看中庭栀子花。
    \end{tabular}
  \end{table}
\end{minipage}
\vspace{1cm}


\noindent\begin{minipage}{\linewidth}
  \ptitle{劝学}
  \addcontentsline{toc}{section}{\makebox[9cm]{劝学\dotfill{} 颜真卿}}
  \pauthor{【唐】颜真卿}
  \vskip-3pt\begin{table}[H]
    \centering
    \begin{tabular}{@{}l@{}}
三更灯火五更鸡,正是男儿读书时。\\
黑发不知勤学早,白首方悔读书迟。
    \end{tabular}
  \end{table}
\end{minipage}
\vspace{1cm}


\noindent\begin{minipage}{\linewidth}
  \ptitle{菊花}
  \addcontentsline{toc}{section}{\makebox[9cm]{菊花\dotfill{} 黄巢}}
  \pauthor{【唐】黄巢}
  \vskip-3pt\begin{table}[H]
    \centering
    \begin{tabular}{@{}l@{}}
待到秋来九月八, 我花开后百花杀。\\
冲天香阵透长安, 满城尽带黄金甲。
    \end{tabular}
  \end{table}
\end{minipage}
\vspace{1cm}


\noindent\begin{minipage}{\linewidth}
  \ptitle{水亭二首(其一)}
  \addcontentsline{toc}{section}{\makebox[9cm]{水亭二首(其一)\dotfill{} 游}}
  \pauthor{【宋  陆】游}
  \vskip-3pt\begin{table}[H]
    \centering
    \begin{tabular}{@{}l@{}}
水亭不受俗尘侵,葛帐筠床弄素琴。\\
一片风光谁画得:红蜻蜓点绿荷心。
    \end{tabular}
  \end{table}
\end{minipage}
\vspace{1cm}


\noindent\begin{minipage}{\linewidth}
  \ptitle{牧童}
  \addcontentsline{toc}{section}{\makebox[9cm]{牧童\dotfill{} 吕岩}}
  \pauthor{【唐】吕岩}
  \vskip-3pt\begin{table}[H]
    \centering
    \begin{tabular}{@{}l@{}}
草铺横野六七里,笛弄晚风三四声。\\
归来饱饭黄昏后,不脱蓑衣卧月明。
    \end{tabular}
  \end{table}
\end{minipage}
\vspace{1cm}


\noindent\begin{minipage}{\linewidth}
  \ptitle{舟过安仁}
  \addcontentsline{toc}{section}{\makebox[9cm]{舟过安仁\dotfill{} 杨万里}}
  \pauthor{【宋】杨万里}
  \vskip-3pt\begin{table}[H]
    \centering
    \begin{tabular}{@{}l@{}}
一叶渔船两小童,收篙停棹坐船中。\\
怪生无雨都张伞,不是遮头是使风。
    \end{tabular}
  \end{table}
\end{minipage}
\vspace{1cm}


\noindent\begin{minipage}{\linewidth}
  \ptitle{晚春}
  \addcontentsline{toc}{section}{\makebox[9cm]{晚春\dotfill{} 韩愈}}
  \pauthor{【唐】韩愈}
  \vskip-3pt\begin{table}[H]
    \centering
    \begin{tabular}{@{}l@{}}
草木知春不久归,百般红紫斗芳菲。\\
杨花榆荚无才思,惟解漫天作雪飞。
    \end{tabular}
  \end{table}
\end{minipage}
\vspace{1cm}


\noindent\begin{minipage}{\linewidth}
  \ptitle{游山西村}
  \addcontentsline{toc}{section}{\makebox[9cm]{游山西村\dotfill{} 陆游}}
  \pauthor{【宋】陆游}
  \vskip-3pt\begin{table}[H]
    \centering
    \begin{tabular}{@{}l@{}}
莫笑农家腊酒浑,丰年留客足鸡豚。\\
山重水复疑无路,柳暗花明又一村。\\
箫鼓追随春社近,衣冠简朴古风存。\\
从今若许闲乘月,拄杖无时夜叩门。
    \end{tabular}
  \end{table}
\end{minipage}
\vspace{1cm}


\noindent\begin{minipage}{\linewidth}
  \ptitle{江畔独步寻花}
  \addcontentsline{toc}{section}{\makebox[9cm]{江畔独步寻花\dotfill{} 杜甫}}
  \pauthor{【唐】杜甫}
  \vskip-3pt\begin{table}[H]
    \centering
    \begin{tabular}{@{}l@{}}
黄师塔前江水东,春光懒困倚微风。\\
桃花一簇开无主,可爱深红爱浅红?
    \end{tabular}
  \end{table}
\end{minipage}
\vspace{1cm}


\noindent\begin{minipage}{\linewidth}
  \ptitle{逢雪宿芙蓉山主人}
  \addcontentsline{toc}{section}{\makebox[9cm]{逢雪宿芙蓉山主人\dotfill{} 刘长卿}}
  \pauthor{【唐】刘长卿}
  \vskip-3pt\begin{table}[H]
    \centering
    \begin{tabular}{@{}l@{}}
日暮苍山远,天寒白屋贫。\\
柴门闻犬吠,风雪夜归人。
    \end{tabular}
  \end{table}
\end{minipage}
\vspace{1cm}


\noindent\begin{minipage}{\linewidth}
  \ptitle{江上}
  \addcontentsline{toc}{section}{\makebox[9cm]{江上\dotfill{} 王安石}}
  \pauthor{【宋】王安石}
  \vskip-3pt\begin{table}[H]
    \centering
    \begin{tabular}{@{}l@{}}
江北秋阴一半开,晚云含雨却低回。\\
青山缭绕疑无路,忽见千帆隐映来。
    \end{tabular}
  \end{table}
\end{minipage}
\vspace{1cm}


\noindent\begin{minipage}{\linewidth}
  \ptitle{秋词}
  \addcontentsline{toc}{section}{\makebox[9cm]{秋词\dotfill{} 刘禹锡}}
  \pauthor{【唐】刘禹锡}
  \vskip-3pt\begin{table}[H]
    \centering
    \begin{tabular}{@{}l@{}}
自古逢秋悲寂寥,我言秋日胜春朝。\\
晴空一鹤排云上,便引诗情到碧宵。
    \end{tabular}
  \end{table}
\end{minipage}
\vspace{1cm}


\noindent\begin{minipage}{\linewidth}
  \ptitle{城东早春}
  \addcontentsline{toc}{section}{\makebox[9cm]{城东早春\dotfill{} 杨巨源}}
  \pauthor{【唐】杨巨源}
  \vskip-3pt\begin{table}[H]
    \centering
    \begin{tabular}{@{}l@{}}
诗家清景在新春,绿柳才黄半未匀。\\
若待上林花似锦,出门俱是看花人。
    \end{tabular}
  \end{table}
\end{minipage}
\vspace{1cm}


\noindent\begin{minipage}{\linewidth}
  \ptitle{送友人}
  \addcontentsline{toc}{section}{\makebox[9cm]{送友人\dotfill{} 李白}}
  \pauthor{【唐】李白}
  \vskip-3pt\begin{table}[H]
    \centering
    \begin{tabular}{@{}l@{}}
青山横北郭,白水绕东城。\\
此地一为别,孤蓬万里征。\\
浮云游子意,落日故人情。\\
挥手自兹去,萧萧班马鸣。
    \end{tabular}
  \end{table}
\end{minipage}
\vspace{1cm}


\noindent\begin{minipage}{\linewidth}
  \ptitle{有约}
  \addcontentsline{toc}{section}{\makebox[9cm]{有约\dotfill{} 赵师秀}}
  \pauthor{【宋】赵师秀}
  \vskip-3pt\begin{table}[H]
    \centering
    \begin{tabular}{@{}l@{}}
黄梅时节家家雨,青草池塘处处蛙。\\
有约不来过夜半,闲敲棋子落灯花。
    \end{tabular}
  \end{table}
\end{minipage}
\vspace{1cm}


\noindent\begin{minipage}{\linewidth}
  \ptitle{秋思}
  \addcontentsline{toc}{section}{\makebox[9cm]{秋思\dotfill{} 张籍}}
  \pauthor{【唐】张籍}
  \vskip-3pt\begin{table}[H]
    \centering
    \begin{tabular}{@{}l@{}}
洛阳城里见秋风,欲作家书意万重。\\
复恐匆匆说不尽,行人临发又开封。
    \end{tabular}
  \end{table}
\end{minipage}
\vspace{1cm}


\noindent\begin{minipage}{\linewidth}
  \ptitle{夜雨寄北}
  \addcontentsline{toc}{section}{\makebox[9cm]{夜雨寄北\dotfill{} 李商隐}}
  \pauthor{【唐】李商隐}
  \vskip-3pt\begin{table}[H]
    \centering
    \begin{tabular}{@{}l@{}}
君问归期未有期,巴山夜雨涨秋池。\\
何当共剪西窗烛,却话巴山夜雨时。
    \end{tabular}
  \end{table}
\end{minipage}
\vspace{1cm}


\noindent\begin{minipage}{\linewidth}
  \ptitle{问刘十九}
  \addcontentsline{toc}{section}{\makebox[9cm]{问刘十九\dotfill{} 白居易}}
  \pauthor{【唐】白居易}
  \vskip-3pt\begin{table}[H]
    \centering
    \begin{tabular}{@{}l@{}}
绿蚁新醅酒,红泥小火炉。\\
晚来天欲雪,能饮一杯无。
    \end{tabular}
  \end{table}
\end{minipage}
\vspace{1cm}


\noindent\begin{minipage}{\linewidth}
  \ptitle{乌衣巷}
  \addcontentsline{toc}{section}{\makebox[9cm]{乌衣巷\dotfill{} 刘禹锡}}
  \pauthor{【唐】刘禹锡}
  \vskip-3pt\begin{table}[H]
    \centering
    \begin{tabular}{@{}l@{}}
朱雀桥边野草花,乌衣巷口夕阳斜。\\
旧时王谢堂前燕,飞入寻常百姓家。
    \end{tabular}
  \end{table}
\end{minipage}
\vspace{1cm}


\noindent\begin{minipage}{\linewidth}
  \ptitle{早梅}
  \addcontentsline{toc}{section}{\makebox[9cm]{早梅\dotfill{} 张谓}}
  \pauthor{【唐】张谓}
  \vskip-3pt\begin{table}[H]
    \centering
    \begin{tabular}{@{}l@{}}
一树寒梅白玉条,迥临林村傍溪桥。\\
不知近水花先发,疑是经冬雪未销。
    \end{tabular}
  \end{table}
\end{minipage}
\vspace{1cm}


\noindent\begin{minipage}{\linewidth}
  \ptitle{芙蓉楼送辛渐}
  \addcontentsline{toc}{section}{\makebox[9cm]{芙蓉楼送辛渐\dotfill{} 王昌龄}}
  \pauthor{【唐】王昌龄}
  \vskip-3pt\begin{table}[H]
    \centering
    \begin{tabular}{@{}l@{}}
寒雨连江夜入吴, 平明送客楚山孤。\\
洛阳亲友如相问, 一片冰心在玉壶。
    \end{tabular}
  \end{table}
\end{minipage}
\vspace{1cm}


\noindent\begin{minipage}{\linewidth}
  \ptitle{终南望余雪}
  \addcontentsline{toc}{section}{\makebox[9cm]{终南望余雪\dotfill{} 祖咏}}
  \pauthor{【唐】祖咏}
  \vskip-3pt\begin{table}[H]
    \centering
    \begin{tabular}{@{}l@{}}
终南阴岭秀,积雪浮云端。\\
林表明霁色,城中增暮寒。
    \end{tabular}
  \end{table}
\end{minipage}
\vspace{1cm}


\noindent\begin{minipage}{\linewidth}
  \ptitle{霜月}
  \addcontentsline{toc}{section}{\makebox[9cm]{霜月\dotfill{} 李商隐}}
  \pauthor{【唐】李商隐}
  \vskip-3pt\begin{table}[H]
    \centering
    \begin{tabular}{@{}l@{}}
初闻征雁已无蝉,百尺楼高水接天。\\
青女素娥俱耐冷,月中霜里斗婵娟。
    \end{tabular}
  \end{table}
\end{minipage}
\vspace{1cm}


\noindent\begin{minipage}{\linewidth}
  \ptitle{竹枝词 其一}
  \addcontentsline{toc}{section}{\makebox[9cm]{竹枝词 其一\dotfill{} 刘禹锡}}
  \pauthor{【唐】刘禹锡}
  \vskip-3pt\begin{table}[H]
    \centering
    \begin{tabular}{@{}l@{}}
杨柳青青江水平,闻郎江上踏歌声 。\\
东边日出西边雨,道是无晴却有晴。
    \end{tabular}
  \end{table}
\end{minipage}
\vspace{1cm}


\noindent\begin{minipage}{\linewidth}
  \ptitle{竹枝词 其二}
  \addcontentsline{toc}{section}{\makebox[9cm]{竹枝词 其二\dotfill{} 刘禹锡}}
  \pauthor{【唐】刘禹锡}
  \vskip-3pt\begin{table}[H]
    \centering
    \begin{tabular}{@{}l@{}}
楚水巴山江雨多,巴人能唱本乡歌。\\
今朝北客思归去,回入纥那披绿罗。
    \end{tabular}
  \end{table}
\end{minipage}
\vspace{1cm}


\noindent\begin{minipage}{\linewidth}
  \ptitle{泊秦淮}
  \addcontentsline{toc}{section}{\makebox[9cm]{泊秦淮\dotfill{} 杜牧}}
  \pauthor{【唐  】杜牧}
  \vskip-3pt\begin{table}[H]
    \centering
    \begin{tabular}{@{}l@{}}
烟笼寒水月笼沙,夜泊秦淮近酒家。\\
商女不知亡国恨,隔江犹唱后庭花。
    \end{tabular}
  \end{table}
\end{minipage}
\vspace{1cm}


\noindent\begin{minipage}{\linewidth}
  \ptitle{赤壁}
  \addcontentsline{toc}{section}{\makebox[9cm]{赤壁\dotfill{} 牧}}
  \pauthor{【唐   杜】牧}
  \vskip-3pt\begin{table}[H]
    \centering
    \begin{tabular}{@{}l@{}}
折戟沉沙铁未销,自将磨洗认前朝。\\
东风不与周郎便,铜雀春深锁二乔。
    \end{tabular}
  \end{table}
\end{minipage}
\vspace{1cm}


\noindent\begin{minipage}{\linewidth}
  \ptitle{己亥杂诗 其五}
  \addcontentsline{toc}{section}{\makebox[9cm]{己亥杂诗 其五\dotfill{} 龚自珍}}
  \pauthor{【清】龚自珍}
  \vskip-3pt\begin{table}[H]
    \centering
    \begin{tabular}{@{}l@{}}
浩荡离愁白日斜,吟鞭东指即天涯。\\
落红不是无情物,化作春泥更护花。
    \end{tabular}
  \end{table}
\end{minipage}
\vspace{1cm}


\noindent\begin{minipage}{\linewidth}
  \ptitle{十一月四日风雨大作}
  \addcontentsline{toc}{section}{\makebox[9cm]{十一月四日风雨大作\dotfill{} 陆游}}
  \pauthor{【南宋】陆游}
  \vskip-3pt\begin{table}[H]
    \centering
    \begin{tabular}{@{}l@{}}
僵卧孤村不自哀,尚思为国戍轮台。\\
夜阑卧听风吹雨,铁马冰河入梦来。
    \end{tabular}
  \end{table}
\end{minipage}
\vspace{1cm}


\noindent\begin{minipage}{\linewidth}
  \ptitle{冬夜读书示子聿}
  \addcontentsline{toc}{section}{\makebox[9cm]{冬夜读书示子聿\dotfill{} 陆游}}
  \pauthor{【南宋 】陆游}
  \vskip-3pt\begin{table}[H]
    \centering
    \begin{tabular}{@{}l@{}}
古人学问无遗力,少壮工夫老始成。\\
纸上得来终觉浅,绝知此事要躬行。
    \end{tabular}
  \end{table}
\end{minipage}
\vspace{1cm}


\noindent\begin{minipage}{\linewidth}
  \ptitle{江南逢李龟年}
  \addcontentsline{toc}{section}{\makebox[9cm]{江南逢李龟年\dotfill{} 杜甫}}
  \pauthor{【唐】杜甫}
  \vskip-3pt\begin{table}[H]
    \centering
    \begin{tabular}{@{}l@{}}
岐王宅里寻常见,崔九堂前几度闻。\\
正是江南好风景,落花时节又逢君。
    \end{tabular}
  \end{table}
\end{minipage}
\vspace{1cm}


\noindent\begin{minipage}{\linewidth}
  \ptitle{临洞庭湖赠张丞相}
  \addcontentsline{toc}{section}{\makebox[9cm]{临洞庭湖赠张丞相\dotfill{} 孟浩然}}
  \pauthor{【唐】孟浩然}
  \vskip-3pt\begin{table}[H]
    \centering
    \begin{tabular}{@{}l@{}}
八月湖水平,涵虚混太清。\\
气蒸云梦泽,波撼岳阳城。\\
欲济无舟楫,端居耻圣明。\\
坐观垂钓者,徒有羡鱼情。\\
\\
水调歌头\\
宋 苏轼\\
明月几时有,把酒问青天。\\
不知天上宫阙,今夕是何年。\\
我欲乘风归去,又恐琼楼玉宇,高处不胜寒。\\
起舞弄清影,何似在人间。\\
\\
转朱阁,低绮户,照无眠。\\
不应有恨,何事长向别时圆?\\
人有悲欢离合,月有阴晴圆缺,此事古难全。\\
但愿人长久,千里共婵娟。
    \end{tabular}
  \end{table}
\end{minipage}
\vspace{1cm}


\noindent\begin{minipage}{\linewidth}
  \ptitle{宋 辛弃疾}
  \addcontentsline{toc}{section}{\makebox[9cm]{宋 辛弃疾\dotfill{} 少年不识愁滋味,爱上层楼。}}
  \pauthor{少年不识愁滋味,爱上层楼。}
  \vskip-3pt\begin{table}[H]
    \centering
    \begin{tabular}{@{}l@{}}
爱上层楼。为赋新词强说愁。\\
而今识尽愁滋味,欲说还休。\\
欲说还休。却道天凉好个秋。
    \end{tabular}
  \end{table}
\end{minipage}
\vspace{1cm}


\noindent\begin{minipage}{\linewidth}
  \ptitle{送杜少府之任蜀州}
  \addcontentsline{toc}{section}{\makebox[9cm]{送杜少府之任蜀州\dotfill{} 王勃}}
  \pauthor{【唐】王勃}
  \vskip-3pt\begin{table}[H]
    \centering
    \begin{tabular}{@{}l@{}}
城阙辅三秦,风烟望五津。\\
与君离别意,同是宦游人。\\
海内存知己,天涯若比邻。\\
无为在歧路,儿女共沾巾。
    \end{tabular}
  \end{table}
\end{minipage}
\vspace{1cm}


\noindent\begin{minipage}{\linewidth}
  \ptitle{闻官军收河南河北}
  \addcontentsline{toc}{section}{\makebox[9cm]{闻官军收河南河北\dotfill{} 杜甫}}
  \pauthor{【唐】杜甫}
  \vskip-3pt\begin{table}[H]
    \centering
    \begin{tabular}{@{}l@{}}
剑外忽传收蓟北,初闻涕泪满衣裳。\\
却看妻子愁何在,漫卷诗书喜欲狂。\\
白日放歌须纵酒,青春作伴好还乡。\\
即从巴峡穿巫峡,便下襄阳向洛阳。
    \end{tabular}
  \end{table}
\end{minipage}
\vspace{1cm}


\noindent\begin{minipage}{\linewidth}
  \ptitle{山亭夏日}
  \addcontentsline{toc}{section}{\makebox[9cm]{山亭夏日\dotfill{} 高骈}}
  \pauthor{【唐 】高骈}
  \vskip-3pt\begin{table}[H]
    \centering
    \begin{tabular}{@{}l@{}}
绿树阴浓夏日长,楼台倒影入池塘。\\
水晶帘动微风起,满架蔷薇一院香。
    \end{tabular}
  \end{table}
\end{minipage}
\vspace{1cm}


\noindent\begin{minipage}{\linewidth}
  \ptitle{菊花}
  \addcontentsline{toc}{section}{\makebox[9cm]{菊花\dotfill{} 元稹}}
  \pauthor{【唐 】元\xpinyin*{\xpinyin{稹}{zhěn}}}
  \vskip-3pt\begin{table}[H]
    \centering
    \begin{tabular}{@{}l@{}}
秋丛绕舍似陶家,遍绕篱边日渐斜。\\
不是花中偏爱菊,此花开尽更无花。
    \end{tabular}
  \end{table}
\end{minipage}
\vspace{1cm}


\noindent\begin{minipage}{\linewidth}
  \ptitle{如梦令}
  \addcontentsline{toc}{section}{\makebox[9cm]{如梦令\dotfill{} 李清照}}
  \pauthor{【宋】李清照}
  \vskip-3pt\begin{table}[H]
    \centering
    \begin{tabular}{@{}l@{}}
常记溪亭日暮,沉醉不知归路。\\
兴尽晚回舟,误入藕花深处。\\
争渡,争渡,惊起一滩鸥鹭。
    \end{tabular}
  \end{table}
\end{minipage}
\vspace{1cm}


\noindent\begin{minipage}{\linewidth}
  \ptitle{朝天子·咏喇叭}
  \addcontentsline{toc}{section}{\makebox[9cm]{朝天子·咏喇叭\dotfill{} 王磐}}
  \pauthor{【明】王磐}
  \vskip-3pt\begin{table}[H]
    \centering
    \begin{tabular}{@{}l@{}}
喇叭,唢呐,曲儿小腔儿大。\\
官船来往乱如麻,全仗你抬声价。\\
军听了军愁,民听了民怕。\\
哪里去辨甚么真共假?\\
眼见的吹翻了这家,吹伤了那家,\\
只吹的水尽鹅飞罢!
    \end{tabular}
  \end{table}
\end{minipage}
\vspace{1cm}


\noindent\begin{minipage}{\linewidth}
  \ptitle{菩萨蛮书江西造口壁}
  \addcontentsline{toc}{section}{\makebox[9cm]{菩萨蛮书江西造口壁\dotfill{} 辛弃疾}}
  \pauthor{【宋】辛弃疾}
  \vskip-3pt\begin{table}[H]
    \centering
    \begin{tabular}{@{}l@{}}
郁孤台下清江水,中间多少行人泪?西北望长安,可怜无数山。\\
\\
青山遮不住,毕竟东流去。江晚正愁余,山深闻鹧鸪。
    \end{tabular}
  \end{table}
\end{minipage}
\vspace{1cm}


\noindent\begin{minipage}{\linewidth}
  \ptitle{秋月}
  \addcontentsline{toc}{section}{\makebox[9cm]{秋月\dotfill{} 程颢}}
  \pauthor{【宋 】程颢}
  \vskip-3pt\begin{table}[H]
    \centering
    \begin{tabular}{@{}l@{}}
清溪流过碧山头,空水澄鲜一色秋。\\
隔断红尘三十里,白云红叶两悠悠。
    \end{tabular}
  \end{table}
\end{minipage}
\vspace{1cm}


\noindent\begin{minipage}{\linewidth}
  \ptitle{六月二十七日望湖楼醉书(二)}
  \addcontentsline{toc}{section}{\makebox[9cm]{六月二十七日望湖楼醉书(二)\dotfill{} 苏轼}}
  \pauthor{【宋】苏轼}
  \vskip-3pt\begin{table}[H]
    \centering
    \begin{tabular}{@{}l@{}}
放生鱼鳖逐人来,无主荷花到处开。\\
水枕能令山俯仰,风船解与月徘徊。
    \end{tabular}
  \end{table}
\end{minipage}
\vspace{1cm}


\noindent\begin{minipage}{\linewidth}
  \ptitle{题榴花}
  \addcontentsline{toc}{section}{\makebox[9cm]{题榴花\dotfill{} 朱熹}}
  \pauthor{【宋 】朱熹}
  \vskip-3pt\begin{table}[H]
    \centering
    \begin{tabular}{@{}l@{}}
五月榴花照眼明,枝间时见子初成。\\
可怜此地无车马,颠倒苍苔落绛英。
    \end{tabular}
  \end{table}
\end{minipage}
\vspace{1cm}


\noindent\begin{minipage}{\linewidth}
  \ptitle{绝句·古木阴中系短篷}
  \addcontentsline{toc}{section}{\makebox[9cm]{绝句·古木阴中系短篷\dotfill{} 志南}}
  \pauthor{【宋】志南}
  \vskip-3pt\begin{table}[H]
    \centering
    \begin{tabular}{@{}l@{}}
古木阴中系短篷,杖藜扶我过桥东。\\
沾衣欲湿杏花雨,吹面不寒杨柳风。
    \end{tabular}
  \end{table}
\end{minipage}
\vspace{1cm}


\noindent\begin{minipage}{\linewidth}
  \ptitle{登飞来峰}
  \addcontentsline{toc}{section}{\makebox[9cm]{登飞来峰\dotfill{} 王安石}}
  \pauthor{【宋 】王安石}
  \vskip-3pt\begin{table}[H]
    \centering
    \begin{tabular}{@{}l@{}}
飞来山上千寻塔,闻说鸡鸣见日升。\\
不畏浮云遮望眼,自缘身在最高层。
    \end{tabular}
  \end{table}
\end{minipage}
\vspace{1cm}


\noindent\begin{minipage}{\linewidth}
  \ptitle{月下独酌}
  \addcontentsline{toc}{section}{\makebox[9cm]{月下独酌\dotfill{} 李白}}
  \pauthor{【唐 】李白}
  \vskip-3pt\begin{table}[H]
    \centering
    \begin{tabular}{@{}l@{}}
花间一壶酒,独酌无相亲。举杯邀明月,对影成三人。\\
月既不解饮,影徒随我身。暂伴月将影,行乐须及春。\\
我歌月徘徊,我舞影零乱。醒时相交欢,醉后各分散。\\
永结无情游,相期邈云汉。
    \end{tabular}
  \end{table}
\end{minipage}
\vspace{1cm}


\noindent\begin{minipage}{\linewidth}
  \ptitle{次北固山下}
  \addcontentsline{toc}{section}{\makebox[9cm]{次北固山下\dotfill{} 王湾}}
  \pauthor{【唐】王湾}
  \vskip-3pt\begin{table}[H]
    \centering
    \begin{tabular}{@{}l@{}}
客路青山外,行舟绿水前。\\
潮平两岸阔,风正一帆悬。\\
海日生残夜,江春入旧年。\\
乡书何处达?归雁洛阳边。
    \end{tabular}
  \end{table}
\end{minipage}
\vspace{1cm}


\noindent\begin{minipage}{\linewidth}
  \ptitle{沁园春·雪}
  \addcontentsline{toc}{section}{\makebox[9cm]{沁园春·雪\dotfill{} 毛泽东}}
  \pauthor{【现代】毛泽东}
  \vskip-3pt\begin{table}[H]
    \centering
    \begin{tabular}{@{}l@{}}
北国风光,千里冰封,万里雪飘。\\
望长城内外,惟余莽莽;大河上下,顿失滔滔。\\
山舞银蛇,原驰蜡象,欲与天公试比高。\\
\\
须晴日,看红装素裹,分外妖娆。\\
江山如此多娇,引无数英雄竞折腰。\\
惜秦皇汉武,略输文采;唐宗宋祖,稍逊风骚。\\
一代天骄,成吉思汗,只识弯弓射大雕。\\
俱往矣,数风流人物,还看今朝。
    \end{tabular}
  \end{table}
\end{minipage}
\vspace{1cm}


\noindent\begin{minipage}{\linewidth}
  \ptitle{峨眉山月歌}
  \addcontentsline{toc}{section}{\makebox[9cm]{峨眉山月歌\dotfill{} 李白}}
  \pauthor{【唐 】李白}
  \vskip-3pt\begin{table}[H]
    \centering
    \begin{tabular}{@{}l@{}}
峨眉山月半轮秋,影入平羌江水流。\\
夜发清溪向三峡,思君不见下渝州。
    \end{tabular}
  \end{table}
\end{minipage}
\vspace{1cm}


\noindent\begin{minipage}{\linewidth}
  \ptitle{七绝·赠父诗}
  \addcontentsline{toc}{section}{\makebox[9cm]{七绝·赠父诗\dotfill{} 毛泽东}}
  \pauthor{【现代】毛泽东}
  \vskip-3pt\begin{table}[H]
    \centering
    \begin{tabular}{@{}l@{}}
孩儿立志出乡关,学不成名誓不还。\\
埋骨何须桑梓地,人生无处不青山。
    \end{tabular}
  \end{table}
\end{minipage}
\vspace{1cm}


\noindent\begin{minipage}{\linewidth}
  \ptitle{忆秦娥·娄山关}
  \addcontentsline{toc}{section}{\makebox[9cm]{忆秦娥·娄山关\dotfill{} 毛泽东}}
  \pauthor{【现代】毛泽东}
  \vskip-3pt\begin{table}[H]
    \centering
    \begin{tabular}{@{}l@{}}
西风烈,长空雁叫霜晨月。\\
霜晨月,马蹄声碎,喇叭声咽。\\
\\
雄关漫道真如铁,而今迈步从头越。\\
从头越,苍山如海,残阳如血。
    \end{tabular}
  \end{table}
\end{minipage}
\vspace{1cm}


\noindent\begin{minipage}{\linewidth}
  \ptitle{钱塘湖春行}
  \addcontentsline{toc}{section}{\makebox[9cm]{钱塘湖春行\dotfill{} 白居易}}
  \pauthor{【唐】白居易}
  \vskip-3pt\begin{table}[H]
    \centering
    \begin{tabular}{@{}l@{}}
孤山寺北贾亭西,水面初平云脚低。\\
几处早莺争暖树,谁家新燕啄春泥。\\
乱花渐欲迷人眼,浅草才能没马蹄。\\
最爱湖东行不足,绿杨阴里白沙堤。
    \end{tabular}
  \end{table}
\end{minipage}
\vspace{1cm}


\noindent\begin{minipage}{\linewidth}
  \ptitle{望岳}
  \addcontentsline{toc}{section}{\makebox[9cm]{望岳\dotfill{} 杜甫}}
  \pauthor{【唐】杜甫}
  \vskip-3pt\begin{table}[H]
    \centering
    \begin{tabular}{@{}l@{}}
岱宗夫如何?齐鲁青未了。\\
造化钟神秀,阴阳割昏晓。\\
荡胸生曾云,决眦入归鸟。\\
会当凌绝顶,一览众山小。
    \end{tabular}
  \end{table}
\end{minipage}
\vspace{1cm}


\noindent\begin{minipage}{\linewidth}
  \ptitle{春望}
  \addcontentsline{toc}{section}{\makebox[9cm]{春望\dotfill{} 杜甫}}
  \pauthor{【唐】杜甫}
  \vskip-3pt\begin{table}[H]
    \centering
    \begin{tabular}{@{}l@{}}
国破山河在,城春草木深。\\
感时花溅泪,恨别鸟惊心。\\
烽火连三月,家书抵万金。\\
白头搔更短,浑欲不胜簪。
    \end{tabular}
  \end{table}
\end{minipage}
\vspace{1cm}


\noindent\begin{minipage}{\linewidth}
  \ptitle{蜀相}
  \addcontentsline{toc}{section}{\makebox[9cm]{蜀相\dotfill{} 杜甫}}
  \pauthor{【唐】杜甫}
  \vskip-3pt\begin{table}[H]
    \centering
    \begin{tabular}{@{}l@{}}
丞相祠堂何处寻,锦官城外柏森森。\\
映阶碧草自春色,隔叶黄鹂空好音。\\
三顾频烦天下计,两朝开济老臣心。\\
出师未捷身先死,长使英雄泪满襟。
    \end{tabular}
  \end{table}
\end{minipage}
\vspace{1cm}


\noindent\begin{minipage}{\linewidth}
  \ptitle{使至塞上}
  \addcontentsline{toc}{section}{\makebox[9cm]{使至塞上\dotfill{} 王维}}
  \pauthor{【唐】王维}
  \vskip-3pt\begin{table}[H]
    \centering
    \begin{tabular}{@{}l@{}}
单车欲问边,属国过居延。\\
征蓬出汉塞,归雁入胡天。\\
大漠孤烟直,长河落日圆。\\
萧关逢候骑,都护在燕然。
    \end{tabular}
  \end{table}
\end{minipage}
\vspace{1cm}


\noindent\begin{minipage}{\linewidth}
  \ptitle{酬乐天扬州初逢席上见赠}
  \addcontentsline{toc}{section}{\makebox[9cm]{酬乐天扬州初逢席上见赠\dotfill{} 刘禹锡}}
  \pauthor{【唐】刘禹锡}
  \vskip-3pt\begin{table}[H]
    \centering
    \begin{tabular}{@{}l@{}}
巴山楚水凄凉地,二十三年弃置身。\\
怀旧空吟闻笛赋,到乡翻似烂柯人。\\
沉舟侧畔千帆过,病树前头万木春。\\
今日听君歌一曲,暂凭杯酒长精神。
    \end{tabular}
  \end{table}
\end{minipage}
\vspace{1cm}


\noindent\begin{minipage}{\linewidth}
  \ptitle{浣溪沙}
  \addcontentsline{toc}{section}{\makebox[9cm]{浣溪沙\dotfill{} 晏殊}}
  \pauthor{【宋】晏殊}
  \vskip-3pt\begin{table}[H]
    \centering
    \begin{tabular}{@{}l@{}}
一曲新词酒一杯,去年天气旧亭台。夕阳西下几时回?\\
\\
无可奈何花落去,似曾相识燕归来。小园香径独徘徊。
    \end{tabular}
  \end{table}
\end{minipage}
\vspace{1cm}


\noindent\begin{minipage}{\linewidth}
  \ptitle{卜算子·送鲍浩然之浙东}
  \addcontentsline{toc}{section}{\makebox[9cm]{卜算子·送鲍浩然之浙东\dotfill{} 王观}}
  \pauthor{【宋】王观}
  \vskip-3pt\begin{table}[H]
    \centering
    \begin{tabular}{@{}l@{}}
水是眼波横,山是眉峰聚。欲问行人去那边?眉眼盈盈处。\\
\\
才始送春归,又送君归去。若到江南赶上春,千万和春住。
    \end{tabular}
  \end{table}
\end{minipage}
\vspace{1cm}


\noindent\begin{minipage}{\linewidth}
  \ptitle{临江仙}
  \addcontentsline{toc}{section}{\makebox[9cm]{临江仙\dotfill{} 杨慎}}
  \pauthor{【明】杨慎}
  \vskip-3pt\begin{table}[H]
    \centering
    \begin{tabular}{@{}l@{}}
滚滚长江东逝水,浪花淘尽英雄。\\
是非成败转头空。青山依旧在,几度夕阳红。\\
\\
白发渔樵江渚上,惯看秋月春风。\\
一壶浊酒喜相逢。古今多少事,都付笑谈中。
    \end{tabular}
  \end{table}
\end{minipage}
\vspace{1cm}


\noindent\begin{minipage}{\linewidth}
  \ptitle{满江红·怒发冲冠}
  \addcontentsline{toc}{section}{\makebox[9cm]{满江红·怒发冲冠\dotfill{} 岳飞}}
  \pauthor{【宋】岳飞}
  \vskip-3pt\begin{table}[H]
    \centering
    \begin{tabular}{@{}l@{}}
怒发冲冠,凭阑处、潇潇雨歇。\\
抬望眼,仰天长啸,壮怀激烈。\\
三十功名尘与土,八千里路云和月。\\
莫等闲,白了少年头,空悲切。\\
\\
靖康耻,犹未雪;臣子恨,何时灭?\\
驾长车,踏破贺兰山缺。\\
壮志饥餐胡虏肉,笑谈渴饮匈奴血。\\
待从头,收拾旧山河,朝天阙。
    \end{tabular}
  \end{table}
\end{minipage}
\vspace{1cm}


\noindent\begin{minipage}{\linewidth}
  \ptitle{破阵子·为陈同甫赋壮词以寄之}
  \addcontentsline{toc}{section}{\makebox[9cm]{破阵子·为陈同甫赋壮词以寄之\dotfill{} 辛弃疾}}
  \pauthor{【宋】辛弃疾}
  \vskip-3pt\begin{table}[H]
    \centering
    \begin{tabular}{@{}l@{}}
醉里挑灯看剑,梦回吹角连营。\\
八百里分麾下炙,五十弦翻塞外声,沙场秋点兵。\\
\\
马作的卢飞快,弓如霹雳弦惊。\\
了却君王天下事,赢得生前身后名。可怜白发生!
    \end{tabular}
  \end{table}
\end{minipage}
\vspace{1cm}


\noindent\begin{minipage}{\linewidth}
  \ptitle{念奴娇·赤壁怀古}
  \addcontentsline{toc}{section}{\makebox[9cm]{念奴娇·赤壁怀古\dotfill{} 苏轼}}
  \pauthor{【宋】苏轼}
  \vskip-3pt\begin{table}[H]
    \centering
    \begin{tabular}{@{}l@{}}
大江东去,浪淘尽,千古风流人物。\\
故垒西边,人道是,三国周郎赤壁。\\
乱石穿空,惊涛拍岸,卷起千堆雪。\\
江山如画,一时多少豪杰。\\
\\
遥想公瑾当年,小乔初嫁了,雄姿英发。\\
羽扇纶巾,谈笑间,樯橹灰飞烟灭。\\
故国神游,多情应笑我,早生华发。\\
人生如梦,一尊还酹江月。
    \end{tabular}
  \end{table}
\end{minipage}
\vspace{1cm}


